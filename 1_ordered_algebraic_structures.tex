\section{Estructuras algebráicas ordenadas}

  % Lemma 89
  \begin{lemma}
    \PN Sean $(P, \leq)$ y $(P^{\prime}, \leq^{\prime})$ posets. Supongamos que $F$ es un isomorfismo de $(P, \leq)$ en
    $(P^{\prime}, \leq^{\prime})$, entonces:

    \begin{enumerate}[a)]
      \item Para cada $S \subseteq P$ y cada $a \in P$, se tiene que $a$ es \textbf{cota superior} (resp.
        \textbf{inferior}) de $S$ si y solo si $F(a)$ es \textbf{cota superior} (resp. \textbf{inferior}) de $F(S)$.
      \item Para cada $S \subseteq P$, se tiene que $\exists \ \sup (S)$ si y solo si $\exists \ \sup (F(S))$ y en el
        caso de que existan tales elementos se tiene que $F(\sup (S)) = \sup (F(S))$.
      \item $P$ tiene $1$ (resp. $0$) si y solo si $P^{\prime }$ tiene $1$ (resp. $0$) y en tal caso tales elementos
        están conectados por $F$.
      \item Para cada $a \in P$, $a$ es \textbf{maximal} (resp. \textbf{minimal}) si y solo si $F(a)$ es
        \textbf{maximal} (resp. \textbf{minimal}).
      \item Para $a, b \in P$, tenemos que $a \prec b$ si y solo si $F(a) \prec^{\prime} F(b)$.
    \end{enumerate}
  \end{lemma}
  \begin{proof}
    \begin{enumerate}[a)]
      \item \begin{tabular}{|c|} \hline $\Rightarrow$ \\\hline \end{tabular} Supongamos que $a$ es \textbf{cota
        superior} de $S$, veamos entonces que $F(a)$ es \textbf{cota superior} de $F(S)$. Sean:
        \begin{itemize}
          \item $x \in F(S)$
          \item $s \in S$ tal que $x = F(s)$.
        \end{itemize}

        \PN Ya que $s \leq a$, tenemos que $x = F(s) \leq^{\prime} F(a)$. Luego, $F(a)$ es \textbf{cota superior}.

        \PN \begin{tabular}{|c|} \hline $\Leftarrow$ \\\hline \end{tabular} Supongamos ahora que $F(a)$ es \textbf{cota
        superior} de $F(S)$ y veamos que entonces $a$ es cota superior de $S$.

        \PN Sea $s \in S$, ya que $F(s) \leq^{\prime} F(a)$, tenemos que $s = F^{-1}(F(s)) \leq^{\prime} F^{-1}(F(a)) =
        a$. Por lo tanto, $a$ es \textbf{cota superior}.

      \item \begin{tabular}{|c|} \hline $\Rightarrow$ \\\hline \end{tabular} Supongamos existe $\sup (S)$. Veamos entonces que $F(\sup (S))$ es el supremo de $F(S)$. Por (a) $F(\sup (S))$ es cota superior de $F(S)$.
      Supongamos $b$ es cota superior de $F(S)$. Entonces $F^{-1}(b)$ es cota superior de $S$, por lo cual $\sup (S)\leq ^{\prime }F^{-1}(b)$, produciendo $F(\sup (S))\leq ^{\prime }b$.

      \PN \begin{tabular}{|c|} \hline $\Leftarrow$ \\\hline \end{tabular} En forma analoga se ve que si existe $\sup (F(S))$, entonces $F^{-1}(\sup (F(S)))$ es el supremo de $S$.

      \item Se desprende de (b) tomando $S=P$.
      \item son dejados como ejercicio
      \item son dejados como ejercicio
    \end{enumerate}
  \end{proof}

  % Lemma 90
  \begin{lemma}
    \PN Dado un reticulado \((L,\leq )\) y elementos \(x,y,z,w\in L\), se cumplen las siguientes.
    (1) \(x\leq x\) \(\mathsf{s}\) \(y\)
    (2) \(x\;\mathsf{i\;}y\leq x\)
    (3) \(x\;\mathsf{s}\;x=x\mathsf{\;i\;}x=x\)
    (4) \(x\;\mathsf{s}\;y=y\;\mathsf{s}\;x\)
    (5) \(x\mathsf{\;i\;}y=y\mathsf{\;i\;}x\)
    (6) \(x\leq y\) si y solo si \(x\;\mathsf{s}\;y=y\) si y solo si \(x \mathsf{\;i\;}y=x\)
    (7) \(x\;\mathsf{s}\;(x\mathsf{\;i\;}y)=x\)
    (8) \(x\mathsf{\;i\;}(x\;\mathsf{s}\;y)=x\)
    (9) \((x\;\mathsf{s}\;y)\;\mathsf{s}\;z=x\;\mathsf{s}\;(y\;\mathsf{s} \;z)\)
    (10) \((x\mathsf{\;i\;}y)\mathsf{\;i\;}z=x\mathsf{\;i\;}(y\mathsf{\;i\; }z)\)
    (11) Si \(x\leq z\) e \(y\leq w\), entonces \(x\;\mathsf{s}\ y\leq z\; \mathsf{s}\ w\) y \(x\mathsf{\;i\;}y\leq z\mathsf{\;i\;}w\)
    (12) \((x\mathsf{\;i\;}y)\;\mathsf{s}\;(x\mathsf{\;i\;}z)\leq x\mathsf{ \;i\;}(y\;\mathsf{s}\;z)\)
  \end{lemma}
  \begin{proof}
    (1), (2), (3), (4), (5) y (6) son consecuencias inmediatas de la definicion de las operaciones \(\mathsf{s}\) e \(\mathsf{i}\).

    (7) Ya que \(x\mathsf{\;i\;}y\leq x\), (6) nos dice que \((x\mathsf{\;i\;}y)\; \mathsf{s}\;x=x\), por lo cual \(x\;\mathsf{s}\;(x\mathsf{\;i\;}y)=x\).

    (8) Similar a (7).

    (9) Notese que \(x\;\mathsf{s}\;(y\;\mathsf{s}\;z)\) es cota superior de \( \{x,y,z\}\) ya que onviamente \(x\leq x\;\mathsf{s}\;(y\;\mathsf{s}\;z)\) y ademas

    \(\displaystyle \begin{array}{rcl} y & \leq & (y\;\mathsf{s}\;z)\leq x\;\mathsf{s}\;(y\;\mathsf{s}\;z) \\ z & \leq & (y\;\mathsf{s}\;z)\leq x\;\mathsf{s}\;(y\;\mathsf{s}\;z) \end{array} \)

    Ya que \(x\;\mathsf{s}\;(y\;\mathsf{s}\;z)\) es cota superior de \(\{x,y\}\), tenemos que \(x\;\mathsf{s}\;y\leq x\;\mathsf{s}\ (y\;\mathsf{s}\;z)\), por lo cual \(x\;\mathsf{s}\;(y\;\mathsf{s}\;z)\) es cota superior del conjunto \(\{x\; \mathsf{s}\;y,z\}\), lo cual dice que \((x\;\mathsf{s}\;y)\;\mathsf{s}\;z\leq x\;\mathsf{s}\;(y\;\mathsf{s}\;z)\). Analogamente se puede probar que \(x\; \mathsf{s}\;(y\;\mathsf{s}\;z)\leq (x\;\mathsf{s}\;y)\;\mathsf{s}\;z\).
    (10) Similar a (9).

    (11) Ya que

    \(\displaystyle \begin{array}{rcl} x & \leq & z\leq z\;\mathsf{s}\;w \\ y & \leq & w\leq z\;\mathsf{s}\;w \end{array} \)

    tenemos que \(z\;\mathsf{s}\;w\) es cota superior de \(\{x,y\}\) lo cual dice que \(x\;\mathsf{s}\;y\leq z\;\mathsf{s}\;w\). La otra desigualdad es analoga.
    (12) Ya que

    \(\displaystyle \begin{array}{rcl} (x\mathsf{\;i\;}y),(x\mathsf{\;i\;}z) & \leq & x \\ (x\mathsf{\;i\;}y),(x\mathsf{\;i\;}z) & \leq & y\;\mathsf{s}\;z \end{array} \)

    tenemos que \((x\;\mathsf{i}\;y),(x\mathsf{\;i\;}z)\leq x\mathsf{\;i\;}(y\; \mathsf{s}\;z)\), por lo cual \((x\mathsf{\;i\;}y)\;\mathsf{s}\;(x\mathsf{\;i\; }z)\leq x\mathsf{\;i\;}(y\;\mathsf{s}\;z)\). \(\Box\)
  \end{proof}

  % Lemma 91
  \begin{lemma}
    Sea \((L,\leq )\) un reticulado. Dados elementos \(x_{1},...,x_{n}\in L\), con \( n\geq 2\), se tiene
    \(\displaystyle \begin{array}{rcl} (...(x_{1}\;\mathsf{s\;}x_{2})\;\mathsf{s\;}...)\;\mathsf{s\;}x_{n} & =& \sup (\{x_{1},...,x_{n}\}) \\ (...(x_{1}\mathsf{\;i\;}x_{2})\mathsf{\;i\;}...)\mathsf{\;i\;}x_{n} & =& \inf (\{x_{1},...,x_{n}\}) \end{array} \)
  \end{lemma}
  \begin{proof}
    Por induccion en \(n\). Claramente el resultado vale para \(n=2\). Supongamos vale para \(n\) y veamos entonces que vale para \(n+1\). Sean \( x_{1},...,x_{n+1}\in L\). Por hipotesis inductiva tenemos que

    (1) \((...(x_{1}\;\mathsf{s}\;x_{2})\;\mathsf{s\;}...)\;\mathsf{s\;} x_{n}=\sup (\{x_{1},...,x_{n}\}).\)
    Veamos entonces que

    (2) \(((...(x_{1}\;\mathsf{s\;}x_{2})\;\mathsf{s\;}...)\;\mathsf{s\;} x_{n})\;\mathsf{s\;}x_{n+1}=\sup (\{x_{1},...,x_{n+1}\}).\)
    Es facil ver que \(((...(x_{1}\;\mathsf{s\;}x_{2})\;\mathsf{s\;} ...)\;\mathsf{s\;}x_{n})\;\mathsf{s\;}x_{n+1}\) es cota superior de \( \{x_{1},...,x_{n+1}\}\). Supongamos que \(z\) es otra cota superior. Ya que \(z\) es tambien cota superior del conjunto \(\{x_{1},...,x_{n}\}\), por (1) tenemos que

    \(\displaystyle (...(x_{1}\;\mathsf{s\;}x_{2})\;\mathsf{s}\;...)\;\mathsf{s\;}x_{n}\leq z. \)

    Pero entonces ya que \(x_{n+1}\leq z\), tenemos que
    \(\displaystyle ((...(x_{1}\;\mathsf{s\;}x_{2})\;\mathsf{s\;}...)\;\mathsf{s\;}x_{n})\; \mathsf{s\;}x_{n+1}\leq z, \)

    con lo cual hemos probado (2). \(\Box\)
  \end{proof}

  % Theorem 92
  \begin{theorem}
    Sea \((L,\mathsf{s},\mathsf{i})\) un reticulado. La relacion binaria definida por:
    \(\displaystyle x\leq y\text{ si y solo si }x\;\mathsf{s}\;y=y \)

    es un orden parcial sobre \(L\) para el cual se cumple:
    \(\displaystyle \begin{array}{rcl} \sup (\{x,y\}) & =& x\;\mathsf{s}\;y \\ \inf (\{x,y\}) & =& x\mathsf{\;i\;}y \end{array} \)
  \end{theorem}
  \begin{proof}
    Dejamos como ejercicio para el lector probar que \(\leq \) es reflexiva y antisimetrica. Veamos que \(\leq \) es transitiva. Supongamos que \(x\leq y\) e \( y\leq z\). Entonces

    \(\displaystyle x\;\mathsf{s\;}z=x\;\mathsf{s\;}(y\;\mathsf{s\;}z)=(x\;\mathsf{s\;}y)\; \mathsf{s\;}z=y\;\mathsf{s\;}z=z, \)

    por lo cual \(x\leq z\). Veamos ahora que \(\sup (\{x,y\})=x\;\mathsf{s\;}y\). Es claro que \(x\;\mathsf{s\;}y\) es una cota superior del conjunto \(\{x,y\}\). Supongamos \(x,y\leq z\). Entonces
    \(\displaystyle (x\;\mathsf{s\;}y)\;\mathsf{s\;}z=x\;\mathsf{s\;}(y\;\mathsf{s\;}z)=x\; \mathsf{s\;}z=z, \)

    por lo que \(x\;\mathsf{s\;}y\leq z\). Es decir que \(x\;\mathsf{s\;}y\) es la menor cota superior.
    Para probar que \(\inf (\{x,y\})=x\mathsf{\;i\;}y\), probaremos que para todo \( u,v\in L\),

    \(\displaystyle u\leq v\text{ si y solo si }u\mathsf{\;i\;}v=u, \)

    lo cual le permitira al lector aplicar un razonamiento similar al usado en el caso de la operacion \(\mathsf{s}\). Supongamos que \(u\;\mathsf{s}\;v=v\). Entonces \(u\mathsf{\;i\;}v=u\mathsf{\;i\;}(v\;\mathsf{s}\;v)=u\). Reciprocamente si \(u\mathsf{\;i\;}v=u\), entonces \(u\;\mathsf{s}\;v=(u\mathsf{ \;i\;}v)\;\mathsf{s}\;v=v\), por lo cual \(u\leq v\). \(\Box\)
  \end{proof}

  % Lemma 93
  \begin{lemma}
    Si \(F:(L,\mathsf{s},\mathsf{i})\rightarrow (L^{\prime },\mathsf{s}^{\prime },\mathsf{i}^{\prime })\) es un homomorfismo biyectivo, entonces \(F\) es un isomorfismo
  \end{lemma}
  \begin{proof}
    Solo falta ver que \(F^{-1}\) es un homomorfismo. Sean \(F(x),F(y)\) dos elementos cualesquiera de \(L^{\prime }\). Tenemos que

    \(\displaystyle F^{-1}(F(x)\;\mathsf{s}^{\prime }\ F(y))=F^{-1}(F(x\mathsf{\;s\;}y))=x \mathsf{\;s\;}y=F^{-1}(F(x))\;\mathsf{s}\ F^{-1}(F(y)) \)

    \(\Box\)
  \end{proof}

  % Lemma 94
  \begin{lemma}
    Sean \((L,\mathsf{s},\mathsf{i})\) y \((L^{\prime },\mathsf{s}^{\prime }, \mathsf{i}^{\prime })\) reticulados y sea \(F:(L,\mathsf{s},\mathsf{i} )\rightarrow (L^{\prime },\mathsf{s}^{\prime },\mathsf{i}^{\prime })\) un homomorfismo. Entonces \(I_{F}\) es un subuniverso de \((L^{\prime },\mathsf{s} ^{\prime },\mathsf{i}^{\prime })\).
  \end{lemma}
  \begin{proof}
    Ya que \(L\) es no vacio tenemos que \(I_{F}\) tambien es no vacio. Sean \(a,b\in I_{F}\). Sean \(x,y\in L\) tales que \(F(x)=a\) y \(F(y)=b\). Se tiene que

    \(\displaystyle \begin{array}{rcl} a\;\mathsf{s}^{\prime }\ b & =& F(x)\;\mathsf{s}^{\prime }\ F(y)=F(x\mathsf{ \;s\;}y)\in I_{F} \\ a\;\mathsf{i}^{\prime }\ b & =& F(x)\;\mathsf{i}^{\prime }\ F(y)=F(x\mathsf{ \;i\;}y)\in I_{F} \end{array} \)

    por lo cual \(I_{F}\) es cerrada bajo \(\mathsf{s}^{\prime }\) e \(\mathsf{i} ^{\prime }\). \(\Box\)
  \end{proof}

  % Lemma 95
  \begin{lemma}
    Sean \((L,\mathsf{s},\mathsf{i})\) y \((L^{\prime },\mathsf{s}^{\prime }, \mathsf{i}^{\prime })\) reticulados y sean \((L\leq )\) y \((L^{\prime },\leq ^{\prime })\) los posets asociados. Sea \(F:L\rightarrow L^{\prime }\) una funcion. Entonces \(F\) es un isomorfismo de \((L,\mathsf{s},\mathsf{i})\) en \( (L^{\prime },\mathsf{s}^{\prime },\mathsf{i}^{\prime })\) si y solo si \(F\) es un isomorfismo de \((L,\leq )\) en \((L^{\prime },\leq ^{\prime })\).
  \end{lemma}
  \begin{proof}
    Supongamos \(F\) es un isomorfismo de \((L,\mathsf{s},\mathsf{i})\) en \( (L^{\prime },\mathsf{s}^{\prime },\mathsf{i}^{\prime })\). Sean \(x,y\in L\), tales que \(x\leq y\). Tenemos que \(y=x\mathsf{\;s\;}y\) por lo cual \(F(y)=F(x \mathsf{\;s\;}y)=F(x)\mathsf{\;s^{\prime }\;}F(y)\), produciendo \(F(x)\leq ^{\prime }F(y)\). En forma similar se puede ver que \(F^{-1}\) es tambien un homomorfismo de \((L^{\prime },\leq ^{\prime })\) en \((L,\leq )\). Si \(F\) es un isomorfismo de \((L,\leq )\) en \((L^{\prime },\leq ^{\prime })\), entonces el Lema 89 nos dice que \(F\) y \(F^{-1}\) respetan las operaciones de supremo e infimo por lo cual \(F\) es un isomorfismo de \((L,\mathsf{s},\mathsf{ i})\) en \((L^{\prime },\mathsf{s}^{\prime },\mathsf{i}^{\prime })\). \(\Box\)
  \end{proof}

  % Lemma 96
  \begin{lemma}
    \((L/\theta ,\mathsf{\tilde{s}},\mathsf{\tilde{\imath}})\) es un reticulado. El orden parcial \(\tilde{\leq}\) asociado a este reticulado cumple
    \(\displaystyle x/\theta \tilde{\leq}y/\theta \text{ sii }y\theta (x\mathsf{\;s\;}y) \)
  \end{lemma}
  \begin{proof}
    Veamos que la estructura \((L/\theta ,\mathsf{\tilde{s}},\mathsf{\tilde{\imath }})\) cumple (I4). Sean \(x/\theta \), \(y/\theta \), \(z/\theta \) elementos cualesquiera de \(L/\theta \). Tenemos que

    \(\displaystyle \begin{array}{ccl} (x/\theta \mathsf{\;\tilde{s}\;}y/\theta )\;\mathsf{\tilde{s}}\;z/\theta & = & (x\mathsf{\;s\;}y)/\theta \;\mathsf{\tilde{s}}\;z/\theta \\ & = & ((x\mathsf{\;s\;}y)\;\mathsf{s}\;z)/\theta \\ & = & (x\mathsf{\;s\;}(y\;\mathsf{s}\;z))/\theta \\ & = & x/\theta \;\mathsf{\tilde{s}}\;(y\;\mathsf{s}\;z)/\theta \\ & = & x/\theta \mathsf{\;\tilde{s}\;}(y/\theta \;\mathsf{\tilde{s}} \;z/\theta ) \end{array} \)

    En forma similar se puede ver que la estructura \((L/\theta ,\mathsf{\tilde{s} },\mathsf{\tilde{\imath}})\) cumple el resto de las identidades que definen reticulado.
  \end{proof}

  % Corollary 97
  \begin{corollary}
    Sea \((L,\mathsf{s},\mathsf{i})\) un reticulado en el cual hay un elemento maximo \(1\) (resp. minimo \(0\)). Entonces si \(\theta \) es una congruencia sobre \((L,\mathsf{s},\mathsf{i})\), \(1/\theta \) (resp. \(0/\theta \)) es un elemento maximo (resp. minimo) de \((L/\theta ,\mathsf{\tilde{s}},\mathsf{ \tilde{\imath}})\).
  \end{corollary}
  \begin{proof}
    Ya que \(1\theta (x\mathsf{\;s\;}1)\), para cada \(x\in L\), tenemos que \( x/\theta \tilde{\leq}1/\theta \), para cada \(x\in L\). \(\Box\)
  \end{proof}

  % Lemma 98
  \begin{lemma}
    Si \(F:(L,\mathsf{s},\mathsf{i})\rightarrow (L^{\prime },\mathsf{s}^{\prime }, \mathsf{i}^{\prime })\) es un homomorfismo de reticulados, entonces \(\ker F\) es una congruencia sobre \((L,\mathsf{s},\mathsf{i})\).
  \end{lemma}
  \begin{proof}
    Dejamos al lector ver que \(\ker F\) es una relacion de equivalencia. Supongamos \(x\ker Fx^{\prime }\) y \(y\ker Fy^{\prime }\). Entonces

    \(\displaystyle F(x\mathsf{\;s\;}y)=F(x)\mathsf{\;s^{\prime }\;}F(y)=F(x^{\prime })\mathsf{ \;s^{\prime }\;}F(y^{\prime })=F(x^{\prime }\mathsf{\;s\;}y^{\prime }) \)

    lo cual nos dice que \((x\mathsf{\;s\;}y)\ker F(x^{\prime }\mathsf{\;s\;} y^{\prime })\). En forma similar tenemos que \((x\mathsf{\;i\;}y)\ker F(x^{\prime }\mathsf{\;i\;}y^{\prime })\). \(\Box\)
  \end{proof}

  % Lemma 99
  \begin{lemma}
    Sea \((L,\mathsf{s},\mathsf{i})\) un reticulado y sea \(\theta \) una congruencia sobre \((L,\mathsf{s},\mathsf{i})\). Entonces \(\pi _{\theta }\) es un homomorfismo de \((L,\mathsf{s},\mathsf{i})\) en \((L/\theta ,\mathsf{\tilde{ s}},\mathsf{\tilde{\imath}})\). Ademas \(\ker \pi _{\theta }=\theta \).
  \end{lemma}
  \begin{proof}
    Sean \(x,y\in L\). Tenemos que

    \(\displaystyle \pi _{\theta }(x\mathsf{\;s\;}y)=(x\mathsf{\;s\;}y)/\theta =x/\theta \mathsf{ \;\tilde{s}\;}y/\theta =\pi _{\theta }(x)\mathsf{\;\tilde{s}\;}\pi _{\theta }(y) \)

    por lo cual \(\pi _{\theta }\) preserva la operacion supremo. Para la operacion infimo es similar. \(\Box\)
  \end{proof}

  % Lemma 100
  \begin{lemma}
    Si \(F:(L,\mathsf{s},\mathsf{i},0,1)\rightarrow (L^{\prime },\mathsf{s} ^{\prime },\mathsf{i}^{\prime },0^{\prime },1^{\prime })\) un homomorfismo biyectivo, entonces \(F\) es un isomorfismo
  \end{lemma}
  \begin{proof}
    Similar a la prueba del Lemma 93. \(\Box\)
  \end{proof}

  % Lemma 101
  \begin{lemma}
    Si \(F:(L,\mathsf{s},\mathsf{i},0,1)\rightarrow (L^{\prime },\mathsf{s} ^{\prime },\mathsf{i}^{\prime },0^{\prime },1^{\prime })\) es un homomorfismo, entonces \(I_{F}\) es un subuniverso de \((L^{\prime },\mathsf{s} ^{\prime },\mathsf{i}^{\prime },0^{\prime },1^{\prime })\).
  \end{lemma}
  \begin{proof}
    Ya que \(F\) es un homomorfismo de \((L,\mathsf{s},\mathsf{i})\) en \( (L^{\prime },\mathsf{s}^{\prime },\mathsf{i}^{\prime })\) tenemos que \(I_{F}\) es subuniverso de \((L^{\prime },\mathsf{s}^{\prime },\mathsf{i}^{\prime })\) lo cual ya que \(0^{\prime },1^{\prime }\in I_{F}\) implica que \(I_{F}\) es un subuniverso de \((L^{\prime },\mathsf{s}^{\prime },\mathsf{i}^{\prime },0^{\prime },1^{\prime })\). \(\Box\)
  \end{proof}

  % Lemma 102
  \begin{lemma}
    Si \(F:(L,\mathsf{s},\mathsf{i},0,1)\rightarrow (L^{\prime },\mathsf{s} ^{\prime },\mathsf{i}^{\prime },0^{\prime },1^{\prime })\) es un homomorfismo de reticulados acotados, entonces \(\ker F\) es una congruencia sobre \((L, \mathsf{s},\mathsf{i},0,1)\).
  \end{lemma}

  % Lemma 103
  \begin{lemma}
    Sea \((L,\mathsf{s},\mathsf{i},0,1)\) un reticulado acotado y \(\theta \) una congruencia sobre \((L,\mathsf{s},\mathsf{i},0,1)\).
    (a) \((L/\theta ,\mathsf{\tilde{s}},\mathsf{\tilde{\imath}},0/\theta ,1/\theta )\) es un reticulado acotado.
    (b) \(\pi _{\theta }\) es un homomorfismo de \((L,\mathsf{s},\mathsf{i} ,0,1)\) en \((L/\theta ,\mathsf{\tilde{s}},\mathsf{\tilde{\imath}},0/\theta ,1/\theta )\) cuyo nucleo es \(\theta \).
  \end{lemma}

  % Lemma 104
  \begin{lemma}
    Si \(F:(L,\mathsf{s},\mathsf{i},^{c},0,1)\rightarrow (L^{\prime },\mathsf{s} ^{\prime },\mathsf{i}^{\prime },^{c^{\prime }},0^{\prime },1^{\prime })\) un homomorfismo biyectivo, entonces \(F\) es un isomorfismo
  \end{lemma}

  % Lemma 105
  \begin{lemma}
    Si \(F:(L,\mathsf{s},\mathsf{i},^{c},0,1)\rightarrow (L^{\prime },\mathsf{s} ^{\prime }\),\(\mathsf{i}^{\prime },^{c^{\prime }},0^{\prime },1^{\prime })\) es un homomorfismo, entonces \(I_{F}\) es un subuniverso de \((L^{\prime }, \mathsf{s}^{\prime }\),\(\mathsf{i}^{\prime },^{c^{\prime }},0^{\prime },1^{\prime })\).
  \end{lemma}

  % Lemma 106
  \begin{lemma}
    Si \(F:(L,\mathsf{s},\mathsf{i},^{c},0,1)\rightarrow (L^{\prime },\mathsf{s} ^{\prime },\mathsf{i}^{\prime },^{c^{\prime }},0^{\prime },1^{\prime })\) es un homomorfismo de reticulados complementados, entonces \(\ker F\) es una congruencia sobre \((L,\mathsf{s},\mathsf{i},^{c},0,1)\)
  \end{lemma}

  % Lema 107
  \begin{lemma}
    Sea \((L,\mathsf{s},\mathsf{i},^{c},0,1)\) un reticulado complementado y sea \( \theta \) una congruencia sobre \((L,\mathsf{s},\mathsf{i},^{c},0,1)\).
    (a) \((L/\theta ,\mathsf{\tilde{s}},\mathsf{\tilde{\imath}},^{\tilde{c} },0/\theta ,1/\theta )\) es un reticulado complementado.
    (b) \(\pi _{\theta }\) es un homomorfismo de \((L,\mathsf{s},\mathsf{i} ,^{c},0,1)\) en \((L/\theta ,\mathsf{\tilde{s}},\mathsf{\tilde{\imath}},^{ \tilde{c}},0/\theta ,1/\theta )\) cuyo nucleo es \(\theta \).
  \end{lemma}

  % Lemma 108
  \begin{lemma}
    Sea \((L,\mathsf{s},\mathsf{i})\) un reticulado. Son equivalentes:
    (1) \(x\mathsf{\;i\;}(y\;\mathsf{s}\;z)=(x\mathsf{\;i\;}y)\;\mathsf{s} \;(x\mathsf{\;i\;}z)\), cualesquiera sean \(x,y,z\in L\)
    (2) \(x\;\mathsf{s}\;(y\mathsf{\;i\;}z)=(x\mathsf{\;s\;}y)\mathsf{\;i\; }(x\;\mathsf{s}\;z)\), cualesquiera sean \(x,y,z\in L\).
  \end{lemma}
  \begin{proof}
    (1)\(\Rightarrow \)(2). Notese que

    \(\displaystyle \begin{array}{rcl} (x\mathsf{\;s\;}y)\mathsf{\;i\;}(x\;\mathsf{s}\;z) & =& ((x\mathsf{\;s\;}y) \mathsf{\;i\;}x)\;\mathsf{s}\;((x\mathsf{\;s\;}y)\mathsf{\;i\;}z) \\ & =& (x\;\mathsf{s}\;(z\mathsf{\;i\;}(x\mathsf{\;s\;}y)) \\ & =& (x\;\mathsf{s}\;((z\;\mathsf{i\;}x)\mathsf{\;s\;}(z\;\mathsf{i\;}y)) \\ & =& (x\;\mathsf{s}\;(z\;\mathsf{i\;}x))\mathsf{\;s\;}(z\;\mathsf{i\;}y) \\ & =& x\mathsf{\;s\;}(z\;\mathsf{i\;}y) \\ & =& x\mathsf{\ s\ }(y\ \mathsf{i\ }z) \end{array} \)

    (2)\(\Rightarrow \)(1) es similar. \(\Box\)
  \end{proof}

  % Lemma 109
  \begin{lemma}
    Si \((L,\mathsf{s},\mathsf{i},0,1)\) un reticulado acotado y distributivo, entonces todo elemento tiene a lo sumo un complemento.
  \end{lemma}
  \begin{proof}
    Supongamos \(x\in L\) tiene complementos \(y,z\). Se tiene

    \(\displaystyle y=y\;\mathsf{i\;}1=y\;\mathsf{i\;}(x\;\mathsf{s\;}z)=(y\;\mathsf{i\;}x)\; \mathsf{s\;}(y\;\mathsf{i\;}z)=0\;\mathsf{s\;}(y\;\mathsf{i\;}z)=y\;\mathsf{ i\;}z, \)

    por lo cual \(y\leq z\). En forma analoga se muestra que \(z\leq y\). \(\Box\)
  \end{proof}

  % Lemma 110
  \begin{lemma}
    Si \(S\) es no vacio, entonces \([S)\) es un filtro. Mas aun si \(F\) es un filtro y \(F\supseteq S\), entonces \(F\supseteq \lbrack S)\).
  \end{lemma}
  \begin{proof}
    Ya que \(S\subseteq \lbrack S)\), tenemos que \([S)\neq \varnothing \). Claramente \([S)\) cumple la propiedad (3). Veamos cumple la (2). Si \(y\geq s_{1}\; \mathsf{i\;}s_{2}\;\mathsf{i\;}...\;\mathsf{i\;}s_{n}\) y \(z\geq t_{1}\; \mathsf{i\;}t_{2}\;\mathsf{i\;}\)...\(\;\mathsf{i\;}t_{m}\), con \( s_{1},s_{2},...,s_{n}\), \(t_{1},t_{2},...,t_{m}\in S\), entonces

    \(\displaystyle y\;\mathsf{i\;}z\geq s_{1}\;\mathsf{i\;}s_{2}\;\mathsf{i\;}...\;\mathsf{i\;} s_{n}\;\mathsf{i\;}t_{1}\;\mathsf{i\;}t_{2}\;\mathsf{i\;}...\;\mathsf{i\;} t_{m}, \)

    lo cual prueba (2).
  \end{proof}

  % Lemma 111
  \begin{lemma}
    (Zorn) Sea \((P,\leq )\) un poset y supongamos cada cadena de \(P\) tiene una cota superior. Entonces hay un elemento maximal en \(P\).
    Un filtro \(F\) de un reticulado \((L,\mathsf{s},\mathsf{i})\) sera llamado primo cuando se cumplan:

    (1) \(F\neq L\)
    (2) \(x\;\mathsf{s\;}y\in F\Rightarrow x\in F\) o \(y\in F\).
  \end{lemma}

  % Theorem 112
  \begin{theorem}
    (Teorema del Filtro Primo) Sea \((L,\mathsf{s},\mathsf{i})\) un reticulado distributivo y \(F\) un filtro. Supongamos \(x_{0}\in L-F\). Entonces hay un filtro primo \(P\) tal que \(x_{0}\notin P\) y \(F\subseteq P\).
  \end{theorem}
  \begin{proof}
    Sea

    \(\displaystyle \mathcal{F}=\{F_{1}:F_{1}\text{ es un filtro, }x_{0}\notin F_{1}\text{ y } F\subseteq F_{1}\}. \)

    Notese que \(\mathcal{F}\neq \varnothing \), por lo cual \((\mathcal{F},\subseteq )\) es un poset. Veamos que cada cadena en \((\mathcal{F},\subseteq )\) tiene una cota superior. Sea \(C\) una cadena. Si \(C=\varnothing \), entonces cualquier elemento de \(\mathcal{F}\) es cota de \(C\). Supongamos entonces \(C\neq \varnothing \). Sea
    \(\displaystyle G=\{x\in L:x\in F_{1},\text{para algun }F_{1}\in C\}. \)

    Veamos que \(G\) es un filtro. Es claro que \(G\) es no vacio. Supongamos que \( x,y\in G\). Sean \(F_{1},F_{2}\in \mathcal{F}\) tales que \(x\in F_{1}\) y \(y\in F_{2}\). Si \(F_{1}\subseteq F_{2}\), entonces ya que \(F_{2}\) es un filtro tenemos que \(x\;\mathsf{i\;}y\in F_{2}\subseteq G\). Si \(F_{2}\subseteq F_{1}\) , entonces tenemos que \(x\;\mathsf{i\;}y\in F_{1}\subseteq G\). Ya que \(C\) es una cadena, tenemos que siempre \(x\;\mathsf{i\;}y\in G\). En forma analoga se prueba la propiedad restante por lo cual tenemos que \(G\) es un filtro. Ademas \(x_{0}\notin G\), por lo que \(G\in \mathcal{F}\) es cota superior de \(C\) . Por el lema de Zorn, \((\mathcal{F},\subseteq )\) tiene un elemento maximal \( P\). Veamos que \(P\) es un filtro primo. Supongamos \(x\;\mathsf{s\;}y\in P\) y \( x,y\notin P\). Entonces ya que \(P\) es maximal tenemos que
    \(\displaystyle x_{0}\in \lbrack P\cup \{x\})\cap \lbrack P\cup \{y\}) \)

    Ya que \(x_{0}\in \lbrack P\cup \{x\})\), tenemos que hay elementos \( p_{1},...,p_{n}\in P\), tales que
    \(\displaystyle x_{0}\geq p_{1}\;\mathsf{i\;}...\;\mathsf{i\;}p_{n}\;\mathsf{i\;}x \)

    Ya que \(x_{0}\in \lbrack P\cup \{y\})\), tenemos que hay elementos \( q_{1},...,q_{m}\in P\), tales que
    \(\displaystyle x_{0}\geq q_{1}\;\mathsf{i\;}...\;\mathsf{i\;}q_{m}\;\mathsf{i\;}y \)

    Si llamamos \(p\) al siguiente elemento de \(P\)
    \(\displaystyle p_{1}\;\mathsf{i\;}...\;\mathsf{i\;}p_{n}\;\mathsf{i\;}q_{1}\;\mathsf{i\;} ...\;\mathsf{i\;}q_{m} \)

    tenemos que
    \(\displaystyle \begin{array}{rcl} x_{0} & \geq & p\;\mathsf{i\;}x \\ x_{0} & \geq & p\;\mathsf{i\;}y \end{array} \)

    Se tiene que \(x_{0}\geq (p\;\mathsf{i\;}x)\;\mathsf{s\;}(p\;\mathsf{i\;} y)=p\;\mathsf{i\;}(x\;\mathsf{s\;}y)\in P\), lo cual es absurdo ya que \( x_{0}\notin P\). \(\Box\)
  \end{proof}

  % Corollary 113
  \begin{corollary}
    Sea \((L,\mathsf{s},\mathsf{i},0,1)\) un reticulado acotado distributivo. Si \( \varnothing \neq S\subseteq L\) es tal que \(s_{1}\;\mathsf{i\;}s_{2}\;\mathsf{ i\;}...\;\mathsf{i\;}s_{n}\neq 0\), para cada \(s_{1},...,s_{n}\in S\), entonces hay un filtro primo que contiene a \(S\).
  \end{corollary}
  \begin{proof}
    Notese que \([S)\neq L\) por lo cual se puede aplicar el Teorema del filtro primo. \(\Box\)
  \end{proof}

  % Lemma 114
  \begin{lemma}
    Sea \((B,\mathsf{s},\mathsf{i},^{c},0,1)\) un algebra de Boole. Entonces para un filtro \(F\subseteq B\) las siguientes son equivalentes:
    (1) \(F\) es primo
    (2) \(x\in F\) o \(x^{c}\in F\), para cada \(x\in B\).
  \end{lemma}
  \begin{proof}
    (1)\(\Rightarrow \)(2). Ya que \(x\;\mathsf{s\;}x^{c}=1\in F\), (2) se cumple si \(F\) es primo.

    (2)\(\Rightarrow \)(1). Supongamos que \(x\;\mathsf{s\;}y\in F\) y que \(x\not\in F\). Entonces por (2), \(x^{c}\in F\) y por lo tanto tenemos que

    \(\displaystyle y\geq x^{c}\;\mathsf{i\;}y=(x^{c}\;\mathsf{i\;}x)\;\mathsf{s\;}(x^{c}\; \mathsf{i\;}y)=x^{c}\;\mathsf{i\;}(x\;\mathsf{s\;}y)\in F, \)

    lo cual dice que \(y\in F\). \(\Box\)
  \end{proof}

  % Lemma 115
  \begin{lemma}
    Sea \((B,\mathsf{s},\mathsf{i},^{c},0,1)\) un algebra de Boole. Supongamos que \(b\neq 0\) y \(a=\inf A\), con \(A\subseteq B\). Entonces si \(b\;\mathsf{i\;}a=0\) , existe un \(e\in A\) tal que \(b\;\mathsf{i\;}e^{c}\neq 0\).
  \end{lemma}
  \begin{proof}
    Supongamos que para cada \(e\in A\), tengamos que \(b\;\mathsf{i\;}e^{c}=0\). Entonces tenemos que para cada \(e\in A\),

    \(\displaystyle b=b\;\mathsf{i\;}(e\;\mathsf{s\;}e^{c})=(b\;\mathsf{i\;}e)\;\mathsf{s\;}(b\; \mathsf{i\;}e^{c})=b\;\mathsf{i\;}e, \)

    lo cual nos dice que \(b\) es cota inferior de \(A\). Pero entonces \(b\leq a\), por lo cual \(b=b\;\mathsf{i\;}a=0\), contradiciendo la hipotesis. \(\Box\)
  \end{proof}

  % Theorem 116
  \begin{theorem}
    (Rasiova y Sikorski) Sea \((B,\mathsf{s},\mathsf{i},^{c},0,1)\) un algebra de Boole. Sea \(x\in B\), \(x\neq 0\). Supongamos que \(A_{1},A_{2},...\) son subconjuntos de \(B\) tales que existe \(\inf (A_{j})\), para cada \(j=1,2....\) Entonces hay un filtro primo \(P\) el cual cumple:
    (a) \(x\in P\)
    (b) \(P\supseteq A_{j}\Rightarrow P\ni \inf (A_{j})\), para cada \( j=1,2,....\)
  \end{theorem}
  \begin{proof}
    Sean \(a_{j}=\inf (A_{j})\), \(j=1,2,...\). Construiremos inductivamente una sucesion \(b_{0},b_{1},...\) de elementos de \(B\) tal que:

    (1) \(b_{0}=x\)
    (2) \(b_{0}\;\mathsf{i\;}\)...\(\;\mathsf{i\;}b_{n}\neq 0\), para cada \( n\geq 0\)
    (3) \(b_{j}=a_{j}\) o \(b_{j}^{c}\in A_{j}\), para cada \(j\geq 1\).
    Definamos \(b_{0}=x\). Supongamos ya definimos \(b_{0},...,b_{n}\), veamos como definir \(b_{n+1}\). Si \((b_{0}\;\mathsf{i\;}...\;\mathsf{i\;} b_{n})\;\mathsf{i\;}a_{n+1}\neq 0\), entonces definamos \(b_{n+1}=a_{n+1}\). Si \((b_{0}\;\mathsf{i\;}...\;\mathsf{i\;}b_{n})\;\mathsf{i\;}a_{n+1}=0\), entonces por el lema anterior, tenemos que hay un \(e\in A_{n+1}\) tal que \( (b_{0}\;\mathsf{i\;}...\;\mathsf{i\;}b_{n})\;\mathsf{i\;}e^{c}\neq 0\), lo cual nos permite definir \(b_{n+1}=e^{c}\).

    Usando (2) se puede probar que el conjunto \(S=\{b_{0},b_{1},...\}\) satisface la hipotesis del primer corolario del Teorema del filtro primo, por lo cual hay un filtro primo \(P\) tal que \(\{b_{0},b_{1},...\}\subseteq P\). Es facil chequear que \(P\) satisface las propiedades (a) y (b). \(\Box\)
  \end{proof}
