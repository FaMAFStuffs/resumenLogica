\section{Estructuras algebráicas ordenadas}

  % Lemma 91: Con prueba. Lemma 1.
  \begin{lemma} \label{lemma_1}
    \PN Sean $\POSET$ y $\POSETPRIMO$ posets. Supongamos que $F$ es un isomorfismo de $\POSET$ en $\POSETPRIMO$,
    entonces:
    \begin{enumerate}[a)]
      \item Para cada $S \subseteq P$ y cada $a \in P$, se tiene que $a$ es \textbf{cota superior} (resp.
        \textbf{inferior}) de $S$ si y solo si $F(a)$ es \textbf{cota superior} (resp. \textbf{inferior}) de $F(S)$.
      \item Para cada $S \subseteq P$, se tiene que existe $\sup (S)$ si y solo si existe $\sup (F(S))$ y en el
        caso de que existan tales elementos se tiene que $F(\sup (S)) = \sup (F(S))$.
      \item $P$ tiene $1$ (resp. $0$) si y solo si $P^{\prime}$ tiene $1$ (resp. $0$) y en tal caso tales elementos
        están conectados por $F$.
      \item Para cada $m \in P$, $m$ es \textbf{maximal} (resp. \textbf{minimal}) si y solo si $F(m)$ es
        \textbf{maximal} (resp. \textbf{minimal}).
      \item Para $a, b \in P$, tenemos que $a \prec b$ si y solo si $F(a) \prec^{\prime} F(b)$.
    \end{enumerate}
  \end{lemma}
  \begin{proof}
    \begin{enumerate}[a)]
      \item Probaremos solo el caso de la \textbf{cota superior}.
        \PN \begin{tabular}{|c|} \hline $\Rightarrow$ \\\hline \end{tabular} Supongamos que $a$ es \textbf{cota
        superior} de $S$, veamos entonces que $F(a)$ es \textbf{cota superior} de $F(S)$. Sean:
        \begin{itemize}
          \item $x \in F(S)$
          \item $s \in S$ tal que $x = F(s)$.
        \end{itemize}

        \PN Ya que $s \leq a$, tenemos que $x = F(s) \leq^{\prime} F(a)$. Luego, $F(a)$ es \textbf{cota superior}.

        \PN \begin{tabular}{|c|} \hline $\Leftarrow$ \\\hline \end{tabular} Supongamos ahora que $F(a)$ es \textbf{cota
        superior} de $F(S)$ y veamos entonces que $a$ es cota superior de $S$.

        \PN Sea $s \in S$, ya que $F(s) \leq^{\prime} F(a)$, tenemos que $s = F^{-1}(F(s)) \leq^{\prime} F^{-1}(F(a)) =
        a$. Por lo tanto, $a$ es \textbf{cota superior}.

      \item \begin{tabular}{|c|} \hline $\Rightarrow$ \\\hline \end{tabular} Supongamos existe $\sup (S)$. Veamos
        que $F(\sup (S))$ es el supremo de $F(S)$. Por el iniciso (a) $F(\sup (S))$ es cota superior de $F(S)$. Veamos
        que es la menor de las cotas superiores. Supongamos $b^{\prime}$ cota superior de $F(S)$, entonces
        $F^{-1}(b^{\prime})$ es cota superior de $S$, es decir, $\sup (S) \leq F^{-1}(b^{\prime})$, produciendo
        $F(\sup (S)) \leq^{\prime} b^{\prime}$. Por lo tanto, $F(\sup (S))$ es el supremo de $F(S)$.

        \PN \begin{tabular}{|c|} \hline $\Leftarrow$ \\\hline \end{tabular} Supongamos existe $\sup (F(S))$. Veamos
        que $F^{-1}(\sup (F(S)))$ es el supremo de $S$. Nuevamente, por el iniciso (a) $F^{-1}(\sup (F(S)))$ es cota
        superior de $S$. Veamos que es la menor de las cotas superiores. Supongamos $b$ cota superior de $S$, entonces
        $F(b)$ es cota superior de $F(S)$, es decir, $\sup (F(S)) \leq F(b)$, produciendo $F^{-1}(\sup (F(S))) \leq b$.
        Por lo tanto, $F^{-1}(\sup (F(S)))$ es el supremo de $S$.

      \item Se desprende del inciso (b) tomando $S = P$.
      \item Probaremos solo el caso \textbf{maximal}.
        \PN \begin{tabular}{|c|} \hline $\Rightarrow$ \\\hline \end{tabular} Supongamos que $m$ es maximal de
        $\POSET$. Veamos que $F(m)$ es maximal de $\POSETPRIMO$. Supongamos que $F(m)$ no es maximal
        de $\POSETPRIMO$, es decir, $\exists b^{\prime} \in P^{\prime} \ F(m) <^{\prime} b^{\prime}$.
        Dado que $F$ es isomorfismo:
        \begin{eqnarray*}
    			F^{-1}(F(m)) < F^{-1}(b^{\prime}) \\
    			m < F^{-1}(b^{\prime})
    		\end{eqnarray*}
        \PN Lo cual es un absurdo, dado que $m$ es maximal de $\POSET$. Por lo tanto, $F(m)$ es maximal de
        $\POSETPRIMO$.

        \PN \begin{tabular}{|c|} \hline $\Leftarrow$ \\\hline \end{tabular} Supongamos que $F(m)$ es maximal de
        $\POSETPRIMO$. Veamos que $m$ es maximal de $\POSET$. Supongamos que $m$ no es maximal
        de $\POSET$, es decir, $\exists b \in P \ m < b$. Dado que $F$ es isomorfismo:
        \[
    			F(m) < F(b)
    		\]
        \PN Lo cual es un absurdo, dado que $F(m)$ es maximal de $\POSETPRIMO$. Por lo tanto, $m$ es
        maximal de $\POSET$.

      \item \begin{tabular}{|c|} \hline $\Rightarrow$ \\\hline \end{tabular} Supongamos $a \prec b$, veamos que $F(a)
        \prec^{\prime} F(b)$. Debemos ver:
        \begin{enumerate}[(1)]
          \item $F(a) <^{\prime} F(b)$
          \item $\nexists z^{\prime}$ tal que $F(a) < z^{\prime} < F(b)$
        \end{enumerate}

        \PN Ya que $a \prec b$, por definición tenemos: \begin{tabular}{|c|} \hline $a < b$ y $\nexists z$ tal que
        $a < z < b$ \\\hline \end{tabular} $(\star)$

        \PN Dado que la función $F$ es un isomorfismo, se cumple (1). Veamos que se cumple (2), supongamos que $\exists
        z^{\prime}$ tal que $F(a) < z^{\prime} < F(b)$. Luego, nuevamente utilizando que $F$ es isomorfismo, tenemos:
        \begin{eqnarray*}
    			F^{-1}(F(a)) < &F^{-1}(z^{\prime})& < F^{-1}(F(b)) \\
    			a < &F^{-1}(z^{\prime})& < b
    		\end{eqnarray*}
        \PN Lo cual, contradice $(\star)$, el absurdo vino de suponer que $\exists z^{\prime}$ tal que
        $F(a) < z^{\prime} < F(b)$, por lo tanto $\nexists z^{\prime}$ tal que $F(a) < z^{\prime} < F(b)$.

        \PN Finalmente, dado que se cumplen los puntos (1) y (2), se cumple también $F(a) \prec^{\prime} F(b)$.

        \PN \begin{tabular}{|c|} \hline $\Leftarrow$ \\\hline \end{tabular} Supongamos $F(a) \prec^{\prime} F(b)$,
        veamos que $a \prec b$.

		    \PN Ya que $F^{-1}: \POSETPRIMO \rightarrow \POSET$ es isomorfismo, por lo ya visto tenemos:
    		\begin{eqnarray*}
    			F^{-1}(F(a)) &\prec& F^{-1}(F(b)) \\
    			a &\prec& b
    		\end{eqnarray*}
    \end{enumerate}
  \end{proof}

  % Lemma 92: Con prueba. Lemma 2.
  \begin{lemma} \label{lemma_2}
    \PN Dado un reticulado $(L, \leq)$ y elementos $x, y, z, w \in L$, se cumplen las siguientes propiedades:
    \begin{multicols}{2}
      \begin{enumerate}[(1)]
        \item $x \leq x \ \SU \ y$
        \item $x \ \IN \ y \leq x$
        \item $x \ \SU \ x = x \ \IN \ x = x$
        \item $x \ \SU \ y = y \ \SU \ x$
        \item $x \ \IN \ y = y \ \IN \ x$
        \item $x \leq y \Leftrightarrow x \ \SU \ y = y \Leftrightarrow x \ \IN \ y = x$
        \item $x \ \SU \ (x \ \IN \ y) = x$
        \item $x \ \IN \ (x \ \SU \ y) = x$
        \item $(x \ \SU \ y) \ \SU \ z = x \ \SU \ (y \ \SU \ z)$
        \item $(x \ \IN \ y) \ \IN \ z = x \ \IN \ (y \ \IN \ z)$
        \item Si $x \leq z$ e $y \leq w$ entonces:
          \begin{itemize}
            \item $x \ \SU \ y \leq z \ \SU \ w$
            \item $x \ \IN \ y \leq z \ \IN \ w$
          \end{itemize}
        \item $(x \ \IN \ y) \ \SU \ (x \ \IN \ z) \leq x \ \IN \ (y \ \SU \ z)$
      \end{enumerate}
    \end{multicols}
  \end{lemma}
  \begin{proof}
    \PN Dado que las propiedades $(1), (2), (3), (4), (5), (6)$, son consecuencia inmediata de las definiciones de $\SU$
    e $\IN$, probaremos solo las restantes.
    \begin{enumerate}
      \begin{multicols}{2}
        \item[(7)]
          \begin{alignat*}{3}
            x \ \IN \ y &\leq& \ x \qquad &\text{Por } (2) \\
            (x \ \IN \ y) \ \SU \ x &=& \ x \qquad &\text{Por } (6) \\
            x \ \SU \ (x \ \IN \ y) &=& \ x \qquad &\text{Por } (4) \\
          \end{alignat*}

        \item[(8)]
          \begin{alignat*}{3}
            x &\leq& \ x \ \SU \ y \qquad \; && \text{Por } (1) \\
            x \ \IN \ (x \ \SU \ y) &=& \ x \qquad\qquad &&\text{Por } (6) \\
            x \ \IN \ (y \ \SU \ x) &=& \ x \qquad\qquad &&\text{Por } (4)
          \end{alignat*}
          \vspace{3mm}
      \end{multicols}

      \item[(9)] Para probar la igualdad probaremos las siguientes desigualdades:
        \begin{itemize}
          \item \begin{tabular}{|c|} \hline $(x \ \SU \ y) \ \SU \ z \leq x \ \SU \ (y \ \SU \ z)$\\\hline \end{tabular}
            \PN Notese que $x \ \SU \ (y \ \SU \ z)$ es cota superior de $\{x, y, z\}$ ya que:
            \begin{eqnarray*}
              x &\leq& x \ \SU \ (y \ \SU \ z) \\
              y &\leq& (y \ \SU \ z) \leq x \ \SU \ (y \ \SU \ z) \\
              z &\leq& (y \ \SU \ z) \leq x \ \SU \ (y \ \SU \ z)
            \end{eqnarray*}

            \PN Por otro lado, $x \ \SU \ (y \ \SU \ z)$ es cota superior de $\{x, y\}$, tenemos que $x \ \SU \ y \leq x
            \ \SU \ (y \ \SU \ z)$, por lo cual $x \ \SU \ (y \ \SU \ z)$ es cota superior del conjunto $\{x \ \SU \ y,
            z\}$, lo cual dice que $(x \ \SU \ y) \ \SU \ z \leq x \ \SU \ (y \ \SU \ z)$.

          \item \begin{tabular}{|c|} \hline $(x \ \SU \ y) \ \SU \ z \geq x \ \SU \ (y \ \SU \ z)$\\\hline \end{tabular}
            \PN Notese que $(x \ \SU \ y) \ \SU \ z$ es cota superior de $\{x, y, z\}$ ya que:
            \begin{eqnarray*}
              x &\leq& x \ \SU \ y \leq (x \ \SU \ y) \ \SU \ z \\
              y &\leq& x \ \SU \ y \leq (x \ \SU \ y) \ \SU \ z \\
              z &\leq& (x \ \SU \ y) \ \SU \ z
            \end{eqnarray*}

            \PN Por otro lado, $(x \ \SU \ y) \ \SU \ z$ es cota superior de $\{y, z\}$, tenemos que $y \ \SU \ z \leq
            (x \ \SU \ y) \ \SU \ z$, por lo cual $(x \ \SU \ y) \ \SU \ z$ es cota superior del conjunto $\{x, y \ \SU
            \ z\}$, lo cual dice que $(x \ \SU \ y) \ \SU \ z \geq x \ \SU \ (y \ \SU \ z)$.
        \end{itemize}

        \PN Por lo tanto, $(x \ \SU \ y) \ \SU \ z = x \ \SU \ (y \ \SU \ z)$

      \item[(10)] Para probar la igualdad probaremos las siguientes desigualdades:
        \begin{itemize}
          \item \begin{tabular}{|c|} \hline $(x \ \IN \ y) \ \IN \ z \leq x \ \IN \ (y \ \IN \ z)$\\\hline \end{tabular}
            \PN Notese que $(x \ \IN \ y) \ \IN \ z$ es cota inferior de $\{x, y, z\}$ ya que:
            \begin{eqnarray*}
              (x \ \IN \ y) \ \IN \ z \leq (x \ \IN \ y) &\leq& x \\
              (x \ \IN \ y) \ \IN \ z \leq (x \ \IN \ y) &\leq& y \\
              (x \ \IN \ y) \ \IN \ z &\leq& z
            \end{eqnarray*}

            \PN Por otro lado, $(x \ \IN \ y) \ \IN \ z$ es cota inferior de $\{y, z\}$, tenemos que $(x \ \IN \ y) \
            \IN \ z \leq y \ \IN \ z$, por lo cual $(x \ \IN \ y) \ \IN \ z$ es cota inferior del conjunto $\{x, y \ \IN
            \ z\}$, lo cual dice que $(x \ \IN \ y) \ \IN \ z \leq x \ \IN \ (y \ \IN \ z)$.

          \item \begin{tabular}{|c|} \hline $(x \ \IN \ y) \ \IN \ z \geq x \ \IN \ (y \ \IN \ z)$\\\hline \end{tabular}
            \PN Notese que $x \ \IN \ (y \ \IN \ z)$ es cota inferior de $\{x, y, z\}$ ya que:
            \begin{eqnarray*}
              x \ \IN \ (y \ \IN \ z) &\leq& x \\
              x \ \IN \ (y \ \IN \ z) \leq (y \ \IN \ z) &\leq& y \\
              x \ \IN \ (y \ \IN \ z) \leq (y \ \IN \ z) &\leq& z
            \end{eqnarray*}

            \PN Por otro lado, $x \ \IN \ (y \ \IN \ z)$ es cota inferior de $\{x, y\}$, tenemos que $x \ \IN \ (y \ \IN
            \ z) \leq x \ \IN \ y$, por lo cual $x \ \IN \ (y \ \IN \ z)$ es cota inferior del conjunto $\{x \ \IN \ y,
            z\}$, lo cual dice que $(x \ \IN \ y) \ \IN \ z \geq x \ \IN \ (y \ \IN \ z)$.
        \end{itemize}

        \PN Por lo tanto, $(x \ \IN \ y) \ \IN \ z = x \ \IN \ (y \ \IN \ z)$

      \item[(11)]
        \begin{multicols}{2}
          \begin{alignat*}{3}
            x &\leq& \ z &\leq& \ z \ \SU \ w \\
            y &\leq& \ w &\leq& \ z \ \SU \ w \\
          \end{alignat*}
          \begin{alignat*}{3}
            \\
            x &\leq& \ z \Rightarrow x \ \IN \ y &\leq& \ z \\
            y &\leq& \ w \Rightarrow x \ \IN \ y &\leq& \ w
          \end{alignat*}
        \end{multicols}
        \PN Luego, $z \ \SU \ w$ es cota superior de $\{x, y\}$ y $x \ \IN \ y$ es cota inferior de $\{z, w\}$, por lo
        tanto, $x \ \SU \ y \leq z \ \SU \ w$ y $x \ \IN \ y \leq z \ \IN \ w$.

      \item[(12)]
        \begin{equation*}
          \left.
          \begin{array}{l}
            (x \ \IN \ y), (x \ \IN \ z) \leq x \\
            (x \ \IN \ y), (x \ \IN \ z) \leq y \ \SU \ z
          \end{array}
          \right \rbrace \Rightarrow (x \ \IN \ y), (x \ \IN \ z) \leq x \ \IN \ (y \ \SU \ z)
        \end{equation*}

        \[
          \therefore (x \ \IN \ y) \ \SU \ (x \ \IN \ z) \leq x \ \IN \ (y \ \SU \ z)
        \]
    \end{enumerate}
  \end{proof}

  % Lemma 93: Con prueba. Lemma 3.
  \begin{lemma} \label{lemma_3}
    \PN Sea $(L, \leq)$ un reticulado, dados elementos $x_{1}, \dotsc, x_{n} \in L$, con $n \geq 2$, se tiene
    \[
      \begin{array}{rcl}
        (\dotsc (x_{1} \ \SU \ x_{2}) \ \SU \ \dotsc) \ \SU \ x_{n} &=& \sup (\{x_{1}, \dotsc, x_{n}\}) \\
        (\dotsc (x_{1} \ \IN \ x_{2}) \ \IN \ \dotsc) \ \IN \ x_{n} &=& \inf (\{x_{1}, \dotsc, x_{n}\})
      \end{array}
    \]
  \end{lemma}
  \begin{proof}
    \PN Probaremos por inducción en $n$.

    \vspace{3mm}
    \PN \underline{Caso Base:} \begin{tabular}{|c|} \hline $n = 2$ \\\hline \end{tabular}
    \[
      \begin{array}{rcl}
        x_{1} \ \SU \ x_{2} &=& \sup(\{x_{1}, x_{2}\}) \\
        x_{1} \ \IN \ x_{2} &=& \inf(\{x_{1}, x_{2}\})
      \end{array}
    \]

    \PN Lo cual vale, dado que es la definición.

    \vspace{3mm}
		\PN \underline{Caso Inductivo:} \begin{tabular}{|c|} \hline $n > 2$ \\\hline \end{tabular}

    \PN Supongamos ahora que vale para $n$ y veamos entonces que vale para $n+1$. Sean $x_{1}, \dotsc, x_{n+1} \in L$,
    por hipótesis inductiva tenemos que:
    \begin{eqnarray*}
      (\dotsc (x_{1} \ \SU \ x_{2}) \ \SU \ \dotsc) \ \SU \ x_{n} &=& \sup(\{x_{1}, \dotsc, x_{n}\}) \ \ (\star_{1}) \\
      (\dotsc (x_{1} \ \IN \ x_{2}) \ \IN \ \dotsc) \ \IN \ x_{n} &=& \inf(\{x_{1}, \dotsc, x_{n}\}) \ \ \ (\star_{2})
    \end{eqnarray*}


    \PN Veamos entonces que vale:
    \begin{eqnarray*}
      ((\dotsc(x_{1} \ \SU \ x_{2}) \ \SU \ \dotsc) \ \SU \ x_{n}) \ \SU \ x_{n+1} &=& \sup(\{x_{1}, \dotsc, x_{n+1}\})
        \ \ (\dag_{1}) \\
      ((\dotsc(x_{1} \ \IN \ x_{2}) \ \IN \ \dotsc) \ \IN \ x_{n}) \ \IN \ x_{n+1} &=& \inf(\{x_{1}, \dotsc, x_{n+1}\})
        \ \ \ (\dag_{2})
    \end{eqnarray*}

    \PN Para ello debemos ver $((\dotsc(x_{1} \ \SU \ x_{2}) \ \SU \ \dotsc) \ \SU \ x_{n}) \ \SU \ x_{n+1}$ es cota
    superior de $\{x_{1}, \dotsc, x_{n+1}\}$ y que es la menor de las cotas superiores. Además, que $((\dotsc(x_{1} \
    \IN \ x_{2}) \ \IN \ \dotsc) \ \IN \ x_{n}) \ \IN \ x_{n+1}$ es cota inferior de $\{x_{1}, \dotsc, x_{n+1}\}$ y que
    es la mayor de las cotas inferiores.

    \vspace{5mm}
    \PN Es fácil ver que $((\dotsc(x_{1} \ \SU \ x_{2}) \ \SU \ \dotsc) \ \SU \ x_{n}) \ \SU \ x_{n+1}$ es cota superior
    de $ \{x_{1}, \dotsc, x_{n+1}\}$. Supongamos que $z$ es otra cota superior de $\{x_{1}, \dotsc, x_{n+1}\}$. Ya que
    $z$ es también cota superior del conjunto $\{x_{1}, \dotsc, x_{n}\}$, por $(\star_{1})$ tenemos que:
    \[
      (\dotsc (x_{1} \ \SU \ x_{2}) \ \SU \ \dotsc) \ \SU \ x_{n} \leq z
    \]

    \PN Además, dado que $x_{n+1} \leq z$, tenemos que:
    \[
      ((\dotsc (x_{1} \ \SU \ x_{2}) \ \SU \ \dotsc) \ \SU \ x_{n}) \ \SU \ x_{n+1} \leq z
    \]

    \PN Por lo tanto, vale $(\dag_{1})$.

    \vspace{5mm}
    \PN Nuevamente, es fácil ver que $((\dotsc(x_{1} \ \IN \ x_{2}) \ \IN \ \dotsc) \ \IN \ x_{n}) \ \IN \ x_{n+1}$ es
    cota inferior de $ \{x_{1}, \dotsc, x_{n+1}\}$. Supongamos que $z^{\prime}$ es otra cota inferior de $\{x_{1},
    \dotsc, x_{n+1}\}$. Ya que $z^{\prime}$ es también cota inferior del conjunto $\{x_{1}, \dotsc, x_{n}\}$, por
    $(\star_{2})$ tenemos que:
    \[
      z^{\prime} \leq (\dotsc (x_{1} \ \IN \ x_{2}) \ \IN \ \dotsc) \ \IN \ x_{n}
    \]

    \PN Además, dado que $z^{\prime} \leq x_{n+1}$, tenemos que:
    \[
      z^{\prime} \leq ((\dotsc (x_{1} \ \IN \ x_{2}) \ \IN \ \dotsc) \ \IN \ x_{n}) \ \IN \ x_{n+1}
    \]

    \PN Por lo tanto, vale $(\dag_{2})$.
  \end{proof}

  % Theorem 94: Con prueba. Theorem 4.
  \begin{theorem} \label{theorem_4}
    \PN Sea $\RET$ un reticulado, la relación binaria definida por:
    \[
      x \leq y \Leftrightarrow x \ \SU \ y = y
    \]
    \PN es un orden parcial sobre $L$ para el cual se cumple:
    \[
      \begin{array}{rcl}
        \sup (\{x, y\}) &=& x \ \SU \ y \\
        \inf (\{x, y\}) &=& x \ \IN \ y
      \end{array}
    \]
  \end{theorem}
  \begin{proof}
    \PN \newline
    \begin{itemize}
      \item \underline{Reflexiva:} Sea $x \in L$ un elemento cualquiera. Luego,
        \begin{equation*}
          \left.
          \begin{array}{l}
            x \ \SU \ x = x \\
            x \ \IN \ x = x
          \end{array}
          \right \rbrace \Rightarrow x \leq x
        \end{equation*}

      \item \underline{Antisimétrica:} Sean $x, y \in L$ elementos cualquieras. Supongamos que $x \leq y$ e $y \leq x$,
        entonces:
        \begin{equation*}
          \left.
          \begin{array}{l}
            x \leq y \Rightarrow x \ \SU \ y = y \\
            y \leq x \Rightarrow x \ \SU \ y = x
          \end{array}
          \right \rbrace \Rightarrow x = y
        \end{equation*}

      \item \underline{Transitiva:} Supongamos que $x \leq y$ e $y \leq z$, es decir, $x \ \SU \ y = y$ y $y \ \SU \ z =
      z$ entonces:
        \[
          x \ \SU \ z = x \ \SU \ (y \ \SU \ z) = (x \ \SU \ y) \ \SU \ z = y \ \SU \ z = z
        \]
        \PN por lo cual $x \leq z$.

        \PN Veamos ahora que $\sup(\{x, y\}) = x \ \SU \ y$. Es claro que $x \ \SU \ y$ es una cota superior del
        conjunto $\{x, y\}$, veamos que es la menor. Supongamos $x, y \leq z$, entonces:
        \[
          (x \ \SU \ y) \ \SU \ z = x \ \SU \ (y \ \SU \ z) = x \ \SU \ z = z
        \]
        \PN por lo que $x\ \SU \ y \leq z$, es decir, $x\ \SU \ y$ es la menor cota superior.

        \PN Resta probar que $\inf(\{x, y\}) = x \ \IN \ y$. Nuevamente, es claro que $x \ \IN \ y$ es una cota inferior
        del conjunto $\{x, y\}$, veamos que es la mayor. Supongamos $z \leq x, y$, entonces:
        \[
          (x \ \IN \ y) \ \IN \ z = x \ \IN \ (y \ \IN \ z) = x \ \IN \ z = z
        \]
        \PN por lo que $z \leq x\ \IN \ y$, es decir, $x\ \IN \ y$ es la mayor cota inferior.
    \end{itemize}
  \end{proof}

  % Lemma 95: Con prueba. Lemma 5.
  \begin{lemma} \label{lemma_5}
    \PN Si $F: \RET \rightarrow \RETPRIMO$ es un homomorfismo biyectivo, entonces $F$ es un isomorfismo.
  \end{lemma}
  \begin{proof}
    \PN Debemos probar que $F^{-1}$ es un homomorfismo. Sean $F(x), F(y)$ dos elementos cualesquiera de $L^{\prime}$,
    tenemos que:
    \begin{multicols}{2}
      \begin{eqnarray*}
        F^{-1}(F(x) \ \SU^{\prime} \ F(y)) &=& F^{-1}(F(x \ \SU \ y)) \\
        &=& x \ \SU \ y \\
        &=& F^{-1}(F(x)) \ \SU \ F^{-1}(F(y))
      \end{eqnarray*}

      \begin{eqnarray*}
        F^{-1}(F(x) \ \IN^{\prime} \ F(y)) &=& F^{-1}(F(x \ \IN \ y)) \\
        &=& x \ \IN \ y \\
        &=& F^{-1}(F(x)) \ \IN \ F^{-1}(F(y))
      \end{eqnarray*}
    \end{multicols}

    \PN Luego, $F^{-1}$ es homomorfismo y por lo tanto $F$ es isomorfismo.
  \end{proof}

  % Lemma 96: Con prueba. Lemma 6.
  \begin{lemma} \label{lemma_6}
    \PN Sean $\RET$ y $\RETPRIMO$ reticulados y sea $F: \RET \rightarrow \RETPRIMO$ un homomorfismo, entonces $I_{F}$ es
    un subuniverso de $\RETPRIMO$.
  \end{lemma}
  \begin{proof}
    \PN Para probar que $I_{F}$ es un subuniverso de $\RETPRIMO$, debemos ver:
    \begin{itemize}
      \item $I_{F} \neq \emptyset$: Ya que $L \neq \emptyset$, tenemos que $I_{F} \neq \emptyset$.
      \item $I_{F}$ es cerrado bajo la operaciones $\SU^{\prime}$ e $\IN^{\prime}$: Sean $a, b \in I_{F}, \ x, y \in L$
      tales que $F(x) = a$ y $F(y) = b$. Se tiene que:
      \begin{alignat*}{3}
        a \ \SU^{\prime} \ b &=& \ F(x) \ \SU^{\prime} \ F(y) &=& \ F(x \ \SU \ y) \in I_{F} \\
        a \ \IN^{\prime} \ b &=& \ F(x) \ \IN^{\prime} \ F(y) &=& \ F(x \ \IN \ y) \in I_{F}
      \end{alignat*}

      \PN por lo cual $I_{F}$ es cerrada bajo $\SU^{\prime}$ e $\IN^{\prime}$.
    \end{itemize}
  \end{proof}

  % Lemma 97: Con prueba. Lemma 7.
  \begin{lemma} \label{lemma_7}
    \PN Sean $\RET$ y $\RETPRIMO$ reticulados y sean $(L, \leq)$ y $(L^{\prime}, \leq^{\prime})$ los posets asociados.
    Sea $F: L \rightarrow L^{\prime}$ una función, entonces $F$ es un isomorfismo de $\RET$ en $ \RETPRIMO$ si y solo si
    $F$ es un isomorfismo de $(L, \leq)$ en $(L^{\prime}, \leq^{\prime})$.
  \end{lemma}
  \begin{proof}
    \PN \begin{tabular}{|c|} \hline $\Rightarrow$ \\\hline \end{tabular} Supongamos que $F$ es un isomorfismo de
    $\RET$ en $ \RETPRIMO$.

    \PN \underline{Para F}: Sean $x, y \in L$ tales que $x \leq y$. Tenemos:
    \begin{eqnarray*}
      y &=& x \ \SU \ y \\
      F(y) &=& F(x \ \SU \ y) \\
      &=& F(x) \ \SU^{\prime} \ F(y) \\
      \therefore F(x) &\leq^{\prime}& F(y)
    \end{eqnarray*}

    \PN \underline{Para $F^{-1}$}: Sean $x^{\prime}, y^{\prime} \in L^{\prime}$ tales que $x^{\prime} \leq^{\prime}
    y^{\prime}$. Tenemos:
    \begin{eqnarray*}
      y^{\prime} &=& x^{\prime} \ \SU^{\prime} \ y^{\prime} \\
      F^{-1}(y^{\prime}) &=& F^{-1}(x^{\prime} \ \SU^{\prime} \ y^{\prime}) \\
      &=& F^{-1}(x^{\prime}) \ \SU \ F^{-1}(y^{\prime}) \\
      \therefore F^{-1}(x^{\prime}) &\leq& F^{-1}(y^{\prime})
    \end{eqnarray*}

    \PN Por lo tanto, $F$ es un isomorfismo de $(L, \leq)$ en $(L^{\prime}, \leq^{\prime})$.

    \PN \begin{tabular}{|c|} \hline $\Leftarrow$ \\\hline \end{tabular} Supongamos ahora que $F$ es un isomorfismo
    de $(L, \leq)$ en $(L^{\prime}, \leq^{\prime})$, entonces el \textbf{Lemma~\ref{lemma_1}} nos dice que $F$ y
    $F^{-1}$ respetan la operaciones de supremo e ínfimo, por lo cual $F$ es un isomorfismo de $\RET$ y
    $\RETPRIMO$.
  \end{proof}

  % Lemma 98: Con prueba. Lemma 8.
  \begin{lemma} \label{lemma_8}
    \PN Sea $\RETCOCIENTADO$ un reticulado. El orden parcial $\tilde{\leq}$ asociado a este reticulado cumple:
    \[
      x/\theta \ \tilde{\leq} \ y/\theta \Leftrightarrow y \ \theta \ (x \ \SU \ y)
    \]
  \end{lemma}
  \begin{proof}
    \PN Veamos que $\RETCOCIENTADO$ satisface las 7 identidades de la
    definición de reticulado. Sean $x/\theta, y/\theta, z/\theta$ elementos cualesquiera de $L/\theta$.
    \begin{multicols}{2}
      \begin{enumerate}
        \item[(I1)] \begin{tabular}{|c|} \hline $x/\theta \ \mathsf{\tilde{s}} \ x/\theta = x/\theta \
          \mathsf{\tilde{\imath}} \ x/\theta = x/\theta$ \\\hline \end{tabular}
          \begin{eqnarray*}
            x/\theta \ \mathsf{\tilde{s}} \ x/\theta &=& (x \ \SU \ x)/\theta \ =\ x /\theta \\
            x/\theta \ \mathsf{\tilde{\imath}} \ x/\theta &=& (x \ \IN \ x)/\theta \ = \ x /\theta
          \end{eqnarray*}
        \item[(I2)] \begin{tabular}{|c|} \hline $x/\theta \ \mathsf{\tilde{s}} \ y/\theta = y/\theta \
          \mathsf{\tilde{s}} \ x/\theta$ \\\hline \end{tabular}
          \begin{eqnarray*}
            x/\theta \ \mathsf{\tilde{s}} \ y/\theta &=& (x \ \SU \ y)/\theta \\
            &=& (y \ \SU \ x)/\theta \\
            &=& y/\theta \ \mathsf{\tilde{s}} \ y/\theta
          \end{eqnarray*}
        \item[(I3)] \begin{tabular}{|c|} \hline $x/\theta \ \mathsf{\tilde{\imath}} \ y/\theta = y/\theta \
          \mathsf{\tilde{\imath}} \ x/\theta$ \\\hline \end{tabular}
          \begin{eqnarray*}
            x/\theta \ \mathsf{\tilde{\imath}} \ y/\theta &=& (x \ \IN \ y)/\theta \\
            &=& (y \ \IN \ x)/\theta \\
            &=& y/\theta \ \mathsf{\tilde{\imath}} \ y/\theta
          \end{eqnarray*}
        \item[(I4)] \begin{tabular}{|c|} \hline $(x/\theta \ \mathsf{\tilde{s}} \ y/\theta) \ \mathsf{\tilde{s}} \
          z/\theta = x/\theta \ \mathsf{\tilde{s}} \ (y/\theta \ \mathsf{\tilde{s}} \ z/\theta)$ \\\hline \end{tabular}
          \begin{eqnarray*}
            (x/\theta \ \mathsf{\tilde{s}} \ y/\theta) \ \mathsf{\tilde{s}} \ z/\theta &=& (x \ \SU \ y)/\theta \
              \mathsf{\tilde{s}} \ z/\theta \\
            &=& ((x \ \SU \ y) \ \SU \ z) /\theta \\
            &=& (x \ \SU \ (y \ \SU \ z)) /\theta \\
            &=& x/\theta \ \mathsf{\tilde{s}} \ (y \ \SU \ z) /\theta \\
            &=& x/\theta \ \mathsf{\tilde{s}} \ (y /\theta \ \mathsf{\tilde{s}} \ z/\theta)
          \end{eqnarray*}
        \item[(I5)] \begin{tabular}{|c|} \hline $(x/\theta \ \mathsf{\tilde{\imath}} \ y/\theta) \
          \mathsf{\tilde{\imath}} \ z/\theta = x/\theta \ \mathsf{\tilde{\imath}} \ (y/\theta \ \mathsf{\tilde{\imath}}
          \ z/\theta)$ \\\hline \end{tabular}
          \begin{eqnarray*}
            (x/\theta \ \mathsf{\tilde{\imath}} \ y/\theta) \ \mathsf{\tilde{\imath}} \ z/\theta &=& (x \ \IN \
              y)/\theta \ \mathsf{\tilde{\imath}} \ z/\theta \\
            &=& ((x \ \IN \ y) \ \IN \ z) /\theta \\
            &=& (x \ \IN \ (y \ \IN \ z)) /\theta \\
            &=& x/\theta \ \mathsf{\tilde{\imath}} \ (y \ \IN \ z) /\theta \\
            &=& x/\theta \ \mathsf{\tilde{\imath}} \ (y /\theta \ \mathsf{\tilde{\imath}} \ z/\theta)
          \end{eqnarray*}
        \item[(I6)] \begin{tabular}{|c|} \hline $x/\theta \ \mathsf{\tilde{s}} \ (x/\theta \ \mathsf{\tilde{\imath}} \
          y/\theta) = x/\theta$ \\\hline \end{tabular}
          \begin{eqnarray*}
            x/\theta \ \mathsf{\tilde{s}} \ (x/\theta \ \mathsf{\tilde{\imath}} \ y/\theta) &=& x/\theta \ \mathsf{\tilde{s}}
              \ (x \ \IN \ y)/\theta \\
            &=& (x \ \SU \ (x \ \IN \ y))/\theta \\
            &=& x/\theta
          \end{eqnarray*}
        \item[(I7)] \begin{tabular}{|c|} \hline $x/\theta \ \mathsf{\tilde{\imath}} \ (x/\theta \ \mathsf{\tilde{s}} \
          y/\theta) = x/\theta$ \\\hline \end{tabular}
          \begin{eqnarray*}
            x/\theta \ \mathsf{\tilde{\imath}} \ (x/\theta \ \mathsf{\tilde{s}} \ y/\theta) &=& x/\theta \
              \mathsf{\tilde{\imath}} \ (x \ \SU \ y)/\theta \\
            &=& (x \ \IN \ (x \ \SU \ y))/\theta \\
            &=& x/\theta
          \end{eqnarray*}
      \end{enumerate}
    \end{multicols}

    \PN Por definición, $x/\theta \ \tilde{\leq} \ y/\theta \Leftrightarrow y/\theta = x/\theta \ \mathsf{\tilde{s}} \
    y/\theta$, por lo cual $x/\theta \ \tilde{\leq} \ y/\theta \Leftrightarrow y/\theta = (x \ \SU \ y)/\theta$ y por lo
    tanto $y \theta (x \ \SU \ y)$.
  \end{proof}

  % Corollary 99: Con prueba. Corollary 9.
  \begin{corollary} \label{corrolary_9}
    \PN Sea $\RET$ un reticulado en el cual hay un elemento máximo $1$ (resp. mínimo $0$), entonces si $\theta$ es una
    congruencia sobre $\RET, 1/\theta$ (resp. $0/\theta$) es un elemento máximo (resp. mínimo) de $\RETCOCIENTADO$.
  \end{corollary}
  \begin{proof}
    \PN Ya que $1 = x \ \SU \ 1$ para cada $x \in L$ tenemos que $1/\theta = x/\theta \ \mathsf{\tilde{s}} \ 1/\theta$,
    es decir, $1/\theta = (x \ \SU \ 1)/\theta$ para cada $x \in L$. Utilizando el \textbf{Lemma~\ref{lemma_8}} tenemos
    que $x/\theta \ \tilde{\leq} \ 1/\theta$, para cada $x \in L$.
  \end{proof}

  % Lemma 100: Con prueba. Lemma 10.
  \begin{lemma} \label{lemma_10}
    \PN Si $F: \RET \rightarrow \RETPRIMO$ es un homomorfismo de reticulados, entonces $\ker F$ es una congruencia sobre
    $\RET$.
  \end{lemma}
  \begin{proof}
    \PN Veamos primero que $\ker F$ es una relación de equivalencia.
    \begin{itemize}
      \item \underline{Reflexiva:} $(x, x) \in$ ker $F$. Trivial pues $F(x) = F(x)$.
      \item \underline{Simétrica:} Si $(x, y) \in$ ker $F \Rightarrow (y, x) \in$ ker $F$.
        \PN Si $(x, y) \in$ ker $F \Rightarrow F(x) = F(y)$. Luego, vale también $F(y) = F(x)$.
      \item \underline{Transitiva:} Si $(x, y), (y, z) \in$ ker $F \Rightarrow (x, z) \in$ ker $F$.
        \begin{equation*}
          \left.
          \begin{array}{l}
            (x, y) \in \text{ ker } F \Rightarrow F(x) = F(y) \\
            (y, z) \in \text{ ker } F \Rightarrow F(y) = F(z)
          \end{array}
          \right \rbrace \Rightarrow F(x) = F(y) = F(z)
        \end{equation*}
        \PN Por lo tanto, $(x, z) \in$ ker $F$.
      \end{itemize}

    \PN Veamos ahora que, si $x \ \ker F \ x^{\prime}$ y $y \ \ker F \ y^{\prime}$ entonces $(x \ \SU \ y) \ \ker F \
    (x^{\prime} \ \SU \ y^{\prime})$ y $(x \ \IN \ y) \ \ker F \ (x^{\prime} \ \IN \ y^{\prime})$, es decir, $F(x \ \SU
    \ y) = F(x^{\prime} \ \SU \ y^{\prime})$ y $F(x \ \IN \ y) = F(x^{\prime} \ \IN \ y^{\prime})$.

    \vspace{3mm}
    \PN Supongamos $x \ \ker F \ x^{\prime}$ y $y \ \ker F \ y^{\prime}$, entonces:
    \begin{alignat*}{4}
      F(x \ \SU \ y) &=& \ F(x) \ \mathsf{s^{\prime}} \ F(y) &=& \ F(x^{\prime}) \ \mathsf{s^{\prime}} \ F(y^{\prime})
        &=& \ F(x^{\prime} \ \SU \ y^{\prime}) \\
      F(x \ \IN \ y) &=& \ F(x) \ \mathsf{i^{\prime}} \ F(y) &=& \ F(x^{\prime}) \ \mathsf{i^{\prime}} \ F(y^{\prime})
        &=& \ F(x^{\prime} \ \IN \ y^{\prime})
    \end{alignat*}
  \end{proof}

  % Lemma 101: Con prueba. Lemma 11.
  \begin{lemma} \label{lemma_11}
    \PN Sea $\RET$ un reticulado y sea $\theta$ una congruencia sobre $\RET$, entonces $\pi_{\theta}$ es un homomorfismo
    de $\RET$ en $\RETCOCIENTADO$. Además $\ker \pi_{\theta} = \theta$.
  \end{lemma}
  \begin{proof}
    \PN Sean $x, y \in L$ elementos cualquiera. Tenemos que:
      \begin{alignat*}{4}
        \pi_{\theta}(x \ \SU \ y) &=& \ (x \ \SU \ y)/\theta &=& \ x/\theta \ \mathsf{\tilde{s}} \ y/\theta &=&
          \ \pi_{\theta}(x) \ \mathsf{\tilde{s}} \ \pi_{\theta}(y) \\
        \pi_{\theta}(x \ \IN \ y) &=& \ (x \ \IN \ y)/\theta &=& \ x/\theta \ \mathsf{\tilde{\imath}} \ y/\theta &=&
          \ \pi_{\theta}(x) \ \mathsf{\tilde{\imath}} \ \pi_{\theta}(y)
      \end{alignat*}
    \PN por lo cual $\pi_{\theta}$ preserva las operaciones de supremo e ínfimo, es decir, es un homomorfismo.
  \end{proof}

  % Lemma 102: Con prueba. Lemma 12.
  \begin{lemma} \label{lemma_12}
    \PN Si $F: \ACOTADO \rightarrow \ACOTADOPRIMO$ un homomorfismo biyectivo, entonces $F$ es un isomorfismo.
  \end{lemma}
  \begin{proof}
      \PN Debemos probar que $F^{-1}$ es un homomorfismo. Sean $F(x), F(y)$ dos elementos cualesquiera de $L^{\prime}$,
      tenemos que:
      \begin{multicols}{2}
        \begin{eqnarray*}
          F^{-1}(F(1)) &=& F^{-1}(1^{\prime}) \\
          F^{-1}(1^{\prime}) &=& 1 \\
          \\
          F^{-1}(F(x) \ \SU^{\prime} \ F(y)) &=& F^{-1}(F(x \ \SU \ y)) \\
          &=& x \ \SU \ y \\
          &=& F^{-1}(F(x)) \ \SU \ F^{-1}(F(y))
        \end{eqnarray*}

        \begin{eqnarray*}
          F^{-1}(F(0)) &=& F^{-1}(0^{\prime}) \\
          F^{-1}(0^{\prime}) &=& 0 \\
          \\
          F^{-1}(F(x) \ \IN^{\prime} \ F(y)) &=& F^{-1}(F(x \ \IN \ y)) \\
          &=& x \ \IN \ y \\
          &=& F^{-1}(F(x)) \ \IN \ F^{-1}(F(y))
        \end{eqnarray*}
      \end{multicols}

      \PN Luego, $F^{-1}$ es homomorfismo y por lo tanto $F$ es isomorfismo.
  \end{proof}

  % Lemma 103: Con prueba. Lemma 13.
  \begin{lemma} \label{lemma_13}
    \PN Si $F: \ACOTADO \rightarrow \ACOTADOPRIMO$ es un homomorfismo, entonces $I_{F}$ es un subuniverso de $\ACOTADOPRIMO$.
  \end{lemma}
  \begin{proof}
    \PN Para probar que $I_{F}$ es un subuniverso de $\ACOTADOPRIMO$, debemos ver:
    \begin{itemize}
      \item $I_{F} \neq \emptyset$: Ya que $L \neq \emptyset$, tenemos que $I_{F} \neq \emptyset$.
      \item Preserva $0$ y $1$: $0, 1 \in I_{F}$
      \item $I_{F}$ es cerrado bajo la operaciones $\SU^{\prime}$ e $\IN^{\prime}$: Sean $a, b \in I_{F}, \ x, y \in L$
      tales que $F(x) = a$ y $F(y) = b$. Se tiene que:
      \begin{alignat*}{3}
        a \ \SU^{\prime} \ b &=& \ F(x) \ \SU^{\prime} \ F(y) &=& \ F(x \ \SU \ y) \in I_{F} \\
        a \ \IN^{\prime} \ b &=& \ F(x) \ \IN^{\prime} \ F(y) &=& \ F(x \ \IN \ y) \in I_{F}
      \end{alignat*}

      \PN por lo cual $I_{F}$ es cerrada bajo $\SU^{\prime}$ e $\IN^{\prime}$.
    \end{itemize}.
  \end{proof}

  % Lemma 104: Con prueba. Lemma 14.
  \begin{lemma} \label{lemma_14}
    \PN Si $F: \ACOTADO \rightarrow \ACOTADOPRIMO$ es un homomorfismo de reticulados acotados, entonces $\ker F$ es una
    congruencia sobre $\ACOTADO$.
  \end{lemma}
  \begin{proof}
    \PN Dado que $F$ es un homomorfismo de $\RET$ en $ \RETPRIMO$ utilizando el \textbf{Lemma~\ref{lemma_10}} tenemos
    que $\ker F$ es una congruencia sobre $\RET$ lo cual por definición, nos dice que $\ker F$ es una congruencia sobre
    $\ACOTADO$.
  \end{proof}

  % Lemma 105: Con prueba. Lemma 15.
  \begin{lemma} \label{lemma_15}
    \PN Sea $\ACOTADO$ un reticulado acotado y $\theta$ una congruencia sobre $\ACOTADO$, entonces:
    \begin{enumerate}[a)]
      \item $\ACOTADOCOCIENTADO$ es un reticulado acotado.
      \item $\pi_{\theta}$ es un homomorfismo de $\ACOTADO$ en $\ACOTADOCOCIENTADO$ cuyo núcleo es $\theta$.
    \end{enumerate}
  \end{lemma}
  \begin{proof}
    \PN \newline
    \begin{enumerate}[a)]
      \item Sabemos por el \textbf{Lemma~\ref{lemma_11}} que $\RETCOCIENTADO$ es un reticulado. Además, por el
      \textbf{Lemma~\ref{lemma_9}} se tiene que $\ACOTADOCOCIENTADO$ es un reticulado acotado.
      \item Por el \textbf{Lemma~\ref{lemma_11}} se tiene que $\pi_{\theta}$ es un homomorfismo de $\RET$ en
      $\RETCOCIENTADO$. Además, dado que $\pi_{\theta}(1) = 1/\theta$ y $\pi_{\theta}(0) = 0/\theta$, tenemos que
      $\pi_{\theta}$ es homomorfismo de $\ACOTADO$ en $\ACOTADOCOCIENTADO$.
    \end{enumerate}
  \end{proof}

  % Lemma 106: Con prueba. Lemma 16.
  \begin{lemma} \label{lemma_16}
    \PN Si $F:\COMPLEMENTADO \rightarrow \COMPLEMENTADOPRIMO$ un homomorfismo biyectivo, entonces $F$ es un isomorfismo.
  \end{lemma}
  \begin{proof}
      \PN Debemos probar que $F^{-1}$ es un homomorfismo. Sean $F(x), F(y)$ dos elementos cualesquiera de $L^{\prime}$,
      tenemos que:
      \begin{multicols}{2}
        \begin{eqnarray*}
          1^{\prime} &=& F(1) \\
          F^{-1}(1^{\prime}) &=& F^{-1}(F(1)) \\
          F^{-1}(1^{\prime}) &=& 1 \\
          \\
          F(x) \ \SU^{\prime} \ F(y) &=& F(x \ \SU \ y) \\
          F^{-1}(F(x) \ \SU^{\prime} \ F(y)) &=& F^{-1}(F(x \ \SU \ y)) \\
          &=& x \ \SU \ y \\
          &=& F^{-1}(F(x)) \ \SU \ F^{-1}(F(y)) \\
          \\
          1^{\prime} &=& F(1) \\
          &=& F(x \ \SU \ x^{c}) \\
          &=& F(x) \ \SU^{\prime} \ F(x^{c}) \\
          &=& F(x) \ \SU^{\prime} \ F(x)^{c^{\prime}} \\
          F^{-}(1^{\prime}) &=& F^{-1}(F(x) \ \SU^{\prime} \ F(x)^{c^{\prime}}) \\
          &=& F^{-1}(F(x)) \ \SU \ F^{-1}(F(x)^{c^{\prime}})) \\
          F^{-}(1^{\prime}) &=& x \ \SU \ F^{-1}(F(x)^{c^{\prime}}) \qquad (\star_{1})
        \end{eqnarray*}

        \begin{eqnarray*}
          0^{\prime} &=& F(0) \\
          F^{-1}(0^{\prime}) &=& F^{-1}(F(0)) \\
          F^{-1}(0^{\prime}) &=& 0 \\
          \\
          F(x) \ \IN^{\prime} \ F(y) &=& F(x \ \IN \ y) \\
          F^{-1}(F(x) \ \IN^{\prime} \ F(y)) &=& F^{-1}(F(x \ \IN \ y)) \\
          &=& x \ \IN \ y \\
          &=& F^{-1}(F(x)) \ \IN \ F^{-1}(F(y)) \\
          \\
          0^{\prime} &=& F(0) \\
          &=& F(x \ \IN \ x^{c}) \\
          &=& F(x) \ \IN^{\prime} \ F(x^{c}) \\
          &=& F(x) \ \IN^{\prime} \ F(x)^{c^{\prime}} \\
          F^{-}(1^{\prime}) &=& F^{-1}(F(x) \ \IN^{\prime} \ F(x)^{c^{\prime}}) \\
          &=& F^{-1}(F(x)) \ \IN \ F^{-1}(F(x)^{c^{\prime}})) \\
          F^{-}(1^{\prime}) &=& x \ \IN \ F^{-1}(F(x)^{c^{\prime}}) \qquad (\star_{2})
        \end{eqnarray*}
      \end{multicols}

      \PN Por lo tanto, de $(\star_{1})$ y $(\star_{2})$ obtenemos:
      \[
        F^{-1}(F(x)^{c}) = x^{c} = F^{-1}(F(x))^{c}
      \]
      \PN Luego, $F^{-1}$ es homomorfismo y por lo tanto $F$ es isomorfismo.
  \end{proof}

  % Lemma 107: Con prueba. Lemma 17.
  \begin{lemma} \label{lemma_17}
    \PN Si $F: \COMPLEMENTADO \rightarrow \COMPLEMENTADOPRIMO$ es un homomorfismo, entonces $I_{F}$ es un subuniverso de
    $\COMPLEMENTADOPRIMO$.
  \end{lemma}
  \begin{proof}
    \PN Para probar que $I_{F}$ es un subuniverso de $\COMPLEMENTADOPRIMO$, debemos ver:
    \begin{itemize}
      \item $I_{F} \neq \emptyset$: Ya que $L \neq \emptyset$, tenemos que $I_{F} \neq \emptyset$.
      \item Preserva $0$ y $1$: $0, 1 \in I_{F}$
      \item $I_{F}$ es cerrado bajo la operaciones $\SU^{\prime}, \IN^{\prime}$ y $^{c^{\prime}}$: Sean $a, b \in I_{F},
      \ x, y \in L$ tales que $F(x) = a$ y $F(y) = b$. Se tiene que:
      \begin{eqnarray*}
        a \ \SU^{\prime} \ b &=& F(x) \ \SU^{\prime} \ F(y) = F(x \ \SU \ y) \in I_{F} \\
        a \ \IN^{\prime} \ b &=& F(x) \ \IN^{\prime} \ F(y) \ = F(x \ \IN \ y) \in I_{F} \\
        a^{c^{\prime}} &=& F(x)^{c^{\prime}} = \ F(x^{c}) \in I_{F}
      \end{eqnarray*}

      \PN por lo cual $I_{F}$ es cerrada bajo $\SU^{\prime}, \IN^{\prime}$ y $^{c^{\prime}}$:
    \end{itemize}.
  \end{proof}

  % Lemma 108: Con prueba. Lemma 18.
  \begin{lemma} \label{lemma_18}
    \PN Si $F: \COMPLEMENTADO \rightarrow \COMPLEMENTADOPRIMO$ es un homomorfismo de reticulados complementados,
    entonces $\ker F$ es una congruencia sobre $\COMPLEMENTADO$.
  \end{lemma}
  \begin{proof}
    \PN Ya que $F$ es un homomorfismo de $\COMPLEMENTADO$ en $\COMPLEMENTADOPRIMO$, tenemos por
    \textbf{Lemma~\ref{lemma_14}} que $ker F$ es una congruencia sobre $\COMPLEMENTADO$, es decir, solo falta probar que
    para todos $x, y \in L$ se tiene que $x / ker F = y / ker F$ implica $x^{c} / ker F = y^{c} / ker F$, veamos esto.
    Supongamos $x / ker F = y / ker F$, es decir, por definición tenemos que $F(x) = F(y)$, luego:
    \begin{eqnarray*}
      F(x) = F(y) &\Leftrightarrow & F(x)^{c} = F(y)^{c} \\
      &\Leftrightarrow & F(x^{c}) = F(y^{c})
    \end{eqnarray*}
    \PN Por lo tanto, $x^{c} / ker F = y^{c} / ker F$.
  \end{proof}

  % Lemma 109: Con prueba. Lemma 19.
  \begin{lemma} \label{lemma_19}
    \PN Sea $\COMPLEMENTADO$ un reticulado complementado y sea $\theta$ una congruencia sobre $\COMPLEMENTADO$.
    \begin{enumerate}[a)]
      \item $\COMPLEMENTADOCOCIENTADO$ es un reticulado complementado.
      \item $\pi_{\theta}$ es un homomorfismo de $\COMPLEMENTADO$ en $\COMPLEMENTADOCOCIENTADO$ cuyo núcleo es $\theta$.
    \end{enumerate}
  \end{lemma}
  \begin{proof}
    \PN \newline
    \begin{enumerate}[a)]
      \item Sabemos por el \textbf{Lemma~\ref{lemma_11}} que $\ACOTADOCOCIENTADO$ es un reticulado acotado, es decir,
      solo nos falta ver que $\COMPLEMENTADOCOCIENTADO$ satisface:
      \begin{itemize}
        \item \begin{tabular}{|c|} \hline $x/\theta \ \mathsf{\tilde{s}} \ (x/\theta)^{\tilde{c}} = 1/\theta$ \\\hline
          \end{tabular} para cada $x/\theta \in L/\theta$: Sea $x/\theta$ un elemento cualquiera de $L/\theta$. Ya que
          $\COMPLEMENTADO$ satisface (I10), tenemos que $x \ \SU \ x^{c} = 1$. Osea que:
          \begin{eqnarray*}
            x \ \SU \ x^{c} = 1 &\Leftrightarrow& (x \ \SU \ x^{c})/\theta = 1/\theta \\
            &\Leftrightarrow& x/\theta \ \mathsf{\tilde{s}} \ x^{c}/\theta = 1/\theta \\
            &\Leftrightarrow& x/\theta \ \mathsf{\tilde{s}} \ (x/\theta)^{\tilde{c}} = 1/\theta
          \end{eqnarray*}
        \item \begin{tabular}{|c|} \hline $x/\theta \ \mathsf{\tilde{\imath}} \ (x/\theta)^{\tilde{c}} = 0/\theta$ \\
          \hline \end{tabular} para cada $x/\theta \in L/\theta$: Sea $x/\theta$ un elemento cualquiera de $L/\theta$.
          Ya que $\COMPLEMENTADO$ satisface (I11), tenemos que $x \ \IN \ x^{c} = 0$. Osea que:
          \begin{eqnarray*}
            x \ \IN \ x^{c} = 0 &\Leftrightarrow& (x \ \IN \ x^{c})/\theta = 0/\theta \\
            &\Leftrightarrow& x/\theta \ \mathsf{\tilde{\imath}} \ x^{c}/\theta = 0/\theta \\
            &\Leftrightarrow& x/\theta \ \mathsf{\tilde{\imath}} \ (x/\theta)^{\tilde{c}} = 0/\theta
          \end{eqnarray*}
      \end{itemize}
      \item Por el Lema \textbf{Lemma~\ref{lemma_15}} tenemos que $\pi_{\theta}$ es un homomorfismo de $\ACOTADO$ en
      $(L/\theta, \mathsf{\tilde{s}}, \mathsf{\tilde{\imath}}, 0/\theta, \linebreak 1/\theta)$, cuyo núcleo es $\theta$.
      Notar que por definición de $^{\tilde{c}}$ tenemos que $x^{c}/\theta = (x/\theta)^{\tilde{c}}$, es decir,
      $\pi_{\theta}(x^{c}) = (\pi_{\theta}(x))^{\tilde{c}}$, cualquiera sea $x \in L$.
    \end{enumerate}
  \end{proof}

  % Lemma 110: Con prueba. Lemma 20.
  \begin{lemma} \label{lemma_20}
    \PN Sea $\RET$ un reticulado. Son equivalentes:
    \begin{enumerate}[(1)]
      \item $x \ \IN \ (y \ \SU \ z) = (x \ \IN \ y) \ \SU \ (x \ \IN \ z)$, cualesquiera sean $x, y, z \in L$
      \item $x \ \SU \ (y \ \IN \ z) = (x \ \SU \ y) \ \IN \ (x \ \SU \ z)$, cualesquiera sean $x, y, z \in L$.
    \end{enumerate}
  \end{lemma}
  \begin{proof}
    \PN \begin{tabular}{|c|} \hline $(1) \Rightarrow (2)$ \\\hline \end{tabular} Notar que:
    \begin{eqnarray*}
      (x \ \SU \ y) \ \IN \ (x \ \SU \ z) &=& ((x \ \SU \ y) \ \IN \ x) \ \SU \ ((x \ \SU \ y) \ \IN \ z) \\
      &=& x \ \SU \ ((x \ \SU \ y) \ \IN \ z) \\
      &=& x \ \SU \ (z \ \IN \ (x \ \SU \ y)) \\
      &=& x \ \SU \ ((z \ \IN \ x) \ \SU \ (z \ \IN \ y)) \\
      &=& (x \ \SU \ (z \ \IN \ x)) \ \SU \ (z \ \IN \ y) \\
      &=& x \ \SU \ (z \ \IN \ y) \\
      &=& x \ \SU \ (y \ \IN \ z)
    \end{eqnarray*}

    \PN \begin{tabular}{|c|} \hline $(2) \Rightarrow (1)$ \\\hline \end{tabular} Notar que:
    \begin{eqnarray*}
      (x \ \IN \ y) \ \SU \ (x \ \IN \ z) &=& ((x \ \IN \ y) \ \SU \ x) \ \IN \ ((x \ \IN \ y) \ \SU \ z) \\
      &=& x \ \IN \ ((x \ \IN \ y) \ \SU \ z) \\
      &=& x \ \IN \ (z \ \SU \ (x \ \IN \ y)) \\
      &=& x \ \IN \ ((z \ \SU \ x) \ \IN \ (z \ \SU \ y)) \\
      &=& (x \ \IN \ (z \ \SU \ x)) \ \IN \ (z \ \SU \ y) \\
      &=& x \ \IN \ (z \ \SU \ y) \\
      &=& x \ \IN \ (y \ \SU \ z)
    \end{eqnarray*}
  \end{proof}

  % Lemma 111: Con prueba. Lemma 21.
  \begin{lemma} \label{lemma_21}
    \PN Si $\ACOTADO$ un reticulado acotado y distributivo, entonces todo elemento tiene a lo sumo un complemento.
  \end{lemma}
  \begin{proof}
    \PN Supongamos $x \in L$ tiene complementos $y, z$. Se tiene $y \ \SU \ x = \ 1 = \ x \ \SU \ z$ y $y \ \IN \ x = \
    0 = \ x \ \IN \ z$. Luego:
    \begin{eqnarray*}
      y &=& y \ \SU \ 0 \\
      &=& y \ \SU \ (x \ \IN \ z) \\
      &=& (y \ \SU \ x) \ \IN \ (y \ \SU \ z) \\
      &=& 1 \ \IN \ (y \ \SU \ z) \\
      &=& (x \ \SU \ z) \ \IN \ (y \ \SU \ z) \\
      &=& (x \ \IN \ y) \ \SU \ z \\
      &=& 0 \ \SU \ z \\
      &=& z
    \end{eqnarray*}
  \end{proof}

  % Lemma 112: Con prueba. Lemma 22.
  \begin{lemma} \label{lemma_22}
    \PN Si $S \neq \emptyset$, entonces $[S)$ es un filtro. Más aún si $F$ es un filtro y $F \supseteq S$, entonces $F
    \supseteq \lbrack S)$, es decir, \lbrack S) es el menor filtro que contiene a S.
  \end{lemma}
  \begin{proof}
    \PN Recordemos:
    \[
      [S) = \{y \in L: y \geq s_{1} \ \IN \ \dotsc \ \IN \ s_{n}, \text{ para algunos } s_{1}, \dotsc, s_{n} \in S,
      n \geq 1\}
    \]

    \begin{enumerate}
      \item \begin{tabular}{|c|} \hline $[S) \neq \emptyset$: \\\hline \end{tabular} Ya que $S \subseteq \lbrack S)$,
        tenemos que $[S) \neq \emptyset$.
      \item \begin{tabular}{|c|} \hline $x, y \in [S) \Rightarrow x \ \IN \ y \in [S)$: \\\hline \end{tabular} Sean $x,
        y$ tales que:
        \begin{eqnarray*}
          x \geq s_{1} \ \IN \ s_{2} \ \IN \ \dotsc \ \IN \ s_{n}, \ \text{ i.e, } x \in [S) \\
          y \geq t_{1} \ \IN \ t_{2} \ \IN \ \dotsc \ \IN \ t_{m}, \ \text{ i.e, } y \in [S)
        \end{eqnarray*}
        \PN con $s_{1}, s_{2}, \dotsc, s_{n}, t_{1}, t_{2}, \dotsc, t_{m} \in S$, entonces:
        \[
          x \ \IN \ y \geq s_{1} \ \IN \ s_{2} \ \IN \ \dotsc \ \IN \  s_{n} \ \IN \ t_{1} \ \IN \ t_{2} \ \IN \ \dotsc
          \ \IN \ t_{m}
        \]
      \item \begin{tabular}{|c|} \hline $x \in [S)$ y $x \leq y \Rightarrow y \in [S)$: \\\hline \end{tabular} Por
        construcción, claramente $[S)$ cumple esta propiedad.
    \end{enumerate}
  \end{proof}

  % Lemma 113: Sin prueba. Lemma 23.
  \begin{lemma} \label{lemma_23}
    \PN (\textbf{Zorn}) Sea $\POSET$ un poset y supongamos que cada cadena de $\POSET$ tiene una cota superior, entonces
    existe un elemento maximal en $\POSET$.
  \end{lemma}

  TODO
  % Theorem 114: Con prueba. Theorem 24.
  \begin{theorem} \label{theorem_24}
    \PN \textbf{(Teorema del Filtro Primo)} Sea $\RET$ un reticulado distributivo y $F$ un filtro. Supongamos $x_{0} \in
    L-F$, entonces hay un filtro primo $P$ tal que $x_{0} \notin P$ y $F \subseteq P$.
  \end{theorem}
  \begin{proof}
    \PN Sea:
    \[
      \mathcal{F} = \{F_{1}: F_{1} \text{ es un filtro, } x_{0} \notin F_{1} \text{ y } F \subseteq F_{1}\}
    \]

    \PN Notar que $\mathcal{F} \neq \emptyset$, por lo cual $(\mathcal{F}, \subseteq)$ es un poset.
    \PN Veamos que cada cadena en $(\mathcal{F}, \subseteq)$ tiene una cota superior. Sea $C$ una cadena.
    \begin{itemize}
      \item Si $C = \emptyset$, entonces cualquier elemento de $\mathcal{F}$ es cota de $C$.
      \item Si $C \neq \emptyset$. Sea:
        \[
          G = \{x \in L: x \in F_{1}, \text{ para algún } F_{1} \in C\}
        \]
    \end{itemize}

    \PN Veamos que $G$ es un filtro.
    \begin{enumerate}
      \item Es claro que $G \neq \emptyset$.
      \item Supongamos que $ x,y\in G$. Sean $F_{1},F_{2}\in \mathcal{F}$ tales que $x\in F_{1}$ y $y\in F_{2}$.
      \begin{itemize}
        \item Si $F_{1}\subseteq F_{2}$, entonces ya que $F_{2}$ es un filtro tenemos que $x \ \IN \ y\in F_{2}\subseteq G$.
        \item Si $F_{2}\subseteq F_{1}$ , entonces tenemos que $x \ \IN \ y \in F_{1} \subseteq G$.
      \end{itemize}
      \PN Ya que $C$ es una cadena, tenemos que siempre $x \ \IN \ y \in G$.

      \item En forma analoga se prueba la propiedad restante ... % TODO
    \end{enumerate}

    \PN Por lo tanto, tenemos que $G$ es un filtro. Además $x_{0} \notin G$, por lo que $G \in \mathcal{F}$ es cota
    superior de $C$. Por el \textbf{Lemma~\ref{lemma_23}}, $(\mathcal{F}, \subseteq)$ tiene un elemento maximal $P$.
    Veamos que $P$ es un filtro primo.

    \vspace{2mm}
    \PN Supongamos $x \ \SU \ y \in P$ y $x, y \notin P$, entonces ya que $P$ es maximal tenemos que:
    \[
      x_{0} \in \lbrack P \cup \{x\}) \cap \lbrack P \cup \{y\})
    \]
    \PN Ya que $x_{0} \in \lbrack P \cup \{x\})$, tenemos que hay elementos $p_{1}, \dotsc, p_{n} \in P$, tales que:
    \[
      x_{0} \geq p_{1} \ \IN \ \dotsc \ \IN \ p_{n} \ \IN \ x
    \]
    \PN Ya que $x_{0} \in \lbrack P \cup \{y\})$, tenemos que hay elementos $q_{1}, \dotsc, q_{m} \in P$, tales que:
    \[
      x_{0} \geq q_{1} \ \IN \ \dotsc \ \IN \ q_{m} \ \IN \ y
    \]
    \PN Denotemos:
    \[
      p = p_{1} \ \IN \ \dotsc \ \IN \ p_{n} \ \IN \ q_{1} \ \IN \  \dotsc \ \IN \ q_{m}
    \]
    \PN tenemos que:
    \begin{eqnarray*}
      x_{0} &\geq& p \ \IN \ x \\
      x_{0} &\geq& p \ \IN \ y
    \end{eqnarray*}
    \PN Se tiene que $x_{0} \geq (p \ \IN \ x) \ \SU \ (p \ \IN \ y) = p \ \IN \ (x \ \SU \ y) \in P$, lo cual es
    absurdo ya que $x_{0} \notin P$.
  \end{proof}

  % Corollary 115: Con prueba. Corollary 25.
  \begin{corollary} \label{corollary_25}
    \PN Sea $\ACOTADO$ un reticulado acotado distributivo. Si $\emptyset \neq S \subseteq L$ es tal que $s_{1} \ \IN \
    s_{2} \ \IN \ \dotsc \ \IN \ s_{n} \neq 0$, para cada $s_{1}, \dotsc, s_{n} \in S$, entonces hay un filtro primo que
    contiene a $S$.
  \end{corollary}
  \begin{proof}
    \PN Dado que $[S) \neq L$, se puede aplicar el \textbf{Theorem~\ref{theorem_24}} (Teorema del filtro primo).
  \end{proof}

  % Lemma 116: Con prueba. Lemma 26.
  \begin{lemma} \label{lemma_26}
    \PN Sea $\BOOLE$ un algebra de Boole, entonces para un filtro $F \subseteq B$ las siguientes son equivalentes:
    \begin{enumerate}[(1)]
      \item $F$ es primo
      \item $x \in F$ ó $x^{c} \in F$, para cada $x \in B$.
    \end{enumerate}
  \end{lemma}
  \begin{proof}
    \begin{tabular}{|c|} \hline $(1) \Rightarrow (2)$\\\hline \end{tabular} Ya que $1 \in F$ por definición de filtro,
      y $1 = x \ \SU \ x^{c}$ entonces $x \ \SU \ x^{c} \in F$. Finalmente, por definición de filtro primo se cumple que
      $x \in F$ ó $x^{c} \in F$.

    \vspace{3mm}
    \begin{tabular}{|c|} \hline $(2) \Rightarrow (1)$\\\hline \end{tabular} Supongamos que $x \ \SU \ y \in F$ y que
    $x \notin F$, entonces por (2), $x^{c} \in F$ y por lo tanto tenemos que:
    \[
      y \geq x^{c} \ \IN \ y = 0 \ \SU \ (x^{c} \ \IN \ y) = (x^{c} \ \IN \ x) \ \SU \ (x^{c} \ \IN \ y) = x^{c} \ \IN \
      (x \ \SU \ y) \in F
    \]
    \PN lo cual dice que $y \in F$.
  \end{proof}

  % Lemma 117: Con prueba. Lemma 27.
  \begin{lemma} \label{lemma_27}
    \PN Sea $\BOOLE$ un álgebra de Boole. Supongamos que $b \neq 0$ y $a = \inf A$, con $A \subseteq B$, entonces si $b
    \ \IN \ a = 0$ existe un $e \in A$ tal que $b \ \IN \ e^{c} \neq 0$.
  \end{lemma}
  \begin{proof}
    \PN Supongamos que para cada $e \in A$, tengamos que $b \ \IN \ e^{c} = 0$, entonces tenemos que para cada
    $e \in A$,
    \[
      b = b \ \IN \ (e \ \SU \ e^{c}) = (b \ \IN \ e)\ \SU \ (b \ \IN \ e^{c}) = b \ \IN \ e
    \]
    \PN es decir, $b \leq e \ \forall e \in A$, lo cual nos dice que $b$ es cota inferior de $A$. Pero si $b \leq a$,
    entonces $b = b \ \IN \ a = 0$, es decir, $b = 0$, lo cual es un absurdo dado que por hipótesis sabíamos que
    $b \neq 0$.
  \end{proof}

  % Theorem 118: Con prueba. Theorem 28.
  \begin{theorem} \label{theorem_28}
    \PN \textbf{(Rasiova y Sikorski)} Sea $\BOOLE$ un álgebra de Boole. Sea $x \in B$, tal que $x \neq 0$. Supongamos
    que $A_{1}, A_{2}, \dotsc$ son subconjuntos de $B$ tales que existe $\inf(A_{j})$, para cada $j = 1, 2, \dotsc$,
    entonces hay un filtro primo $P$ el cual cumple:
    \begin{enumerate}[a)]
      \item $x \in P$
      \item $A_{j} \subseteq P \Rightarrow \inf(A_{j}) \in P$, para cada $j = 1, 2, \; \dotsc$
    \end{enumerate}
  \end{theorem}
  \begin{proof}
    \PN Sea $a_{j} = \inf(A_{j})$, para $j = 1, 2, \; \dotsc $ construiremos inductivamente una sucesión $b_{0},
    b_{1}, \dotsc$ de elementos de $B$ tal que:
    \begin{itemize}
      \item $b_{0} = x$
      \item $b_{0} \ \IN \ \dotsc \ \IN \ b_{n} \neq 0$, para cada $n \geq 0$
      \item $b_{j} = a_{j}$ ó $b_{j}^{c} \in A_{j}$, para cada $j \geq 1$
    \end{itemize}
    \begin{enumerate}[(1)]
      \item Definamos $b_{0} = x$
      \item Supongamos ya definimos $b_{0}, \dotsc, b_{n}$, veamos como definir $b_{n+1}$.
        \begin{itemize}
          \item Si $(b_{0} \ \IN \ \dotsc \ \IN \  b_{n}) \ \IN \ a_{n+1}\neq 0$, entonces definamos $b_{n+1} = a_{n+1}$.
          \item Si $(b_{0} \ \IN \ \dotsc \ \IN \ b_{n}) \ \IN \ a_{n+1}=0$, entonces por el
            \textbf{Lemma~\ref{lemma_27}}, tenemos que hay un $e \in A_{n+1}$ tal que $(b_{0} \ \IN \ \dotsc \ \IN \
            b_{n}) \ \IN \ e^{c}\neq 0$, lo cual nos permite definir $b_{n+1} = e^{c}$.
        \end{itemize}
    \end{enumerate}

    \PN Dado que el conjunto $S = \{b_{0}, b_{1}, \dotsc\}$ satisface la hipótesis del
    \textbf{Corollary~\ref{corollary_25}}, por lo tanto hay un filtro primo $P$ tal que $\{b_{0}, b_{1}, \dotsc\}
    \subseteq P$, el cual satisface las propiedades (a) y (b) dado que así lo construimos.
  \end{proof}
