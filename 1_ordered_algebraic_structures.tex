\section{Estructuras algebráicas ordenadas}

  % Lemma 89: Con prueba. Lemma 1.
  \begin{lemma}
    \PN Sean $(P, \leq)$ y $(P^{\prime}, \leq^{\prime})$ posets. Supongamos que $F$ es un isomorfismo de $(P, \leq)$ en
    $(P^{\prime}, \leq^{\prime})$, entonces:

    \begin{enumerate}[a)]
      \item Para cada $S \subseteq P$ y cada $a \in P$, se tiene que $a$ es \textbf{cota superior} (resp.
        \textbf{inferior}) de $S$ si y solo si $F(a)$ es \textbf{cota superior} (resp. \textbf{inferior}) de $F(S)$.
      \item Para cada $S \subseteq P$, se tiene que $\exists \ \sup (S)$ si y solo si $\exists \ \sup (F(S))$ y en el
        caso de que existan tales elementos se tiene que $F(\sup (S)) = \sup (F(S))$.
      \item $P$ tiene $1$ (resp. $0$) si y solo si $P^{\prime }$ tiene $1$ (resp. $0$) y en tal caso tales elementos
        están conectados por $F$.
      \item Para cada $m \in P$, $m$ es \textbf{maximal} (resp. \textbf{minimal}) si y solo si $F(m)$ es
        \textbf{maximal} (resp. \textbf{minimal}).
      \item Para $a, b \in P$, tenemos que $a \prec b$ si y solo si $F(a) \prec^{\prime} F(b)$.
    \end{enumerate}
  \end{lemma}
  \begin{proof}
    \begin{enumerate}[a)]
      \item Probaremos solo el caso de la \textbf{cota superior}.
        \PN \begin{tabular}{|c|} \hline $\Rightarrow$ \\\hline \end{tabular} Supongamos que $a$ es \textbf{cota
        superior} de $S$, veamos entonces que $F(a)$ es \textbf{cota superior} de $F(S)$. Sean:
        \begin{itemize}
          \item $x \in F(S)$
          \item $s \in S$ tal que $x = F(s)$.
        \end{itemize}

        \PN Ya que $s \leq a$, tenemos que $x = F(s) \leq^{\prime} F(a)$. Luego, $F(a)$ es \textbf{cota superior}.

        \PN \begin{tabular}{|c|} \hline $\Leftarrow$ \\\hline \end{tabular} Supongamos ahora que $F(a)$ es \textbf{cota
        superior} de $F(S)$ y veamos entonces que $a$ es cota superior de $S$.

        \PN Sea $s \in S$, ya que $F(s) \leq^{\prime} F(a)$, tenemos que $s = F^{-1}(F(s)) \leq^{\prime} F^{-1}(F(a)) =
        a$. Por lo tanto, $a$ es \textbf{cota superior}.

      \item \begin{tabular}{|c|} \hline $\Rightarrow$ \\\hline \end{tabular} Supongamos existe $\sup (S)$. Veamos
        que $F(\sup (S))$ es el supremo de $F(S)$. Por el iniciso (a) $F(\sup (S))$ es cota superior de $F(S)$. Veamos
        que es la menor de las cotas superiores. Supongamos $b^{\prime}$ cota superior de $F(S)$, entonces
        $F^{-1}(b^{\prime})$ es cota superior de $S$, es decir, $\sup (S) \leq F^{-1}(b^{\prime})$, produciendo
        $F(\sup (S)) \leq^{\prime} b^{\prime}$. Por lo tanto, $F(\sup (S))$ es el supremo de $F(S)$.

        \PN \begin{tabular}{|c|} \hline $\Leftarrow$ \\\hline \end{tabular} Supongamos existe $\sup (F(S))$. Veamos
        que $F^{-1}(\sup (F(S)))$ es el supremo de $S$. Nuevamente, por el iniciso (a) $F^{-1}(\sup (F(S)))$ es cota
        superior de $S$. Veamos que es la menor de las cotas superiores. Supongamos $b$ cota superior de $S$, entonces
        $F(b)$ es cota superior de $F(S)$, es decir, $\sup (F(S)) \leq F(b)$, produciendo $F^{-1}(\sup (F(S))) \leq b$.
        Por lo tanto, $F^{-1}(\sup (F(S)))$ es el supremo de $S$.

      \item Se desprende del inciso (b) tomando $S = P$.
      \item Probaremos solo el caso \textbf{maximal}.
        \PN \begin{tabular}{|c|} \hline $\Rightarrow$ \\\hline \end{tabular} Supongamos que $m$ es maximal de
        $(P, \leq)$. Veamos que $F(m)$ es maximal de $(P^{\prime}, \leq^{\prime})$. Supongamos que $F(m)$ no es maximal
        de $(P^{\prime}, \leq^{\prime})$, es decir, $F(m) <^{\prime} b^{\prime} \ \forall b^{\prime} \in P^{\prime}$.
        Dado que $F$ es isomorfismo:
        \begin{eqnarray*}
    			F^{-1}(F(m)) < F^{-1}(b^{\prime}) \\
    			m < F^{-1}(b^{\prime})
    		\end{eqnarray*}
        \PN Lo cual es un absurdo, dado que $m$ es maximal de $(P, \leq)$. Por lo tanto, $F(m)$ es maximal de
        $(P^{\prime}, \leq^{\prime})$.

        \PN \begin{tabular}{|c|} \hline $\Leftarrow$ \\\hline \end{tabular} Supongamos que $F(m)$ es maximal de
        $(P^{\prime}, \leq^{\prime})$. Veamos que $m$ es maximal de $(P, \leq)$. Supongamos que $m$ no es maximal
        de $(P, \leq)$, es decir, $m < b \ \forall b \in P$. Dado que $F$ es isomorfismo:
        \[
    			F(m) < F(b)
    		\]
        \PN Lo cual es un absurdo, dado que $F(m)$ es maximal de $(P^{\prime}, \leq^{\prime})$. Por lo tanto, $m$ es
        maximal de $(P, \leq)$.

      \item \begin{tabular}{|c|} \hline $\Rightarrow$ \\\hline \end{tabular} Supongamos $a \prec b$, veamos que $F(a)
        \prec^{\prime} F(b)$. Debemos ver:
        \begin{enumerate}[1)]
          \item $F(a) <^{\prime} F(b)$
          \item $\nexists z^{\prime}$ tal que $F(a) < z^{\prime} < F(b)$
        \end{enumerate}

        \PN Ya que $a \prec b$, por definición tenemos: \begin{tabular}{|c|} \hline $a < b$ y $\nexists z$ tal que
        $a < z < b$ \\\hline \end{tabular} $(\star)$

        \PN Dado que la función $F$ es un isomorfismo, se cumple (1). Veamos que se cumple (2), supongamos que $\exists
        z^{\prime}$ tal que $F(a) < z^{\prime} < F(b)$. Luego, nuevamente utilizando que $F$ es isomorfismo, tenemos:
        \begin{eqnarray*}
    			F^{-1}(F(a)) < &F^{-1}(z^{\prime})& < F^{-1}(F(b)) \\
    			a < &F^{-1}(z^{\prime})& < b
    		\end{eqnarray*}
        \PN Lo cual, contradice $(\star)$, el absurdo vino de suponer que $\exists z^{\prime}$ tal que
        $F(a) < z^{\prime} < F(b)$, por lo tanto $\nexists z^{\prime}$ tal que $F(a) < z^{\prime} < F(b)$.

        \PN Finalmente, dado que se cumplen los puntos (1) y (2), se cumple también $F(a) \prec^{\prime} F(b)$.

        \PN \begin{tabular}{|c|} \hline $\Leftarrow$ \\\hline \end{tabular} Supongamos $F(a) \prec^{\prime} F(b)$,
        veamos que $a \prec b$.

		    \PN Ya que $F^{-1}: (P^{\prime}, \leq^{\prime}) \rightarrow (P, \leq)$ es isomorfismo, por lo ya visto tenemos:
    		\begin{eqnarray*}
    			F^{-1}(F(a)) &\prec& F^{-1}(F(b)) \\
    			a &\prec& b
    		\end{eqnarray*}
    \end{enumerate}
  \end{proof}

  % % Lemma 90: Con prueba. Lemma 2.
  % \begin{lemma}
  %   \PN Dado un reticulado $(L, \leq)$ y elementos $x, y, z, w \in L$, se cumplen las siguientes propiedades:
  %   \begin{enumerate}
  %     \item $x \leq x \ \mathsf{s} \ y$
  %     \item $x \ \mathsf{i} \ y \leq x$
  %     \item $x \ \mathsf{s} \ x = x \ \mathsf{i} \ x = x$
  %     \item $x \ \mathsf{s} \ y = y \ \mathsf{s} \ x$
  %     \item $x \ \mathsf{i} \ y = y \ \mathsf{i} \ x$
  %     \item $x \leq y \Leftrightarrow x \ \mathsf{s} \ y = y \Leftrightarrow x \ \mathsf{i} \ y = x$
  %     \item $x \ \mathsf{s} \ (x \ \mathsf{i} \ y) = x$
  %     \item $x \ \mathsf{i}(x \ \mathsf{s} \ y) = x$
  %     \item $(x \ \mathsf{s} \ y) \ \mathsf{s} \ z = x \ \mathsf{s} \ (y \ \mathsf{s} \ z)$
  %     \item $(x \ \mathsf{i} \ y) \ \mathsf{i} \ z = x \ \mathsf{i} \ (y \ \mathsf{i} \ z)$
  %     \item Si $x \leq z$ e $y \leq w \Rightarrow x \ \mathsf{s} \ y \leq z \ \mathsf{s} \ w$ y $x \ \mathsf{i} \ y \leq
  %       z \ \mathsf{i} \ w$
  %     \item $(x \ \mathsf{i} \ y) \ \mathsf{s} \ (x \ \mathsf{i} \ z) \leq x \ \mathsf{i} \ (y \ \mathsf{s} \ z)$
  %   \end{enumerate}
  % \end{lemma}
  % \begin{proof}
  %   \begin{enumerate}
  %     \item \begin{tabular}{|c|} \hline $x \leq x \ \mathsf{s} \ y$ \\\hline \end{tabular}
  %       \begin{eqnarray*}
  %         \textup{Si } x \leq y &\Rightarrow& x \leq x \ \mathsf{s} \ y = y \Rightarrow x \leq y \\
  %         \textup{Si } y \leq x &\Rightarrow& x \leq x \ \mathsf{s} \ y = x \Rightarrow x \leq x
  %       \end{eqnarray*}
  %     \item \begin{tabular}{|c|} \hline $x \ \mathsf{i} \ y \leq x$ \\\hline \end{tabular}
  %       \begin{eqnarray*}
  %         \textup{Si } x \leq y \Rightarrow x \ \mathsf{i} \ y &\leq& x \\
  %         x &\leq& x \\
  %         \textup{Si } y \leq x \Rightarrow x \ \mathsf{i} \ y &\leq& x \\
  %         y &\leq& x
  %       \end{eqnarray*}
  %     \item \begin{tabular}{|c|} \hline $x \ \mathsf{s} \ x = x \ \mathsf{i} \ x = x$ \\\hline \end{tabular}
  %       \begin{eqnarray*}
  %         x \ \mathsf{s} \ x &=& x \ \mathsf{i} \ x = x \\
  %         x &=& x = x
  %       \end{eqnarray*}
  %       % TODO
  %     \item \begin{tabular}{|c|} \hline $x \ \mathsf{s} \ y = y \ \mathsf{s} \ x$ \\\hline \end{tabular}
  %       \begin{eqnarray*}
  %         \textup{Si } x \leq y \Rightarrow x \ \mathsf{s} \ y &=& y \ \mathsf{s} \ x \\
  %         y &=& y \\
  %         \textup{Si } y \leq x \Rightarrow x \ \mathsf{s} \ y &=& y \ \mathsf{s} \ x \\
  %         x &=& x
  %       \end{eqnarray*}
  %     \item \begin{tabular}{|c|} \hline $x \ \mathsf{i} \ y = y \ \mathsf{i} \ x$ \\\hline \end{tabular}
  %       \begin{eqnarray*}
  %         \textup{Si } x \leq y \Rightarrow x \ \mathsf{i} \ y &=& y \ \mathsf{i} \ x \\
  %         x &=& x \\
  %         \textup{Si } y \leq x \Rightarrow x \ \mathsf{i} \ y &=& y \ \mathsf{i} \ x \\
  %         y &=& y
  %       \end{eqnarray*}
  %     \item \begin{tabular}{|c|} \hline $x \leq y \Leftrightarrow x \ \mathsf{s} \ y = y \Leftrightarrow x \ \mathsf{i}
  %       \ y = x$ \\\hline \end{tabular}
  %       % TODO
  %     \item Ya que $x\mathsf{\;i\;}y\leq x$, (6) nos dice que $(x\mathsf{\;i\;}y)\; \mathsf{s}\;x=x$, por lo cual $x\;\mathsf{s}\;(x\mathsf{\;i\;}y)=x$.
  %     \item Similar a (7).
  %     \item Notese que $x\;\mathsf{s}\;(y\;\mathsf{s}\;z)$ es cota superior de $ \{x,y,z\}$ ya que onviamente $x\leq x\;\mathsf{s}\;(y\;\mathsf{s}\;z)$ y ademas
  %       $\displaystyle \begin{array}{rcl} y & \leq & (y\;\mathsf{s}\;z)\leq x\;\mathsf{s}\;(y\;\mathsf{s}\;z) \\ z & \leq & (y\;\mathsf{s}\;z)\leq x\;\mathsf{s}\;(y\;\mathsf{s}\;z) \end{array} $
  %
  %       Ya que $x\;\mathsf{s}\;(y\;\mathsf{s}\;z)$ es cota superior de $\{x,y\}$, tenemos que $x\;\mathsf{s}\;y\leq x\;\mathsf{s}\ (y\;\mathsf{s}\;z)$, por lo cual $x\;\mathsf{s}\;(y\;\mathsf{s}\;z)$ es cota superior del conjunto $\{x\; \mathsf{s}\;y,z\}$, lo cual dice que $(x\;\mathsf{s}\;y)\;\mathsf{s}\;z\leq x\;\mathsf{s}\;(y\;\mathsf{s}\;z)$. Analogamente se puede probar que $x\; \mathsf{s}\;(y\;\mathsf{s}\;z)\leq (x\;\mathsf{s}\;y)\;\mathsf{s}\;z$.
  %     \item Similar a (9).
  %     \item Ya que
  %       $\displaystyle \begin{array}{rcl} x & \leq & z\leq z\;\mathsf{s}\;w \\ y & \leq & w\leq z\;\mathsf{s}\;w \end{array} $
  %
  %       tenemos que $z\;\mathsf{s}\;w$ es cota superior de $\{x,y\}$ lo cual dice que $x\;\mathsf{s}\;y\leq z\;\mathsf{s}\;w$. La otra desigualdad es analoga.
  %     \item Ya que
  %       $\displaystyle \begin{array}{rcl} (x\mathsf{\;i\;}y),(x\mathsf{\;i\;}z) & \leq & x \\ (x\mathsf{\;i\;}y),(x\mathsf{\;i\;}z) & \leq & y\;\mathsf{s}\;z \end{array} $
  %
  %       tenemos que $(x\;\mathsf{i}\;y),(x\mathsf{\;i\;}z)\leq x\mathsf{\;i\;}(y\; \mathsf{s}\;z)$, por lo cual $(x\mathsf{\;i\;}y)\;\mathsf{s}\;(x\mathsf{\;i\; }z)\leq x\mathsf{\;i\;}(y\;\mathsf{s}\;z)$.
  %   \end{enumerate}
  % \end{proof}
  %
  % % Lemma 91: Con prueba. Lemma 3.
  % \begin{lemma}
  %   \PN Sea $(L, \leq)$ un reticulado, dados elementos $x_{1}, \dotsc, x_{n} \in L$, con $n \geq 2$, se tiene
  %   \[
  %     \begin{array}{rcl}
  %       (\dotsc (x_{1} \ \mathsf{s} \ x_{2}) \ \mathsf{s} \ \dotsc) \ \mathsf{s} \ x_{n} &=& \sup (\{x_{1}, \dotsc,
  %         x_{n}\}) \\
  %       (\dotsc (x_{1} \ \mathsf{i} \ x_{2}) \ \mathsf{i} \ \dotsc) \ \mathsf{i} \ x_{n} &=& \inf (\{x_{1}, \dotsc,
  %       x_{n}\})
  %     \end{array}
  %   \]
  % \end{lemma}
  % \begin{proof}
  %   \PN Probaremos por inducción en $n$.
  %
  %   \vspace{3mm}
  %   \PN \underline{Caso Base:} \begin{tabular}{|c|} \hline $n = 2$ \\\hline \end{tabular}
  %
  %   \vspace{3mm}
	% 	\PN \underline{Caso Inductivo:} \begin{tabular}{|c|} \hline $n > 2$ \\\hline \end{tabular}
  %
  %   \PN Supongamos ahora que vale para $n$ y veamos entonces que vale para $n+1$. Sean $x_{1}, \dotsc, x_{n+1} \in L$,
  %   por hipótesis inductiva tenemos que:
  %   \[
  %     \begin{tabular}{|c|} \hline $(\dotsc (x_{1} \ \mathsf{s} \ x_{2}) \ \mathsf{s} \ \dotsc) \ \mathsf{s} \ x_{n} =
  %     \sup (\{x_{1}, \dotsc, x_{n}\})$ \\\hline \end{tabular} \ (\star)
  %   \]
  %
  %   \PN Veamos entonces que:
  %
  %   (2) $((...(x_{1}\;\mathsf{s\;}x_{2})\;\mathsf{s\;}...)\;\mathsf{s\;} x_{n})\;\mathsf{s\;}x_{n+1}=\sup (\{x_{1}, \dotsc, x_{n+1}\}).$
  %   Es facil ver que $((...(x_{1}\;\mathsf{s\;}x_{2})\;\mathsf{s\;} ...)\;\mathsf{s\;}x_{n})\;\mathsf{s\;}x_{n+1}$ es cota superior de $ \{x_{1}, \dotsc, x_{n+1}\}$. Supongamos que $z$ es otra cota superior. Ya que $z$ es tambien cota superior del conjunto $\{x_{1}, \dotsc, x_{n}\}$, por (1) tenemos que
  %
  %   $\displaystyle (...(x_{1}\;\mathsf{s\;}x_{2})\;\mathsf{s}\;...)\;\mathsf{s\;}x_{n}\leq z. $
  %
  %   Pero entonces ya que $x_{n+1}\leq z$, tenemos que
  %   $\displaystyle ((...(x_{1}\;\mathsf{s\;}x_{2})\;\mathsf{s\;}...)\;\mathsf{s\;}x_{n})\; \mathsf{s\;}x_{n+1}\leq z, $
  %
  %   con lo cual hemos probado (2).
  % \end{proof}
  %
  % % Theorem 92: Con prueba. Theorem 4.
  % \begin{theorem}
  %   \PN Sea $(L, \mathsf{s}, \mathsf{i})$ un reticulado, la relación binaria definida por:
  %   \[
  %     x \leq y \Leftrightarrow x \ \mathsf{s} \ y=y
  %   \]
  %
  %   \PN es un orden parcial sobre $L$ para el cual se cumple:
  %   \[
  %     \begin{array}{rcl}
  %       \sup (\{x, y\}) &=& x \ \mathsf{s} \ y \\
  %       \inf (\{x, y\}) &=& x \ \mathsf{i} \ y
  %     \end{array}
  %   \]
  % \end{theorem}
  % \begin{proof}
  %   \begin{itemize}
  %     \item \underline{Reflexiva:}
  %     \item \underline{Antisimétrica:}
  %     \item \underline{Transitiva:} Supongamos que $x \leq y$ e $y \leq z$, entonces:
  %       \[
  %         x \ \mathsf{s} \ z = x \ \mathsf{s} \ (y \ \mathsf{s} \ z) = (x \ \mathsf{s} \ y) \ \mathsf{s} \ z = y \ \mathsf{s} \ z = z
  %       \]
  %
  %       \PN por lo cual $x \leq z$. Veamos ahora que $\sup (\{x,y\})=x\;\mathsf{s\;}y$. Es claro que $x\;\mathsf{s\;}y$ es una cota superior del conjunto $\{x,y\}$. Supongamos $x,y\leq z$. Entonces
  %     $\displaystyle (x\;\mathsf{s\;}y)\;\mathsf{s\;}z=x\;\mathsf{s\;}(y\;\mathsf{s\;}z)=x\; \mathsf{s\;}z=z, $
  %
  %     por lo que $x\;\mathsf{s\;}y\leq z$. Es decir que $x\;\mathsf{s\;}y$ es la menor cota superior.
  %     Para probar que $\inf (\{x,y\})=x\mathsf{\;i\;}y$, probaremos que para todo $ u,v\in L$,
  %
  %     $\displaystyle u\leq v\text{ si y solo si }u\mathsf{\;i\;}v=u, $
  %
  %     lo cual le permitira al lector aplicar un razonamiento similar al usado en el caso de la operacion $\mathsf{s}$. Supongamos que $u\;\mathsf{s}\;v=v$. Entonces $u\mathsf{\;i\;}v=u\mathsf{\;i\;}(v\;\mathsf{s}\;v)=u$. Reciprocamente si $u\mathsf{\;i\;}v=u$, entonces $u\;\mathsf{s}\;v=(u\mathsf{ \;i\;}v)\;\mathsf{s}\;v=v$, por lo cual $u\leq v$.
  %   \end{itemize}
  % \end{proof}
  %
  % % Lemma 93: Con prueba. Lemma 5.
  % \begin{lemma}
  %   \PN Si $F: (L, \mathsf{s}, \mathsf{i}) \rightarrow (L^{\prime}, \mathsf{s}^{\prime}, \mathsf{i}^{\prime})$ es un
  %   homomorfismo biyectivo, entonces $F$ es un isomorfismo.
  % \end{lemma}
  % \begin{proof}
  %   \PN Debemos probar que $F^{-1}$ es un homomorfismo. Sean $F(x), F(y)$ dos elementos cualesquiera de $L^{\prime}$,
  %   tenemos que:
  %   \begin{eqnarray*}
  %     F^{-1}(F(x) \ \mathsf{s}^{\prime} \ F(y)) &=& F^{-1}(F(x \ \mathsf{s} \ y)) \\
  %     &=& x \ \mathsf{s} \ y \\
  %     &=& F^{-1}(F(x)) \ \mathsf{s} \ F^{-1}(F(y))
  %   \end{eqnarray*}
  %
  %   \PN Luego, $F^{-1}$ es homomorfismo y por lo tanto $F$ es isomorfismo.
  % \end{proof}
  %
  % % Lemma 94: Con prueba. Lemma 6.
  % \begin{lemma}
  %   \PN Sean $(L, \mathsf{s}, \mathsf{i})$ y $(L^{\prime}, \mathsf{s}^{\prime}, \mathsf{i}^{\prime})$ reticulados y sea
  %   $F: (L, \mathsf{s}, \mathsf{i}) \rightarrow (L^{\prime}, \mathsf{s}^{\prime}, \mathsf{i}^{\prime})$ un homomorfismo,
  %   entonces $I_{F}$ es un subuniverso de $(L^{\prime}, \mathsf{s}^{\prime}, \mathsf{i}^{\prime})$.
  % \end{lemma}
  % \begin{proof}
  %   Ya que $L$ es no vacio tenemos que $I_{F}$ tambien es no vacio. Sean $a,b\in I_{F}$. Sean $x,y\in L$ tales que $F(x)=a$ y $F(y)=b$. Se tiene que
  %   \begin{eqnarray*}
  %     a\;\mathsf{s}^{\prime }\ b &=& F(x)\;\mathsf{s}^{\prime }\ F(y)=F(x\mathsf{ \;s\;}y)\in I_{F} \\
  %     a\;\mathsf{i}^{\prime }\ b & =& F(x)\;\mathsf{i}^{\prime }\ F(y)=F(x\mathsf{ \;i\;}y)\in I_{F}
  %   \end{eqnarray*}
  %
  %   \PN por lo cual $I_{F}$ es cerrada bajo $\mathsf{s}^{\prime }$ e $\mathsf{i} ^{\prime }$.
  % \end{proof}
  %
  % % Lemma 95: Con prueba. Lemma 7.
  % \begin{lemma}
  %   \PN Sean $(L,\mathsf{s},\mathsf{i})$ y $(L^{\prime },\mathsf{s}^{\prime }, \mathsf{i}^{\prime })$ reticulados y sean $(L\leq )$ y $(L^{\prime },\leq ^{\prime })$ los posets asociados. Sea $F:L\rightarrow L^{\prime }$ una funcion. Entonces $F$ es un isomorfismo de $(L,\mathsf{s},\mathsf{i})$ en $ (L^{\prime },\mathsf{s}^{\prime },\mathsf{i}^{\prime })$ si y solo si $F$ es un isomorfismo de $(L,\leq )$ en $(L^{\prime },\leq ^{\prime })$.
  % \end{lemma}
  % \begin{proof}
  %   Supongamos $F$ es un isomorfismo de $(L,\mathsf{s},\mathsf{i})$ en $ (L^{\prime },\mathsf{s}^{\prime },\mathsf{i}^{\prime })$. Sean $x,y\in L$, tales que $x\leq y$. Tenemos que $y=x\mathsf{\;s\;}y$ por lo cual $F(y)=F(x \mathsf{\;s\;}y)=F(x)\mathsf{\;s^{\prime }\;}F(y)$, produciendo $F(x)\leq ^{\prime }F(y)$. En forma similar se puede ver que $F^{-1}$ es tambien un homomorfismo de $(L^{\prime },\leq ^{\prime })$ en $(L,\leq )$. Si $F$ es un isomorfismo de $(L,\leq )$ en $(L^{\prime },\leq ^{\prime })$, entonces el Lema 89 nos dice que $F$ y $F^{-1}$ respetan las operaciones de supremo e infimo por lo cual $F$ es un isomorfismo de $(L,\mathsf{s},\mathsf{ i})$ en $(L^{\prime },\mathsf{s}^{\prime },\mathsf{i}^{\prime })$. $\Box$
  % \end{proof}
  %
  % % Lemma 96: Con prueba. Lemma 8.
  % \begin{lemma}
  %   $(L/\theta ,\mathsf{\tilde{s}},\mathsf{\tilde{\imath}})$ es un reticulado. El orden parcial $\tilde{\leq}$ asociado a este reticulado cumple
  %   $\displaystyle x/\theta \tilde{\leq}y/\theta \text{ sii }y\theta (x\mathsf{\;s\;}y) $
  % \end{lemma}
  % \begin{proof}
  %   Veamos que la estructura $(L/\theta ,\mathsf{\tilde{s}},\mathsf{\tilde{\imath }})$ cumple (I4). Sean $x/\theta $, $y/\theta $, $z/\theta $ elementos cualesquiera de $L/\theta $. Tenemos que
  %
  %   $\displaystyle \begin{array}{ccl} (x/\theta \mathsf{\;\tilde{s}\;}y/\theta )\;\mathsf{\tilde{s}}\;z/\theta & = & (x\mathsf{\;s\;}y)/\theta \;\mathsf{\tilde{s}}\;z/\theta \\ & = & ((x\mathsf{\;s\;}y)\;\mathsf{s}\;z)/\theta \\ & = & (x\mathsf{\;s\;}(y\;\mathsf{s}\;z))/\theta \\ & = & x/\theta \;\mathsf{\tilde{s}}\;(y\;\mathsf{s}\;z)/\theta \\ & = & x/\theta \mathsf{\;\tilde{s}\;}(y/\theta \;\mathsf{\tilde{s}} \;z/\theta ) \end{array} $
  %
  %   En forma similar se puede ver que la estructura $(L/\theta ,\mathsf{\tilde{s} },\mathsf{\tilde{\imath}})$ cumple el resto de las identidades que definen reticulado.
  % \end{proof}
  %
  % % Corollary 97: Con prueba. Corollary 9.
  % \begin{corollary}
  %   Sea $(L,\mathsf{s},\mathsf{i})$ un reticulado en el cual hay un elemento maximo $1$ (resp. minimo $0$). Entonces si $\theta $ es una congruencia sobre $(L,\mathsf{s},\mathsf{i})$, $1/\theta $ (resp. $0/\theta $) es un elemento maximo (resp. minimo) de $(L/\theta ,\mathsf{\tilde{s}},\mathsf{ \tilde{\imath}})$.
  % \end{corollary}
  % \begin{proof}
  %   Ya que $1\theta (x\mathsf{\;s\;}1)$, para cada $x\in L$, tenemos que $ x/\theta \tilde{\leq}1/\theta $, para cada $x\in L$. $\Box$
  % \end{proof}
  %
  % % Lemma 98: Con prueba. Lemma 10.
  % \begin{lemma}
  %   Si $F:(L,\mathsf{s},\mathsf{i})\rightarrow (L^{\prime },\mathsf{s}^{\prime }, \mathsf{i}^{\prime })$ es un homomorfismo de reticulados, entonces $\ker F$ es una congruencia sobre $(L,\mathsf{s},\mathsf{i})$.
  % \end{lemma}
  % \begin{proof}
  %   Dejamos al lector ver que $\ker F$ es una relacion de equivalencia. Supongamos $x\ker Fx^{\prime }$ y $y\ker Fy^{\prime }$. Entonces
  %
  %   $\displaystyle F(x\mathsf{\;s\;}y)=F(x)\mathsf{\;s^{\prime }\;}F(y)=F(x^{\prime })\mathsf{ \;s^{\prime }\;}F(y^{\prime })=F(x^{\prime }\mathsf{\;s\;}y^{\prime }) $
  %
  %   lo cual nos dice que $(x\mathsf{\;s\;}y)\ker F(x^{\prime }\mathsf{\;s\;} y^{\prime })$. En forma similar tenemos que $(x\mathsf{\;i\;}y)\ker F(x^{\prime }\mathsf{\;i\;}y^{\prime })$. $\Box$
  % \end{proof}
  %
  % % Lemma 99: Con prueba. Lemma 11.
  % \begin{lemma}
  %   Sea $(L,\mathsf{s},\mathsf{i})$ un reticulado y sea $\theta $ una congruencia sobre $(L,\mathsf{s},\mathsf{i})$. Entonces $\pi _{\theta }$ es un homomorfismo de $(L,\mathsf{s},\mathsf{i})$ en $(L/\theta ,\mathsf{\tilde{ s}},\mathsf{\tilde{\imath}})$. Ademas $\ker \pi _{\theta }=\theta $.
  % \end{lemma}
  % \begin{proof}
  %   Sean $x,y\in L$. Tenemos que
  %
  %   $\displaystyle \pi _{\theta }(x\mathsf{\;s\;}y)=(x\mathsf{\;s\;}y)/\theta =x/\theta \mathsf{ \;\tilde{s}\;}y/\theta =\pi _{\theta }(x)\mathsf{\;\tilde{s}\;}\pi _{\theta }(y) $
  %
  %   por lo cual $\pi _{\theta }$ preserva la operacion supremo. Para la operacion infimo es similar. $\Box$
  % \end{proof}
  %
  % % Lemma 100: Con prueba. Lemma 12.
  % \begin{lemma}
  %   Si $F:(L,\mathsf{s},\mathsf{i},0,1)\rightarrow (L^{\prime },\mathsf{s} ^{\prime },\mathsf{i}^{\prime },0^{\prime },1^{\prime })$ un homomorfismo biyectivo, entonces $F$ es un isomorfismo
  % \end{lemma}
  % \begin{proof}
  %   Similar a la prueba del Lemma 93. $\Box$
  % \end{proof}
  %
  % % Lemma 101: Con prueba. Lemma 13.
  % \begin{lemma}
  %   Si $F:(L,\mathsf{s},\mathsf{i},0,1)\rightarrow (L^{\prime },\mathsf{s} ^{\prime },\mathsf{i}^{\prime },0^{\prime },1^{\prime })$ es un homomorfismo, entonces $I_{F}$ es un subuniverso de $(L^{\prime },\mathsf{s} ^{\prime },\mathsf{i}^{\prime },0^{\prime },1^{\prime })$.
  % \end{lemma}
  % \begin{proof}
  %   Ya que $F$ es un homomorfismo de $(L,\mathsf{s},\mathsf{i})$ en $ (L^{\prime },\mathsf{s}^{\prime },\mathsf{i}^{\prime })$ tenemos que $I_{F}$ es subuniverso de $(L^{\prime },\mathsf{s}^{\prime },\mathsf{i}^{\prime })$ lo cual ya que $0^{\prime },1^{\prime }\in I_{F}$ implica que $I_{F}$ es un subuniverso de $(L^{\prime },\mathsf{s}^{\prime },\mathsf{i}^{\prime },0^{\prime },1^{\prime })$. $\Box$
  % \end{proof}
  %
  % % Lemma 102: Sin prueba. Lemma 14.
  % \begin{lemma}
  %   \PN Si $F: (L, \mathsf{s}, \mathsf{i}, 0, 1) \rightarrow (L^{\prime}, \mathsf{s}^{\prime}, \mathsf{i}^{\prime},
  %   0^{\prime}, 1^{\prime})$ es un homomorfismo de reticulados acotados, entonces $\ker F$ es una congruencia sobre $(L,
  %   \mathsf{s}, \mathsf{i}, 0, 1)$.
  % \end{lemma}
  %
  % % Lemma 103: Sin prueba. Lemma 15.
  % \begin{lemma}
  %   \PN Sea $(L, \mathsf{s}, \mathsf{i}, 0, 1)$ un reticulado acotado y $\theta$ una congruencia sobre $(L, \mathsf{s},
  %   \mathsf{i}, 0, 1)$, entonces:
  %   \begin{enumerate}[a)]
  %     \item $(L/\theta, \mathsf{\tilde{s}}, \mathsf{\tilde{\imath}}, 0/\theta, 1/\theta)$ es un reticulado acotado.
  %     \item $\pi_{\theta}$ es un homomorfismo de $(L, \mathsf{s}, \mathsf{i}, 0, 1)$ en $(L/\theta, \mathsf{\tilde{s}},
  %       \mathsf{\tilde{\imath}}, 0/\theta, 1/\theta)$ cuyo núcleo es $\theta$.
  %   \end{enumerate}
  % \end{lemma}
  %
  % % Lemma 104: Sin prueba. Lemma 16.
  % \begin{lemma}
  %   \PN Si $F:(L, \mathsf{s}, \mathsf{i}, ^{c}, 0, 1) \rightarrow (L^{\prime}, \mathsf{s}^{\prime}, \mathsf{i}^{\prime},
  %   ^{c^{\prime}}, 0^{\prime}, 1^{\prime})$ un homomorfismo biyectivo, entonces $F$ es un isomorfismo.
  % \end{lemma}
  %
  % % Lemma 105: Sin prueba. Lemma 17.
  % \begin{lemma}
  %   \PN Si $F: (L, \mathsf{s}, \mathsf{i}, ^{c}, 0, 1) \rightarrow (L^{\prime}, \mathsf{s}^{\prime },
  %   \mathsf{i}^{\prime}, ^{c^{\prime}}, 0^{\prime}, 1^{\prime})$ es un homomorfismo, entonces $I_{F}$ es un subuniverso
  %   de $(L^{\prime}, \mathsf{s}^{\prime}, \mathsf{i}^{\prime}, ^{c^{\prime}}, 0^{\prime}, 1^{\prime})$.
  % \end{lemma}
  %
  % % Lemma 106: Sin prueba. Lemma 18.
  % \begin{lemma}
  %   \PN Si $F: (L, \mathsf{s}, \mathsf{i}, ^{c}, 0, 1) \rightarrow (L^{\prime}, \mathsf{s}^{\prime},
  %   \mathsf{i}^{\prime}, ^{c^{\prime}}, 0^{\prime}, 1^{\prime})$ es un homomorfismo de reticulados complementados,
  %   entonces $\ker F$ es una congruencia sobre $(L, \mathsf{s}, \mathsf{i}, ^{c}, 0, 1)$.
  % \end{lemma}
  %
  % % Lemma 107: Sin prueba. Lemma 19.
  % \begin{lemma}
  %   \PN Sea $(L, \mathsf{s}, \mathsf{i}, ^{c}, 0, 1)$ un reticulado complementado y sea $\theta$ una congruencia sobre
  %   $(L, \mathsf{s}, \mathsf{i}, ^{c}, 0, 1)$.
  %   \begin{enumerate}[a)]
  %     \item $(L/\theta, \mathsf{\tilde{s}}, \mathsf{\tilde{\imath}}, ^{\tilde{c}}, 0/\theta, 1/\theta)$ es un reticulado
  %       complementado.
  %     \item $\pi_{\theta}$ es un homomorfismo de $(L, \mathsf{s}, \mathsf{i}, ^{c}, 0, 1)$ en $(L/\theta,
  %       \mathsf{\tilde{s}}, \mathsf{\tilde{\imath}}, ^{\tilde{c}}, 0/\theta, 1/\theta)$ cuyo núcleo es $\theta$.
  %   \end{enumerate}
  % \end{lemma}
  %
  % % Lemma 108: Con prueba. Lemma 20.
  % \begin{lemma}
  %   Sea $(L,\mathsf{s},\mathsf{i})$ un reticulado. Son equivalentes:
  %   (1) $x\mathsf{\;i\;}(y\;\mathsf{s}\;z)=(x\mathsf{\;i\;}y)\;\mathsf{s} \;(x\mathsf{\;i\;}z)$, cualesquiera sean $x,y,z\in L$
  %   (2) $x\;\mathsf{s}\;(y\mathsf{\;i\;}z)=(x\mathsf{\;s\;}y)\mathsf{\;i\; }(x\;\mathsf{s}\;z)$, cualesquiera sean $x,y,z\in L$.
  % \end{lemma}
  % \begin{proof}
  %   (1)$\Rightarrow $(2). Notese que
  %
  %   $\displaystyle \begin{array}{rcl} (x\mathsf{\;s\;}y)\mathsf{\;i\;}(x\;\mathsf{s}\;z) & =& ((x\mathsf{\;s\;}y) \mathsf{\;i\;}x)\;\mathsf{s}\;((x\mathsf{\;s\;}y)\mathsf{\;i\;}z) \\ & =& (x\;\mathsf{s}\;(z\mathsf{\;i\;}(x\mathsf{\;s\;}y)) \\ & =& (x\;\mathsf{s}\;((z\;\mathsf{i\;}x)\mathsf{\;s\;}(z\;\mathsf{i\;}y)) \\ & =& (x\;\mathsf{s}\;(z\;\mathsf{i\;}x))\mathsf{\;s\;}(z\;\mathsf{i\;}y) \\ & =& x\mathsf{\;s\;}(z\;\mathsf{i\;}y) \\ & =& x\mathsf{\ s\ }(y\ \mathsf{i\ }z) \end{array} $
  %
  %   (2)$\Rightarrow $(1) es similar. $\Box$
  % \end{proof}
  %
  % % Lemma 109: Con prueba. Lemma 21.
  % \begin{lemma}
  %   Si $(L,\mathsf{s},\mathsf{i},0,1)$ un reticulado acotado y distributivo, entonces todo elemento tiene a lo sumo un complemento.
  % \end{lemma}
  % \begin{proof}
  %   Supongamos $x\in L$ tiene complementos $y,z$. Se tiene
  %
  %   $\displaystyle y=y\;\mathsf{i\;}1=y\;\mathsf{i\;}(x\;\mathsf{s\;}z)=(y\;\mathsf{i\;}x)\; \mathsf{s\;}(y\;\mathsf{i\;}z)=0\;\mathsf{s\;}(y\;\mathsf{i\;}z)=y\;\mathsf{ i\;}z, $
  %
  %   por lo cual $y\leq z$. En forma analoga se muestra que $z\leq y$. $\Box$
  % \end{proof}
  %
  % % Lemma 110: Con prueba. Lemma 22.
  % \begin{lemma}
  %   Si $S$ es no vacio, entonces $[S)$ es un filtro. Mas aun si $F$ es un filtro y $F\supseteq S$, entonces $F\supseteq \lbrack S)$.
  % \end{lemma}
  % \begin{proof}
  %   Ya que $S\subseteq \lbrack S)$, tenemos que $[S)\neq \varnothing $. Claramente $[S)$ cumple la propiedad (3). Veamos cumple la (2). Si $y\geq s_{1}\; \mathsf{i\;}s_{2}\;\mathsf{i\;}...\;\mathsf{i\;}s_{n}$ y $z\geq t_{1}\; \mathsf{i\;}t_{2}\;\mathsf{i\;}$...$\;\mathsf{i\;}t_{m}$, con $ s_{1},s_{2}, \dotsc, s_{n}$, $t_{1},t_{2}, \dotsc, t_{m}\in S$, entonces
  %
  %   $\displaystyle y\;\mathsf{i\;}z\geq s_{1}\;\mathsf{i\;}s_{2}\;\mathsf{i\;}...\;\mathsf{i\;} s_{n}\;\mathsf{i\;}t_{1}\;\mathsf{i\;}t_{2}\;\mathsf{i\;}...\;\mathsf{i\;} t_{m}, $
  %
  %   lo cual prueba (2).
  % \end{proof}
  %
  % % Lemma 111: Sin prueba. Lemma 23.
  % \begin{lemma}
  %   \PN (\textbf{Zorn}) Sea $(P, \leq)$ un poset y supongamos que cada cadena de $P$ tiene una cota superior, entonces
  %   existe un elemento maximal en $P$. Un filtro $F$ de un reticulado $(L, \mathsf{s}, \mathsf{i})$ será llamado primo
  %   cuando se cumplan:
  %   \begin{enumerate}
  %     \item $F \neq L$
  %     \item $x \ \mathsf{s} \ y \in F \Rightarrow x \in F$ ó $y \in F$.
  %   \end{enumerate}
  % \end{lemma}
  %
  % % Theorem 112: Con prueba. Theorem 24.
  % \begin{theorem}
  %   (Teorema del Filtro Primo) Sea $(L,\mathsf{s},\mathsf{i})$ un reticulado distributivo y $F$ un filtro. Supongamos $x_{0}\in L-F$. Entonces hay un filtro primo $P$ tal que $x_{0}\notin P$ y $F\subseteq P$.
  % \end{theorem}
  % \begin{proof}
  %   Sea
  %
  %   $\displaystyle \mathcal{F}=\{F_{1}:F_{1}\text{ es un filtro, }x_{0}\notin F_{1}\text{ y } F\subseteq F_{1}\}. $
  %
  %   Notese que $\mathcal{F}\neq \varnothing $, por lo cual $(\mathcal{F},\subseteq )$ es un poset. Veamos que cada cadena en $(\mathcal{F},\subseteq )$ tiene una cota superior. Sea $C$ una cadena. Si $C=\varnothing $, entonces cualquier elemento de $\mathcal{F}$ es cota de $C$. Supongamos entonces $C\neq \varnothing $. Sea
  %   $\displaystyle G=\{x\in L:x\in F_{1},\text{para algun }F_{1}\in C\}. $
  %
  %   Veamos que $G$ es un filtro. Es claro que $G$ es no vacio. Supongamos que $ x,y\in G$. Sean $F_{1},F_{2}\in \mathcal{F}$ tales que $x\in F_{1}$ y $y\in F_{2}$. Si $F_{1}\subseteq F_{2}$, entonces ya que $F_{2}$ es un filtro tenemos que $x\;\mathsf{i\;}y\in F_{2}\subseteq G$. Si $F_{2}\subseteq F_{1}$ , entonces tenemos que $x\;\mathsf{i\;}y\in F_{1}\subseteq G$. Ya que $C$ es una cadena, tenemos que siempre $x\;\mathsf{i\;}y\in G$. En forma analoga se prueba la propiedad restante por lo cual tenemos que $G$ es un filtro. Ademas $x_{0}\notin G$, por lo que $G\in \mathcal{F}$ es cota superior de $C$ . Por el lema de Zorn, $(\mathcal{F},\subseteq )$ tiene un elemento maximal $ P$. Veamos que $P$ es un filtro primo. Supongamos $x\;\mathsf{s\;}y\in P$ y $ x,y\notin P$. Entonces ya que $P$ es maximal tenemos que
  %   $\displaystyle x_{0}\in \lbrack P\cup \{x\})\cap \lbrack P\cup \{y\}) $
  %
  %   Ya que $x_{0}\in \lbrack P\cup \{x\})$, tenemos que hay elementos $ p_{1}, \dotsc, p_{n}\in P$, tales que
  %   $\displaystyle x_{0}\geq p_{1}\;\mathsf{i\;}...\;\mathsf{i\;}p_{n}\;\mathsf{i\;}x $
  %
  %   Ya que $x_{0}\in \lbrack P\cup \{y\})$, tenemos que hay elementos $ q_{1}, \dotsc, q_{m}\in P$, tales que
  %   $\displaystyle x_{0}\geq q_{1}\;\mathsf{i\;}...\;\mathsf{i\;}q_{m}\;\mathsf{i\;}y $
  %
  %   Si llamamos $p$ al siguiente elemento de $P$
  %   $\displaystyle p_{1}\;\mathsf{i\;}...\;\mathsf{i\;}p_{n}\;\mathsf{i\;}q_{1}\;\mathsf{i\;} ...\;\mathsf{i\;}q_{m} $
  %
  %   tenemos que
  %   $\displaystyle \begin{array}{rcl} x_{0} & \geq & p\;\mathsf{i\;}x \\ x_{0} & \geq & p\;\mathsf{i\;}y \end{array} $
  %
  %   Se tiene que $x_{0}\geq (p\;\mathsf{i\;}x)\;\mathsf{s\;}(p\;\mathsf{i\;} y)=p\;\mathsf{i\;}(x\;\mathsf{s\;}y)\in P$, lo cual es absurdo ya que $ x_{0}\notin P$. $\Box$
  % \end{proof}
  %
  % % Corollary 113: Con prueba. Corollary 25.
  % \begin{corollary}
  %   Sea $(L,\mathsf{s},\mathsf{i},0,1)$ un reticulado acotado distributivo. Si $ \varnothing \neq S\subseteq L$ es tal que $s_{1}\;\mathsf{i\;}s_{2}\;\mathsf{ i\;}...\;\mathsf{i\;}s_{n}\neq 0$, para cada $s_{1}, \dotsc, s_{n}\in S$, entonces hay un filtro primo que contiene a $S$.
  % \end{corollary}
  % \begin{proof}
  %   Notese que $[S)\neq L$ por lo cual se puede aplicar el Teorema del filtro primo. $\Box$
  % \end{proof}
  %
  % % Lemma 114: Con prueba. Lemma 26.
  % \begin{lemma}
  %   Sea $(B,\mathsf{s},\mathsf{i},^{c},0,1)$ un algebra de Boole. Entonces para un filtro $F\subseteq B$ las siguientes son equivalentes:
  %   (1) $F$ es primo
  %   (2) $x\in F$ o $x^{c}\in F$, para cada $x\in B$.
  % \end{lemma}
  % \begin{proof}
  %   (1)$\Rightarrow $(2). Ya que $x\;\mathsf{s\;}x^{c}=1\in F$, (2) se cumple si $F$ es primo.
  %
  %   (2)$\Rightarrow $(1). Supongamos que $x\;\mathsf{s\;}y\in F$ y que $x\not\in F$. Entonces por (2), $x^{c}\in F$ y por lo tanto tenemos que
  %
  %   $\displaystyle y\geq x^{c}\;\mathsf{i\;}y=(x^{c}\;\mathsf{i\;}x)\;\mathsf{s\;}(x^{c}\; \mathsf{i\;}y)=x^{c}\;\mathsf{i\;}(x\;\mathsf{s\;}y)\in F, $
  %
  %   lo cual dice que $y\in F$. $\Box$
  % \end{proof}
  %
  % % Lemma 115: Con prueba. Lemma 27.
  % \begin{lemma}
  %   Sea $(B,\mathsf{s},\mathsf{i},^{c},0,1)$ un algebra de Boole. Supongamos que $b\neq 0$ y $a=\inf A$, con $A\subseteq B$. Entonces si $b\;\mathsf{i\;}a=0$ , existe un $e\in A$ tal que $b\;\mathsf{i\;}e^{c}\neq 0$.
  % \end{lemma}
  % \begin{proof}
  %   Supongamos que para cada $e\in A$, tengamos que $b\;\mathsf{i\;}e^{c}=0$. Entonces tenemos que para cada $e\in A$,
  %
  %   $\displaystyle b=b\;\mathsf{i\;}(e\;\mathsf{s\;}e^{c})=(b\;\mathsf{i\;}e)\;\mathsf{s\;}(b\; \mathsf{i\;}e^{c})=b\;\mathsf{i\;}e, $
  %
  %   lo cual nos dice que $b$ es cota inferior de $A$. Pero entonces $b\leq a$, por lo cual $b=b\;\mathsf{i\;}a=0$, contradiciendo la hipotesis. $\Box$
  % \end{proof}
  %
  % % Theorem 116: Con prueba. Theorem 28.
  % \begin{theorem}
  %   (Rasiova y Sikorski) Sea $(B,\mathsf{s},\mathsf{i},^{c},0,1)$ un algebra de Boole. Sea $x\in B$, $x\neq 0$. Supongamos que $A_{1},A_{2},...$ son subconjuntos de $B$ tales que existe $\inf (A_{j})$, para cada $j=1,2....$ Entonces hay un filtro primo $P$ el cual cumple:
  %   (a) $x\in P$
  %   (b) $P\supseteq A_{j}\Rightarrow P\ni \inf (A_{j})$, para cada $ j=1,2,....$
  % \end{theorem}
  % \begin{proof}
  %   Sean $a_{j}=\inf (A_{j})$, $j=1,2,...$. Construiremos inductivamente una sucesion $b_{0},b_{1},...$ de elementos de $B$ tal que:
  %
  %   (1) $b_{0}=x$
  %   (2) $b_{0}\;\mathsf{i\;}$...$\;\mathsf{i\;}b_{n}\neq 0$, para cada $ n\geq 0$
  %   (3) $b_{j}=a_{j}$ o $b_{j}^{c}\in A_{j}$, para cada $j\geq 1$.
  %   Definamos $b_{0}=x$. Supongamos ya definimos $b_{0}, \dotsc, b_{n}$, veamos como definir $b_{n+1}$. Si $(b_{0}\;\mathsf{i\;}...\;\mathsf{i\;} b_{n})\;\mathsf{i\;}a_{n+1}\neq 0$, entonces definamos $b_{n+1}=a_{n+1}$. Si $(b_{0}\;\mathsf{i\;}...\;\mathsf{i\;}b_{n})\;\mathsf{i\;}a_{n+1}=0$, entonces por el lema anterior, tenemos que hay un $e\in A_{n+1}$ tal que $ (b_{0}\;\mathsf{i\;}...\;\mathsf{i\;}b_{n})\;\mathsf{i\;}e^{c}\neq 0$, lo cual nos permite definir $b_{n+1}=e^{c}$.
  %
  %   Usando (2) se puede probar que el conjunto $S=\{b_{0},b_{1},...\}$ satisface la hipotesis del primer corolario del Teorema del filtro primo, por lo cual hay un filtro primo $P$ tal que $\{b_{0},b_{1},...\}\subseteq P$. Es facil chequear que $P$ satisface las propiedades (a) y (b). $\Box$
  % \end{proof}
