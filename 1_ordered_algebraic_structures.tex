\section{Estructuras algebráicas ordenadas}

  % % Lemma 89: Con prueba. Lemma 1.
  % \begin{lemma} \label{lemma_1}
  %   \PN Sean $(P, \leq)$ y $(P^{\prime}, \leq^{\prime})$ posets. Supongamos que $F$ es un isomorfismo de $(P, \leq)$ en
  %   $(P^{\prime}, \leq^{\prime})$, entonces:
  %
  %   \begin{enumerate}[a)]
  %     \item Para cada $S \subseteq P$ y cada $a \in P$, se tiene que $a$ es \textbf{cota superior} (resp.
  %       \textbf{inferior}) de $S$ si y solo si $F(a)$ es \textbf{cota superior} (resp. \textbf{inferior}) de $F(S)$.
  %     \item Para cada $S \subseteq P$, se tiene que $\exists \ \sup (S)$ si y solo si $\exists \ \sup (F(S))$ y en el
  %       caso de que existan tales elementos se tiene que $F(\sup (S)) = \sup (F(S))$.
  %     \item $P$ tiene $1$ (resp. $0$) si y solo si $P^{\prime}$ tiene $1$ (resp. $0$) y en tal caso tales elementos
  %       están conectados por $F$.
  %     \item Para cada $m \in P$, $m$ es \textbf{maximal} (resp. \textbf{minimal}) si y solo si $F(m)$ es
  %       \textbf{maximal} (resp. \textbf{minimal}).
  %     \item Para $a, b \in P$, tenemos que $a \prec b$ si y solo si $F(a) \prec^{\prime} F(b)$.
  %   \end{enumerate}
  % \end{lemma}
  % \begin{proof}
  %   \begin{enumerate}[a)]
  %     \item Probaremos solo el caso de la \textbf{cota superior}.
  %       \PN \begin{tabular}{|c|} \hline $\Rightarrow$ \\\hline \end{tabular} Supongamos que $a$ es \textbf{cota
  %       superior} de $S$, veamos entonces que $F(a)$ es \textbf{cota superior} de $F(S)$. Sean:
  %       \begin{itemize}
  %         \item $x \in F(S)$
  %         \item $s \in S$ tal que $x = F(s)$.
  %       \end{itemize}
  %
  %       \PN Ya que $s \leq a$, tenemos que $x = F(s) \leq^{\prime} F(a)$. Luego, $F(a)$ es \textbf{cota superior}.
  %
  %       \PN \begin{tabular}{|c|} \hline $\Leftarrow$ \\\hline \end{tabular} Supongamos ahora que $F(a)$ es \textbf{cota
  %       superior} de $F(S)$ y veamos entonces que $a$ es cota superior de $S$.
  %
  %       \PN Sea $s \in S$, ya que $F(s) \leq^{\prime} F(a)$, tenemos que $s = F^{-1}(F(s)) \leq^{\prime} F^{-1}(F(a)) =
  %       a$. Por lo tanto, $a$ es \textbf{cota superior}.
  %
  %     \item \begin{tabular}{|c|} \hline $\Rightarrow$ \\\hline \end{tabular} Supongamos existe $\sup (S)$. Veamos
  %       que $F(\sup (S))$ es el supremo de $F(S)$. Por el iniciso (a) $F(\sup (S))$ es cota superior de $F(S)$. Veamos
  %       que es la menor de las cotas superiores. Supongamos $b^{\prime}$ cota superior de $F(S)$, entonces
  %       $F^{-1}(b^{\prime})$ es cota superior de $S$, es decir, $\sup (S) \leq F^{-1}(b^{\prime})$, produciendo
  %       $F(\sup (S)) \leq^{\prime} b^{\prime}$. Por lo tanto, $F(\sup (S))$ es el supremo de $F(S)$.
  %
  %       \PN \begin{tabular}{|c|} \hline $\Leftarrow$ \\\hline \end{tabular} Supongamos existe $\sup (F(S))$. Veamos
  %       que $F^{-1}(\sup (F(S)))$ es el supremo de $S$. Nuevamente, por el iniciso (a) $F^{-1}(\sup (F(S)))$ es cota
  %       superior de $S$. Veamos que es la menor de las cotas superiores. Supongamos $b$ cota superior de $S$, entonces
  %       $F(b)$ es cota superior de $F(S)$, es decir, $\sup (F(S)) \leq F(b)$, produciendo $F^{-1}(\sup (F(S))) \leq b$.
  %       Por lo tanto, $F^{-1}(\sup (F(S)))$ es el supremo de $S$.
  %
  %     \item Se desprende del inciso (b) tomando $S = P$.
  %     \item Probaremos solo el caso \textbf{maximal}.
  %       \PN \begin{tabular}{|c|} \hline $\Rightarrow$ \\\hline \end{tabular} Supongamos que $m$ es maximal de
  %       $(P, \leq)$. Veamos que $F(m)$ es maximal de $(P^{\prime}, \leq^{\prime})$. Supongamos que $F(m)$ no es maximal
  %       de $(P^{\prime}, \leq^{\prime})$, es decir, $F(m) <^{\prime} b^{\prime} \ \forall b^{\prime} \in P^{\prime}$.
  %       Dado que $F$ es isomorfismo:
  %       \begin{eqnarray*}
  %   			F^{-1}(F(m)) < F^{-1}(b^{\prime}) \\
  %   			m < F^{-1}(b^{\prime})
  %   		\end{eqnarray*}
  %       \PN Lo cual es un absurdo, dado que $m$ es maximal de $(P, \leq)$. Por lo tanto, $F(m)$ es maximal de
  %       $(P^{\prime}, \leq^{\prime})$.
  %
  %       \PN \begin{tabular}{|c|} \hline $\Leftarrow$ \\\hline \end{tabular} Supongamos que $F(m)$ es maximal de
  %       $(P^{\prime}, \leq^{\prime})$. Veamos que $m$ es maximal de $(P, \leq)$. Supongamos que $m$ no es maximal
  %       de $(P, \leq)$, es decir, $m < b \ \forall b \in P$. Dado que $F$ es isomorfismo:
  %       \[
  %   			F(m) < F(b)
  %   		\]
  %       \PN Lo cual es un absurdo, dado que $F(m)$ es maximal de $(P^{\prime}, \leq^{\prime})$. Por lo tanto, $m$ es
  %       maximal de $(P, \leq)$.
  %
  %     \item \begin{tabular}{|c|} \hline $\Rightarrow$ \\\hline \end{tabular} Supongamos $a \prec b$, veamos que $F(a)
  %       \prec^{\prime} F(b)$. Debemos ver:
  %       \begin{enumerate}[1)]
  %         \item $F(a) <^{\prime} F(b)$
  %         \item $\nexists z^{\prime}$ tal que $F(a) < z^{\prime} < F(b)$
  %       \end{enumerate}
  %
  %       \PN Ya que $a \prec b$, por definición tenemos: \begin{tabular}{|c|} \hline $a < b$ y $\nexists z$ tal que
  %       $a < z < b$ \\\hline \end{tabular} $(\star)$
  %
  %       \PN Dado que la función $F$ es un isomorfismo, se cumple (1). Veamos que se cumple (2), supongamos que $\exists
  %       z^{\prime}$ tal que $F(a) < z^{\prime} < F(b)$. Luego, nuevamente utilizando que $F$ es isomorfismo, tenemos:
  %       \begin{eqnarray*}
  %   			F^{-1}(F(a)) < &F^{-1}(z^{\prime})& < F^{-1}(F(b)) \\
  %   			a < &F^{-1}(z^{\prime})& < b
  %   		\end{eqnarray*}
  %       \PN Lo cual, contradice $(\star)$, el absurdo vino de suponer que $\exists z^{\prime}$ tal que
  %       $F(a) < z^{\prime} < F(b)$, por lo tanto $\nexists z^{\prime}$ tal que $F(a) < z^{\prime} < F(b)$.
  %
  %       \PN Finalmente, dado que se cumplen los puntos (1) y (2), se cumple también $F(a) \prec^{\prime} F(b)$.
  %
  %       \PN \begin{tabular}{|c|} \hline $\Leftarrow$ \\\hline \end{tabular} Supongamos $F(a) \prec^{\prime} F(b)$,
  %       veamos que $a \prec b$.
  %
	% 	    \PN Ya que $F^{-1}: (P^{\prime}, \leq^{\prime}) \rightarrow (P, \leq)$ es isomorfismo, por lo ya visto tenemos:
  %   		\begin{eqnarray*}
  %   			F^{-1}(F(a)) &\prec& F^{-1}(F(b)) \\
  %   			a &\prec& b
  %   		\end{eqnarray*}
  %   \end{enumerate}
  % \end{proof}
  %
  % % Lemma 90: Con prueba. Lemma 2.
  % \begin{lemma}
  %   \PN Dado un reticulado $(L, \leq)$ y elementos $x, y, z, w \in L$, se cumplen las siguientes propiedades:
  %   \begin{multicols}{2}
  %     \begin{enumerate}[(1)]
  %       \item $x \leq x \ \SU \ y$
  %       \item $x \ \IN \ y \leq x$
  %       \item $x \ \SU \ x = x \ \IN \ x = x$
  %       \item $x \ \SU \ y = y \ \SU \ x$
  %       \item $x \ \IN \ y = y \ \IN \ x$
  %       \item $x \leq y \Leftrightarrow x \ \SU \ y = y \Leftrightarrow x \ \IN \ y = x$
  %       \item $x \ \SU \ (x \ \IN \ y) = x$
  %       \item $x \ \IN \ (x \ \SU \ y) = x$
  %       \item $(x \ \SU \ y) \ \SU \ z = x \ \SU \ (y \ \SU \ z)$
  %       \item $(x \ \IN \ y) \ \IN \ z = x \ \IN \ (y \ \IN \ z)$
  %       \item Si $x \leq z$ e $y \leq w$ entonces:
  %         \begin{itemize}
  %           \item $x \ \SU \ y \leq z \ \SU \ w$
  %           \item $x \ \IN \ y \leq z \ \IN \ w$
  %         \end{itemize}
  %       \item $(x \ \IN \ y) \ \SU \ (x \ \IN \ z) \leq x \ \IN \ (y \ \SU \ z)$
  %     \end{enumerate}
  %   \end{multicols}
  % \end{lemma}
  % \begin{proof}
  %   \PN Dado que las propiedades $(1), (2), (3), (4), (5), (6)$, son consecuencia inmediata de las definiciones de $\SU$
  %   e $\IN$, probaremos solo las restantes.
  %   \begin{enumerate}
  %     \begin{multicols}{2}
  %       \item[(7)]
  %         \begin{alignat*}{3}
  %           x \ \IN \ y &\leq& x \qquad &\text{Por } (2) \\
  %           (x \ \IN \ y) \ \SU \ x &=& x \qquad &\text{Por } (6) \\
  %           x \ \SU \ (x \ \IN \ y) &=& x \qquad &\text{Por } (3)
  %         \end{alignat*}
  %
  %       \item[(8)]
  %         \begin{alignat*}{3}
  %           x &\leq& x \ \SU \ y \qquad \; && \text{Por } (1) \\
  %           x \ \IN \ (y \ \SU \ x) &=& x \qquad\qquad &&\text{Por } (6)
  %         \end{alignat*}
  %         \vspace{3mm}
  %     \end{multicols}
  %
  %     \item[(9)] Para probar la igualdad probaremos las siguientes desigualdades:
  %       \begin{itemize}
  %         \item \begin{tabular}{|c|} \hline $(x \ \SU \ y) \ \SU \ z \leq x \ \SU \ (y \ \SU \ z)$\\\hline \end{tabular}
  %           \PN Notese que $x \ \SU \ (y \ \SU \ z)$ es cota superior de $\{x, y, z\}$ ya que:
  %           \begin{eqnarray*}
  %             x &\leq& x \ \SU \ (y \ \SU \ z) \\
  %             y &\leq& (y \ \SU \ z) \leq x \ \SU \ (y \ \SU \ z) \\
  %             z &\leq& (y \ \SU \ z) \leq x \ \SU \ (y \ \SU \ z)
  %           \end{eqnarray*}
  %
  %           \PN Por otro lado, $x \ \SU \ (y \ \SU \ z)$ es cota superior de $\{x, y\}$, tenemos que $x \ \SU \ y \leq x
  %           \ \SU \ (y \ \SU \ z)$, por lo cual $x \ \SU \ (y \ \SU \ z)$ es cota superior del conjunto $\{x \ \SU \ y,
  %           z\}$, lo cual dice que $(x \ \SU \ y) \ \SU \ z \leq x \ \SU \ (y \ \SU \ z)$.
  %
  %         \item \begin{tabular}{|c|} \hline $(x \ \SU \ y) \ \SU \ z \geq x \ \SU \ (y \ \SU \ z)$\\\hline \end{tabular}
  %           \PN Notese que $(x \ \SU \ y) \ \SU \ z$ es cota superior de $\{x, y, z\}$ ya que:
  %           \begin{eqnarray*}
  %             x &\leq& x \ \SU \ y \leq (x \ \SU \ y) \ \SU \ z \\
  %             y &\leq& x \ \SU \ y \leq (x \ \SU \ y) \ \SU \ z \\
  %             z &\leq& (x \ \SU \ y) \ \SU \ z
  %           \end{eqnarray*}
  %
  %           \PN Por otro lado, $(x \ \SU \ y) \ \SU \ z$ es cota superior de $\{y, z\}$, tenemos que $y \ \SU \ z \leq
  %           (x \ \SU \ y) \ \SU \ z$, por lo cual $(x \ \SU \ y) \ \SU \ z$ es cota superior del conjunto $\{x, y \ \SU
  %           \ z\}$, lo cual dice que $(x \ \SU \ y) \ \SU \ z \geq x \ \SU \ (y \ \SU \ z)$.
  %       \end{itemize}
  %
  %       \PN Por lo tanto, $(x \ \SU \ y) \ \SU \ z = x \ \SU \ (y \ \SU \ z)$
  %
  %     \item[(10)] Para probar la igualdad probaremos las siguientes desigualdades:
  %       \begin{itemize}
  %         \item \begin{tabular}{|c|} \hline $(x \ \IN \ y) \ \IN \ z \leq x \ \IN \ (y \ \IN \ z)$\\\hline \end{tabular}
  %           \PN Notese que $x \ \IN \ (y \ \IN \ z)$ es cota inferior de $\{x, y, z\}$ ya que:
  %           \begin{eqnarray*}
  %             x \ \IN \ (y \ \IN \ z) &\leq& x \\
  %             (y \ \IN \ z) \leq x \ \IN \ (y \ \IN \ z) &\leq& y \\
  %             z &\leq& (y \ \IN \ z) \leq x \ \IN \ (y \ \IN \ z)
  %           \end{eqnarray*}
  %
  %           \PN Por otro lado, $x \ \IN \ (y \ \IN \ z)$ es cota inferior de $\{x, y\}$, tenemos que $x \ \IN \ y \leq x
  %           \ \IN \ (y \ \IN \ z)$, por lo cual $x \ \IN \ (y \ \IN \ z)$ es cota inferior del conjunto $\{x \ \IN \ y,
  %           z\}$, lo cual dice que $(x \ \IN \ y) \ \IN \ z \leq x \ \IN \ (y \ \IN \ z)$.
  %
  %         \item \begin{tabular}{|c|} \hline $(x \ \IN \ y) \ \IN \ z \geq x \ \IN \ (y \ \IN \ z)$\\\hline \end{tabular}
  %           \PN Notese que $(x \ \IN \ y) \ \IN \ z$ es cota inferior de $\{x, y, z\}$ ya que:
  %           \begin{eqnarray*}
  %             x &\leq& x \ \IN \ y \leq (x \ \SU \ y) \ \IN \ z \\
  %             y &\leq& x \ \IN \ y \leq (x \ \SU \ y) \ \IN \ z \\
  %             z &\leq& (x \ \IN \ y) \ \IN \ z
  %           \end{eqnarray*}
  %
  %           \PN Por otro lado, $(x \ \IN \ y) \ \IN \ z$ es cota inferior de $\{y, z\}$, tenemos que $y \ \IN \ z \leq
  %           (x \ \IN \ y) \ \IN \ z$, por lo cual $(x \ \IN \ y) \ \IN \ z$ es cota inferior del conjunto $\{x, y \ \IN
  %           \ z\}$, lo cual dice que $(x \ \IN \ y) \ \IN \ z \geq x \ \IN \ (y \ \IN \ z)$.
  %       \end{itemize}
  %
  %       \PN Por lo tanto, $(x \ \IN \ y) \ \IN \ z = x \ \IN \ (y \ \IN \ z)$
  %
  %     \item[(11)]
  %       \begin{multicols}{2}
  %         \begin{alignat*}{3}
  %           x &\leq& z &\leq& z \ \SU \ w \\
  %           y &\leq& w &\leq& z \ \SU \ w \\
  %         \end{alignat*}
  %         \begin{alignat*}{3}
  %           \\
  %           x &\leq& z \Rightarrow x \ \IN \ y &\leq& z \\
  %           y &\leq& w \Rightarrow x \ \IN \ y &\leq& w
  %         \end{alignat*}
  %       \end{multicols}
  %       \PN Luego, $z \ \SU \ w$ es cota superior de $\{x, y\}$ y $x \ \IN \ y$ es cota inferior de $\{z, w\}$, por lo
  %       tanto, $x \ \SU \ y \leq z \ \SU \ w$ y $x \ \IN \ y \leq z \ \IN \ w$.
  %
  %     \item[(12)]
  %       \begin{equation*}
  %         \left.
  %         \begin{array}{l}
  %           (x \ \IN \ y), (x \ \IN \ z) \leq x \\
  %           (x \ \IN \ y), (x \ \IN \ z) \leq y \ \SU \ z
  %         \end{array}
  %         \right \rbrace \Rightarrow (x \ \IN \ y), (x \ \IN \ z) \leq x \ \IN \ (y \ \SU \ z)
  %       \end{equation*}
  %
  %       \[
  %         \therefore (x \ \IN \ y) \ \SU \ (x \ \IN \ z) \leq x \ \IN \ (y \ \SU \ z)
  %       \]
  %   \end{enumerate}
  % \end{proof}
  %
  % % Lemma 91: Con prueba. Lemma 3.
  % \begin{lemma}
  %   \PN Sea $(L, \leq)$ un reticulado, dados elementos $x_{1}, \dotsc, x_{n} \in L$, con $n \geq 2$, se tiene
  %   \[
  %     \begin{array}{rcl}
  %       (\dotsc (x_{1} \ \SU \ x_{2}) \ \SU \ \dotsc) \ \SU \ x_{n} &=& \sup (\{x_{1}, \dotsc, x_{n}\}) \\
  %       (\dotsc (x_{1} \ \IN \ x_{2}) \ \IN \ \dotsc) \ \IN \ x_{n} &=& \inf (\{x_{1}, \dotsc, x_{n}\})
  %     \end{array}
  %   \]
  % \end{lemma}
  % \begin{proof}
  %   \PN Probaremos por inducción en $n$.
  %
  %   \vspace{3mm}
  %   \PN \underline{Caso Base:} \begin{tabular}{|c|} \hline $n = 2$ \\\hline \end{tabular}
  %   \[
  %     \begin{array}{rcl}
  %       x_{1} \ \SU \ x_{2} &=& \sup(\{x_{1}, x_{2}\}) \\
  %       x_{1} \ \IN \ x_{2} &=& \inf(\{x_{1}, x_{2}\})
  %     \end{array}
  %   \]
  %
  %   \PN Lo cual vale, dado que es la definición.
  %
  %   \vspace{3mm}
	% 	\PN \underline{Caso Inductivo:} \begin{tabular}{|c|} \hline $n > 2$ \\\hline \end{tabular}
  %
  %   \PN Supongamos ahora que vale para $n$ y veamos entonces que vale para $n+1$. Sean $x_{1}, \dotsc, x_{n+1} \in L$,
  %   por hipótesis inductiva tenemos que:
  %   \begin{eqnarray*}
  %     (\dotsc (x_{1} \ \SU \ x_{2}) \ \SU \ \dotsc) \ \SU \ x_{n} &=& \sup(\{x_{1}, \dotsc, x_{n}\}) \ \ (\star_{1}) \\
  %     (\dotsc (x_{1} \ \IN \ x_{2}) \ \IN \ \dotsc) \ \IN \ x_{n} &=& \inf(\{x_{1}, \dotsc, x_{n}\}) \ \ \ (\star_{2})
  %   \end{eqnarray*}
  %
  %
  %   \PN Veamos entonces que:
  %   \begin{eqnarray*}
  %     ((\dotsc(x_{1} \ \SU \ x_{2}) \ \SU \ \dotsc) \ \SU \ x_{n}) \ \SU \ x_{n+1} &=& \sup(\{x_{1}, \dotsc, x_{n+1}\})
  %       \ \ (\dag_{1}) \\
  %     ((\dotsc(x_{1} \ \IN \ x_{2}) \ \IN \ \dotsc) \ \IN \ x_{n}) \ \IN \ x_{n+1} &=& \inf(\{x_{1}, \dotsc, x_{n+1}\})
  %       \ \ \ (\dag_{2})
  %   \end{eqnarray*}
  %
  %   \PN Para ello debemos ver $((\dotsc(x_{1} \ \SU \ x_{2}) \ \SU \ \dotsc) \ \SU \ x_{n}) \ \SU \ x_{n+1}$ es cota
  %   superior de $\{x_{1}, \dotsc, x_{n+1}\}$ y que es la menor de las cotas superiores. Además, que $((\dotsc(x_{1} \
  %   \IN \ x_{2}) \ \IN \ \dotsc) \ \IN \ x_{n}) \ \IN \ x_{n+1}$ es cota inferior de $\{x_{1}, \dotsc, x_{n+1}\}$ y que
  %   es la mayor de las cotas inferiores.
  %
  %   \vspace{5mm}
  %   \PN Es fácil ver que $((\dotsc(x_{1} \ \SU \ x_{2}) \ \SU \ \dotsc) \ \SU \ x_{n}) \ \SU \ x_{n+1}$ es cota superior
  %   de $ \{x_{1}, \dotsc, x_{n+1}\}$. Supongamos que $z$ es otra cota superior de $\{x_{1}, \dotsc, x_{n+1}\}$. Ya que
  %   $z$ es también cota superior del conjunto $\{x_{1}, \dotsc, x_{n}\}$, por $(\star_{1})$ tenemos que:
  %   \[
  %     (\dotsc (x_{1} \ \SU \ x_{2}) \ \SU \ \dotsc) \ \SU \ x_{n} \leq z
  %   \]
  %
  %   \PN Además, dado que $x_{n+1} \leq z$, tenemos que:
  %   \[
  %     ((\dotsc (x_{1} \ \SU \ x_{2}) \ \SU \ \dotsc) \ \SU \ x_{n}) \ \SU \ x_{n+1} \leq z
  %   \]
  %
  %   \PN Por lo tanto, vale $(\dag_{1})$.
  %
  %   \vspace{5mm}
  %   \PN Nuevamente, es fácil ver que $((\dotsc(x_{1} \ \IN \ x_{2}) \ \IN \ \dotsc) \ \IN \ x_{n}) \ \IN \ x_{n+1}$ es
  %   cota inferior de $ \{x_{1}, \dotsc, x_{n+1}\}$. Supongamos que $z^{\prime}$ es otra cota inferior de $\{x_{1},
  %   \dotsc, x_{n+1}\}$. Ya que $z^{\prime}$ es también cota inferior del conjunto $\{x_{1}, \dotsc, x_{n}\}$, por
  %   $(\star_{2})$ tenemos que:
  %   \[
  %     z^{\prime} \leq (\dotsc (x_{1} \ \IN \ x_{2}) \ \IN \ \dotsc) \ \IN \ x_{n}
  %   \]
  %
  %   \PN Además, dado que $z^{\prime} \leq x_{n+1}$, tenemos que:
  %   \[
  %     z^{\prime} \leq ((\dotsc (x_{1} \ \IN \ x_{2}) \ \IN \ \dotsc) \ \IN \ x_{n}) \ \IN \ x_{n+1}
  %   \]
  %
  %   \PN Por lo tanto, vale $(\dag_{2})$.
  % \end{proof}
  %
  % % Theorem 92: Con prueba. Theorem 4.
  % \begin{theorem}
  %   \PN Sea $(L, \SU, \IN)$ un reticulado, la relación binaria definida por:
  %   \[
  %     x \leq y \Leftrightarrow x \ \SU \ y = y
  %   \]
  %
  %   \PN es un orden parcial sobre $L$ para el cual se cumple:
  %   \[
  %     \begin{array}{rcl}
  %       \sup (\{x, y\}) &=& x \ \SU \ y \\
  %       \inf (\{x, y\}) &=& x \ \IN \ y
  %     \end{array}
  %   \]
  % \end{theorem}
  % \begin{proof}
  %   \begin{itemize}
  %     \item \underline{Reflexiva:} Sea $x \in L$ un elemento cualquiera. Luego,
  %       \begin{equation*}
  %         \left.
  %         \begin{array}{l}
  %           x \ \SU \ x = x \\
  %           x \ \IN \ x = x
  %         \end{array}
  %         \right \rbrace \Rightarrow x \leq x
  %       \end{equation*}
  %
  %     \item \underline{Antisimétrica:} Sean $x, y \in L$ elementos cualquieras. Supongamos que $x \leq y$ e $y \leq x$,
  %       entonces:
  %       \begin{equation*}
  %         \left.
  %         \begin{array}{l}
  %           x \leq y \Rightarrow x \ \SU \ y = y \\
  %           y \leq x \Rightarrow x \ \SU \ y = x
  %         \end{array}
  %         \right \rbrace \Rightarrow x = y
  %       \end{equation*}
  %
  %     \item \underline{Transitiva:} Supongamos que $x \leq y$ e $y \leq z$, entonces:
  %       \[
  %         x \ \SU \ z = x \ \SU \ (y \ \SU \ z) = (x \ \SU \ y) \ \SU \ z = y \ \SU \ z = z
  %       \]
  %
  %       \PN por lo cual $x \leq z$.
  %
  %       \PN Veamos ahora que $\sup(\{x, y\}) = x \ \SU \ y$. Es claro que $x \ \SU \ y$ es una cota superior del
  %       conjunto $\{x, y\}$, veamos que es la menor. Supongamos $x, y \leq z$, entonces:
  %       \[
  %         (x \ \SU \ y) \ \SU \ z = x \ \SU \ (y \ \SU \ z) = x \ \SU \ z = z
  %       \]
  %       \PN por lo que $x\ \SU \ y \leq z$, es decir, $x\ \SU \ y$ es la menor cota superior.
  %
  %       \PN Resta probar que $\inf(\{x, y\}) = x \ \IN \ y$. Nuevamente, es claro que $x \ \IN \ y$ es una cota inferior
  %       del conjunto $\{x, y\}$, veamos que es la mayor. Supongamos $z \leq x, y$, entonces:
  %       \[
  %         (x \ \IN \ y) \ \IN \ z = x \ \IN \ (y \ \IN \ z) = x \ \IN \ z = z
  %       \]
  %       \PN por lo que $z \leq x\ \IN \ y$, es decir, $x\ \IN \ y$ es la mayor cota inferior.
  %   \end{itemize}
  % \end{proof}
  %
  % % Lemma 93: Con prueba. Lemma 5.
  % \begin{lemma}
  %   \PN Si $F: (L, \SU, \IN) \rightarrow (L^{\prime}, \SU^{\prime}, \IN^{\prime})$ es un
  %   homomorfismo biyectivo, entonces $F$ es un isomorfismo.
  % \end{lemma}
  % \begin{proof}
  %   \PN Debemos probar que $F^{-1}$ es un homomorfismo. Sean $F(x), F(y)$ dos elementos cualesquiera de $L^{\prime}$,
  %   tenemos que:
  %   \begin{multicols}{2}
  %     \begin{eqnarray*}
  %       F^{-1}(F(x) \ \SU^{\prime} \ F(y)) &=& F^{-1}(F(x \ \SU \ y)) \\
  %       &=& x \ \SU \ y \\
  %       &=& F^{-1}(F(x)) \ \SU \ F^{-1}(F(y))
  %     \end{eqnarray*}
  %
  %     \begin{eqnarray*}
  %       F^{-1}(F(x) \ \IN^{\prime} \ F(y)) &=& F^{-1}(F(x \ \IN \ y)) \\
  %       &=& x \ \IN \ y \\
  %       &=& F^{-1}(F(x)) \ \IN \ F^{-1}(F(y))
  %     \end{eqnarray*}
  %   \end{multicols}
  %
  %   \PN Luego, $F^{-1}$ es homomorfismo y por lo tanto $F$ es isomorfismo.
  % \end{proof}
  %
  % % Lemma 94: Con prueba. Lemma 6.
  % \begin{lemma}
  %   \PN Sean $(L, \SU, \IN)$ y $(L^{\prime}, \SU^{\prime}, \IN^{\prime})$ reticulados y sea $F: (L, \SU, \IN)
  %   \rightarrow (L^{\prime}, \SU^{\prime}, \IN^{\prime})$ un homomorfismo, entonces $I_{F}$ es un subuniverso de
  %   $(L^{\prime}, \SU^{\prime}, \IN^{\prime})$.
  % \end{lemma}
  % \begin{proof}
  %   \PN Ya que $L \neq \emptyset$, tenemos que $I_{F} \neq \emptyset$. Sean $a, b \in I_{F}, \ x, y \in L$ tales que
  %   $F(x) = a$ y $F(y) = b$. Se tiene que:
  %   \begin{alignat*}{3}
  %     a \ \SU^{\prime} \ b &=& F(x) \ \SU^{\prime} \ F(y) &=& F(x \ \SU \ y) \in I_{F} \\
  %     a \ \IN^{\prime} \ b &=& F(x) \ \IN^{\prime} \ F(y) &=& F(x \ \IN \ y) \in I_{F}
  %   \end{alignat*}
  %
  %   \PN por lo cual $I_{F}$ es cerrada bajo $\SU^{\prime}$ e $\IN^{\prime}$.
  % \end{proof}
  %
  % % Lemma 95: Con prueba. Lemma 7.
  % \begin{lemma}
  %   \PN Sean $(L, \SU, \IN)$ y $(L^{\prime}, \SU^{\prime}, \IN^{\prime})$ reticulados y sean $(L, \leq)$ y $(L^{\prime},
  %   \leq^{\prime})$ los posets asociados. Sea $F: L \rightarrow L^{\prime}$ una función, entonces $F$ es un isomorfismo
  %   de $(L, \SU, \IN)$ en $ (L^{\prime}, \SU^{\prime}, \IN^{\prime})$ si y solo si $F$ es un isomorfismo de $(L, \leq)$
  %   en $(L^{\prime}, \leq^{\prime})$.
  % \end{lemma}
  % \begin{proof}
  %   \PN \begin{tabular}{|c|} \hline $\Rightarrow$ \\\hline \end{tabular} Supongamos que $F$ es un isomorfismo de
  %   $(L, \SU, \IN)$ en $ (L^{\prime}, \SU^{\prime}, \IN^{\prime})$.
  %
  %   \PN Sean $x, y \in L$ tales que $x \leq y$. Tenemos:
  %   \begin{eqnarray*}
  %     y &=& x \ \SU \ y \\
  %     F(y) &=& F(x \ \SU \ y) \\
  %     &=& F(x) \ \SU^{\prime} \ F(y) \\
  %     \therefore F(x) &\leq^{\prime}& F(y)
  %   \end{eqnarray*}
  %
  %   \PN Sean $x^{\prime}, y^{\prime} \in L^{\prime}$ tales que $x^{\prime} \leq^{\prime} y^{\prime}$. Tenemos:
  %   \begin{eqnarray*}
  %     y^{\prime} &=& x^{\prime} \ \SU^{\prime} \ y^{\prime} \\
  %     F^{-1}(y^{\prime}) &=& F^{-1}(x^{\prime} \ \SU^{\prime} \ y^{\prime}) \\
  %     &=& F^{-1}(x^{\prime}) \ \SU \ F^{-1}(y^{\prime}) \\
  %     \therefore F^{-1}(x) &\leq& F^{-1}(y)
  %   \end{eqnarray*}
  %
  %   \PN Por lo tanto, $F$ es un isomorfismo de $(L, \leq)$ en $(L^{\prime}, \leq^{\prime})$.
  %
  %   \PN \begin{tabular}{|c|} \hline $\Leftarrow$ \\\hline \end{tabular} Supongamos ahora que $F$ es un isomorfismo
  %   de $(L, \leq)$ en $(L^{\prime}, \leq^{\prime})$, entonces el \textbf{Lemma ~\ref{lemma_1}} nos dice que $F$ y
  %   $F^{\prime}$ respetan la operaciones de supremo e ínfimo, por lo cual $F$ es un isomorfismo de $(L, \SU, \IN)$ y
  %   $(L^{\prime}, \SU^{\prime}, \IN^{\prime})$.
  % \end{proof}
  %
  % % Lemma 96: Con prueba. Lemma 8. TODO
  % \begin{lemma}
  %   \PN Sea $(L/\theta, \mathsf{\tilde{s}}, \mathsf{\tilde{\imath}})$ un reticulado. El orden parcial $\tilde{\leq}$
  %   asociado a este reticulado cumple:
  %   \[
  %     x/\theta \ \tilde{\leq} \ y/\theta \Leftrightarrow y \ \theta \ (x \ \SU \ y)
  %   \]
  % \end{lemma}
  % \begin{proof}
  %   \PN Veamos que $(L/\theta, \mathsf{\tilde{s}}, \mathsf{\tilde{\imath}})$ satisface las 7 identidades de la
  %   definición de reticulado. Sean $x/\theta, y/\theta, z/\theta$ elementos cualesquiera de $L/\theta$.
  %   \begin{multicols}{2}
  %     \begin{enumerate}
  %       \item[(I1)] \begin{tabular}{|c|} \hline $x/\theta \ \mathsf{\tilde{s}} \ x/\theta = x/\theta \
  %         \mathsf{\tilde{\imath}} \ x/\theta = x/\theta$ \\\hline \end{tabular}
  %       \item[(I2)] \begin{tabular}{|c|} \hline $x/\theta \ \mathsf{\tilde{s}} \ y/\theta = y/\theta \
  %         \mathsf{\tilde{s}} \ x/\theta$ \\\hline \end{tabular}
  %       \item[(I3)] \begin{tabular}{|c|} \hline $x/\theta \ \mathsf{\tilde{\imath}} \ y/\theta = y/\theta \
  %         \mathsf{\tilde{\imath}} \ x/\theta$ \\\hline \end{tabular}
  %       \item[(I4)] \begin{tabular}{|c|} \hline $(x/\theta \ \mathsf{\tilde{s}} \ y/\theta) \ \mathsf{\tilde{s}} \
  %         z/\theta = x/\theta \ \mathsf{\tilde{s}} \ (y/\theta \ \mathsf{\tilde{s}} \ z/\theta)$ \\\hline \end{tabular}
  %
  %         \((x\mathsf{\;s\;}y)\;\mathsf{s}\;z=x\;\mathsf{s}\;(y\;\mathsf{s} \;z)\)
  %         \begin{eqnarray*}
  %           (x/\theta \ \mathsf{\tilde{s}} \ y/\theta) \ \mathsf{\tilde{s}} \ z/\theta &=& (x \ \mathsf{s} \ y)/\theta \
  %             \mathsf{\tilde{s}} \ z/\theta \\
  %           &=& ((x \ \SU \ y) \ \SU \ z) /\theta \\
  %           &=& (x \ \SU \ (y \ \SU \ z)) /\theta \\
  %           &=& x/\theta \ \mathsf{\tilde{s}} \ (y \ \SU \ z) /\theta \\
  %           &=& x/\theta \ \mathsf{\tilde{s}} \ (y /\theta \ \mathsf{\tilde{s}} \ z/\theta)
  %         \end{eqnarray*}
  %       \item[(I5)] \begin{tabular}{|c|} \hline $(x/\theta \ \mathsf{\tilde{\imath}} \ y/\theta) \
  %         \mathsf{\tilde{\imath}} \ z/\theta = x/\theta \ \mathsf{\tilde{\imath}} \ (y/\theta \ \mathsf{\tilde{\imath}}
  %         \ z/\theta)$ \\\hline \end{tabular}
  %       \item[(I6)] \begin{tabular}{|c|} \hline $x/\theta \ \mathsf{\tilde{s}} \ (x/\theta \ \mathsf{\tilde{\imath}} \
  %         y/\theta) = x/\theta$ \\\hline \end{tabular}
  %       \item[(I7)] \begin{tabular}{|c|} \hline $x/\theta \ \mathsf{\tilde{\imath}} \ (x/\theta \ \mathsf{\tilde{s}} \
  %         y/\theta) = x/\theta$ \\\hline \end{tabular}
  %     \end{enumerate}
  %   \end{multicols}
  % \end{proof}
  %
  % Corollary 97: Con prueba. Corollary 9.
  \begin{corollary}
    \PN Sea $(L, \SU, \IN)$ un reticulado en el cual hay un elemento máximo $1$ (resp. mínimo $0$), entonces si $\theta$
    es una congruencia sobre $(L, \SU, \IN), 1/\theta$ (resp. $0/\theta$) es un elemento máximo (resp. mínimo) de
    $(L/\theta, \mathsf{\tilde{s}}, \mathsf{\tilde{\imath}})$.
  \end{corollary}
  \begin{proof}
    \PN Ya que $1 \ \theta \ (x \ \SU \ 1)$, para cada $x \in L$, tenemos que $x/\theta \ \tilde{\leq} \ 1/\theta$, para
    cada $x \in L$.
  \end{proof}

  % % Lemma 98: Con prueba. Lemma 10.
  % \begin{lemma}
  %   \PN Si $F: (L, \SU, \IN) \rightarrow (L^{\prime}, \SU^{\prime}, \IN^{\prime})$ es un homomorfismo de reticulados,
  %   entonces $\ker F$ es una congruencia sobre $(L, \SU, \IN)$.
  % \end{lemma}
  % \begin{proof}
  %   Dejamos al lector ver que $\ker F$ es una relacion de equivalencia. Supongamos $x\ker Fx^{\prime}$ y $y\ker Fy^{\prime}$. Entonces
  %
  %   $\displaystyle F(x\mathsf{\;s\;}y)=F(x)\mathsf{\;s^{\prime}\;}F(y)=F(x^{\prime})\mathsf{ \;s^{\prime}\;}F(y^{\prime})=F(x^{\prime}\mathsf{\;s\;}y^{\prime}) $
  %
  %   lo cual nos dice que $(x\mathsf{\;s\;}y)\ker F(x^{\prime}\mathsf{\;s\;} y^{\prime})$. En forma similar tenemos que $(x \ \IN \ y)\ker F(x^{\prime} \ \IN \ y^{\prime})$. $\Box$
  % \end{proof}
  %
  % % Lemma 99: Con prueba. Lemma 11.
  % \begin{lemma}
  %   \PN Sea $(L, \SU, \IN)$ un reticulado y sea $\theta$ una congruencia sobre $(L, \SU, \IN)$, entonces $\pi_{\theta}$
  %   es un homomorfismo de $(L, \SU, \IN)$ en $(L/\theta, \mathsf{\tilde{s}}, \mathsf{\tilde{\imath}})$. Además $\ker
  %   \pi_{\theta} = \theta$.
  % \end{lemma}
  % \begin{proof}
  %   Sean $x,y\in L$. Tenemos que
  %
  %   $\displaystyle \pi _{\theta }(x\mathsf{\;s\;}y)=(x\mathsf{\;s\;}y)/\theta =x/\theta \mathsf{ \;\tilde{s}\;}y/\theta =\pi _{\theta }(x)\mathsf{\;\tilde{s}\;}\pi _{\theta }(y) $
  %
  %   por lo cual $\pi _{\theta }$ preserva la operacion supremo. Para la operacion infimo es similar. $\Box$
  % \end{proof}
  %
  % % Lemma 100: Con prueba. Lemma 12.
  % \begin{lemma}
  %   \PN Si $F: (L, \SU, \IN, 0, 1) \rightarrow (L^{\prime}, \SU^{\prime}, \IN^{\prime}, 0^{\prime}, 1^{\prime})$ un
  %   homomorfismo biyectivo, entonces $F$ es un isomorfismo.
  % \end{lemma}
  % \begin{proof}
  %   Similar a la prueba del Lemma 93. $\Box$
  % \end{proof}
  %
  % % Lemma 101: Con prueba. Lemma 13.
  % \begin{lemma}
  %   \PN Si $F: (L, \SU, \IN, 0, 1) \rightarrow (L^{\prime}, \SU^{\prime}, \IN^{\prime}, 0^{\prime}, 1^{\prime})$ es un
  %   homomorfismo, entonces $I_{F}$ es un subuniverso de $(L^{\prime}, \SU^{\prime}, \IN^{\prime}, 0^{\prime},
  %   1^{\prime})$.
  % \end{lemma}
  % \begin{proof}
  %   Ya que $F$ es un homomorfismo de $(L,\SU,\IN)$ en $ (L^{\prime},\SU^{\prime},\IN^{\prime})$ tenemos que $I_{F}$ es subuniverso de $(L^{\prime},\SU^{\prime},\IN^{\prime})$ lo cual ya que $0^{\prime},1^{\prime}\in I_{F}$ implica que $I_{F}$ es un subuniverso de $(L^{\prime},\SU^{\prime},\IN^{\prime},0^{\prime},1^{\prime})$. $\Box$
  % \end{proof}
  %
  % % Lemma 102: Sin prueba. Lemma 14.
  % \begin{lemma}
  %   \PN Si $F: (L, \SU, \IN, 0, 1) \rightarrow (L^{\prime}, \SU^{\prime}, \IN^{\prime},
  %   0^{\prime}, 1^{\prime})$ es un homomorfismo de reticulados acotados, entonces $\ker F$ es una congruencia sobre $(L,
  %   \SU, \IN, 0, 1)$.
  % \end{lemma}
  %
  % % Lemma 103: Sin prueba. Lemma 15.
  % \begin{lemma}
  %   \PN Sea $(L, \SU, \IN, 0, 1)$ un reticulado acotado y $\theta$ una congruencia sobre $(L, \SU, \IN, 0, 1)$, entonces:
  %   \begin{enumerate}[a)]
  %     \item $(L/\theta, \mathsf{\tilde{s}}, \mathsf{\tilde{\imath}}, 0/\theta, 1/\theta)$ es un reticulado acotado.
  %     \item $\pi_{\theta}$ es un homomorfismo de $(L, \SU, \IN, 0, 1)$ en $(L/\theta, \mathsf{\tilde{s}},
  %       \mathsf{\tilde{\imath}}, 0/\theta, 1/\theta)$ cuyo núcleo es $\theta$.
  %   \end{enumerate}
  % \end{lemma}
  %
  % % Lemma 104: Sin prueba. Lemma 16.
  % \begin{lemma}
  %   \PN Si $F:(L, \SU, \IN, ^{c}, 0, 1) \rightarrow (L^{\prime}, \SU^{\prime}, \IN^{\prime}, ^{c^{\prime}}, 0^{\prime},
  %   1^{\prime})$ un homomorfismo biyectivo, entonces $F$ es un isomorfismo.
  % \end{lemma}
  %
  % % Lemma 105: Sin prueba. Lemma 17.
  % \begin{lemma}
  %   \PN Si $F: (L, \SU, \IN, ^{c}, 0, 1) \rightarrow (L^{\prime}, \SU^{\prime}, \IN^{\prime}, ^{c^{\prime}}, 0^{\prime},
  %   1^{\prime})$ es un homomorfismo, entonces $I_{F}$ es un subuniverso de $(L^{\prime}, \SU^{\prime}, \IN^{\prime},
  %   ^{c^{\prime}}, 0^{\prime}, 1^{\prime})$.
  % \end{lemma}
  %
  % % Lemma 106: Sin prueba. Lemma 18.
  % \begin{lemma}
  %   \PN Si $F: (L, \SU, \IN, ^{c}, 0, 1) \rightarrow (L^{\prime}, \SU^{\prime}, \IN^{\prime}, ^{c^{\prime}}, 0^{\prime},
  %   1^{\prime})$ es un homomorfismo de reticulados complementados, entonces $\ker F$ es una congruencia sobre $(L, \SU,
  %   \IN, ^{c}, 0, 1)$.
  % \end{lemma}
  %
  % % Lemma 107: Sin prueba. Lemma 19.
  % \begin{lemma}
  %   \PN Sea $(L, \SU, \IN, ^{c}, 0, 1)$ un reticulado complementado y sea $\theta$ una congruencia sobre $(L, \SU, \IN,
  %   ^{c}, 0, 1)$.
  %   \begin{enumerate}[a)]
  %     \item $(L/\theta, \mathsf{\tilde{s}}, \mathsf{\tilde{\imath}}, ^{\tilde{c}}, 0/\theta, 1/\theta)$ es un reticulado
  %       complementado.
  %     \item $\pi_{\theta}$ es un homomorfismo de $(L, \SU, \IN, ^{c}, 0, 1)$ en $(L/\theta, \mathsf{\tilde{s}},
  %     \mathsf{\tilde{\imath}}, ^{\tilde{c}}, 0/\theta, 1/\theta)$ cuyo núcleo es $\theta$.
  %   \end{enumerate}
  % \end{lemma}
  %
  % % Lemma 108: Con prueba. Lemma 20.
  % \begin{lemma}
  %   \PN Sea $(L, \SU, \IN)$ un reticulado. Son equivalentes:
  %   \begin{enumerate}[(1)]
  %     \item $x \ \IN \ (y \ \SU \ z) = (x \ \IN \ y) \ \SU \ (x \ \IN \ z)$, cualesquiera sean $x, y, z \in L$
  %     \item $x \ \SU \ (y \ \IN \ z) = (x \ \SU \ y) \ \IN \ (x \ \SU \ z)$, cualesquiera sean $x, y, z \in L$.
  %   \end{enumerate}
  % \end{lemma}
  % \begin{proof}
  %   (1)$\Rightarrow $(2). Notese que
  %
  %   $\displaystyle \begin{array}{rcl} (x\mathsf{\;s\;}y) \ \IN \ (x\;\SU\;z) & =& ((x\mathsf{\;s\;}y)  \ \IN \ x)\;\SU\;((x\mathsf{\;s\;}y) \ \IN \ z) \\ & =& (x\;\SU\;(z \ \IN \ (x\mathsf{\;s\;}y)) \\ & =& (x\;\SU\;((z \ \IN \ x)\mathsf{\;s\;}(z \ \IN \ y)) \\ & =& (x\;\SU\;(z \ \IN \ x))\mathsf{\;s\;}(z \ \IN \ y) \\ & =& x\mathsf{\;s\;}(z \ \IN \ y) \\ & =& x\mathsf{\ s\ }(y\ \mathsf{i\ }z) \end{array} $
  %
  %   (2)$\Rightarrow $(1) es similar. $\Box$
  % \end{proof}
  %
  % % Lemma 109: Con prueba. Lemma 21.
  % \begin{lemma}
  %   \PN Si $(L, \SU, \IN, 0, 1)$ un reticulado acotado y distributivo, entonces todo elemento tiene a lo sumo un
  %   complemento.
  % \end{lemma}
  % \begin{proof}
  %   Supongamos $x\in L$ tiene complementos $y,z$. Se tiene
  %
  %   % $\displaystyle y=y \ \IN \ 1=y \ \IN \ (x\ \SU \z)=(y \ \IN \ x)\; \mathsf{s\;}(y \ \IN \ z)=0\ \SU \(y \ \IN \ z)=y\;\mathsf{ i\;}z, $
  %
  %   por lo cual $y\leq z$. En forma analoga se muestra que $z\leq y$. $\Box$
  % \end{proof}
  %
  % % Lemma 110: Con prueba. Lemma 22.
  % \begin{lemma}
  %   \PN Si $S \neq \emptyset$, entonces $[S)$ es un filtro. Más aún si $F$ es un filtro y $F \supseteq S$, entonces $F
  %   \supseteq \lbrack S)$.
  % \end{lemma}
  % \begin{proof}
  %   Ya que $S\subseteq \lbrack S)$, tenemos que $[S)\neq \varnothing $. Claramente $[S)$ cumple la propiedad (3). Veamos cumple la (2). Si $y\geq s_{1}\; \mathsf{i\;}s_{2} \ \IN \ \dotsc \ \IN \ s_{n}$ y $z\geq t_{1}\; \mathsf{i\;}t_{2} \ \IN \ $\dotsc$ \ \IN \ t_{m}$, con $ s_{1},s_{2}, \dotsc, s_{n}$, $t_{1},t_{2}, \dotsc, t_{m}\in S$, entonces
  %
  %   $\displaystyle y \ \IN \ z\geq s_{1} \ \IN \ s_{2} \ \IN \ \dotsc \ \IN \  s_{n} \ \IN \ t_{1} \ \IN \ t_{2} \ \IN \ \dotsc \ \IN \  t_{m}, $
  %
  %   lo cual prueba (2).
  % \end{proof}
  %
  % % Lemma 111: Sin prueba. Lemma 23.
  % \begin{lemma}
  %   \PN (\textbf{Zorn}) Sea $(P, \leq)$ un poset y supongamos que cada cadena de $P$ tiene una cota superior, entonces
  %   existe un elemento maximal en $P$. Un filtro $F$ de un reticulado $(L, \SU, \IN)$ será llamado primo
  %   cuando se cumplan:
  %   \begin{enumerate}
  %     \item $F \neq L$
  %     \item $x \ \SU \ y \in F \Rightarrow x \in F$ ó $y \in F$.
  %   \end{enumerate}
  % \end{lemma}
  %
  % % Theorem 112: Con prueba. Theorem 24.
  % \begin{theorem}
  %   \PN (Teorema del Filtro Primo) Sea $(L, \SU, \IN)$ un reticulado distributivo y $F$ un filtro. Supongamos $x_{0} \in
  %   L-F$, entonces hay un filtro primo $P$ tal que $x_{0} \notin P$ y $F \subseteq P$.
  % \end{theorem}
  % \begin{proof}
  %   Sea
  %
  %   $\displaystyle \mathcal{F}=\{F_{1}:F_{1}\text{ es un filtro, }x_{0}\notin F_{1}\text{ y } F\subseteq F_{1}\}. $
  %
  %   Notese que $\mathcal{F}\neq \varnothing $, por lo cual $(\mathcal{F},\subseteq )$ es un poset. Veamos que cada cadena en $(\mathcal{F},\subseteq )$ tiene una cota superior. Sea $C$ una cadena. Si $C=\varnothing $, entonces cualquier elemento de $\mathcal{F}$ es cota de $C$. Supongamos entonces $C\neq \varnothing $. Sea
  %   $\displaystyle G=\{x\in L:x\in F_{1},\text{para algun }F_{1}\in C\}. $
  %
  %   Veamos que $G$ es un filtro. Es claro que $G$ es no vacio. Supongamos que $ x,y\in G$. Sean $F_{1},F_{2}\in \mathcal{F}$ tales que $x\in F_{1}$ y $y\in F_{2}$. Si $F_{1}\subseteq F_{2}$, entonces ya que $F_{2}$ es un filtro tenemos que $x \ \IN \ y\in F_{2}\subseteq G$. Si $F_{2}\subseteq F_{1}$ , entonces tenemos que $x \ \IN \ y\in F_{1}\subseteq G$. Ya que $C$ es una cadena, tenemos que siempre $x \ \IN \ y\in G$. En forma analoga se prueba la propiedad restante por lo cual tenemos que $G$ es un filtro. Ademas $x_{0}\notin G$, por lo que $G\in \mathcal{F}$ es cota superior de $C$ . Por el lema de Zorn, $(\mathcal{F},\subseteq )$ tiene un elemento maximal $ P$. Veamos que $P$ es un filtro primo. Supongamos $x\ \SU \y\in P$ y $ x,y\notin P$. Entonces ya que $P$ es maximal tenemos que
  %   $\displaystyle x_{0}\in \lbrack P\cup \{x\})\cap \lbrack P\cup \{y\}) $
  %
  %   Ya que $x_{0}\in \lbrack P\cup \{x\})$, tenemos que hay elementos $ p_{1}, \dotsc, p_{n}\in P$, tales que
  %   $\displaystyle x_{0}\geq p_{1} \ \IN \ \dotsc \ \IN \ p_{n} \ \IN \ x $
  %
  %   Ya que $x_{0}\in \lbrack P\cup \{y\})$, tenemos que hay elementos $ q_{1}, \dotsc, q_{m}\in P$, tales que
  %   $\displaystyle x_{0}\geq q_{1} \ \IN \ \dotsc \ \IN \ q_{m} \ \IN \ y $
  %
  %   Si llamamos $p$ al siguiente elemento de $P$
  %   $\displaystyle p_{1} \ \IN \ \dotsc \ \IN \ p_{n} \ \IN \ q_{1} \ \IN \  \dotsc \ \IN \ q_{m} $
  %
  %   tenemos que
  %   $\displaystyle \begin{array}{rcl} x_{0} & \geq & p \ \IN \ x \\ x_{0} & \geq & p \ \IN \ y \end{array} $
  %
  %   Se tiene que $x_{0}\geq (p \ \IN \ x)\ \SU \(p \ \IN \  y)=p \ \IN \ (x\ \SU \y)\in P$, lo cual es absurdo ya que $ x_{0}\notin P$. $\Box$
  % \end{proof}
  %
  % % Corollary 113: Con prueba. Corollary 25.
  % \begin{corollary}
  %   \PN Sea $(L,\SU,\IN,0,1)$ un reticulado acotado distributivo. Si $\emptyset \neq S \subseteq L$ es tal que $s_{1}
  %   \ \IN \ s_{2} \ \IN \ \dotsc \ \IN \ s_{n} \neq 0$, para cada $s_{1}, \dotsc, s_{n} \in S$, entonces hay un filtro
  %   primo que contiene a $S$.
  % \end{corollary}
  % \begin{proof}
  %   Notese que $[S)\neq L$ por lo cual se puede aplicar el Teorema del filtro primo. $\Box$
  % \end{proof}
  %
  % % Lemma 114: Con prueba. Lemma 26.
  % \begin{lemma}
  %   \PN Sea $(B, \SU, \IN, ^{c}, 0, 1)$ un algebra de Boole, entonces para un filtro $F \subseteq B$ las siguientes son
  %   equivalentes:
  %   \begin{enumerate}
  %     \item $F$ es primo
  %     \item $x \in F$ ó $x^{c} \in F$, para cada $x \in B$.
  %   \end{enumerate}
  % \end{lemma}
  % \begin{proof}
  %   (1)$\Rightarrow $(2). Ya que $x\ \SU \x^{c}=1\in F$, (2) se cumple si $F$ es primo.
  %
  %   (2)$\Rightarrow $(1). Supongamos que $x\ \SU \y\in F$ y que $x\not\in F$. Entonces por (2), $x^{c}\in F$ y por lo tanto tenemos que
  %
  %   $\displaystyle y\geq x^{c} \ \IN \ y=(x^{c} \ \IN \ x)\ \SU \(x^{c}\; \mathsf{i\;}y)=x^{c} \ \IN \ (x\ \SU \y)\in F, $
  %
  %   lo cual dice que $y\in F$. $\Box$
  % \end{proof}
  %
  % % Lemma 115: Con prueba. Lemma 27.
  % \begin{lemma}
  %   Sea $(B,\SU,\IN,^{c},0,1)$ un algebra de Boole. Supongamos que $b\neq 0$ y $a=\inf A$, con $A\subseteq B$. Entonces si $b \ \IN \ a=0$ , existe un $e\in A$ tal que $b \ \IN \ e^{c}\neq 0$.
  % \end{lemma}
  % \begin{proof}
  %   Supongamos que para cada $e\in A$, tengamos que $b \ \IN \ e^{c}=0$. Entonces tenemos que para cada $e\in A$,
  %
  %   $\displaystyle b=b \ \IN \ (e\ \SU \e^{c})=(b \ \IN \ e)\ \SU \(b\; \mathsf{i\;}e^{c})=b \ \IN \ e, $
  %
  %   lo cual nos dice que $b$ es cota inferior de $A$. Pero entonces $b\leq a$, por lo cual $b=b \ \IN \ a=0$, contradiciendo la hipotesis. $\Box$
  % \end{proof}
  %
  % % Theorem 116: Con prueba. Theorem 28.
  % \begin{theorem}
  %   \PN (Rasiova y Sikorski) Sea $(B, \SU, \IN, ^{c}, 0, 1)$ un algebra de Boole. Sea $x \in B$, $x \neq 0$. Supongamos
  %   que $A_{1}, A_{2}, \dotsc$ son subconjuntos de $B$ tales que existe $\inf(A_{j})$, para cada $j = 1, 2 \dotsc$,
  %   entonces hay un filtro primo $P$ el cual cumple:
  %   \begin{enumerate}[a)]
  %     \item $x \in P$
  %     \item $P \supseteq A_{j} \Rightarrow P \ni \inf (A_{j})$, para cada $ j = 1, 2, \dotsc.$
  %   \end{enumerate}
  % \end{theorem}
  % \begin{proof}
  %   Sean $a_{j}=\inf (A_{j})$, $j=1,2,\dotsc$. Construiremos inductivamente una sucesion $b_{0},b_{1},\dotsc$ de elementos de $B$ tal que:
  %
  %   (1) $b_{0}=x$
  %   (2) $b_{0} \ \IN \ $\dotsc$ \ \IN \ b_{n}\neq 0$, para cada $ n\geq 0$
  %   (3) $b_{j}=a_{j}$ o $b_{j}^{c}\in A_{j}$, para cada $j\geq 1$.
  %   Definamos $b_{0}=x$. Supongamos ya definimos $b_{0}, \dotsc, b_{n}$, veamos como definir $b_{n+1}$. Si $(b_{0} \ \IN \ \dotsc \ \IN \  b_{n}) \ \IN \ a_{n+1}\neq 0$, entonces definamos $b_{n+1}=a_{n+1}$. Si $(b_{0} \ \IN \ \dotsc \ \IN \ b_{n}) \ \IN \ a_{n+1}=0$, entonces por el lema anterior, tenemos que hay un $e\in A_{n+1}$ tal que $ (b_{0} \ \IN \ \dotsc \ \IN \ b_{n}) \ \IN \ e^{c}\neq 0$, lo cual nos permite definir $b_{n+1}=e^{c}$.
  %
  %   Usando (2) se puede probar que el conjunto $S=\{b_{0},b_{1},\dotsc\}$ satisface la hipotesis del primer corolario del Teorema del filtro primo, por lo cual hay un filtro primo $P$ tal que $\{b_{0},b_{1},\dotsc\}\subseteq P$. Es facil chequear que $P$ satisface las propiedades (a) y (b). $\Box$
  % \end{proof}
