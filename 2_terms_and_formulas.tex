\section{Términos y fórmulas}

  % Lemma 119. Con prueba. Lemma 29.
  \begin{lemma} \label{lemma_29}
    \PN Supongamos $t \in T_{k}^{\tau}$, con $k \geq 1$, entonces ya sea $t \in Var \cup \mathcal{C}$ ó $t = f(t_{1},
    \dotsc, t_{n})$, con $f \in \mathcal{F}_{n}, n \geq 1, \; t_{1}, \dotsc, t_{n} \in T_{k-1}^{\tau}$.
  \end{lemma}
  \begin{proof}
    \PN Probaremos este lema por inducción en $k$.

    \vspace{3mm}
    \PN \underline{Caso Base:} \begin{tabular}{|c|} \hline $k = 1$ \\\hline \end{tabular} Es directo, ya que por
    definición:
    \[
      T_{1}^{\tau} = Var \cup \mathcal{C} \cup \{f(t_{1}, t_{2}, \dotsc t_{n}): f \in \mathcal{F}_n, n \geq 1, t_{1},
      t_{2}, \dotsc t_{n} \in T_{0}^{\tau}\}
    \]

		\PN \underline{Caso Inductivo:} \begin{tabular}{|c|} \hline $k \Rightarrow k + 1$ \\\hline \end{tabular} Sea $t \in
    T_{k+1}^{\tau}$. Por definición de $ T_{k+1}^{\tau}$ tenemos que:
    \begin{itemize}
      \item $t \in T_{k}^{\tau}$ ó
      \item $t = f(t_{1}, \dotsc, t_{n})$ con $f \in \mathcal{F}_{n}, n \geq 1$ y $t_{1}, \dotsc, t_{n}\in T_{k}^{\tau}$.
    \end{itemize}

    \PN Si se da que $t \in T_{k}^{\tau}$, entonces podemos aplicar hipótesis inductiva y usar que $T_{k-1}^{\tau}
    \subseteq T_{k}^{\tau}$.
  \end{proof}

  % Lemma 120. Nada. Lemma 30.
  \begin{lemma}
    \PN Este lema no se evalua.
  \end{lemma}

  % Lemma 121. Nada. Lemma 31.
  \begin{lemma}
    \PN Este lema no se evalua.
  \end{lemma}

  % Lemma 122. Nada. Lemma 32.
  \begin{lemma}
    \PN Este lema no se evalua.
  \end{lemma}

  % Lemma 123. Nada. Lemma 33.
  \begin{lemma}
    \PN Este lema no se evalua.
  \end{lemma}

  % Theorem 124. Sin prueba. Lemma 34.
  \begin{theorem} \label{lemma_34}
    \PN \textbf{(Lectura única de terminos)}. Dado $t \in \TAU$ se da una de las siguientes:
    \begin{enumerate}[(1)]
      \item $t \in Var \cup \mathcal{C}$
      \item Hay únicos $n \geq 1, \ f \in \mathcal{F}_{n}, \ t_{1}, \dotsc, t_{n} \in \TAU$ tales que $t = f(t_{1},
        \dotsc, t_{n})$.
    \end{enumerate}
  \end{theorem}

  % Lemma 125. Sin prueba. Lemma 35.
  \begin{lemma} \label{lemma_35}
    \PN Sean $r, s, t \in \TAU$.
    \begin{enumerate}[(a)]
      \item Si $s \neq t = f(t_{1}, \dotsc, t_{n})$ y $s$ ocurre en $t$, entonces dicha ocurrencia sucede dentro de
        algún $t_{j}$, $j = 1, \dotsc, n$.
      \item Si $r, s$ ocurren en $t$, entonces dichas ocurrencias son disjuntas o una ocurre dentro de otra. En
        particular, las distintas ocurrencias de $r$ en $t$ son disjuntas.
      \item Si $t^{\prime}$ es el resultado de reemplazar una ocurrencia de $s$ en $t$ por $r$, entonces $t^{\prime} \in
        \TAU$.
    \end{enumerate}
  \end{lemma}

  % Lemma 126. Con prueba. Lemma 36.
  \begin{lemma} \label{lemma_36}
    \PN Supongamos $\varphi \in F_{k}^{\tau}$, con $k \geq 1$, entonces $\varphi$ es de alguna de las siguientes formas:
    \begin{itemize}
      \item $\varphi = (t \equiv s)$, con $t, s \in \TAU$.
      \item $\varphi = r(t_{1}, \dotsc, t_{n})$, con $r \in \mathcal{R}_{n}, t_{1}, \dotsc, t_{n} \in \TAU$.
      \item $\varphi = (\varphi_{1} \eta \varphi_{2})$, con $\eta \in \{\wedge, \vee, \rightarrow, \leftrightarrow\}, \
        \varphi_{1}, \varphi_{2} \in F_{k-1}^{\tau}$.
      \item $\varphi = \lnot \varphi_{1}$, con $\varphi_{1} \in F_{k-1}^{\tau}$.
      \item $\varphi = Qv\varphi_{1}$, con $Q \in \{\forall, \exists\}, \ v \in Var$ y $\varphi_{1} \in F_{k-1}^{\tau}$.
    \end{itemize}
    \PN Llamaremos ($\star$) a la lista anterior.
  \end{lemma}
  \begin{proof}
    \PN Probaremos este teorema por inducción en $k$, utilizando la definición del conjunto $\FT$.

    \vspace{3mm}
    \PN \underline{Caso Base:}
    \[
      \varphi \ \in \ \{(t \equiv s): t, s \in \TAU\} \ \cup \ \{r(t_{1}, \dotsc, t_{n}): r \in \mathcal{R}_{n}, n \geq
      1, t_{1}, \dotsc, t_{n} \in \TAU\}
    \]
    \PN por lo que $\varphi$ es de alguna de las siguientes formas:
    \begin{itemize}
      \item $\varphi = (t \equiv s)$, con $t, s \in \TAU$.
      \item $\varphi = r(t_{1}, \dotsc, t_{n})$, con $r \in \mathcal{R}_{n}, t_{1}, \dotsc, t_{n} \in \TAU$.
    \end{itemize}

    \vspace{3mm}
		\PN \underline{Caso Inductivo:} Supongamos que si $\varphi \in F_{k-1}^{\tau}$ entonces $\varphi$ es de alguna de
    las formas de ($\star$). Probaremos que si $\varphi \in F_{k}^{\tau}$ entonces $\varphi$ también es de alguna de las
    formas de la lista ($\star$).
    \begin{eqnarray*}
      \varphi \ \in \ F_{k-1}^{\tau} &\cup& \{\lnot \varphi: \varphi \in F_{k-1}^{\tau}\} \ \cup \ \{(\varphi \eta \psi):
        \varphi, \psi \in F_{k-1}^{\tau}, \eta \in \{\vee, \wedge, \rightarrow, \leftrightarrow\}\} \\
      &\cup& \ \{Qv\varphi: \varphi \in F_{k-1}^{\tau}, v \in Var, Q \in \{\forall, \exists\}\}
    \end{eqnarray*}

    \PN Luego, si $\varphi \in F_{k-1}^{\tau}$ aplicando HI y el hecho de que $F_{k-2}^{\tau} \subseteq F_{k-1}^{\tau}$,
    obtenemos que $\varphi$ es de alguna de las formas de la lista anterior. Caso contrario, se dá alguna de las
    siguientes:
    \begin{itemize}
      \item $\varphi = (\varphi_{1} \eta \varphi_{2})$, con $\varphi_{1}, \varphi_{2} \in F_{k-1}^{\tau}, \eta \in
      \{\wedge, \vee, \rightarrow, \leftrightarrow\}$.
      \item $\varphi = \lnot \varphi_{1}$, con $\varphi_{1} \in F_{k-1}^{\tau}$.
      \item $\varphi = Qv\varphi_{1}$, con $Q \in \{\forall, \exists\}, \ v \in Var$ y $\varphi_{1} \in F_{k-1}^{\tau}$.
    \end{itemize}
  \end{proof}

  % Lemma 127. Nada. Lemma 37.
  \begin{lemma}
    \PN Este lema no se evalua.
  \end{lemma}

  % Lemma 128. Nada. Lemma 38.
  \begin{lemma}
    \PN Este lema no se evalua.
  \end{lemma}

  % Lemma 129. Nada. Lemma 39.
  \begin{lemma}
    \PN Este lema no se evalua.
  \end{lemma}

  % Proposition 130. Nada. Lemma 40.
  \begin{proposition}
    \PN Este proposición no se evalua.
  \end{proposition}

  % Theorem 131. Sin prueba. Lemma 41.
  \begin{theorem} \label{lemma_41}
    \PN \textbf{(Lectura única de fórmulas)} Dada $\varphi \in \FT$ se da una y solo una de las siguientes:
    \begin{enumerate}[(1)]
      \item $\varphi = (t \equiv s)$, con $t, s \in \TAU$
      \item $\varphi = r(t_{1}, \dotsc, t_{n})$, con $r \in \mathcal{R}_{n}, t_{1}, \dotsc, t_{n} \in \TAU$
      \item $\varphi = (\varphi_{1} \eta \varphi_{2})$, con $\eta \in \{\wedge, \vee, \rightarrow, \leftrightarrow\}, \
        \varphi_{1}, \varphi_{2} \in \FT$
      \item $\varphi = \lnot \varphi_{1}$, con $\varphi_{1} \in \FT$
      \item $\varphi = Qv \varphi_{1}$, con $Q \in \{\forall, \exists\}, \ \varphi_{1} \in \FT$ y $v \in Var$.
    \end{enumerate}

    \PN Más aún, en todos los puntos tales descomposiciones son únicas.
  \end{theorem}

  % Lemma 132. Sin prueba. Lemma 42.
  \begin{lemma} \label{lemma_42}
    \PN Sea $\tau$ un tipo.
    \begin{enumerate}[(a)]
      \item Las fórmulas atómicas no tienen subfórmulas propias.
      \item Si $\varphi$ ocurre propiamente en $(\psi \eta \varphi)$, entonces tal ocurrencia es en $\psi$ ó en
        $\varphi$.
      \item Si $\varphi$ ocurre propiamente en $\lnot \psi$, entonces tal ocurrencia es en $\psi$.
      \item Si $\varphi$ ocurre propiamente en $Qx_{k} \psi$, entonces tal ocurrencia es en $\psi$.
      \item Si $\varphi_{1}, \varphi_{2}$ ocurren en $\varphi$, entonces dichas ocurrencias son disjuntas o una contiene
        a la otra.
      \item Si $\lambda^{\prime}$ es el resultado de reemplazar alguna ocurrencia de $\varphi$ en $\lambda$ por $\psi$,
        entonces $\lambda^{\prime} \in \FT$.
    \end{enumerate}
  \end{lemma}
