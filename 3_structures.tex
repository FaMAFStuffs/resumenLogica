\section{Estructuras}

  % % Lemma 131. Con prueba. Lemma 43.
  % \begin{lemma}
  %   \PN Sea $\mathbf{A}$ una estructura de tipo $\tau$ y sea $t\in \TAU$. Supongamos que $\vec{a}, \vec{b}$ son
  %   asignaciones tales que $a_{i} = b_{i},$ cada vez que $x_{i}$ ocurra en $t$, entonces $t^{\mathbf{A}}[\vec{a}] =
  %   t^{\mathbf{A}}[\vec{b}]$.
  % \end{lemma}
  % \begin{proof}
  %   Sea
  %
  %   - Teo$_{k}$: El lema vale para $t\in T_{k}^{\tau }$.
  %   Teo$_{0}$ es facil de probar. Veamos Teo$_{k}\Rightarrow $Teo$_{k+1}$. Supongamos $t\in T_{k+1}^{\tau }-T_{k}^{\tau }$ y sean $\vec{a},\vec{b}$ asignaciones tales que $a_{i}=b_{i},$ cada vez que $x_{i}$ ocurra en $t$. Notese que $t=f(t_{1}, \dotsc, t_{n})$, con $f\in \mathcal{F}_{n},\;n\geq 1$ y $ t_{1}, \dotsc, t_{n}\in \TAU$. Notese que para cada $j=1, \dotsc, n$, tenemos que $a_{i}=b_{i},$ cada vez que $x_{i}$ ocurra en $t_{j}$, lo cual por Teo$_{k}$ nos dice que
  %
  %   $\displaystyle t_{j}^{\mathbf{A}}[\vec{a}]=t_{j}^{\mathbf{A}}[\vec{b}]\text{, }j=1, \dotsc, n $
  %
  %   Se tiene entonces que
  %   $\displaystyle \begin{array}{ccl} t^{\mathbf{A}}[\vec{a}] & = & i(f)(t_{1}^{\mathbf{A}}[\vec{a}], \dotsc, t_{n}^{ \mathbf{A}}[\vec{a}])\text{ (por def de }t^{\mathbf{A}}[\vec{a}]\text{)} \\ & = & i(f)(t_{1}^{\mathbf{A}}[\vec{b}], \dotsc, t_{n}^{\mathbf{A}}[\vec{b}]) \\ & = & t^{\mathbf{A}}[\vec{b}]\text{ (por def de }t^{\mathbf{A}}[\vec{a}] \text{)} \end{array} $
  %
  %   $\Box$
  % \end{proof}
  %
  % % Lemma 132. Con prueba. Lemma 44.
  % \begin{lemma}
  %   \begin{enumerate}[(a)]
  %     \item $Li((t \equiv s)) = \{v \in Var: v$ ocurre en $t$ ó $v$ ocurre en $s\}$
  %     \item $Li(r(t_{1}, \dotsc, ,t_{n})) = \{v \in Var: v$ ocurre en algún $t_{i}\}$
  %     \item $Li(\lnot \varphi) = Li(\varphi)$
  %     \item $Li((\varphi \eta \psi)) = Li(\varphi) \cup Li(\psi)$
  %     \item $Li(Qx_{j}\varphi) = Li(\varphi)-\{x_{j}\}$
  %   \end{enumerate}
  % \end{lemma}
  % \begin{proof}
  %   (a) y (b) son triviales de las definiciones, teniendo en cuenta que si una variable $v$ ocurre en $(t\equiv s)$ (resp. en $r(t_{1}, \dotsc, t_{n})$) entonces $v$ ocurre en $t$ o $v$ ocurre en $s$ (resp.$v$ ocurre en algun $ t_{i}$)
  %
  %   (d) Supongamos $v\in Li((\varphi \eta \psi ))$, entonces hay un $i$ tal que $ v$ ocurre libremente en $(\varphi \eta \psi )$ a partir de $i$. Por definicion tenemos que ya sea $v$ ocurre libremente en $\varphi $ a partir de $i-1$ o $v$ ocurre libremente en $\psi $ a partir de $i-\left\vert (\varphi \eta \right\vert $, con lo cual $v\in Li(\varphi )\cup Li(\psi )$
  %
  %   Supongamos ahora que $v\in Li(\varphi )\cup Li(\psi )$. S.p.d.g. supongamos $ v\in Li(\psi )$. Por definicion tenemos que hay un $i$ tal que $v$ ocurre libremente en $\psi $ a partir de $i$. Pero notese que esto nos dice por definicion que $v$ ocurre libremente en $(\varphi \eta \psi )$ a partir de $ i+\left\vert (\varphi \eta \right\vert $ con lo cual $v\in Li((\varphi \eta \psi ))$.
  %
  %   (c) es similar a (d)
  %
  %   (e) Supongamos $v\in Li(Qx_{j}\varphi )$, entonces hay un $i$ tal que $v$ ocurre libremente en $Qx_{j}\varphi $ a partir de $i$. Por definicion tenemos que $v\neq x_{j}$ y $v$ ocurre libremente en $\varphi $ a partir de $ i-\left\vert Qx_{j}\right\vert $, con lo cual $v\in Li(\varphi )-\{x_{j}\}$
  %
  %   Supongamos ahora que $v\in Li(\varphi )-\{x_{j}\}$. Por definicion tenemos que hay un $i$ tal que $v$ ocurre libremente en $\varphi $ a partir de $i$. Ya que $v\neq x_{j}$ esto nos dice por definicion que $v$ ocurre libremente en $Qx_{j}\varphi $ a partir de $i+\left\vert Qx_{j}\right\vert $, con lo cual $v\in Li(Qx_{j}\varphi )$. $\Box$
  % \end{proof}
  %
  % % Lemma 133. Con prueba. Lemma 45.
  % \begin{lemma}
  %   \PN Supongamos que $\vec{a}, \vec{b}$ son asignaciones tales que si $x_{i} \in Li(\varphi)$, entonces $a_{i} =
  %   b_{i}$, entonces $\mathbf{A} \models \varphi \lbrack \vec{a}] \Leftrightarrow \mathbf{A} \models \varphi \lbrack
  %   \vec{b}]$.
  % \end{lemma}
  % \begin{proof}
  %   Probaremos por induccion en $k$ que el lema vale para cada $\varphi \in F_{k}^{\tau }.$ El caso $k=0$ se desprende del Lema 131. Veamos que Teo$_{k}$ implica Teo$_{k+1}.$ Sea $\varphi \in F_{k+1}^{\tau }-F_{k}^{\tau }.$ Hay varios casos:
  %
  %   CASO $\varphi =(\varphi_{1}\wedge \varphi_{2})$.
  %
  %   Ya que $Li(\varphi_{i})\subseteq Li(\varphi )$, $i=1,2$, Teo$_{k}$ nos dice que $\mathbf{A} \models \varphi_{i}[\vec{a}]$ sii $\mathbf{A} \models \varphi_{i}[\vec{b}]$, para $i=1,2$. Se tiene entonces que
  %
  %   $\displaystyle \begin{array}{l} \mathbf{A} \models \varphi \lbrack \vec{a}] \\ \ \ \Updownarrow \text{ (por (3) en la def de }\mathbf{A} \models \varphi \lbrack \vec{a}]\text{)} \\ \mathbf{A} \models \varphi_{1}[\vec{a}]\text{ y }\mathbf{A} \models \varphi_{2}[\vec{a}] \\ \ \ \Updownarrow \text{ (por Teo}_{k}\text{)} \\ \mathbf{A} \models \varphi_{1}[\vec{b}]\text{ y }\mathbf{A} \models \varphi_{2}[\vec{b}] \\ \ \ \Updownarrow \text{(por (3) en la def de }\mathbf{A} \models \varphi \lbrack \vec{a}]\text{)} \\ \mathbf{A} \models \varphi \lbrack \vec{b}] \end{array} $
  %
  %   CASO $\varphi =(\varphi_{1}\wedge \varphi_{2})$.
  %
  %   Es completamente similar al anterior.
  %
  %   CASO $\varphi =\lnot \varphi_{1}.$
  %
  %   Es completamente similar al anterior.
  %
  %   CASO $\varphi =\forall x_{j}\varphi_{1}.$
  %
  %   Supongamos $\mathbf{A} \models \varphi \lbrack \vec{a}]$. Entonces por (8) en la def de $\mathbf{A} \models \varphi \lbrack \vec{a}]$ se tiene que $\mathbf{A} \models \varphi_{1}[\downarrow _{j}^{a}(\vec{a})]$, para todo $a\in A$. Notese que $\downarrow _{j}^{a}(\vec{a})$ y $\downarrow _{j}^{a}(\vec{b})$ coinciden en toda $x_{i}$ de $x_{i}\in Li(\varphi_{1})\subseteq Li(\varphi_{1})\cup \{x_{j}\}$, con lo cual por Teo$_{k}$ se tiene que $\mathbf{A} \models \varphi_{1}[\downarrow _{j}^{a}(\vec{b})]$, para todo $a\in A$, lo cual por (8) en la def de $\mathbf{A} \models \varphi \lbrack \vec{a}]$ nos dice que $\mathbf{A} \models \varphi \lbrack \vec{b}]$. La prueba de que $\mathbf{A} \models \varphi \lbrack \vec{b}]$ implica que $ \mathbf{A} \models \varphi \lbrack \vec{a}]$ es similar.
  %
  %   CASO $\varphi =\exists x_{j}\varphi_{1}$.
  %
  %   Es similar al anterior. $\Box$
  % \end{proof}
  %
  % % Corollary 134. Sin prueba. Lemma 46.
  % \begin{corollary}
  %   \PN Si $\varphi$ es una sentencia, entonces $\mathbf{A} \models \varphi \lbrack \vec{a}] \Leftrightarrow \mathbf{A}
  %   \models \varphi \lbrack \vec{b}]$, cualesquiera sean las asignaciones $\vec{a}, \vec{b}$.
  % \end{corollary}
  %
  % % Lemma 135. Con prueba. Lemma 47.
  % \begin{lemma}
  %   \begin{enumerate}[(a)]
  %     \item Si $Li(\varphi) \cup Li(\psi) \subseteq \{x_{i_{1}}, \dotsc, x_{i_{n}}\}$, entonces $\varphi \thicksim \psi$
  %     si y solo si la sentencia $\forall x_{i_{1}} \dotsc \forall x_{i_{n}}(\varphi \leftrightarrow \psi)$ es
  %     universalmente válida.
  %     \item Si $\varphi_{i} \thicksim \psi_{i}, i = 1, 2$, entonces $\lnot \varphi_{1} \thicksim \lnot \psi_{1},
  %     (\varphi_{1} \eta \varphi_{2}) \thicksim (\psi_{1} \eta \psi_{2})$ y $Qv\varphi_{1} \thicksim Qv \psi_{1}$.
  %     \item Si $\varphi \thicksim \psi$ y $\alpha^{\prime}$ es el resultado de reemplazar en una fórmula $\alpha$
  %     algunas (posiblemente $0$) ocurrencias de $\varphi$ por $\psi$, entonces $\alpha \thicksim \alpha^{\prime}$.
  %   \end{enumerate}
  % \end{lemma}
  % \begin{proof}
  %   Tenemos que
  %
  %   $\displaystyle \begin{array}{l} \varphi \thicksim \psi \\ \ \ \Updownarrow \text{ (por (6) de la def de}\models \text{)} \\ \mathbf{A} \models (\varphi \leftrightarrow \psi )[\vec{a}]\text{, para todo } \mathbf{A}\text{ y toda }\vec{a}\in A^{\mathbf{N}} \\ \ \ \Updownarrow \\ \mathbf{A} \models (\varphi \leftrightarrow \psi )[\downarrow _{i_{n}}^{a}(\vec{a })]\text{, para todo }\mathbf{A}\text{, }a\in A\text{ y toda }\vec{a}\in A^{ \mathbf{N}} \\ \ \ \Updownarrow (\text{por (8) de la def de}\models ) \\ \mathbf{A} \models \forall x_{i_{n}}(\varphi \leftrightarrow \psi )[\vec{a}] \text{, para todo }\mathbf{A}\text{ y toda }\vec{a}\in A^{\mathbf{N}} \\ \ \ \Updownarrow \\ \mathbf{A} \models \forall x_{i_{n}}(\varphi \leftrightarrow \psi )[\downarrow _{i_{n-1}}^{a}(\vec{a})]\text{, para todo }\mathbf{A}\text{, }a\in A\text{ y toda }\vec{a}\in A^{\mathbf{N}} \\ \ \ \Updownarrow \text{ (por (8) de la def de}\models \text{)} \\ \mathbf{A} \models \forall x_{i_{n-1}}\forall x_{i_{n}}(\varphi \leftrightarrow \psi )[\vec{a}]\text{, para todo }\mathbf{A}\text{ y toda }\vec{a}\in A^{ \mathbf{N}} \\ \ \ \Updownarrow \\ \ \ \ \ \vdots \\ \ \ \Updownarrow \\ \mathbf{A} \models \forall x_{i_{1}}...\forall x_{i_{n}}(\varphi \leftrightarrow \psi )[\vec{a}]\text{, para todo }\mathbf{A}\text{ y toda }\vec{a}\in A^{ \mathbf{N}} \\ \ \ \Updownarrow \\ \forall x_{i_{1}}...\forall x_{i_{n}}(\varphi \leftrightarrow \psi )\text{ es universalmente valida} \end{array} $
  %
  %   (b) Es dejado al lector.
  %
  %   (c) Por induccion en el $k$ tal que $\alpha \in F_{k}^{\tau }$. $\Box$
  % \end{proof}
  %
  % % Lemma 136. Con prueba. Lemma 48.
  % \begin{lemma}
  %   \PN Sea $F: \mathbf{A} \rightarrow \mathbf{B}$ un homomorfismo, entonces:
  %   \[
  %     F(t^{\mathbf{A}}[(a_{1}, a_{2}, \dotsc)] = t^{\mathbf{B}}[F(a_{1}), F(a_{2}), \dotsc)]
  %   \]
  %   \PN para cada $t \in \TAU, (a_{1}, a_{2}, \dotsc) \in A^{\mathbf{N}}$.
  % \end{lemma}
  % \begin{proof}
  %   Sea
  %
  %   - Teo$_{k}$: Si $F:\mathbf{A} \rightarrow \mathbf{B}$ es un homomorfismo, entonces
  %   $\displaystyle F(t^{\mathbf{A}}[(a_{1},a_{2},...)]=t^{\mathbf{B}}[F(a_{1}),F(a_{2}),...)] $
  %
  %   para cada $t\in T_{k}^{\tau }$, $(a_{1},a_{2},...)\in A^{\mathbf{N}}$.
  %   Teo$_{0}$ es trivial. Veamos que Teo$_{k}$ implica Teo$_{k+1}$. Supongamos que vale Teo$_{k}$ y supongamos $F:\mathbf{A} \rightarrow \mathbf{B}$ es un homomorfismo, $t\in T_{k+1}^{\tau }-T_{k}^{\tau }$ y $\vec{a} =(a_{1},a_{2},...)\in A^{\mathbf{N}}$. Denotemos $(F(a_{1}),F(a_{2}),...)$ con $F(\vec{a})$. Por Lema 117, $t=f(t_{1}, \dotsc, t_{n})$, con $n\geq 1 $,$\;f\in \mathcal{F}_{n}$ y $t_{1}, \dotsc, t_{n}\in T_{k}^{\tau }$. Tenemos entonces
  %
  %   $\displaystyle \begin{array}{ccl} F(t^{\mathbf{A}}[\vec{a}]) & = & F(f(t_{1}, \dotsc, t_{n})^{\mathbf{A}}[\vec{a}]) \\ & = & F(f^{\mathbf{A}}(t_{1}^{\mathbf{A}}[\vec{a}], \dotsc, t_{n}^{\mathbf{A}}[ \vec{a}])) \\ & = & f^{\mathbf{B}}(F(t_{1}^{\mathbf{A}}[\vec{a}]), \dotsc, F(t_{n}^{\mathbf{A}}[ \vec{a}])) \\ & = & f^{\mathbf{B}}(t_{1}^{\mathbf{B}}[F(\vec{a})], \dotsc, t_{n}^{\mathbf{B}}[F( \vec{a})])) \\ & = & f(t_{1}, \dotsc, t_{n})^{\mathbf{B}}[F(\vec{a})] \\ & = & t^{\mathbf{B}}[F(\vec{a})] \end{array} $
  %
  %   $\Box$
  % \end{proof}
  %
  % % Lemma 137. Sin prueba. Lemma 49.
  % \begin{lemma}
  %   \PN Supongamos que $F: \mathbf{A} \rightarrow \mathbf{B}$ es un isomorfismo. Sea $\varphi \in F^{\tau}$, entonces:
  %   \[
  %     \mathbf{A} \models \varphi \lbrack (a_{1}, a_{2}, \dotsc)] \Leftrightarrow \mathbf{B} \models \varphi \lbrack
  %     (F(a_{1}), F(a_{2}), \dotsc)]
  %   \]
  %   \PN para cada $(a_{1}, a_{2}, \dotsc) \in A^{\mathbf{N}}$. En particular $\mathbf{A}$ y $\mathbf{B}$ satisfacen las
  %   mismas sentencias de tipo $\tau$.
  % \end{lemma}
  %
  % % Lemma 138. Con prueba. Lemma 50.
  % \begin{lemma}
  %   \PN Si $F: \mathbf{A} \rightarrow \mathbf{B}$ es un homomorfismo biyectivo, entonces $F$ es un isomorfismo.
  % \end{lemma}
  % \begin{proof}
  %   Solo falta probar que $F^{-1}$ es un homomorfismo. Supongamos que $c\in \mathcal{C}$. Ya que $F(c^{\mathbf{A}})=c^{\mathbf{B}}$, tenemos que $ F^{-1}(c^{\mathbf{B}})=c^{\mathbf{A}}$, por lo cual $F^{-1}$ cumple (1) de la definicion de homomorfismo. Supongamos ahora que $f\in \mathcal{F}_{n}$ y sean $b_{1}, \dotsc, b_{n}\in B$. Sean $a_{1}, \dotsc, a_{n}\in A$ tales que $ F(a_{i})=b_{i}$, $i=1, \dotsc, n$. Tenemos que
  %
  %   $\displaystyle \begin{array}{ccl} F^{-1}(f^{\mathbf{B}}(b_{1}, \dotsc, b_{n})) & = & F^{-1}(f^{\mathbf{B} }(F(a_{1}), \dotsc, F(a_{n}))) \\ & = & F^{-1}(F(f^{\mathbf{A}}(a_{1}, \dotsc, a_{n})) \\ & = & f^{\mathbf{A}}(a_{1}, \dotsc, a_{n}) \\ & = & f^{\mathbf{A}}(F^{-1}(b_{1}), \dotsc, F^{-1}(b_{n})) \end{array} $
  %
  %   por lo cual $F^{-1}$ satisface (2) de la definicion de homomorfismo $\Box$
  % \end{proof}
  %
  % % Lemma 139. Con prueba. Lemma 51.
  % \begin{lemma}
  %   \PN Si $F: \mathbf{A} \rightarrow \mathbf{B}$ es un homomorfismo, entonces $I_{F}$ es un subuniverso de
  %   $\mathbf{B}$.
  % \end{lemma}
  % \begin{proof}
  %   Ya que $A\neq \varnothing ,$ tenemos que $I_{F}\neq \varnothing .$ Es claro que $ c^{\mathbf{B}}=F(c^{\mathbf{A}})\in I_{F},$ para cada $c\in \mathcal{C}$. Sea $f\in \mathcal{F}_{n}$ y sean $b_{1}, \dotsc, b_{n}\in I_{F}$ Sean $ a_{1}, \dotsc, a_{n}$ tales que $F(a_{i})=b_{i},$ $i=1, \dotsc, n$. Tenemos que
  %
  %   $\displaystyle f^{\mathbf{B}}(b_{1}, \dotsc, b_{n})=f^{\mathbf{B}}(F(a_{1}), \dotsc, F(a_{n}))=F(f^{ \mathbf{A}}(a_{1}, \dotsc, a_{n}))\in I_{F} $
  %
  %   por lo cual $I_{F}$ es cerrada bajo $f^{\mathbf{B}}$.
  % \end{proof}
  %
  % % Lemma 140. Con prueba. Lemma 52.
  % \begin{lemma}
  %   \PN Si $F: \mathbf{A} \rightarrow \mathbf{B}$ es un homomorfismo, entonces $\ker F$ es una congruencia sobre
  %   $\mathbf{A}$.
  % \end{lemma}
  % \begin{proof}
  %   Sea $f\in \mathcal{F}_{n}$. Supongamos que $a_{1}, \dotsc, a_{n},b_{1}, \dotsc, b_{n} \in A$ son tales que $a_{1}\ker Fb_{1}, \dotsc, a_{n}\ker Fb_{n}$. Tenemos entonces que
  %
  %   $\displaystyle \begin{array}{ccl} F(f^{\mathbf{A}}(a_{1}, \dotsc, a_{n})) & = & f^{\mathbf{B} }(F(a_{1}), \dotsc, F(a_{n})) \\ & = & f^{\mathbf{B}}(F(b_{1}), \dotsc, F(b_{n})) \\ & = & F(f^{\mathbf{B}}(b_{1}, \dotsc, b_{n})) \end{array} $
  %
  %   lo cual nos dice que $f^{\mathbf{A}}(a_{1}, \dotsc, a_{n})\ker Ff^{\mathbf{B} }(b_{1}, \dotsc, b_{n})$ $\Box$
  % \end{proof}
  %
  % % Lemma 141. Con prueba. Lemma 53.
  % \begin{lemma}
  %   \PN $\pi_{\theta}: \mathbf{A} \rightarrow \mathbf{A}/\theta$ es un homomorfismo cuyo núcleo es $\theta$.
  % \end{lemma}
  % \begin{proof}
  %   Sea $c\in \mathcal{C}$. Tenemos que
  %
  %   $\displaystyle \pi _{\theta }(c^{\mathbf{A}})=c^{\mathbf{A}}/\theta =c^{\mathbf{A}/\theta } $
  %
  %   Sea $f\in \mathcal{F}_{n}$, con $n\geq 1$ y sean $a_{1}, \dotsc, a_{n}\in A$. Tenemos que
  %   $\displaystyle \begin{array}{ccl} \pi _{\theta }(f^{\mathbf{A}}(a_{1}, \dotsc, a_{n})) & = & f^{\mathbf{A} }(a_{1}, \dotsc, a_{n})/\theta \\ & = & f^{\mathbf{A}/\theta }(a_{1}/\theta , \dotsc, a_{n}/\theta ) \\ & = & f^{\mathbf{A}/\theta }(\pi _{\theta }(a_{1}), \dotsc, \pi _{\theta }(a_{n})) \end{array} $
  %
  %   con lo cual $\pi _{\theta }$ es un homomorfismo. Es trivial que $\ker \pi _{\theta }=\theta $ $\Box$
  % \end{proof}
  %
  % % Corollary 142. Con prueba. Lemma 54.
  % \begin{corollary}
  %   \PN Para cada $t \in \TAU, \vec{a} \in A^{\mathbf{N}}$, se tiene que $t^{\mathbf{A}/\theta}[(a_{1}/\theta,
  %   a_{2}/\theta, \dotsc)] = t^{\mathbf{A}}[(a_{1}, a_{2}, \dotsc)]/\theta$.
  % \end{corollary}
  % \begin{proof}
  %   Ya que $\pi _{\theta }$ es un homomorfismo, se puede aplicar el Lema 136. $\Box$
  % \end{proof}
  %
  % % Theorem 143. Con prueba. Lemma 55.
  % \begin{theorem}
  %   \PN Sea $F: \mathbf{A} \rightarrow \mathbf{B}$ un homomorfismo sobreyectivo, entonces:
  %   \begin{eqnarray*}
  %     A/\ker F &\rightarrow& B \\
  %     a/\ker F &\rightarrow& F(a)
  %   \end{eqnarray*}
  %   \PN define sin ambiguedad una función $\bar{F}$ la cual es un isomorfismo de $\mathbf{A}/\ker F$ en $\mathbf{B}$.
  % \end{theorem}
  % \begin{proof}
  %   Notese que la definicion de $\bar{F}$ es inambigua ya que si $a/\ker F=a^{\prime }/\ker F$, entonces $F(a)=F(a^{\prime }).$ Ya que $F$ es sobre, tenemos que $\bar{F}$ lo es. Supongamos que $\bar{F}(a/\ker F)=\bar{F} (a^{\prime }/\ker F).$ Claramente entonces tenemos que $F(a)=F(a^{\prime })$ , lo cual nos dice que $a/\ker F=a^{\prime }/\ker F$. Esto prueba que $\bar{F }$ es inyectiva. Para ver que $\bar{F}$ es un isomorfismo, por el Lema 138, basta con ver que $\bar{F}$ es un homomorfismo. Sea $c\in \mathcal{C}$. Tenemos que
  %
  %   $\displaystyle \bar{F}(c^{\mathbf{A}/\ker F})=\bar{F}(c^{\mathbf{A}}/\ker F)=F(c^{\mathbf{A} })=c^{\mathbf{B}} $
  %
  %   Sea $f\in \mathcal{F}_{n}$. Sean $a_{1}, \dotsc, a_{n}\in A$. Tenemos que
  %   $\displaystyle \begin{array}{ccl} \bar{F}(f^{\mathbf{A}/\ker F}(a_{1}/\ker F, \dotsc, a_{n}/\ker F)) & = & \bar{F} (f^{\mathbf{A}}(a_{1}, \dotsc, a_{n})/\ker F) \\ & = & F(f^{\mathbf{A}}(a_{1}, \dotsc, a_{n})) \\ & = & f^{\mathbf{B}}(F(a_{1}), \dotsc, F(a_{n})) \\ & = & f^{\mathbf{B}}(\bar{F}(a_{1}/\ker F), \dotsc, \bar{F}(a_{n}/\ker F)) \end{array} $
  %
  %   con lo cual $\bar{F}$ cunple (2) de la definicion de homomorfismo
  % \end{proof}
  %
  % % Lemma 144. Con prueba. Lemma 56.
  % \begin{lemma}
  %   \PN Los mapeos $\pi_{1}: A \times B \rightarrow A$ y $\pi_{2}: A \times B \rightarrow A$ son homomorfismos.
  % \end{lemma}
  % \begin{proof}
  %   Veamos que $\pi _{1}$ es un homomorfismo. Primero notese que si $c\in \mathcal{C}$, entonces
  %
  %   $\displaystyle \pi _{1}(c^{\mathbf{A}\times \mathbf{B}})=\pi _{1}((c^{\mathbf{A}},c^{ \mathbf{B}}))=c^{\mathbf{A}} $
  %
  %   Sea $f\in \mathcal{F}_{n}$, con $n\geq 1$ y sean $ (a_{1},b_{1}), \dotsc, (a_{n},b_{n})\in A\times B$. Tenemos que
  %   $\displaystyle \begin{array}{ccl} \pi _{1}(f^{\mathbf{A}\times \mathbf{B}}((a_{1},b_{1}), \dotsc, (a_{n},b_{n})) & = & \pi _{1}((f^{\mathbf{A}}(a_{1}, \dotsc, a_{n}),f^{\mathbf{B}}(b_{1}, \dotsc, b_{n})) \\ & = & f^{\mathbf{A}}(a_{1}, \dotsc, a_{n}) \\ & = & f^{\mathbf{A}}(\pi _{1}(a_{1},b_{1}), \dotsc, \pi _{1}(a_{n},b_{n})) \end{array} $
  %
  %   con lo cual hemos probado que $\pi _{1}$ cumple (2) de la definicion de homomorfismo
  % \end{proof}
  %
  % % Lemma 145. Con prueba. Lemma 57.
  % \begin{lemma}
  %   \PN Para cada $t \in \TAU, ((a_{1}, b_{1}), (a_{2}, b_{2}), \dotsc) \in (A \times B)^{\mathbf{N}}$, se tiene que
  %   $t^{\mathbf{A} \times \mathbf{B}}[((a_{1}, b_{1}), (a_{2}, b_{2}), \dotsc)] = (t^{\mathbf{A}}[(a_{1}, a_{2},
  %   \dotsc)], t^{\mathbf{B}}[(b_{1}, b_{2}, \dotsc)])$.
  % \end{lemma}
  %
  % % Lemma 146. Con prueba. Lemma 58.
  % \begin{lemma}
  %   \PN Sean $w_{1}, \dotsc, w_{k}$ variables, todas distintas. Sean $v_{1}, \dotsc, v_{n}$ variables, todas distintas.
  %   Supongamos $t =_{d} t(w_{1}, \dotsc, w_{k}), s_{1} =_{d} s_{1}(v_{1}, \dotsc, v_{n}), \dotsc, s_{k} =_{d} s_{k}
  %   (v_{1}, \dotsc, v_{n})$, entonces:
  %   \begin{enumerate}[(a)]
  %     \item $t(s_{1}, \dotsc, s_{k}) =_{d} t(s_{1}, \dotsc, s_{k})(v_{1}, \dotsc, v_{n})$
  %     \item Para cada estructura $\mathbf{A}$ y $a_{1}, \dotsc, a_{n} \in A$, se tiene que:
  %       \[
  %         t(s_{1}, \dotsc, s_{k})^{\mathbf{A}}[a_{1}, \dotsc, a_{n}] = t^{\mathbf{A}}[s_{1}^{\mathbf{A}}[a_{1}, \dotsc,
  %         a_{n}], \dotsc, s_{k}^{\mathbf{A}}[a_{1}, \dotsc, a_{n}]]
  %       \]
  %   \end{enumerate}
  % \end{lemma}
  % \begin{proof}
  %   Probaremos que valen (a) y (b), por induccion en el $l$ tal que $t\in T_{l}^{\tau }.$ El caso $l=0$ es dejado al lector. Supongamos entonces que valen (a) y (b) siempre que $t\in T_{l}^{\tau }$ y veamos que entonces valen (a) y (b) cuando $t\in T_{l+1}^{\tau }-T_{l}^{\tau }$. Hay $f\in \mathcal{F} _{m}$ y $t_{1}, \dotsc, t_{m}\in T_{l}^{\tau }$ tales que $ t_{1}=_{d}t_{1}(w_{1}, \dotsc, w_{k}), \dotsc, t_{m}=_{d}t_{m}(w_{1}, \dotsc, w_{k})$ y $ t=f(t_{1}, \dotsc, t_{m})$. Notese que por (a) de la HI tenemos que
  %
  %   $\displaystyle t_{i}(s_{1}, \dotsc, s_{k})=_{d}t_{i}(s_{1}, \dotsc, s_{k})(v_{1}, \dotsc, v_{n})\text{, } i=1, \dotsc, m $
  %
  %   lo cual ya que
  %   $\displaystyle t(s_{1}, \dotsc, s_{k})=f(t_{1}(s_{1}, \dotsc, s_{k}), \dotsc, t_{m}(s_{1}, \dotsc, s_{k})) $
  %
  %   nos dice que
  %   $\displaystyle t(s_{1}, \dotsc, s_{k})=_{d}t(s_{1}, \dotsc, s_{k})(v_{1}, \dotsc, v_{n}) $
  %
  %   obteniendo asi (a). Para probar (b) notemos que por (b) de la hipotesis inductiva
  %   $\displaystyle t_{j}(s_{1}, \dotsc, s_{k})^{\mathbf{A}}[\vec{a}]=t_{j}^{\mathbf{A}}[s_{1}^{ \mathbf{A}}[\vec{a}], \dotsc, s_{k}^{\mathbf{A}}[\vec{a}]],j=1, \dotsc, m $
  %
  %   lo cual nos dice que
  %   $\displaystyle \begin{array}{ccl} t(s_{1}, \dotsc, s_{k})^{\mathbf{A}}[\vec{a}] & = & f(t_{1}(s_{1}, \dotsc, s_{k}), \dotsc, t_{m}(s_{1}, \dotsc, s_{k}))^{\mathbf{A}}[\vec{a}] \\ & = & f^{\mathbf{A}}(t_{1}(s_{1}, \dotsc, s_{k})^{\mathbf{A}}[\vec{a} ], \dotsc, t_{m}(s_{1}, \dotsc, s_{k})^{\mathbf{A}}[\vec{a}]) \\ & = & f^{\mathbf{A}}(t_{1}^{\mathbf{A}}[s_{1}^{\mathbf{A}}[\vec{a} ], \dotsc, s_{k}^{\mathbf{A}}[\vec{a}]], \dotsc, t_{m}^{\mathbf{A}}[s_{1}^{\mathbf{A}}[ \vec{a}], \dotsc, s_{k}^{\mathbf{A}}[\vec{a}]]) \\ & = & t^{\mathbf{A}}[s_{1}^{\mathbf{A}}[\vec{a}], \dotsc, s_{k}^{\mathbf{A}}[\vec{ a}]] \end{array} $
  % \end{proof}
  %
  % % Lemma 147. Con prueba. Lemma 59.
  % \begin{lemma}
  %   \PN Si $Qv$ ocurre en $\varphi$ a partir de $i$, entonces hay una única fórmula $\psi$ tal que $Qv\psi$ ocurre en
  %   $\varphi$ a partir de $i$.
  % \end{lemma}
  % \begin{proof}
  %   Por induccion en el $k$ tal que $\varphi \in F^{\tau }$.
  % \end{proof}
  % 
  % % Lemma 148. Con prueba. Lemma 60.
  % \begin{lemma}
  %   \PN Sean $w_{1}, \dotsc, w_{k}$ variables, todas distintas. Sean $v_{1}, \dotsc, v_{n}$ variables, todas distintas.
  %   Supongamos $\varphi =_{d} \varphi(w_{1}, \dotsc, w_{k}), t_{1} =_{d} t_{1}(v_{1}, \dotsc, v_{n}), \dotsc, t_{k}
  %   =_{d}t_{k}(v_{1}, \dotsc, v_{n})$ son tales que cada $w_{j}$ es sustituible por $t_{j}$ en $\varphi$, entonces:
  %   \begin{enumerate}[(a)]
  %     \item $\varphi(t_{1}, \dotsc, t_{k}) =_{d} \varphi(t_{1}, \dotsc, t_{k})(v_{1}, \dotsc, v_{n})$
  %     \item Para cada estructura $\mathbf{A}$ y $\vec{a} \in A^{n}$ se tiene:
  %       \[
  %         \mathbf{A} \models \varphi(t_{1}, \dotsc, t_{k})[\vec{a}] \Leftrightarrow \mathbf{A} \models \varphi \lbrack
  %         t_{1}^{\mathbf{A}}[\vec{a}], \dotsc, t_{k}^{ \mathbf{A}}[\vec{a}]]
  %       \]
  %   \end{enumerate}
  % \end{lemma}
  % \begin{proof}
  %   Probaremos que se dan (a) y (b), por induccion en el $l$ tal que $\varphi \in F_{l}^{\tau }.$ El caso $l=0$ es una consecuencia directa del Lema 146. Supongamos (a) y (b) valen para cada $\varphi \in F_{l}^{\tau } $ y sea $\varphi \in F_{l+1}^{\tau }-F_{l}^{\tau }.$ Notese que se puede suponer que cada $v_{i}$ ocurre en algun $t_{i},$ y que cada $w_{i}\in Li(\varphi )$, ya que para cada $\varphi ,$ el caso general se desprende del caso con estas restricciones. Hay varios casos
  %
  %   CASO $\varphi =\forall w\varphi_{1},$ con $w\not\in \{w_{1}, \dotsc, w_{k}\}$ y $ \varphi_{1}=_{d}\varphi_{1}(w_{1}, \dotsc, w_{k},w)$
  %
  %   Notese que cada $w_{j}\in Li(\varphi_{1})$. Ademas notese que $ w\not\in \{v_{1}, \dotsc, v_{n}\}$ ya que de lo contrario $w$ ocurriria en algun $ t_{j}$, y entonces $w_{j}$ no seria sustituible por $t_{j}$ en $\varphi $. Sean
  %
  %   $\displaystyle \begin{array}{ccc} \tilde{t}_{1} & = & t_{1} \\ & \vdots & \\ \tilde{t}_{k} & = & t_{k} \\ \tilde{t}_{k+1} & = & w \end{array} $
  %
  %   Notese que
  %   $\displaystyle \tilde{t}_{j}=_{d}\tilde{t}_{j}(v_{1}, \dotsc, v_{n},w) $
  %
  %   Por (a) de la hipotesis inductiva tenemos que
  %   $\displaystyle Li(\varphi_{1}(t_{1}, \dotsc, t_{k},w))=Li(\varphi_{1}(\tilde{t}_{1}, \dotsc, \tilde{ t}_{k},\tilde{t}_{k+1}))\subseteq \{v_{1}, \dotsc, v_{n},w\} $
  %
  %   y por lo tanto
  %   $\displaystyle Li(\varphi(t_{1}, \dotsc, t_{k}))\subseteq \{v_{1}, \dotsc, v_{n}\} $
  %
  %   lo cual prueba (a). Finalmente notese que
  %   $\displaystyle \begin{array}{c} \mathbf{A} \models \varphi(t_{1}, \dotsc, t_{k})\mathbf{[}\vec{a}] \\ \Updownarrow \\ \mathbf{A} \models \varphi_{1}(\tilde{t}_{1}, \dotsc, \tilde{t}_{k},\tilde{t} _{k+1})[\vec{a},a],\text{ para todo }a\in A \\ \Updownarrow \\ \mathbf{A} \models \varphi_{1}[\tilde{t}_{1}^{\mathbf{A}}[\vec{a},a], \dotsc,  \tilde{t}_{k}^{\mathbf{A}}[\vec{a},a],\tilde{t}_{k+1}^{\mathbf{A}}[\vec{a} ,a]],\text{ para todo }a\in A \\ \Updownarrow \\ \mathbf{A} \models \varphi_{1}[t_{1}^{\mathbf{A}}[\vec{a}], \dotsc, t_{k}^{ \mathbf{A}}[\vec{a}],a],\text{ para todo }a\in A \\ \Updownarrow \\ \mathbf{A} \models \varphi \lbrack t_{1}^{\mathbf{A}}[\vec{a}], \dotsc, t_{k}^{ \mathbf{A}}[\vec{a}]] \end{array} $
  %
  %   lo cual pueba (b). El caso del cuantificador $\exists $ es analogo y los casos de nexos logicos son directos.
  % \end{proof}
