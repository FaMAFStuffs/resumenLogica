\section{Estructuras}

  % Lemma 133. Con prueba. Lemma 43.
  \begin{lemma} \label{lemma_43}
    \PN Sea $\mathbf{A}$ una estructura de tipo $\tau$ y sea $t\in \TAU$. Supongamos que $\vec{a}, \vec{b}$ son
    asignaciones tales que $a_{i} = b_{i}$ cada vez que $x_{i}$ ocurra en $t$, entonces $t^{\mathbf{A}}[\vec{a}] =
    t^{\mathbf{A}}[\vec{b}]$.
  \end{lemma}
  \begin{proof}
    Sea

    - Teo$_{k}$: El lema vale para $t\in T_{k}^{\tau }$.
    Teo$_{0}$ es facil de probar. Veamos Teo$_{k}\Rightarrow $Teo$_{k+1}$. Supongamos $t\in T_{k+1}^{\tau }-T_{k}^{\tau }$
    y sean $\vec{a},\vec{b}$ asignaciones tales que $a_{i}=b_{i}$ cada vez que $x_{i}$ ocurra en $t$. Notese que
    $t=f(t_{1}, \dotsc, t_{n})$, con $f\in \mathcal{F}_{n},\;n\geq 1 \ \text{y} \  t_{1}, \dotsc, t_{n}\in \TAU$. Notese que para
    cada $j=1, \dotsc, n$, tenemos que $a_{i}=b_{i}$ cada vez que $x_{i}$ ocurra en $t_{j}$, lo cual por Teo$_{k}$ nos
    dice que

    $t_{j}^{\mathbf{A}}[\vec{a}]=t_{j}^{\mathbf{A}}[\vec{b}]\text{, }j=1, \dotsc, n $

    Se tiene entonces que
    \[
      \begin{array}{ccl}
        t^{\mathbf{A}}[\vec{a}] &=& i(f)(t_{1}^{\mathbf{A}}[\vec{a}], \dotsc, t_{n}^{ \mathbf{A}}[\vec{a}])\text{ (por def de }t^{\mathbf{A}}[\vec{a}]\text{)} \\
        &=& i(f)(t_{1}^{\mathbf{A}}[\vec{b}], \dotsc, t_{n}^{\mathbf{A}}[\vec{b}]) \\
        &=& t^{\mathbf{A}}[\vec{b}]\text{ (por def de }t^{\mathbf{A}}[\vec{a}] \text{)}
      \end{array}
    \]
  \end{proof}

  % Lemma 134. Con prueba. Lemma 44.
  \begin{lemma} \label{lemma_44}
    \PN \newline
    \begin{enumerate}[(a)]
      \item $Li((t \equiv s)) = \{v \in Var: v$ ocurre en $t$ ó $v$ ocurre en $s\}$
      \item $Li(r(t_{1}, \dotsc, ,t_{n})) = \{v \in Var: v$ ocurre en algún $t_{i}\}$
      \item $Li(\lnot \varphi) = Li(\varphi)$
      \item $Li((\varphi \eta \psi)) = Li(\varphi) \cup Li(\psi)$
      \item $Li(Qx_{j}\varphi) = Li(\varphi)-\{x_{j}\}$
    \end{enumerate}
  \end{lemma}
  \begin{proof}
    \begin{enumerate}[(a)]
      \item son triviales de las definiciones, teniendo en cuenta que si una variable $v$ ocurre en $(t\equiv s)$ (resp. en $r(t_{1}, \dotsc, t_{n})$) entonces $v$ ocurre en $t \ \text{o} \ v$ ocurre en $s$ (resp.$v$ ocurre en algun $ t_{i}$)
      \item son triviales de las definiciones, teniendo en cuenta que si una variable $v$ ocurre en $(t\equiv s)$ (resp. en $r(t_{1}, \dotsc, t_{n})$) entonces $v$ ocurre en $t \ \text{o} \ v$ ocurre en $s$ (resp.$v$ ocurre en algun $ t_{i}$)
      \item es similar a (d)
      \item Supongamos $v\in Li((\varphi \eta \psi ))$, entonces hay un $i$ tal que $ v$ ocurre libremente en $(\varphi \eta \psi )$ a partir de $i$. Por definicion tenemos que ya sea $v$ ocurre libremente en $\varphi $ a partir de $i-1 \ \text{o} \ v$ ocurre libremente en $\psi $ a partir de $i-\left\vert (\varphi \eta \right\vert $, con lo cual $v\in Li(\varphi )\cup Li(\psi )$
      Supongamos ahora que $v\in Li(\varphi )\cup Li(\psi )$. S.p.d.g. supongamos $ v\in Li(\psi )$. Por definicion tenemos que hay un $i$ tal que $v$ ocurre libremente en $\psi $ a partir de $i$. Pero notese que esto nos dice por definicion que $v$ ocurre libremente en $(\varphi \eta \psi )$ a partir de $ i+\left\vert (\varphi \eta \right\vert $ con lo cual $v\in Li((\varphi \eta \psi ))$.
      \item Supongamos $v\in Li(Qx_{j}\varphi )$, entonces hay un $i$ tal que $v$ ocurre libremente en $Qx_{j}\varphi $ a partir de $i$. Por definicion tenemos que $v\neq x_{j} \ \text{y} \ v$ ocurre libremente en $\varphi $ a partir de $ i-\left\vert Qx_{j}\right\vert $, con lo cual $v\in Li(\varphi )-\{x_{j}\}$
      Supongamos ahora que $v\in Li(\varphi )-\{x_{j}\}$. Por definicion tenemos que hay un $i$ tal que $v$ ocurre libremente en $\varphi $ a partir de $i$. Ya que $v\neq x_{j}$ esto nos dice por definicion que $v$ ocurre libremente en $Qx_{j}\varphi $ a partir de $i+\left\vert Qx_{j}\right\vert $, con lo cual $v\in Li(Qx_{j}\varphi )$.
    \end{enumerate}
  \end{proof}

  % Lemma 135. Con prueba. Lemma 45.
  \begin{lemma} \label{lemma_45}
    \PN Supongamos que $\vec{a}, \vec{b}$ son asignaciones tales que si $x_{i} \in Li(\varphi)$, entonces $a_{i} =
    b_{i}$, entonces $\mathbf{A} \models \varphi \lbrack \vec{a}] \Leftrightarrow \mathbf{A} \models \varphi \lbrack
    \vec{b}]$.
  \end{lemma}
  \begin{proof}
    Probaremos por induccion en $k$ que el lema vale para cada $\varphi \in F_{k}^{\tau }.$ El caso $k=0$ se desprende del Lema 131. Veamos que Teo$_{k}$ implica Teo$_{k+1}.$ Sea $\varphi \in F_{k+1}^{\tau }-F_{k}^{\tau }.$ Hay varios casos:

    CASO $\varphi = (\varphi_{1}\wedge \varphi_{2})$.

    Ya que $Li(\varphi_{i})\subseteq Li(\varphi )$, $i=1,2$, Teo$_{k}$ nos dice que $\mathbf{A} \models \varphi_{i}[\vec{a}]$ sii $\mathbf{A} \models \varphi_{i}[\vec{b}]$, para $i=1,2$. Se tiene entonces que

    $\displaystyle \begin{array}{l} \mathbf{A} \models \varphi \lbrack \vec{a}] \\ \ \ \Updownarrow \text{ (por (3) en la def de }\mathbf{A} \models \varphi \lbrack \vec{a}]\text{)} \\ \mathbf{A} \models \varphi_{1}[\vec{a}]\text{ y }\mathbf{A} \models \varphi_{2}[\vec{a}] \\ \ \ \Updownarrow \text{ (por Teo}_{k}\text{)} \\ \mathbf{A} \models \varphi_{1}[\vec{b}]\text{ y }\mathbf{A} \models \varphi_{2}[\vec{b}] \\ \ \ \Updownarrow \text{(por (3) en la def de }\mathbf{A} \models \varphi \lbrack \vec{a}]\text{)} \\ \mathbf{A} \models \varphi \lbrack \vec{b}] \end{array} $

    CASO $\varphi = (\varphi_{1}\wedge \varphi_{2})$.

    Es completamente similar al anterior.

    CASO $\varphi =\lnot \varphi_{1}.$

    Es completamente similar al anterior.

    CASO $\varphi =\forall x_{j}\varphi_{1}.$

    Supongamos $\mathbf{A} \models \varphi \lbrack \vec{a}]$. Entonces por (8) en la def de $\mathbf{A} \models \varphi \lbrack \vec{a}]$ se tiene que $\mathbf{A} \models \varphi_{1}[\downarrow _{j}^{a}(\vec{a})]$, para todo $a\in A$. Notese que $\downarrow _{j}^{a}(\vec{a}) \ \text{y} \ \downarrow _{j}^{a}(\vec{b})$ coinciden en toda $x_{i}$ de $x_{i}\in Li(\varphi_{1})\subseteq Li(\varphi_{1})\cup \{x_{j}\}$, con lo cual por Teo$_{k}$ se tiene que $\mathbf{A} \models \varphi_{1}[\downarrow _{j}^{a}(\vec{b})]$, para todo $a\in A$, lo cual por (8) en la def de $\mathbf{A} \models \varphi \lbrack \vec{a}]$ nos dice que $\mathbf{A} \models \varphi \lbrack \vec{b}]$. La prueba de que $\mathbf{A} \models \varphi \lbrack \vec{b}]$ implica que $ \mathbf{A} \models \varphi \lbrack \vec{a}]$ es similar.

    CASO $\varphi =\exists x_{j}\varphi_{1}$.

    Es similar al anterior.
  \end{proof}

  % Corollary 136. Sin prueba. Lemma 46.
  \begin{corollary} \label{corollary_46}
    \PN Si $\varphi$ es una sentencia, entonces $\mathbf{A} \models \varphi \lbrack \vec{a}] \Leftrightarrow \mathbf{A}
    \models \varphi \lbrack \vec{b}]$, cualesquiera sean las asignaciones $\vec{a}, \vec{b}$.
  \end{corollary}

  % Lemma 137. Con prueba. Lemma 47.
  \begin{lemma} \label{lemma_47}
    \PN \newline
    \begin{enumerate}[(a)]
      \item Si $Li(\varphi) \cup Li(\psi) \subseteq \{x_{i_{1}}, \dotsc, x_{i_{n}}\}$, entonces $\varphi \thicksim \psi$
      si y solo si la sentencia $\forall x_{i_{1}} \dotsc \forall x_{i_{n}}(\varphi \leftrightarrow \psi)$ es
      universalmente válida.
      \item Si $\varphi_{i} \thicksim \psi_{i}, i = 1, 2$, entonces $\lnot \varphi_{1} \thicksim \lnot \psi_{1},
      (\varphi_{1} \eta \varphi_{2}) \thicksim (\psi_{1} \eta \psi_{2}) \ \text{y} \ Qv\varphi_{1} \thicksim Qv \psi_{1}$.
      \item Si $\varphi \thicksim \psi \ \text{y} \ \alpha^{\prime}$ es el resultado de reemplazar en una fórmula $\alpha$
      algunas (posiblemente $0$) ocurrencias de $\varphi$ por $\psi$, entonces $\alpha \thicksim \alpha^{\prime}$.
    \end{enumerate}
  \end{lemma}
  \begin{proof}
    \begin{enumerate}[(a)]
      \item Tenemos que
        \[
          \begin{array}{l}
            \varphi \thicksim \psi \\
            \ \ \Updownarrow \text{ (por (6) de la def de}\models \text{)} \\
            \mathbf{A} \models (\varphi \leftrightarrow \psi )[\vec{a}]\text{, para todo } \mathbf{A}\text{ y toda }\vec{a}\in A^{\mathbb{N}} \\
            \ \ \Updownarrow \\
            \mathbf{A} \models (\varphi \leftrightarrow \psi )[\downarrow _{i_{n}}^{a}(\vec{a })]\text{, para todo }\mathbf{A}\text{, }a\in A\text{ y toda }\vec{a}\in A^{ \mathbb{N}} \\ \ \ \Updownarrow (\text{por (8) de la def de}\models ) \\ \mathbf{A} \models \forall x_{i_{n}}(\varphi \leftrightarrow \psi )[\vec{a}] \text{, para todo }\mathbf{A}\text{ y toda }\vec{a}\in A^{\mathbb{N}} \\ \ \ \Updownarrow \\ \mathbf{A} \models \forall x_{i_{n}}(\varphi \leftrightarrow \psi )[\downarrow _{i_{n-1}}^{a}(\vec{a})]\text{, para todo }\mathbf{A}\text{, }a\in A\text{ y toda }\vec{a}\in A^{\mathbb{N}} \\ \ \ \Updownarrow \text{ (por (8) de la def de}\models \text{)} \\ \mathbf{A} \models \forall x_{i_{n-1}}\forall x_{i_{n}}(\varphi \leftrightarrow \psi )[\vec{a}]\text{, para todo }\mathbf{A}\text{ y toda }\vec{a}\in A^{ \mathbb{N}} \\ \ \ \Updownarrow \\ \ \ \ \ \vdots \\ \ \ \Updownarrow \\ \mathbf{A} \models \forall x_{i_{1}}...\forall x_{i_{n}}(\varphi \leftrightarrow \psi )[\vec{a}]\text{, para todo }\mathbf{A}\text{ y toda }\vec{a}\in A^{ \mathbb{N}} \\ \ \ \Updownarrow \\ \forall x_{i_{1}}...\forall x_{i_{n}}(\varphi \leftrightarrow \psi )\text{ es universalmente valida}
          \end{array}
        \]
      \item Es dejado al lector.
      \item Por induccion en el $k$ tal que $\alpha \in F_{k}^{\tau }$.
    \end{enumerate}
  \end{proof}

  % Lemma 138. Con prueba. Lemma 48.
  \begin{lemma} \label{lemma_48}
    \PN Sea $F: \mathbf{A} \rightarrow \mathbf{B}$ un homomorfismo, entonces:
    \[
      F(t^{\mathbf{A}}[(a_{1}, a_{2}, \dotsc)] = t^{\mathbf{B}}[F(a_{1}), F(a_{2}), \dotsc)]
    \]
    \PN para cada $t \in \TAU, (a_{1}, a_{2}, \dotsc) \in A^{\mathbb{N}}$.
  \end{lemma}
  \begin{proof}
    Sea

    - Teo$_{k}$: Si $F:\mathbf{A} \rightarrow \mathbf{B}$ es un homomorfismo, entonces
    $\displaystyle F(t^{\mathbf{A}}[(a_{1},a_{2},...)]=t^{\mathbf{B}}[F(a_{1}),F(a_{2}),...)] $

    para cada $t\in T_{k}^{\tau }$, $(a_{1},a_{2},...)\in A^{\mathbb{N}}$.
    Teo$_{0}$ es trivial. Veamos que Teo$_{k}$ implica Teo$_{k+1}$. Supongamos que vale Teo$_{k}$ y supongamos $F:\mathbf{A} \rightarrow \mathbf{B}$ es un homomorfismo, $t\in T_{k+1}^{\tau }-T_{k}^{\tau } \ \text{y} \ \vec{a} = (a_{1},a_{2},...)\in A^{\mathbb{N}}$. Denotemos $(F(a_{1}),F(a_{2}),...)$ con $F(\vec{a})$. Por Lema 117, $t=f(t_{1}, \dotsc, t_{n})$, con $n\geq 1 $$\;f\in \mathcal{F}_{n} \ \text{y} \ t_{1}, \dotsc, t_{n}\in T_{k}^{\tau }$. Tenemos entonces

    \[
      \begin{array}{ccl}
        F(t^{\mathbf{A}}[\vec{a}]) &=& F(f(t_{1}, \dotsc, t_{n})^{\mathbf{A}}[\vec{a}]) \\
        &=& F(f^{\mathbf{A}}(t_{1}^{\mathbf{A}}[\vec{a}], \dotsc, t_{n}^{\mathbf{A}}[ \vec{a}])) \\
        &=& f^{\mathbf{B}}(F(t_{1}^{\mathbf{A}}[\vec{a}]), \dotsc, F(t_{n}^{\mathbf{A}}[ \vec{a}])) \\
        &=& f^{\mathbf{B}}(t_{1}^{\mathbf{B}}[F(\vec{a})], \dotsc, t_{n}^{\mathbf{B}}[F( \vec{a})])) \\
        &=& f(t_{1}, \dotsc, t_{n})^{\mathbf{B}}[F(\vec{a})] \\
        &=& t^{\mathbf{B}}[F(\vec{a})]
      \end{array}
    \]
  \end{proof}

  % Lemma 139. Sin prueba. Lemma 49.
  \begin{lemma} \label{lemma_49}
    \PN Supongamos que $F: \mathbf{A} \rightarrow \mathbf{B}$ es un isomorfismo. Sea $\varphi \in F^{\tau}$, entonces:
    \[
      \mathbf{A} \models \varphi \lbrack (a_{1}, a_{2}, \dotsc)] \Leftrightarrow \mathbf{B} \models \varphi \lbrack
      (F(a_{1}), F(a_{2}), \dotsc)]
    \]
    \PN para cada $(a_{1}, a_{2}, \dotsc) \in A^{\mathbb{N}}$. En particular $\mathbf{A} \ \text{y} \ \mathbf{B}$ satisfacen las
    mismas sentencias de tipo $\tau$.
  \end{lemma}

  % Lemma 140. Con prueba. Lemma 50.
  \begin{lemma} \label{lemma_50}
    \PN Supongamos que $\tau$ es algebraico, si $F: \mathbf{A} \rightarrow \mathbf{B}$ es un homomorfismo biyectivo,
    entonces $F$ es un isomorfismo.
  \end{lemma}
  \begin{proof}
    Solo falta probar que $F^{-1}$ es un homomorfismo. Supongamos que $c\in \mathcal{C}$. Ya que $F(c^{\mathbf{A}})=c^{\mathbf{B}}$, tenemos que $ F^{-1}(c^{\mathbf{B}})=c^{\mathbf{A}}$, por lo cual $F^{-1}$ cumple (1) de la definicion de homomorfismo. Supongamos ahora que $f\in \mathcal{F}_{n}$ y sean $b_{1}, \dotsc, b_{n}\in B$. Sean $a_{1}, \dotsc, a_{n}\in A$ tales que $ F(a_{i})=b_{i}$, $i=1, \dotsc, n$. Tenemos que

    $\displaystyle \begin{array}{ccl} F^{-1}(f^{\mathbf{B}}(b_{1}, \dotsc, b_{n})) & = & F^{-1}(f^{\mathbf{B} }(F(a_{1}), \dotsc, F(a_{n}))) \\ & = & F^{-1}(F(f^{\mathbf{A}}(a_{1}, \dotsc, a_{n})) \\ & = & f^{\mathbf{A}}(a_{1}, \dotsc, a_{n}) \\ & = & f^{\mathbf{A}}(F^{-1}(b_{1}), \dotsc, F^{-1}(b_{n})) \end{array} $

    por lo cual $F^{-1}$ satisface (2) de la definicion de homomorfismo
  \end{proof}

  % Lemma 141. Con prueba. Lemma 51.
  \begin{lemma} \label{lemma_51}
    \PN Supongamos que $\tau$ es algebraico, si $F: \mathbf{A} \rightarrow \mathbf{B}$ es un homomorfismo, entonces
    $I_{F}$ es un subuniverso de $\mathbf{B}$.
  \end{lemma}
  \begin{proof}
    Ya que $A\neq \varnothing $ tenemos que $I_{F}\neq \varnothing .$ Es claro que $ c^{\mathbf{B}}=F(c^{\mathbf{A}})\in I_{F}$ para cada $c\in \mathcal{C}$. Sea $f\in \mathcal{F}_{n}$ y sean $b_{1}, \dotsc, b_{n}\in I_{F}$ Sean $ a_{1}, \dotsc, a_{n}$ tales que $F(a_{i})=b_{i}$ $i=1, \dotsc, n$. Tenemos que

    $\displaystyle f^{\mathbf{B}}(b_{1}, \dotsc, b_{n})=f^{\mathbf{B}}(F(a_{1}), \dotsc, F(a_{n}))=F(f^{ \mathbf{A}}(a_{1}, \dotsc, a_{n}))\in I_{F} $

    por lo cual $I_{F}$ es cerrada bajo $f^{\mathbf{B}}$.
  \end{proof}

  % Lemma 142. Con prueba. Lemma 52.
  \begin{lemma} \label{lemma_52}
    \PN Supongamos que $\tau$ es algebraico, si $F: \mathbf{A} \rightarrow \mathbf{B}$ es un homomorfismo, entonces
    $\ker F$ es una congruencia sobre $\mathbf{A}$.
  \end{lemma}
  \begin{proof}
    Sea $f\in \mathcal{F}_{n}$. Supongamos que $a_{1}, \dotsc, a_{n},b_{1}, \dotsc, b_{n} \in A$ son tales que $a_{1}\ker Fb_{1}, \dotsc, a_{n}\ker Fb_{n}$. Tenemos entonces que

    $\displaystyle \begin{array}{ccl} F(f^{\mathbf{A}}(a_{1}, \dotsc, a_{n})) & = & f^{\mathbf{B} }(F(a_{1}), \dotsc, F(a_{n})) \\ & = & f^{\mathbf{B}}(F(b_{1}), \dotsc, F(b_{n})) \\ & = & F(f^{\mathbf{B}}(b_{1}, \dotsc, b_{n})) \end{array} $

    lo cual nos dice que $f^{\mathbf{A}}(a_{1}, \dotsc, a_{n})\ker Ff^{\mathbf{B} }(b_{1}, \dotsc, b_{n})$
  \end{proof}

  % Lemma 143. Con prueba. Lemma 53.
  \begin{lemma} \label{lemma_53}
    \PN $\pi_{\theta}: \mathbf{A} \rightarrow \mathbf{A}/\theta$ es un homomorfismo cuyo núcleo es $\theta$.
  \end{lemma}
  \begin{proof}
    Sea $c\in \mathcal{C}$. Tenemos que

    \[
      \pi_{\theta}(c^{\mathbf{A}}) = c^{\mathbf{A}}/\theta = c^{\mathbf{A}/\theta}
    \]

    Sea $f\in \mathcal{F}_{n}$, con $n\geq 1$ y sean $a_{1}, \dotsc, a_{n}\in A$. Tenemos que

    \[
      \begin{array}{ccl}
        \pi_{\theta}(f^{\mathbf{A}}(a_{1}, \dotsc, a_{n})) &=& f^{\mathbf{A}}(a_{1}, \dotsc, a_{n})/\theta \\
        &=& f^{\mathbf{A}/\theta}(a_{1}/\theta, \dotsc, a_{n}/\theta) \\
        &=& f^{\mathbf{A}/\theta}(\pi_{\theta}(a_{1}), \dotsc, \pi_{\theta}(a_{n}))
      \end{array}
    \]
    con lo cual $\pi _{\theta }$ es un homomorfismo. Es trivial que $\ker \pi _{\theta }=\theta $.
  \end{proof}

  % Corollary 144. Con prueba. Lemma 54.
  \begin{corollary} \label{corollary_54}
    \PN Para cada $t \in \TAU, \vec{a} \in A^{\mathbb{N}}$, se tiene que $t^{\mathbf{A}/\theta}[(a_{1}/\theta,
    a_{2}/\theta, \dotsc)] = t^{\mathbf{A}}[(a_{1}, a_{2}, \dotsc)]/\theta$.
  \end{corollary}
  \begin{proof}
    Ya que $\pi _{\theta }$ es un homomorfismo, se puede aplicar el Lema 136.
  \end{proof}

  % Theorem 145. Con prueba. Lemma 55.
  \begin{theorem} \label{theorem_55}
    \PN Sea $F: \mathbf{A} \rightarrow \mathbf{B}$ un homomorfismo sobreyectivo, entonces:
    \begin{eqnarray*}
      A/\ker F &\rightarrow& B \\
      a/\ker F &\rightarrow& F(a)
    \end{eqnarray*}
    \PN define sin ambiguedad una función $\bar{F}$ la cual es un isomorfismo de $\mathbf{A}/\ker F$ en $\mathbf{B}$.
  \end{theorem}
  \begin{proof}
    Notese que la definicion de $\bar{F}$ es inambigua ya que si $a/\ker F=a^{\prime }/\ker F$, entonces $F(a)=F(a^{\prime }).$ Ya que $F$ es sobre, tenemos que $\bar{F}$ lo es. Supongamos que $\bar{F}(a/\ker F)=\bar{F} (a^{\prime }/\ker F).$ Claramente entonces tenemos que $F(a)=F(a^{\prime })$ , lo cual nos dice que $a/\ker F=a^{\prime }/\ker F$. Esto prueba que $\bar{F }$ es inyectiva. Para ver que $\bar{F}$ es un isomorfismo, por el Lema 138, basta con ver que $\bar{F}$ es un homomorfismo. Sea $c\in \mathcal{C}$. Tenemos que

    $\displaystyle \bar{F}(c^{\mathbf{A}/\ker F})=\bar{F}(c^{\mathbf{A}}/\ker F)=F(c^{\mathbf{A} })=c^{\mathbf{B}} $

    Sea $f\in \mathcal{F}_{n}$. Sean $a_{1}, \dotsc, a_{n}\in A$. Tenemos que
    \[
      \begin{array}{ccl}
        \bar{F}(f^{\mathbf{A}/\ker F}(a_{1}/\ker F, \dotsc, a_{n}/\ker F)) &=& \bar{F}(f^{\mathbf{A}}(a_{1}, \dotsc,
          a_{n})/\ker F) \\
        &=& F(f^{\mathbf{A}}(a_{1}, \dotsc, a_{n})) \\
        &=& f^{\mathbf{B}}(F(a_{1}), \dotsc, F(a_{n})) \\
        &=& f^{\mathbf{B}}(\bar{F}(a_{1}/\ker F), \dotsc, \bar{F}(a_{n}/\ker F))
      \end{array}
    \]

    con lo cual $\bar{F}$ cunple (2) de la definicion de homomorfismo.
  \end{proof}

  % Lemma 146. Con prueba. Lemma 56.
  \begin{lemma} \label{lemma_56}
    \PN Los mapeos $\pi_{1}: \mathbf{A} \times \mathbf{B} \rightarrow \mathbf{A} \ \text{y} \ \pi_{2}: \mathbf{A} \times
    \mathbf{B} \rightarrow \mathbf{A}$ son homomorfismos.
  \end{lemma}
  \begin{proof}
    Veamos que $\pi _{1}$ es un homomorfismo. Primero notese que si $c\in \mathcal{C}$, entonces

    $\displaystyle \pi _{1}(c^{\mathbf{A}\times \mathbf{B}})=\pi _{1}((c^{\mathbf{A}},c^{ \mathbf{B}}))=c^{\mathbf{A}} $

    Sea $f\in \mathcal{F}_{n}$, con $n\geq 1$ y sean $ (a_{1},b_{1}), \dotsc, (a_{n},b_{n})\in A\times B$. Tenemos que
    \[
      \begin{array}{ccl}
        \pi_{1}(f^{\mathbf{A} \times \mathbf{B}}((a_{1}, b_{1}), \dotsc, (a_{n}, b_{n})) &=& \pi_{1}((f^{\mathbf{A}}
          (a_{1}, \dotsc, a_{n}), f^{\mathbf{B}}(b_{1}, \dotsc, b_{n})) \\
        &=& f^{\mathbf{A}}(a_{1}, \dotsc, a_{n}) \\
        &=& f^{\mathbf{A}}(\pi_{1}(a_{1}, b_{1}), \dotsc, \pi_{1}(a_{n}, b_{n}))
      \end{array}
    \]

    con lo cual hemos probado que $\pi _{1}$ cumple (2) de la definicion de homomorfismo
  \end{proof}

  % Lemma 147. Con prueba. Lemma 57.
  \begin{lemma} \label{lemma_57}
    \PN Para cada $t \in \TAU, ((a_{1}, b_{1}), (a_{2}, b_{2}), \dotsc) \in (A \times B)^{\mathbb{N}}$, se tiene que
    $t^{\mathbf{A} \times \mathbf{B}}[((a_{1}, b_{1}), \linebreak (a_{2}, b_{2}), \dotsc)] = (t^{\mathbf{A}}[(a_{1},
    a_{2}, \dotsc)], t^{\mathbf{B}}[(b_{1}, b_{2}, \dotsc)])$.
  \end{lemma}

  % Lemma 148. Con prueba. Lemma 58.
  \begin{lemma} \label{lemma_58}
    \PN Sea $\tau$ un tipo cualquiera y supongamos $t \in \TAU$. Si $t =_{d} t(v_{1}, \dotsc, v_{n})$ entonces se da
    alguna de las siguientes:
    \begin{enumerate}
      \item $t = c$ para algún $c \in \mathcal{C}$.
      \item $t = v_{j}$ para algún j.
      \item $t = f(t_{1}, \dotsc, t_{m})$, con $f \in \mathcal{F}_{m} \ \text{y} \ t_{1}, \dotsc, t_{m} \in T_{k-1}^{\tau}$ tales
      que las variables que ocurren en cada uno de ellos están en $\{v_{1}, \dotsc, v_{n}\}$.
    \end{enumerate}
  \end{lemma}
  \begin{proof}
  \end{proof}

  % Lemma 149. Con prueba. Lemma 59.
  \begin{lemma} \label{lemma_59}
    \PN Sea $\tau$ un tipo cualquiera y supongamos $t \in \TAU$. Si $t =_{d} t(v_{1}, \dotsc, v_{n})$. Sea $\mathbf{A}$
    un modelo de tipo $\tau$. Sean $a_{1}, \dotsc, a_{n} \in A$. Se tiene que:
    \begin{enumerate}
      \item Si $t = c$ entonces $t^{\mathbf{A}}[a_{1}, \dotsc, a_{n}]= c^{\mathbf{A}}$.
      \item Si $t = v_{j}$ entonces $t^{\mathbf{A}}[a_{1}, \dotsc, a_{n}]= a_{j}$.
      \item Si $t = f(t_{1}, \dotsc, t_{m})$, con $f \in \mathcal{F}_{m} \ \text{y} \ t_{1}, \dotsc, t_{m} \in \TAU$, entonces:
      \[
        t^{\mathbf{A}}[a_{1}, \dotsc, a_{n}]= f^{\mathbf{A}}(t_{1}^{\mathbf{A}}[a_{1}, \dotsc, a_{n}], \dotsc,
        t_{m}^{\mathbf{A}}[a_{1}, \dotsc, a_{n}])
      \]
    \end{enumerate}
  \end{lemma}
  \begin{proof}
    \begin{enumerate}[(1)]
      \item trivial
      \item trivial
      \item Sea $\vec{b}$ una asignación tal que a cada $v_{i}$le asigna el valor $a_{i}$. Por definición tenemos EquivElim
      \begin{eqnarray*}
        t^{\mathbf{A}}[a_{1}, \dotsc, a_{n}] &=& t^{\mathbf{A}}[\vec{b}] \\
        &=& f^{\mathbf{A}}(t_{1}^{\mathbf{A}}[\vec{b}], \dotsc, t_{m}^{\mathbf{A}}[\vec{b}]) \\
        &=& f^{\mathbf{A}}(t_{1}^{\mathbf{A}}[a_{1}, \dotsc, a_{n}], \dotsc, t_{m}^{\mathbf{A}}[a_{1}, \dotsc, a_{n}])
      \end{eqnarray*}
    \end{enumerate}
  \end{proof}

  % Lemma 150. Con prueba. Lemma 60.
  \begin{lemma} \label{lemma_60}
    \PN \textbf{(De reemplazo para términos)}. Supongamos $t =_{d} t(w_{1}, \dotsc, w_{k}), s_{1} =_{d} s_{1}(v_{1},
    \dotsc, v_{n}), \dotsc, s_{k} =_{d} s_{k}(v_{1}, \dotsc, v_{n})$. Todas las variables de $t(s_{1}, \dotsc, s_{k})$
    están en $\{v_{1}, \dotsc, v_{n}\}$ y si declaramos $t(s_{1}, \dotsc, s_{k}) =_{d} t(s_{1}, \dotsc, s_{k})(v_{1},
    \dotsc, v_{n})$, entonces para cada estructura $\mathbf{A} \ \text{y} \ a_{1}, \dotsc a_{n} \in A$, se tiene:
    \[
      t(s_{1}, \dotsc, s_{k})^{\mathbf{A}}[a_{1}, \dotsc, a_{n}] = t^{\mathbf{A}}[s_{1}^{\mathbf{A}}[a_{1}, \dotsc,
      a_{n}], \dotsc, s_{k}^{\mathbf{A}}[a_{1}, \dotsc, a_{n}]]
    \]
  \end{lemma}
  \begin{proof}
    Probaremos que valen (a) y (b), por induccion en el $l$ tal que $t\in T_{l}^{\tau }.$ El caso $l=0$ es dejado al lector. Supongamos entonces que valen (a) y (b) siempre que $t\in T_{l}^{\tau }$ y veamos que entonces valen (a) y (b) cuando $t\in T_{l+1}^{\tau }-T_{l}^{\tau }$. Hay $f\in \mathcal{F} _{m} \ \text{y} \ t_{1}, \dotsc, t_{m}\in T_{l}^{\tau }$ tales que $ t_{1}=_{d}t_{1}(w_{1}, \dotsc, w_{k}), \dotsc, t_{m}=_{d}t_{m}(w_{1}, \dotsc, w_{k}) \ \text{y} \  t=f(t_{1}, \dotsc, t_{m})$. Notese que por (a) de la HI tenemos que

    $\displaystyle t_{i}(s_{1}, \dotsc, s_{k})=_{d}t_{i}(s_{1}, \dotsc, s_{k})(v_{1}, \dotsc, v_{n})\text{, } i=1, \dotsc, m $

    lo cual ya que
    $\displaystyle t(s_{1}, \dotsc, s_{k})=f(t_{1}(s_{1}, \dotsc, s_{k}), \dotsc, t_{m}(s_{1}, \dotsc, s_{k})) $

    nos dice que
    $\displaystyle t(s_{1}, \dotsc, s_{k})=_{d}t(s_{1}, \dotsc, s_{k})(v_{1}, \dotsc, v_{n}) $

    obteniendo asi (a). Para probar (b) notemos que por (b) de la hipotesis inductiva
    $\displaystyle t_{j}(s_{1}, \dotsc, s_{k})^{\mathbf{A}}[\vec{a}]=t_{j}^{\mathbf{A}}[s_{1}^{ \mathbf{A}}[\vec{a}], \dotsc, s_{k}^{\mathbf{A}}[\vec{a}]],j=1, \dotsc, m $

    lo cual nos dice que
    $\displaystyle \begin{array}{ccl} t(s_{1}, \dotsc, s_{k})^{\mathbf{A}}[\vec{a}] & = & f(t_{1}(s_{1}, \dotsc, s_{k}), \dotsc, t_{m}(s_{1}, \dotsc, s_{k}))^{\mathbf{A}}[\vec{a}] \\ & = & f^{\mathbf{A}}(t_{1}(s_{1}, \dotsc, s_{k})^{\mathbf{A}}[\vec{a} ], \dotsc, t_{m}(s_{1}, \dotsc, s_{k})^{\mathbf{A}}[\vec{a}]) \\ & = & f^{\mathbf{A}}(t_{1}^{\mathbf{A}}[s_{1}^{\mathbf{A}}[\vec{a} ], \dotsc, s_{k}^{\mathbf{A}}[\vec{a}]], \dotsc, t_{m}^{\mathbf{A}}[s_{1}^{\mathbf{A}}[ \vec{a}], \dotsc, s_{k}^{\mathbf{A}}[\vec{a}]]) \\ & = & t^{\mathbf{A}}[s_{1}^{\mathbf{A}}[\vec{a}], \dotsc, s_{k}^{\mathbf{A}}[\vec{ a}]] \end{array} $
  \end{proof}

  % Lemma 151. Con prueba. Lemma 61.
  \begin{lemma} \label{lemma_61}
    \PN Sea $\tau$ un tipo cualquiera y supongamos $\varphi \in F^{\tau}$. Si $\varphi =_{d} \varphi(v_{1}, \dotsc,
    v_{n})$, entonces se da una y solo una de las siguientes:
    \begin{enumerate}[(1)]
      \item $\varphi = (t \equiv s)$, con $t, s \in \TAU$, únicos y tales que las variables que ocurren en $t$ o en $s$
      están todas en $ \{v_{1}, \dotsc, v_{n}\}$.
      \item $\varphi = r(t_{1}, \dotsc, t_{m})$, con $r \in \mathcal{R}_{m} \ \text{y} \ t_{1}, \dotsc, t_{m} \in \TAU$, únicos y
      tales que las variables que ocurren en cada $t_{i}$ están todas en $\{v_{1}, \dotsc, v_{n}\}$.
      \item $\varphi = (\varphi_{1} \wedge \varphi_{2})$, con $\varphi_{1}, \varphi_{2} \in F^{\tau}$, únicas y tales
      que $Li(\varphi_{1}) \cup Li(\varphi_{2}) \subseteq \{v_{1}, \dotsc, v_{n}\}$.
      \item $\varphi = (\varphi_{1} \vee \varphi_{2})$, con $\varphi_{1}, \varphi_{2} \in F^{\tau}$, únicas y tales que
      $Li(\varphi_{1}) \cup Li(\varphi_{2}) \subseteq \{v_{1}, \dotsc, v_{n}\}$.
      \item $\varphi = (\varphi_{1} \rightarrow \varphi_{2})$, con $\varphi_{1}, \varphi_{2} \in F^{\tau}$, únicas y
      tales que $Li(\varphi_{1}) \cup Li(\varphi_{2}) \subseteq \{v_{1}, \dotsc, v_{n}\}$.
      \item $\varphi = (\varphi_{1} \leftrightarrow \varphi_{2})$, con $ \varphi_{1}, \varphi_{2} \in F^{\tau}$, únicas
      y tales que $Li(\varphi_{1}) \cup Li(\varphi_{2}) \subseteq \{v_{1}, \dotsc, v_{n}\}$.
      \item $\varphi = \lnot \varphi_{1}$, con $\varphi_{1} \in F^{\tau}$, única y tal que $Li(\varphi_{1}) \subseteq
      \{v_{1}, \dotsc, v_{n}\}$.
      \item $\varphi = \forall v_{j} \varphi_{1}$, con $v_{j} \in \{v_{1}, \dotsc, v_{n}\} \ \text{y} \ \varphi_{1} \in
      F^{\tau}$, únicas y tales que $ Li(\varphi_{1}) \subseteq \{v_{1}, \dotsc, v_{n}\}$.
      \item $\varphi = \forall v \varphi_{1}$, con $v \in Var-\{v_{1}, \dotsc, v_{n}\} \ \text{y} \ \varphi_{1}\ in F^{\tau}$,
      únicas y tales que $ Li(\varphi_{1}) \subseteq \{v_{1}, \dotsc, v_{n}, v\}$.
      \item $\varphi = \exists v_{j} \varphi_{1}$, con $v_{j} \in \{v_{1}, \dotsc, v_{n}\} \ \text{y} \ \varphi_{1} \in
      F^{\tau}$, únicas y tales que $ Li(\varphi_{1}) \subseteq \{v_{1}, \dotsc, v_{n}\}$.
      \item $\varphi = \exists v \varphi_{1}$, con $v \in Var-\{v_{1}, \dotsc, v_{n}\} \ \text{y} \ \varphi_{1} \in F^{\tau}$,
      únicas y tales que $ Li(\varphi_{1}) \subseteq \{v_{1}, \dotsc, v_{n}, v\}$.
    \end{enumerate}
  \end{lemma}
  \begin{proof}
    Induccion en el k tal que $\varphi \in F_{k}^{\tau}$
  \end{proof}

  % Lemma 152. Con prueba. Lemma 62.
  \begin{lemma} \label{lemma_62}
    \PN Supongamos $\varphi =_{d} \varphi(v_{1}, \dotsc, v_{n})$. Sea $\mathbf{A} = (A, i)$ un modelo de tipo $\tau$ y
    sean $a_{1}, \dotsc, a_{n} \in A$, entonces:
    \begin{enumerate}
      \item Si $\varphi = (t \equiv s)$, entonces:
      \begin{center}
        $\mathbf{A} \models \varphi \lbrack a_{1}, \dotsc, a_{n}] \ \text{si y solo si} \ t^{\mathbf{A}}[a_{1}, \dotsc,
        a_{n}]=s^{\mathbf{A}}[a_{1}, \dotsc, a_{n}]$
      \end{center}
      \item Si $\varphi = r(t_{1}, \dotsc, t_{m})$, entonces:
      \begin{equation*}
        \mathbf{A} \models \varphi \lbrack a_{1}, \dotsc, a_{n}] \ \text{si y solo si} \ (t_{1}^{A}[a_{1}, \dotsc,
        a_{n}], \dotsc, t_{m}^{A}[a_{1}, \dotsc, a_{n}])\in r^{A}
      \end{equation*}
      \item Si $\varphi = (\varphi_{1} \wedge \varphi_{2})$ entonces:
      \[
        \mathbf{A} \models \varphi \lbrack a_{1}, \dotsc, a_{n}] \ \text{si y solo si} \ \mathbf{A} \models \varphi_{1}
        [a_{1}, \dotsc, a_{n}] \ \text{y} \ \mathbf{A} \models \varphi_{2}[a_{1}, \dotsc, a_{n}]
      \]
      \item Si $\varphi = (\varphi_{1} \vee \varphi_{2})$ entonces:
      \[
        \mathbf{A} \models \varphi \lbrack a_{1}, \dotsc, a_{n}] \ \text{si y solo si} \ \mathbf{A} \models \varphi_{1}
        [a_{1}, \dotsc, a_{n}] \ \text{o} \ \mathbf{A} \models \varphi_{2}[a_{1}, \dotsc, a_{n}]
      \]
      \item Si $\varphi = (\varphi_{1} \rightarrow \varphi_{2})$ entonces:
      \[
        \mathbf{A} \models \varphi \lbrack a_{1}, \dotsc, a_{n}] \ \text{si y solo si} \ \mathbf{A} \models \varphi_{2}
        [a_{1}, \dotsc, a_{n}] \ \text{o} \ \mathbf{A} \not\models \varphi_{1}[a_{1}, \dotsc, a_{n}]
      \]
      \item Si $\varphi = (\varphi_{1} \leftrightarrow \varphi_{2})$ entonces:
      \begin{eqnarray*}
        \mathbf{A} \models \varphi \lbrack a_{1}, \dotsc, a_{n}] \ \text{si y solo si ya sea} \ && \mathbf{A} \models
        \varphi_{1}[a_{1}, \dotsc, a_{n}] \ \text{y} \ \mathbf{A} \models \varphi_{2}[a_{1}, \dotsc, a_{n}] \ \text{o}
        \\
        && \mathbf{A} \not\models \varphi_{1}[a_{1}, \dotsc, a_{n}] \ \text{y} \ \mathbf{A} \not\models \varphi_{2}
        [a_{1}, \dotsc, a_{n}]
      \end{eqnarray*}
      \item Si $\varphi = \lnot \varphi_{1}$ entonces:
      \[
        \mathbf{A} \models \varphi \lbrack a_{1}, \dotsc, a_{n}] \ \text{si y solo si} \ \mathbf{A} \not\models
        \varphi_{1}[a_{1}, \dotsc, a_{n}]
      \]
      \item Si $\varphi = \forall v\varphi_{1}$ con $v\not\in \{v_{1}, \dotsc, v_{n}\} \ \text{y} \ \varphi_{1} =_{d}
      \varphi_{1}(v_{1}, \dotsc, v_{n}, v)$ entonces:
      \[
        \mathbf{A} \models \varphi \lbrack a_{1}, \dotsc, a_{n}] \ \text{si y solo si} \ \mathbf{A} \models \varphi_{1}
        [a_{1}, \dotsc, a_{n},a] \ \text{para todo} \ a \in A
      \]
      \item Si $\varphi = \forall v_{j}\varphi_{1}$ entonces:
      \[
        \mathbf{A} \models \varphi \lbrack a_{1}, \dotsc, a_{n}] \ \text{si y solo si} \ \mathbf{A} \models \varphi_{1}
        [a_{1}, \dotsc, a, \dotsc, a_{n}] \ \text{para todo} \ a \in A
      \]
      \item) Si $\varphi = \exists v\varphi_{1}$ con $v\not\in \{v_{1}, \dotsc, v_{n}\} \ \text{y} \ \varphi_{1} =_{d}
      \varphi_{1}(v_{1}, \dotsc, v_{n}, v)$ entonces:
      \[
        \mathbf{A} \models \varphi \lbrack a_{1}, \dotsc, a_{n}] \ \text{si y solo si} \ \mathbf{A} \models \varphi_{1}
        [a_{1}, \dotsc, a_{n}, a] \ \text{para algún} \ a \in A
      \]
      \item) Si $\varphi = \exists v_{j}\varphi_{1}$ entonces:
      \[
        \mathbf{A} \models \varphi \lbrack a_{1}, \dotsc, a_{n}] \ \text{si y solo si} \ \mathbf{A} \models \varphi_{1}
        [a_{1}, \dotsc, a, \dotsc, a_{n}] \ \text{para algún} \ a \in A
      \]
    \end{enumerate}
  \end{lemma}
  \begin{proof}
  \end{proof}

  % Lemma 153. Con prueba. Lemma 63.
  \begin{lemma} \label{lemma_63}
    \PN Si $Qv$ ocurre en $\varphi$ a partir de $i$, entonces hay una única fórmula $\psi$ tal que $Qv\psi$ ocurre en
    $\varphi$ a partir de $i$.
  \end{lemma}
  \begin{proof}
    Por induccion en el $k$ tal que $\varphi \in F^{\tau}$.
  \end{proof}

  % Lemma 154. Con prueba. Lemma 64.
  \begin{lemma} \label{lemma_64}
    \PN Supongamos $\varphi =_{d} \varphi(w_{1}, \dotsc, w_{k}), t_{1} =_{d} t_{1}(v_{1}, \dotsc, v_{n}), \dotsc, t_{k}
    =_{d} t_{k}(v_{1}, \dotsc, v_{n})$ y supongamos además que cada $w_{j}$ es sustituible por $t_{j}$ en $\varphi$,
    entonces:
    \begin{enumerate}[(a)]
      \item $Li(\varphi(t_{1}, \dotsc, t_{k})) \subseteq \{v_{1}, \dotsc, v_{n}\}$
      \item Si declaramos $\varphi(t_{1}, \dotsc, t_{k}) =_{d} \varphi(t_{1}, \dotsc, t_{k})(v_{1}, \dotsc, v_{n})$,
      entonces para cada estructura $\mathbf{A} \ \text{y} \ \vec{a} \in A^{n}$ se tiene:
        \[
          \mathbf{A} \models \varphi(t_{1}, \dotsc, t_{k})[\vec{a}] \ \text{si y solo si} \ \mathbf{A} \models \varphi
          \lbrack t_{1}^{\mathbf{A}}[\vec{a}], \dotsc, t_{k}^{ \mathbf{A}}[\vec{a}]]
        \]
    \end{enumerate}
  \end{lemma}
  \begin{proof}
    Probaremos que se dan (a) y (b), por induccion en el $l$ tal que $\varphi \in F_{l}^{\tau }.$ El caso $l=0$ es una consecuencia directa del Lema 146. Supongamos (a) y (b) valen para cada $\varphi \in F_{l}^{\tau } $ y sea $\varphi \in F_{l+1}^{\tau }-F_{l}^{\tau }.$ Notese que se puede suponer que cada $v_{i}$ ocurre en algun $t_{i}$ y que cada $w_{i}\in Li(\varphi )$, ya que para cada $\varphi $ el caso general se desprende del caso con estas restricciones. Hay varios casos

    CASO $\varphi =\forall w\varphi_{1}$ con $w\not\in \{w_{1}, \dotsc, w_{k}\} \ \text{y} \  \varphi_{1}=_{d}\varphi_{1}(w_{1}, \dotsc, w_{k},w)$

    Notese que cada $w_{j}\in Li(\varphi_{1})$. Ademas notese que $ w\not\in \{v_{1}, \dotsc, v_{n}\}$ ya que de lo contrario $w$ ocurriria en algun $ t_{j}$, y entonces $w_{j}$ no seria sustituible por $t_{j}$ en $\varphi $. Sean

    $\displaystyle \begin{array}{ccc} \tilde{t}_{1} & = & t_{1} \\ & \vdots & \\ \tilde{t}_{k} & = & t_{k} \\ \tilde{t}_{k+1} & = & w \end{array} $

    Notese que
    $\displaystyle \tilde{t}_{j}=_{d}\tilde{t}_{j}(v_{1}, \dotsc, v_{n},w) $

    Por (a) de la hipotesis inductiva tenemos que
    $\displaystyle Li(\varphi_{1}(t_{1}, \dotsc, t_{k},w))=Li(\varphi_{1}(\tilde{t}_{1}, \dotsc, \tilde{ t}_{k},\tilde{t}_{k+1}))\subseteq \{v_{1}, \dotsc, v_{n},w\} $

    y por lo tanto
    $\displaystyle Li(\varphi(t_{1}, \dotsc, t_{k}))\subseteq \{v_{1}, \dotsc, v_{n}\} $

    lo cual prueba (a). Finalmente notese que
    $\displaystyle \begin{array}{c} \mathbf{A} \models \varphi(t_{1}, \dotsc, t_{k})\mathbf{[}\vec{a}] \\ \Updownarrow \\ \mathbf{A} \models \varphi_{1}(\tilde{t}_{1}, \dotsc, \tilde{t}_{k},\tilde{t} _{k+1})[\vec{a},a],\text{ para todo }a\in A \\ \Updownarrow \\ \mathbf{A} \models \varphi_{1}[\tilde{t}_{1}^{\mathbf{A}}[\vec{a},a], \dotsc,  \tilde{t}_{k}^{\mathbf{A}}[\vec{a},a],\tilde{t}_{k+1}^{\mathbf{A}}[\vec{a} ,a]],\text{ para todo }a\in A \\ \Updownarrow \\ \mathbf{A} \models \varphi_{1}[t_{1}^{\mathbf{A}}[\vec{a}], \dotsc, t_{k}^{ \mathbf{A}}[\vec{a}],a],\text{ para todo }a\in A \\ \Updownarrow \\ \mathbf{A} \models \varphi \lbrack t_{1}^{\mathbf{A}}[\vec{a}], \dotsc, t_{k}^{ \mathbf{A}}[\vec{a}]] \end{array} $

    lo cual pueba (b). El caso del cuantificador $\exists $ es analogo y los casos de nexos logicos son directos.
  \end{proof}
