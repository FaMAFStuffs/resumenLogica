\section{Teorias de primer orden}

  % Lemma 155. Con prueba. Lemma 65.
  \begin{lemma}
    \PN Si $(\varphi_{1}, \varphi_{2}) \in Generaliz^{\tau}$, entonces el nombre de constante $c$ del cual habla la
    definición de $Generaliz^{\tau}$ está unívocamente determinado por el par $(\varphi_{1}, \varphi_{2})$.
  \end{lemma}
  \begin{proof}
    \PN Recordemos la definición de $Generaliz^{\tau}$.
    \begin{eqnarray*}
      Generaliz^{\tau} &=& \{(\psi, \forall v \tilde{\psi}): \psi \in S^{\tau}, \ v \ \text{no ocurre en} \ \psi \
      \text{y existe } \ c \in \mathcal{C} \ \text{tal que} \ \tilde{\psi} = \text{resultado de} \\
      && \text{reemplazar} \ \text{en} \ \psi \ \text{cada ocurrencia de} \ c \ \text{por} \ v\}
    \end{eqnarray*}

    \PN Notese que $c$ es el único nombre de constante que ocurre en $\varphi_{1}$ y no ocurre en $\varphi_{2}$.
  \end{proof}

  % Lemma 156. Con prueba. Lemma 66.
  \begin{lemma}
    \PN Si $(\varphi_{1}, \varphi_{2}) \in Elec^{\tau}$, entonces el nombre de constante $e$ del cual habla la
    definición de $Elec^{\tau}$ está unívocamente determinado por el par $(\varphi_{1}, \varphi_{2})$.
  \end{lemma}
  \begin{proof}
    \PN Recordemos la definición de $Elec^{\tau}$.
    \begin{eqnarray*}
      Elect^{\tau} = \{(\exists v \varphi(v), \varphi(e)): \varphi =_{d} \varphi(v), \ Li(\varphi) = \{v\} \ \text{y} \
      e \in \mathcal{C} \ \text{no ocurre en} \ \varphi\}
    \end{eqnarray*}

    \PN Notese que $e$ es el único nombre de constante que ocurre en $\varphi_{1}$ y no ocurre en $\varphi_{2}$.
  \end{proof}

  % Lemma 157. Con prueba. Lemma 67.
  \begin{lemma}
    \PN Todas las reglas excepto las reglas de elección y generalización son universales en el sentido que si $\varphi$
    se deduce de $\psi_{1}, \dotsc, \psi_{k}$ por alguna de estas reglas, entonces $\left((\psi_{1} \wedge \dotsc \wedge
    \psi_{k}) \rightarrow \varphi \right)$ es una sentencia universalmente válida.
  \end{lemma}
  \begin{proof}
    \begin{itemize}
      \item \textbf{DISJUNCIÓN-INTRODUCCIÓN:}
      \item \textbf{CONJUNCIÓN-INTRODUCCIÓN:}
      \item \textbf{CONJUNCIÓN-ELIMINACIÓN:}
      \item \textbf{EQUIVALENCIA-INTRODUCCIÓN:}
      \item \textbf{EQUIVALENCIA-ELIMINACIÓN:}
      \item \textbf{PARTICULARIZACIÓN:}

      \item \textbf{EXISTENCIA:} Supongamos $\varphi =_{d}\varphi (v)$, $t\in T_{c}^{\tau }$ y $\mathbf{A}$ es una
      estructura de tipo $\tau $ tal que $\mathbf{A}\models \varphi (t)$. Sea $t^{\mathbf{A}}$ el valor que toma $t$ en
      $\mathbf{A}$. Por el Lema 148 tenemos que $\mathbf{A}\models \varphi \left\lbrack t^{\mathbf{A}}\right\rbrack $,
      por lo cual tenemos que $\mathbf{A}\models \exists v\varphi (v)$.

      \item \textbf{EVOCACIÓN:}
      \item \textbf{ABSURDO:}
      \item \textbf{CONMUTATIVIDAD:}
      \item \textbf{MODUS-PONENS:}
      \item \textbf{DIVISIÓN-POR-CASOS:}

      \item \textbf{REEMPLAZO:} Debemos probar que si $(\psi_{1},\psi_{2},\varphi )\in Reemp^{\tau }=Reemp1^{\tau }\cup Reemp2^{\tau }$ , entonces $\left( (\psi_{1}\wedge \psi_{2})\rightarrow \varphi \right) $ es una sentencia universalmente valida. El caso en el que $(\psi_{1},\psi_{2},\varphi )\in Reemp1^{\tau }$ es facil y lo dejaremos al lector. Para el caso en el que $(\psi_{1},\psi_{2},\varphi )\in Reemp2^{\tau }$ nos hara falta un resultado un poco mas general. Veamos por induccion en $k$ que si se dan las siguienes condiciones

      - $\alpha \in F_{k}^{\tau }$ y $\varphi ,\psi \in F^{\tau }$
      - $\mathbf{A}$ es una estructura de tipo $\tau $
      - $\overline{\alpha }=$ resultado de reemplazar en $\alpha $ una ocurrencia de $\varphi$ por $\psi$,
      - $\mathbf{A}\models \varphi \left\lbrack \vec{a}\right\rbrack $ si y solo si $ \mathbf{A}\models \psi \left\lbrack \vec{a}\right\rbrack $, para cada $\vec{a}\in A^{ \mathbb{N}}$
      entonces se da que

      - $\mathbf{A}\models \alpha \left\lbrack \vec{a}\right\rbrack $ si y solo si $ \mathbf{A}\models \overline{\alpha }\left\lbrack \vec{a}\right\rbrack $, para cada $\vec{ a}\in A^{\mathbb{N}}$.
      CASO $k=0.$

      Entonces $\alpha $ es atomica y por lo tanto ya que $\alpha $ es la unica subformula de $\alpha $, la situacion es facil de probar.

      CASO $\alpha =\forall x_{i}\alpha _{1}.$

      Si $\varphi =\alpha $, entonces la situacion es facil de probar. Si $\varphi \neq \alpha $, entonces la ocurrencia de $\varphi$ a reemplazar sucede en $\alpha _{1}$ y por lo tanto $\overline{\alpha }=\forall x_{i} \overline{\alpha _{1}}.$ Se tiene entonces que para un $\vec{a}$ dado,

      $\displaystyle \begin{array}{c} \mathbf{A}\models \alpha \left\lbrack \vec{a}\right\rbrack \\ \Updownarrow \\ \mathbf{A}\models \alpha _{1}\left\lbrack \downarrow _{i}^{a}\vec{a}\right\rbrack ,\text{ para cada }a\in A \\ \Updownarrow \\ \mathbf{A}\models \overline{\alpha _{1}}\left\lbrack \downarrow _{i}^{a}\vec{a} \right\rbrack ,\text{ para cada }a\in A \\ \Updownarrow \\ \mathbf{A}\models \overline{\alpha }\left\lbrack \vec{a}\right\rbrack \end{array} $

      CASO $\alpha =(\alpha _{1}\vee \alpha _{2})$.
      Si $\varphi =\alpha $, entonces la situacion es facil de probar. Supongamos $\varphi \neq \alpha $ y supongamos que la ocurrencia de $\varphi$ a reemplazar sucede en $\alpha _{1}$. Entonces $\overline{\alpha }=( \overline{\alpha _{1}}\vee \alpha _{2})$ y tenemos que

      $\displaystyle \begin{array}{c} \mathbf{A}\models \alpha \left\lbrack \vec{a}\right\rbrack \\ \Updownarrow \\ \mathbf{A}\models \alpha _{1}\left\lbrack \vec{a}\right\rbrack \text{ o }\mathbf{A} \models \alpha _{2}\left\lbrack \vec{a}\right\rbrack \\ \Updownarrow \\ \mathbf{A}\models \overline{\alpha _{1}}\left\lbrack \vec{a}\right\rbrack \text{ o } \mathbf{A}\models \alpha _{2}\left\lbrack \vec{a}\right\rbrack \\ \Updownarrow \\ \mathbf{A}\models \overline{\alpha }\left\lbrack \vec{a}\right\rbrack \end{array} $

      \item \textbf{TRANSITIVIDAD:}
    \end{itemize}
  \end{proof}

  % Lemma 158. Con prueba. Lemma 68.
  \begin{lemma}
    \PN Sea $\pmb{\varphi} \in S^{\tau +}$, hay únicos $n \geq 1$ y $\varphi_{1}, \dotsc, \varphi_{n} \in S^{\tau}$
    tales que $\pmb{\varphi} = \varphi_{1} \dotsc \varphi_{n}$.
  \end{lemma}
  \begin{proof}
    \PN Debemos probar la unicidad de $\varphi_{1}, \dotsc, \varphi_{n}$, pero esto vale utilizando la
    \textbf{Proposition~\ref{proposition_1}} Solo hay que probar la unicidad la cual sigue de la Proposicion 128.
  \end{proof}

  % Lemma 159. Con prueba. Lemma 69.
  \begin{lemma}
    \PN Sea $\mathbf{J} \in Just^{+}$, hay únicos $n \geq 1$ y $J_{1}, \dotsc, J_{n} \in Just$ tales que $\mathbf{J} =
    J_{1} \dotsc J_{n}$.
  \end{lemma}
  \begin{proof}
    \PN Supongamos $J_{1}, \dotsc, J_{n}$, $J_{1}^{\prime}, \dotsc, J_{m}^{\prime}$, con $ n,m\geq 1$, son
    justificaciones tales que $J_{1}\dotsc J_{n}=J_{1}^{\prime}\dotsc J_{m}^{\prime}$. Es facil ver que entonces tenemos
    $J_{1}=J_{1}^{\prime}$, por lo cual $J_{2}\dotsc J_{n}=J_{2}^{\prime}\dotsc J_{m}^{\prime}$. Un argumento inductivo
    nos dice que entonces $n=m$ y $J_{i}=J_{i}^{\prime}$, $ i=1, \dotsc, n$
  \end{proof}

  % Lemma 160. Con prueba. Lemma 70.
  \begin{lemma}
    \PN Sea $(\pmb{\varphi}, \mathbf{J})$ una prueba de $\varphi$ en $(\Sigma, \tau)$.
    \begin{enumerate}
      \item Sea $m \in \mathbb{N}$ tal que $\mathbf{J}_{i} \neq \mathrm{HIPOTESIS}\bar{m}$, para cada $i = 1, \dotsc,
      n(\pmb{\varphi})$. Supongamos que $\mathbf{J}_{i} = \mathrm{HIPOTESIS}\bar{k}$ y que $\mathbf{J}_{j} =
      \mathrm{TESIS}\bar{k} \alpha$, con $\lbrack\alpha\rbrack_{1} \notin Num$. Sea $\mathbf{\tilde{J}}$ el resultado de
      reemplazar en $\mathbf{J}$ la justificación $\mathbf{J}_{i}$ por $\mathrm{HIPOTESIS}\bar{m}$ y reemplazar la
      justificación $\mathbf{J}_{j}$ por $\mathrm{TESIS}\bar{m}\alpha$, entonces $(\pmb{\varphi}, \mathbf{\tilde{J}})$
      es una prueba de $\varphi$ en $(\Sigma, \tau)$.
      \item Sea $\mathcal{C}_{1}$ el conjunto de nombres de constante que ocurren en alguna $\pmb{\varphi}_{i}$ y que
      no pertenecen a $\mathcal{C}$. Sea $e \in \mathcal{C}_{1}-\mathcal{C}$. Sea $\tilde{e} \notin \mathcal{C} \cup
      \mathcal{C}_{1}$ tal que $(\mathcal{C} \cup (\mathcal{C}_{1}-\{e\}) \cup \{\tilde{e}\}, \mathcal{F}, \mathcal{R},
      a)$ es un tipo. Sea $\tilde{\varphi}_{i} =$ resultado de reemplazar en $\pmb{\varphi}_{i}$ cada ocurrencia de
      $e$ por $\tilde{e}$, entonces $(\pmb{\tilde{\varphi}}_{1} \dotsc \pmb{\tilde{\varphi}}_{n(\mathbf{\varphi})},
      \mathbf{J})$ es una prueba de $\varphi$ en $(\Sigma, \tau)$.
    \end{enumerate}
  \end{lemma}
  \begin{proof}
    \begin{enumerate}
      \item Obvio.
      \item Sean

      $\displaystyle \begin{array}{rcl} \tau _{1} & =& (\mathcal{C}\cup \mathcal{C}_{1},\mathcal{F},\mathcal{R},a) \\ \tau _{2} & =& (\mathcal{C}\cup (\mathcal{C}_{1}-\{e\})\cup \{\tilde{e}\}, \mathcal{F},\mathcal{R},a) \end{array} $

      Para cada $c\in \mathcal{C}\cup (\mathcal{C}_{1}-\{e\})$ definamos $\tilde{c} =c$. Notese que el mapeo $c\rightarrow \tilde{c}$ es una biyeccion entre el conjunto de nombres de constante de $\tau _{1}$ y el conjunto de nombres de cte de $\tau _{2}$. Para cada $t\in T^{\tau _{1}}$ sea $\tilde{t}=$ resultado de reemplazar en $t$ cada ocurrencia de $c$ por $\tilde{c}$, para cada $c\in \mathcal{C}\cup \mathcal{C}_{1}$. Analogamente para una formula $ \psi \in F^{\tau _{1}}$, sea $\tilde{\psi}=$ resultado de reemplazar en $ \psi $ cada ocurrencia de $c$ por $\tilde{c}$, para cada $c\in \mathcal{C} \cup \mathcal{C}_{1}$. Notese que los mapeos $t\rightarrow \tilde{t}$ y $ \psi \rightarrow \tilde{\psi}$ son biyecciones naturales entre $T^{\tau _{1}} $ y $T^{\tau _{2}}$ y entre $F^{\tau _{1}}$ y $F^{\tau _{2}}$, respectivamente. Notese que cualesquiera sean $\psi_{1},\psi_{2}\in F^{\tau _{1}}$, tenemos que $\psi_{1}$ se deduce de $\psi_{2}$ por la regla de generalizacion con constante $c$ sii $\tilde{\psi}_{1}$ se deduce de $\tilde{\psi}_{2}$ por la regla de generalizacion con constante $\tilde{c} $. Para las otras reglas sucede lo mismo. Notese tambien que $c$ ocurre en $ \psi $ sii $\tilde{c}$ ocurre en $\tilde{\psi}.$ Mas aun notese que $c$ depende de $d$ en $(\mathbf{\varphi },\mathbf{J})$ sii $\tilde{c}$ depende de $\tilde{d}$ en $(\mathbf{\tilde{\varphi}},\mathbf{J})$, donde $\mathbf{ \tilde{\varphi}}=\widetilde{\mathbf{\varphi }_{1}}\dotsc\widetilde{\mathbf{ \varphi }_{n(\mathbf{\varphi })}}$. Ahora es facil chequear que $(\mathbf{ \tilde{\varphi}},\mathbf{J})$ es una prueba de $\varphi$ en $(\Sigma, \tau) $ basandose en que $(\mathbf{\varphi },\mathbf{J})$ es una prueba de $\varphi$ en $(\Sigma, \tau)$.
    \end{enumerate}
  \end{proof}

  % Lemma 161. Con prueba. Lemma 71.
  \begin{lemma}
    \PN Sea $(\Sigma, \tau)$ una teoría.
    \begin{enumerate}
      \item Si $(\Sigma, \tau) \vdash \varphi_{1}, \dotsc, \varphi_{n}$ y $(\Sigma \cup \{\varphi_{1}, \dotsc,
      \varphi_{n}\},\tau ) \vdash \varphi$ entonces $(\Sigma, \tau) \vdash \varphi$.
      \item Si $(\Sigma, \tau) \vdash \varphi_{1}, \dotsc, \varphi_{n}$ y $\varphi$ se deduce por alguna regla universal
      a partir de $\varphi_{1}, \dotsc, \varphi_{n}$, entonces $(\Sigma, \tau) \vdash \varphi$.
      \item Si $(\Sigma, \tau)$ es inconsistente, entonces $(\Sigma, \tau) \vdash \varphi$, para toda sentencia
      $\varphi$.
      \item Si $(\Sigma, \tau)$ es consistente y $(\Sigma, \tau) \vdash \varphi$, entonces $(\Sigma \cup \{\varphi\},
      \tau)$ es consistente.
      \item $(\Sigma, \tau) \vdash (\varphi \rightarrow \psi)$ si y solo si $ (\Sigma \cup \{\varphi\}, \tau) \vdash
      \psi$.
      \item Si $(\Sigma, \tau) \not \vdash \lnot \varphi$, entonces $(\Sigma \cup \{\varphi\}, \tau)$ es consistente.
    \end{enumerate}
  \end{lemma}
  \begin{proof}
    \begin{enumerate}
      \item Haremos el caso $n=2.$ Supongamos entonces que $(\Sigma, \tau)\vdash \varphi_{1},\varphi_{2}$ y $(\Sigma \cup \{\varphi_{1},\varphi_{2}\},\tau )\vdash \varphi $. Para $i=1,2$, sea $(\varphi_{1}^{i}\dotsc\varphi_{n_{i}}^{i},J_{1}^{i}\dotsc J_{n_{i}}^{i})$ una prueba de $ \varphi_{i}$ en $(\Sigma, \tau)$. Sea $(\psi_{1}\dotsc\psi_{n},J_{1}\dotsc J_{n})$ una prueba de $\varphi$ en $(\Sigma \cup \{\varphi_{1},\varphi_{2}\},\tau )$. Notese que por el Lema 154 podemos suponer que estas tres pruebas no comparten ningun nombre de constante auxiliar y que tampoco comparten numeros asociados a hipotesis o tesis. Para cada $i=1, \dotsc, n$, definamos $\widetilde{J_{i}}$ de la siguiente manera.

      - Si $\psi_{i}=\varphi_{1}$ y $J_{i}=\mathrm{AXIOMAPROPIO}$, entonces $\widetilde{J_{i}}=\mathrm{EVOCACION}(\overline{n_{1}})$
      - Si $\psi_{i}=\varphi_{2}$ y $J_{i}=\mathrm{AXIOMAPROPIO}$, entonces $\widetilde{J_{i}}=\mathrm{EVOCACION}(\overline{n_{1}+n_{2}})$.
      - Si $\psi_{i}\notin \{\varphi_{1},\varphi_{2}\}$ y $J_{i}=\mathrm{ AXIOMAPROPIO}$, entonces $\widetilde{J_{i}}=\mathrm{AXIOMAPROPIO}$.
      - Si $J_{i}=\mathrm{AXIOMALOGICO}$, entonces $\widetilde{J_{i}}= \mathrm{AXIOMALOGICO}$
      - Si $J_{i}=\mathrm{CONCLUSION}$, entonces $\widetilde{J_{i}}=\mathrm{ CONCLUSION}$.
      - Si $J_{i}=\mathrm{HIPOTESIS}\bar{k}$, entonces $\widetilde{J_{i}}= \mathrm{HIPOTESIS}\bar{k}$
      - Si $J_{i}=\alpha P(\overline{l_{1}}, \dotsc, \overline{l_{k}})$, con $ \alpha \in \{\varepsilon \}\cup \{\mathrm{TESIS}\bar{k}:k\in \mathbb{N}\}$, entonces $\widetilde{J_{i}}=\alpha P(\overline{l_{1}+n_{1}+n_{2}}, \dotsc,  \overline{l_{k}+n_{1}+n_{2}})$
      Para cada $i=1, \dotsc, n_{2}$, definamos $\widetilde{J_{i}^{2}}$ de la siguiente manera.

      - Si $J_{i}^{2}=\mathrm{AXIOMAPROPIO}$, entonces $\widetilde{J_{i}^{2} }=\mathrm{AXIOMAPROPIO}$
      - Si $J_{i}^{2}=\mathrm{AXIOMALOGICO}$, entonces $\widetilde{J_{i}^{2} }=\mathrm{AXIOMALOGICO}$
      - Si $J_{i}^{2}=\mathrm{CONCLUSION}$, entonces $\widetilde{J_{i}^{2}}= \mathrm{CONCLUSION}$.
      - Si $J_{i}^{2}=\mathrm{HIPOTESIS}\bar{k}$, entonces $\widetilde{ J_{i}^{2}}=\mathrm{HIPOTESIS}\bar{k}$
      - Si $J_{i}^{2}=\alpha P(\overline{l_{1}}, \dotsc, \overline{l_{k}})$, con $\alpha \in \{\varepsilon \}\cup \{\mathrm{TESIS}\bar{k}:k\in \mathbb{N}\}$, entonces $\widetilde{J_{i}^{2}}=\alpha P(\overline{l_{1}+n_{1}}, \dotsc,  \overline{l_{k}+n_{1}})$
      Es facil chequear que

      $\displaystyle (\varphi_{1}^{1}\dotsc\varphi_{n_{1}}^{1}\varphi_{1}^{2}\dotsc\varphi_{n_{2}}^{2}\psi_{1}\dotsc\psi_{n},J_{1}^{1}\dotsc J_{n_{1}}^{1}\widetilde{ J_{1}^{2}}\dotsc\widetilde{J_{n_{2}}^{2}}\widetilde{J_{1}}\dotsc\widetilde{J_{n}}) $

      es una prueba de $\varphi$ en $(\Sigma, \tau)$

      \item Supongamos que $(\Sigma, \tau)\vdash \varphi_{1}, \dotsc, \varphi_{n}$ y que $\varphi$ se deduce por regla R a partir de $\varphi_{1}, \dotsc, \varphi_{n}$, con R universal. Notese que

      $\displaystyle \begin{array}{llll} 1.\; & \varphi_{1} & & \text{AXIOMAPROPIO} \\ 2.\; & \varphi_{2} & & \text{AXIOMAPROPIO} \\ \vdots & \vdots & & \vdots \\ n. & \varphi_{n} & & \text{AXIOMAPROPIO} \\ n+1 & \varphi & & \text{R}(\bar{1}, \dotsc, \bar{n}) \end{array} $

      es una prueba de $\varphi$ en $(\Sigma \cup \{\varphi_{1}, \dotsc, \varphi_{n}\},\tau )$, lo cual por (1) nos dice que $(\Sigma, \tau)\vdash \varphi $ .
      \item Si $(\Sigma, \tau)$ es inconsistente, entonces por definicion tenemos que $(\Sigma, \tau)\vdash \psi \wedge \lnot \psi $ para alguna sentencia $ \psi $. Dada una sentencia cualquiera $\varphi$ tenemos que $\varphi$ se deduce por la regla del absurdo a partir de $\psi \wedge \lnot \psi $ con lo cual (2) nos dice que $(\Sigma, \tau)\vdash \varphi $

      \item Supongamos $(\Sigma, \tau)$ es consistente y $(\Sigma, \tau)\vdash \varphi $. Si $(\Sigma \cup \{\varphi \},\tau )$ fuera inconsistente, entonces $(\Sigma \cup \{\varphi \},\tau )\vdash \psi \wedge \lnot \psi $, para alguna sentencia $\psi$, lo cual por (1) nos diria que $(\Sigma, \tau)\vdash \psi \wedge \lnot \psi $, es decir nos diria que $(\Sigma, \tau)$ es inconsistente.

      \item Supongamos $(\Sigma, \tau)\vdash (\varphi \rightarrow \psi )$. Entonces tenemos que $(\Sigma \cup \{\varphi \},\tau )\vdash (\varphi \rightarrow \psi ),\varphi $, lo cual por (2) nos dice que $(\Sigma \cup \{\varphi \},\tau )\vdash \psi $. Supongamos ahora que $(\Sigma \cup \{\varphi \},\tau )\vdash \psi $. Sea $(\varphi_{1}\dotsc\varphi_{n},J_{1}\dotsc,J_{n})$ una prueba de $\psi$ en $(\Sigma \cup \{\varphi \},\tau )$. Notese que podemos suponer que $J_{n}$ es de la forma $P(\overline{l_{1}}, \dotsc, \overline{l_{k}})$ . Definimos $\widetilde{J_{i}}=$ $\mathrm{TESIS}\bar{m}P(\overline{l_{1}+1} , \dotsc, \overline{l_{k}+1}$, donde $m$ es tal que ninguna $J_{i}$ es igual a $ \mathrm{HIPOTESIS}\bar{m}$. Para cada $i=1, \dotsc, n-1$, definamos $\widetilde{ J_{i}}$ de la siguiente manera.

      - Si $\varphi_{i}=\varphi $ y $J_{i}=\mathrm{AXIOMAPROPIO}$, entonces $\widetilde{J_{i}}=\mathrm{EVOCACION}(1)$
      - Si $\varphi_{i}\neq \varphi $ y $J_{i}=\mathrm{AXIOMAPROPIO}$, entonces $\widetilde{J_{i}}=\mathrm{AXIOMAPROPIO}$
      - Si $J_{i}=\mathrm{AXIOMALOGICO}$, entonces $\widetilde{J_{i}}= \mathrm{AXIOMALOGICO}$
      - Si $J_{i}=\mathrm{CONCLUSION}$, entonces $\widetilde{J_{i}}=\mathrm{ CONCLUSION}$
      - Si $J_{i}=\mathrm{HIPOTESIS}\bar{k}$ entonces $\widetilde{J_{i}}= \mathrm{HIPOTESIS}\bar{k}$
      - Si $J_{i}=\alpha P(\overline{l_{1}}, \dotsc, \overline{l_{k}})$, con $ \alpha \in \{\varepsilon \}\cup \{\mathrm{TESIS}\bar{k}:k\in \mathbb{N}\}$, entonces $\widetilde{J_{i}}=\alpha P(\overline{l_{1}+1}, \dotsc, \overline{l_{k}+1 })$
      Es facil chequear que

      $\displaystyle (\varphi \varphi_{1}\dotsc\varphi_{n}(\varphi \rightarrow \psi ),\text{ HIPOTESIS}\bar{m}\widetilde{J_{1}}\dotsc\widetilde{J_{n}}\text{CONCLUSION}) $

      es una prueba de $(\varphi \rightarrow \psi )$ en $(\Sigma, \tau)$
    \end{enumerate}
  \end{proof}

  % Lemma 162. Sin prueba. Theorem 72.
  \begin{theorem}
    \PN \textbf{(Corrección)} $(\Sigma, \tau) \vdash \varphi$ implica $(\Sigma, \tau) \models \varphi$.
  \end{theorem}

  % Lemma 163. Con prueba. Corollary 73.
  \begin{corollary}
    \PN Si $(\Sigma, \tau)$ tiene un modelo, entonces $(\Sigma, \tau)$ es consistente.
  \end{corollary}
  \begin{proof}
    Supongamos $\mathbf{A}$ es un modelo de $(\Sigma, \tau).$ Si $(\Sigma, \tau)$ fuera inconsistente, tendriamos que hay una $\varphi \in S^{t}$ tal que $ (\Sigma, \tau)\vdash (\varphi \wedge \lnot \varphi )$, lo cual por el Teorema de Correccion nos diria que $\mathbf{A}\models (\varphi \wedge \lnot \varphi )$
  \end{proof}

  % Lemma 164. Con prueba. Lemma 74.
  \begin{lemma}
    \PN $\dashv \vdash$ es una relación de equivalencia.
  \end{lemma}
  \begin{proof}
    \PN \newline
    \begin{itemize}
      \item \textit{Reflexiva:} La relacion es reflexiva ya que $(\varphi \leftrightarrow \varphi )$ es un axioma logico.
      \item \textit{Simétrica:} Supongamos que $\varphi \dashv \vdash \psi $, es decir $(\Sigma, \tau)\vdash \left( \varphi \leftrightarrow \psi \right) $. Ya que $((\varphi \leftrightarrow \psi )\leftrightarrow (\psi \leftrightarrow \varphi ))$ es un axioma logico, tenemos que $(\Sigma, \tau)\vdash ((\varphi \leftrightarrow \psi )\leftrightarrow (\psi \leftrightarrow \varphi ))$. Notese que $\left( \psi \leftrightarrow \varphi \right) $ se deduce de $((\varphi \leftrightarrow \psi )\leftrightarrow (\psi \leftrightarrow \varphi ))$ y $(\varphi \leftrightarrow \psi )$ por la regla de reemplazo, lo cual por (2) del Lema 155 nos dice que $(\Sigma, \tau)\vdash \left( \psi \leftrightarrow \varphi \right) $.

      \item \textit{Transitiva:}   Analogamente, usando la regla de transitividad se puede probar que $\dashv \vdash $ es transitiva.
    \end{itemize}
  \end{proof}

  % Lemma 165. Con prueba. Lemma 75.
  \begin{lemma}
    \PN Dada una teoria $T = (\Sigma, \tau)$, se tiene que:
    \begin{enumerate}[(1)]
      \item $\{\varphi \in S^{\tau}: \varphi \ \text{es un teorema de T}\} \in S^{\tau}/ \dashv \vdash_{T}$
      \item $\{\varphi \in S^{\tau}: \varphi \ \text{es refutable en T}\} \in S^{\tau}/ \dashv \vdash_{T}$
    \end{enumerate}
  \end{lemma}
  \begin{proof}
  \end{proof}

  % Lemma 166. Con prueba. Lemma 76.
  \begin{lemma}
    \PN Sea $T = (\Sigma, \tau)$ una teoría, entonces $(S^{\tau}/\mathrm{\dashv \vdash}, \SU^{T}, \IN^{T}, 0^{T},
    1^{T})$ es un álgebra de Boole.
  \end{lemma}
  \begin{proof}
  \end{proof}

  % Lemma 167. Con prueba. Lemma 77.
  \begin{lemma}
    \PN Sea T una teoría y sea $\leq^{T}$ el orden parcial asociado al álgebra de Boole $\mathcal{A}_{T}$ (es decir
    $[\varphi]_{T} \leq^{T} [\psi]_{T}$ si y solo si $[\varphi]_{T} \ \SU^{T} \ [\psi]_{T} = [\psi]_{T})$, entonces se
    tiene que:
    \[
      [\varphi]_{T} \leq^{T} [\psi]_{T} \text{ si y solo si} \ T \vdash (\varphi \rightarrow \psi)
    \]
  \end{lemma}
  \begin{proof}
    En virtud de los lemas anteriores solo falta probar que

    $\displaystyle \begin{array}{rcl} \lbrack \varphi \rbrack\;\mathsf{s}\;\lbrack\varphi \rbrack^{c} & =& 1 \\ \lbrack \varphi \rbrack\;\mathsf{i}\;\lbrack\varphi \rbrack^{c} & =& 0 \end{array} $

    Dejamos al lector la prueba de estas igualdades.
  \end{proof}

  % Lemma 168. Con prueba. Lemma 78.
  \begin{lemma}
    \PN Sean $\tau = (\mathcal{C}, \mathcal{F}, \mathcal{R}, a)$ y $\tau^{\prime} = (\mathcal{C}^{\prime},
    \mathcal{F}^{\prime}, \mathcal{R}^{\prime}, a^{\prime})$ tipos.
    \begin{enumerate}
      \item Si $\mathcal{C} \subseteq \mathcal{C}^{\prime}, \mathcal{F} \subseteq \mathcal{F}^{\prime}, \mathcal{R}
      \subseteq \mathcal{R}^{\prime}$ y $a^{\prime}\mid_{\mathcal{F} \cup \mathcal{R}} = a$, entonces $(\Sigma, \tau)
      \vdash \varphi$ implica $(\Sigma, \tau^{\prime}) \vdash \varphi$.
      \item Si $\mathcal{C} \subseteq \mathcal{C}^{\prime}, \mathcal{F} = \mathcal{F}^{\prime}, \mathcal{R} =
      \mathcal{R}^{\prime}$ y $a^{\prime} = a$, entonces $(\Sigma, \tau^{\prime}) \vdash \varphi$ implica $(\Sigma,
      \tau) \vdash \varphi$, cada vez que $\Sigma \cup \{\varphi\} \subseteq S^{\tau}$.
    \end{enumerate}
  \end{lemma}
  \begin{proof}
    \PN \newline
    \begin{enumerate}[(1)]
      \item Supongamos $(\Sigma, \tau)\vdash \varphi $. Entonces hay una prueba $ (\varphi_{1}\dotsc\varphi_{n},J_{1}\dotsc J_{n})$ de $\varphi$ en $(\Sigma, \tau)$. Sea $\mathcal{C}_{1}$ el conjunto de nombres de constante que ocurren en alguna $\varphi_{i}$ y que no pertenecen a $\mathcal{C}.$ Notese que aplicando varias veces el Lema 154 podemos obtener una prueba $ (\tilde{\varphi}_{1}\dotsc\tilde{\varphi}_{n},J_{1}\dotsc J_{n})$ de $\varphi$ en $(\Sigma, \tau)$ la cual cumple que los nombres de constante que ocurren en alguna $\psi_{i}$ y que no pertenecen a $\mathcal{C}$ no pertenecen a $ \mathcal{C}^{\prime}$. Pero entonces $(\tilde{\varphi}_{1}\dotsc\tilde{\varphi} _{n},J_{1}\dotsc J_{n})$ es una prueba de $\varphi$ en $(\Sigma ,\tau ^{\prime})$, con lo cual $(\Sigma ,\tau ^{\prime})\vdash \varphi $

      \item Supongamos $(\Sigma ,\tau ^{\prime})\vdash \varphi $. Entonces hay una prueba $(\mathbf{\varphi },\mathbf{J})$ de $\varphi$ en $(\Sigma ,\tau ^{\prime})$. Veremos que $(\mathbf{\varphi },\mathbf{J})$ es una prueba de $\varphi$ en $(\Sigma, \tau)$. Ya que $(\mathbf{\varphi },\mathbf{J})$ es una prueba de $\varphi$ en $(\Sigma ,\tau ^{\prime})$ hay un conjunto finito $\mathcal{C}_{1}$, disjunto con $\mathcal{C}^{\prime}$, tal que $( \mathcal{C}^{\prime}\cup \mathcal{C}_{1},\mathcal{F},\mathcal{R},a)$ es un tipo y cada $\mathbf{\varphi }_{i}$ es una sentencia de tipo $(\mathcal{C} ^{\prime}\cup \mathcal{C}_{1},\mathcal{F},\mathcal{R},a)$. Notese que $ \widetilde{\mathcal{C}_{1}}=\mathcal{C}_{1}\cup (\mathcal{C}^{\prime}- \mathcal{C})$ es tal que $(\mathcal{C}\cup \widetilde{\mathcal{C}_{1}}, \mathcal{F},\mathcal{R},a)$ es un tipo y cada $\mathbf{\varphi }_{i}$ es una sentencia de tipo $(\mathcal{C}\cup \widetilde{\mathcal{C}_{1}},\mathcal{F}, \mathcal{R},a)$, con lo cual $(\mathbf{\varphi },\mathbf{J})$ cunple el punto 1. de la definicion de prueba. Todos los otros puntos se cumplen en forma directa, exepto los puntos 4(f) y 4(g)i para los cuales es necesario notar que $\mathcal{C}\subseteq \mathcal{C}^{\prime}$.
    \end{enumerate}
  \end{proof}

  % Lemma 169. Con prueba. Lemma 79.
  \begin{lemma}
    \PN Sea $(\Sigma, \tau)$ una teoría y supongamos que $\tau$ tiene una cantidad infinita de nombres de constante que
    no ocurren en las sentencias de $\Sigma$, entonces para cada formula $\varphi =_{d} \varphi(v)$, se tiene que
    $\lbrack \forall v\varphi (v) \rbrack_{T} = \inf(\{\lbrack \varphi(t) \rbrack_{T}: t$ es un término cerrado$\})$.
  \end{lemma}
  \begin{proof}
    Primero notese que $\lbrack\forall v\;\varphi (v)\rbrack\leq \lbrack \varphi (t)\rbrack$, para todo termino cerrado $t$, ya que podemos dar la siguiente prueba:

    $\displaystyle \begin{array}{cllll} 1. & \forall v\;\varphi (v) & & & \text{HIPOTESIS}1 \\ 2. & \varphi (t) & & & \text{TESIS}1\text{PARTICULARIZACION}(1) \\ 3. & (\forall v\;\varphi (v)\rightarrow \varphi (t)) & & & \text{CONCLUSION } \end{array} $

    Supongamos ahora que $\lbrack\psi \rbrack\leq \lbrack \varphi (t)\rbrack$, para todo termino cerrado $t.$ Por hipotesis hay un nombre de cte $c\in \mathcal{C}$ el cual no ocurre en los elementos de $\Sigma \cup \{\psi ,\varphi (v)\}.$ Ya que $ \lbrack\psi \rbrack\leq \lbrack \varphi (c)\rbrack$, hay una prueba $(\varphi_{1}\dotsc\varphi_{n},J_{1}\dotsc J_{n})$ de $\left( \psi \rightarrow \varphi (c)\right) $ en $(\Sigma, \tau)$. Pero entonces es facil de chequear que la siguiente es una prueba en $(\Sigma ,(\mathcal{C}-\{c\},\mathcal{F},\mathcal{R},a))$ de $ \left( \psi \rightarrow \forall v\;\varphi (v)\right) $:
    $\displaystyle \begin{array}{rlcl} 1. & \varphi_{1} & & J_{1} \\ 2. & \varphi_{2} & & J_{2} \\ \vdots & \vdots & & \vdots \\ n. & \varphi_{n}=\left( \psi \rightarrow \varphi (c)\right) & & J_{n} \\ n+1. & \psi & & \text{HIPOTESIS}\bar{m} \\ n+2. & \varphi (c) & & \text{MODUSPONENS}(\bar{n},\overline{n+1}) \\ n+3. & \forall v\varphi (v) & & \text{TESIS}\bar{m}\text{GENERALIZACION}( \overline{n+2}) \\ n+4. & \left( \psi \rightarrow \forall v\varphi (v)\right) & & \text{ CONCLUSION} \end{array} $

    (con $m$ elejido suficientemente grande). Por el Lema 162 tenemos entonces que $(\Sigma, \tau)\vdash \left( \psi \rightarrow \forall v\;\varphi (v)\right) $
  \end{proof}

  % Lemma 170. Con prueba. Lemma 80.
  \begin{lemma}
    \PN \textbf{(Coincidencia)} Sean $\tau$ y $\tau^{\prime}$ dos tipos cualesquiera y sea $\tau_{\cap}$ dado por:
    \begin{itemize}
      \item $\mathcal{C}_{\cap} = \mathcal{C} \cap \mathcal{C}^{\prime}$
      \item $\mathcal{F}_{\cap} = \{f \in \mathcal{F} \cap \mathcal{F}^{\prime}: a(f) = a^{\prime}(f)\}$
      \item $\mathcal{R}_{\cap} = \{r \in \mathcal{R} \cap \mathcal{R}^{\prime}: a(r) = a^{\prime}(r)\}$
      \item $a_{\cap} = a\mid_{\mathcal{F}_{\cap} \cup \mathcal{R}_{\cap}}$
    \end{itemize}

    \PN Sean $\mathbf{A}$ y $\mathbf{A}^{\prime}$ modelos de tipo $\tau$ y $\tau^{\prime}$ respectivamente. Supongamos
    que $A = A^{\prime}$ y que $c^{\mathbf{A}} = c^{\mathbf{A}^{\prime}}$, para cada $c \in \mathcal{C}_{\cap},
    f^{\mathbf{A}} = f^{\mathbf{A}^{\prime}}$, para cada $f \in \mathcal{F}_{\cap}$ y $r^{\mathbf{A}} =
    r^{\mathbf{A}^{\prime}}$, para cada $r \in \mathcal{R}_{\cap}$, entonces:
    \begin{enumerate}[(a)]
      \item Para cada $t =_{d} t(\vec{v}) \in T^{\tau_{\cap}}$ se tiene que $t^{\mathbf{A}} \lbrack \vec{a} \rbrack =
      t^{\mathbf{A}^{\prime}} \lbrack \vec{a} \rbrack$, para cada $\vec{a} \in A^{n}$.
      \item Para cada $\varphi =_{d} \varphi (\vec{v}) \in F^{\tau_{\cap}}$ se tiene que:
      \[
        \mathbf{A} \models \varphi \lbrack \vec{a} \rbrack \ \text{si y solo si} \ \mathbf{A}^{\prime} \models \varphi
        \lbrack \vec{a} \rbrack
      \]
      \item Si $\Sigma \cup \{\varphi\} \subseteq S^{\tau_{\cap}}$, entonces:
      \[
        (\Sigma, \tau) \models \varphi \ \text{si y solo si} \ (\Sigma, \tau^{\prime}) \models \varphi
      \]
    \end{enumerate}
  \end{lemma}
  \begin{proof}
    (a) y (b) son directos por induccion.

    (c) Supongamos que $(\Sigma, \tau)\models \varphi $. Sea $\mathbf{A} ^{\prime}$ un modelo de $\tau ^{\prime}$ tal que $\mathbf{A}^{\prime}\models \Sigma $. Sea $a\in A^{\prime}$ un elemento fijo. Sea $\mathbf{A}$ el modelo de tipo $\tau $ definido de la siguiente manera

    - universo de $\mathbf{A}=$ $A^{\prime}$
    - $c^{\mathbf{A}}=c^{\mathbf{A}^{\prime}}$, para cada $c\in \mathcal{ C}_{\cap }$,
    - $f^{\mathbf{A}}=f^{\mathbf{A}^{\prime}}$, para cada $f\in \mathcal{ F}_{\cap }$
    - $r^{\mathbf{A}}=r^{\mathbf{A}^{\prime}}$, para cada $r\in \mathcal{ R}_{\cap }$
    - $c^{\mathbf{A}}=a$, para cada $c\in \mathcal{C}-\widetilde{\mathcal{ C}}$
    - $f^{\mathbf{A}}(a_{1}, \dotsc, a_{a(f)})=a$, para cada $f\in \mathcal{F}- \mathcal{F}_{\cap }$, $a_{1}, \dotsc, a_{a(f)}\in A^{\prime}$
    - $r^{\mathbf{A}}=\varnothing $, para cada $r\in \mathcal{R-R}_{\cap }$
    Ya que $\mathbf{A}^{\prime}\models \Sigma $, (b) nos dice que $ \mathbf{A}\models \Sigma $, lo cual nos dice que $\mathbf{A}\models \varphi $ . Nuevamente por (b) tenemos que $\mathbf{A}^{\prime}\models \varphi $, con lo cual hemos probado que $(\Sigma ,\tau ^{\prime})\models \varphi $
  \end{proof}

  % Lemma 171. Con prueba. Lemma 81.
  \begin{lemma}
    \PN Sea $\tau$ un tipo. Hay una infinitupla $(\gamma_{1}, \gamma_{2}, \dotsc) \in F^{\tau_{\mathbb{N}}}$ tal que:
    \begin{enumerate}
      \item $\lvert Li(\gamma_{j}) \rvert \leq 1$, para cada $j = 1, 2, \dotsc$
      \item Si $\lvert Li(\gamma )\rvert \leq 1$, entonces $\gamma = \gamma_{j}$, para algún $j \in \mathbb{N}$
    \end{enumerate}
  \end{lemma}
  \begin{proof}
    Notese que las formulas de tipo $\tau $ son palabras de algun alfabeto finito $A$. Dado un orden total estricto $< $ para $A$, podemos definir

    $\displaystyle \begin{array}{rcl} \gamma _{1} & =& \min\nolimits_{\alpha }^{< }\left( \alpha \in F^{\tau }\wedge \lvert Li(\alpha )\rvert \leq 1\right) \\ \gamma _{t+1} & =& \min\nolimits_{\alpha }^{< }\left( \alpha \in F^{\tau }\wedge \lvert Li(\alpha )\rvert \leq 1\wedge (\forall i\in \omega )_{i\leq t}\alpha \neq \gamma _{i}\right) \end{array} $

    Claramente esta sucesion cumple (1) y es facil ver que tambien se cumple la propiedad (2).
  \end{proof}

  % Theorem 172. Con prueba. Theorem 82.
  \begin{theorem}
    \PN \textbf{(Completitud) (G{\"o}del)} Sea $T = (\Sigma, \tau)$ una teoría de primer orden. Si $T \models \varphi$
    entonces $T \vdash \varphi$.
  \end{theorem}
  % \begin{proof} % TODO error de compilación.
  %   Primero probaremos completitud para el caso en que $\tau $ tiene una cantidad infinita de nombres de cte que no ocurren en las sentencias de $ \Sigma $. Lo probaremos por el absurdo, es decir supongamos que $\varphi_{0} $ es tal que $(\Sigma, \tau)\models \varphi_{0}$ y $(\Sigma, \tau)\not\vdash \varphi_{0}.$ Notese que ya que $(\Sigma, \tau)\not\vdash \varphi_{0}$, tenemos que $\lbrack\lnot \varphi_{0}\rbrack\not=0^{\mathcal{A}_{(\Sigma, \tau)}}.$ Para cada $j\in \mathbb{N}$, sea $w_{j}\in Var$ tal que $ Li(\gamma _{j})\subseteq \{w_{j}\}$. Para cada $j$, declaremos $\gamma _{j}=_{d}\gamma _{j}(w_{j})$. Notese que por el Lema 163 tenemos que $\inf \{\lbrack\gamma _{j}(t)\rbrack:t\in T_{c}^{\tau }\}=\lbrack\forall w_{j}\gamma _{j}(w_{j})\rbrack$, para cada $j=1,2,\dotsc$. Por el Teorema de Rasiova y Sikorski tenemos que hay un filtro primo de $\mathcal{A}_{(\Sigma, \tau)}$ , $\mathcal{U}$ el cual cumple:
  %
  %   (a) $\lbrack\lnot \varphi_{0}\rbrack\in \mathcal{U}$
  %   (b) para cada $j\in \mathbb{N}$, $\{\lbrack\gamma _{j}(t)\rbrack:t\in T_{c}^{\tau }\}\subseteq \mathcal{U}$ implica que $\lbrack\forall w_{j}\gamma _{j}(w_{j})\rbrack\in \mathcal{U}$
  %   Ya que la sucesion de las $\gamma _{i}$ cubre todas las formulas con a lo sumo una variable libre, podemos reescribir la propiedad (b) de la siguiente manera
  %
  %   (b)$^{\prime}$ para cada $\varphi =_{d}\varphi (v)\in F^{\tau }$, si $\{\lbrack\varphi (t)\rbrack:t\in T_{c}^{\tau }\}\subseteq \mathcal{U}$ entonces $ \lbrack\forall v\varphi (v)\rbrack\in \mathcal{U}$
  %   Definamos sobre $T_{c}^{\tau }$ la siguiente relacion:
  %
  %   $\displaystyle t\bowtie s\text{ si y solo si }\lbrack(t\equiv s)\rbrack\in \mathcal{U}\text{.} $
  %
  %   Veamos entonces que:
  %   (1) $\bowtie $ es de equivalencia.
  %   (2) Para cada $\varphi =_{d}\varphi (v_{1}, \dotsc, v_{n})\in F^{\tau }$, $ t_{1}, \dotsc, t_{n},s_{1}, \dotsc, s_{n}\in T_{c}^{\tau }$, si $t_{1}\bowtie s_{1}$, $ t_{2}\bowtie s_{2}$, $\dotsc$, $t_{n}\bowtie s_{n}$, entonces $\lbrack\varphi (t_{1}, \dotsc, t_{n})\rbrack\in \mathcal{U}$ si y solo si $\lbrack\varphi (s_{1}, \dotsc, s_{n})\rbrack\in \mathcal{U}$.
  %   (3) Para cada $f\in \mathcal{F}_{n}$, $ t_{1}, \dotsc, t_{n},s_{1}, \dotsc, s_{n}\in T_{c}^{\tau }$,
  %   $\displaystyle t_{1}\bowtie s_{1},t_{2}\bowtie s_{2}, \dotsc, \;t_{n}\bowtie s_{n}\text{ implica }f(t_{1}, \dotsc, t_{n})\bowtie f(s_{1}, \dotsc, s_{n}). $
  %
  %   Probaremos (2). Notese que
  %
  %   $\displaystyle (\Sigma, \tau)\vdash \left( (t_{1}\equiv s_{1})\wedge (t_{2}\equiv s_{2})\wedge \dotsc\wedge (t_{n}\equiv s_{n})\wedge \varphi (t_{1}, \dotsc, t_{n})\right) \rightarrow \varphi (s_{1}, \dotsc, s_{n}) $
  %
  %   lo cual nos dice que
  %   $\displaystyle \lbrack (t_{1}\equiv s_{1})\rbrack \ \IN \ \lbrack(t_{2}\equiv s_{2})\rbrack \ \IN \  \dotsc \ \IN \ \lbrack(t_{n}\equiv s_{n})\rbrack \ \IN \ \lbrack\varphi (t_{1}, \dotsc, t_{n})\rbrack\leq \lbrack \varphi (s_{1}, \dotsc, s_{n})\rbrack $
  %
  %   de lo cual se desprende que
  %   $\displaystyle \lbrack \varphi (t_{1}, \dotsc, t_{n})\rbrack\in \mathcal{U}\text{ implica }\lbrack\varphi (s_{1}, \dotsc, s_{n})\rbrack\in \mathcal{U} $
  %
  %   ya que $\mathcal{U}$ es un filtro. La otra implicacion es analoga.
  %   Para probar (3) podemos tomar $\varphi =\left( f(v_{1}, \dotsc, v_{n})\equiv f(s_{1}, \dotsc, s_{n})\right) $ y aplicar (2).
  %
  %   Definamos ahora un modelo $\mathbf{A}_{\mathcal{U}}$ de tipo $\tau $ de la siguiente manera:
  %
  %   - Universo de $\mathbf{A}_{\mathcal{U}}=T_{c}^{\tau }/\mathrm{\bowtie }$
  %   - $f^{\mathbf{A}_{\mathcal{U}}}(t_{1}/\mathrm{\bowtie }, \dotsc, t_{n}/ \mathrm{\bowtie })=f(t_{1}, \dotsc, t_{n})/\mathrm{\bowtie }$, $f\in \mathcal{F} _{n}$, $t_{1}, \dotsc, t_{n}\in T_{c}^{\tau }\;$
  %   - $r^{\mathbf{A}_{\mathcal{U}}}=\{(t_{1}/\mathrm{\bowtie }, \dotsc, t_{n}/ \mathrm{\bowtie }):\lbrackr(t_{1}, \dotsc, t_{n})\rbrack\in \mathcal{U}\}$, $r\in \mathcal{R} _{n}.$
  %   Notese que la definicion de $f^{\mathbf{A}_{\mathcal{U}}}$ es inambigua por (3). Probaremos las siguientes propiedades basicas:
  %
  %   (4) Para cada $t=_{d}t(v_{1}, \dotsc, v_{n})\in \TAU$, $ t_{1}, \dotsc, t_{n}\in T_{c}^{\tau }$, tenemos que
  %   $\displaystyle t^{\mathbf{A}_{\mathcal{U}}}\lbrack t_{1}/\mathrm{\bowtie }, \dotsc, t_{n}/\mathrm{ \bowtie }\rbrack=t(t_{1}, \dotsc, t_{n})/\mathrm{\bowtie } $
  %
  %   (5) Para cada $\varphi =_{d}\varphi (v_{1}, \dotsc, v_{n})\in F^{\tau }$, $ t_{1}, \dotsc, t_{n}\in T_{c}^{\tau }$, tenemos que
  %   $\displaystyle \mathbf{A}_{\mathcal{U}}\models \varphi \lbrack t_{1}/\mathrm{\bowtie } , \dotsc, t_{n}/\mathrm{\bowtie }\rbrack\text{ si y solo si }\lbrack\varphi (t_{1}, \dotsc, t_{n})\rbrack\in \mathcal{U}. $
  %
  %   La prueba de (4) es directa por induccion. Probaremos (5) por induccion en el $k$ tal que $\varphi \in F_{k}^{\tau }$. El caso $k=0$ es dejado al lector. Supongamos (5) vale para $\varphi \in F_{k}^{\tau }$. Sea $\varphi =_{d}\varphi (v_{1}, \dotsc, v_{n})\in F_{k+1}^{\tau }-F_{k}^{\tau }.$ Hay varios casos:
  %
  %   CASO $\varphi (v_{1}, \dotsc, v_{n})=\left( \varphi_{1}(v_{1}, \dotsc, v_{n})\vee \varphi_{2}(v_{1}, \dotsc, v_{n})\right) .$
  %
  %   Tenemos
  %
  %   $\displaystyle \begin{array}{c} \mathbf{A}_{\mathcal{U}}\models \varphi \lbrack t_{1}/\mathrm{\bowtie } , \dotsc, t_{n}/\mathrm{\bowtie }\rbrack \\ \Updownarrow \\ \mathbf{A}_{\mathcal{U}}\models \varphi_{1}\lbrack t_{1}/\mathrm{\bowtie } , \dotsc, t_{n}/\mathrm{\bowtie }\rbrack\text{ o }\mathbf{A}_{\mathcal{U}}\models \varphi_{2}\lbrack t_{1}/\mathrm{\bowtie }, \dotsc, t_{n}/\mathrm{\bowtie }\rbrack \\ \Updownarrow \\ \lbrack \varphi_{1}(t_{1}, \dotsc, t_{n})\rbrack\in \mathcal{U}\text{ o }\lbrack\varphi_{2}(t_{1}, \dotsc, t_{n})\rbrack\in \mathcal{U} \\ \Updownarrow \\ \lbrack \varphi_{1}(t_{1}, \dotsc, t_{n})\rbrack\ \mathsf{s\ }\lbrack\varphi_{2}(t_{1}, \dotsc, t_{n})\rbrack\in \mathcal{U} \\ \Updownarrow \\ \lbrack \left( \varphi_{1}(t_{1}, \dotsc, t_{n})\vee \varphi_{2}(t_{1}, \dotsc, t_{n})\right) \rbrack\in \mathcal{U} \\ \Updownarrow \\ \lbrack \varphi (t_{1}, \dotsc, t_{n})\rbrack\in \mathcal{U}. \end{array} $
  %
  %   CASO $\varphi (v_{1}, \dotsc, v_{n})=\forall v\varphi_{1}(v_{1}, \dotsc, v_{n},v).$
  %   Tenemos
  %
  %   $\displaystyle \begin{array}{c} \mathbf{A}_{\mathcal{U}}\models \varphi \lbrack t_{1}/\mathrm{\bowtie } , \dotsc, t_{n}/\mathrm{\bowtie }\rbrack \\ \Updownarrow \\ \mathbf{A}_{\mathcal{U}}\models \varphi_{1}\lbrack t_{1}/\mathrm{\bowtie } , \dotsc, t_{n}/\mathrm{\bowtie },t/\mathrm{\bowtie }\rbrack\text{, para todo }t\in T_{c}^{\tau } \\ \Updownarrow \\ \lbrack \varphi_{1}(t_{1}, \dotsc, t_{n},t)\rbrack\in \mathcal{U}\text{, para todo } t\in T_{c}^{\tau } \\ \Updownarrow \\ \lbrack \forall v\varphi_{1}(t_{1}, \dotsc, t_{n},v)\rbrack\in \mathcal{U} \\ \Updownarrow \\ \lbrack \varphi (t_{1}, \dotsc, t_{n})\rbrack\in \mathcal{U}. \end{array} $
  %
  %   CASO $\varphi (v_{1}, \dotsc, v_{n})=\exists v\varphi_{1}(v_{1}, \dotsc, v_{n},v).$
  %   Tenemos
  %
  %   $\displaystyle \begin{array}{c} \mathbf{A}_{\mathcal{U}}\models \varphi \lbrack t_{1}/\mathrm{\bowtie } , \dotsc, t_{n}/\mathrm{\bowtie }\rbrack \\ \Updownarrow \\ \mathbf{A}_{\mathcal{U}}\models \varphi_{1}\lbrack t_{1}/\mathrm{\bowtie } , \dotsc, t_{n}/\mathrm{\bowtie },t/\mathrm{\bowtie }\rbrack\text{, para algun }t\in T_{c}^{\tau } \\ \Updownarrow \\ \lbrack \varphi_{1}(t_{1}, \dotsc, t_{n},t)\rbrack\in \mathcal{U}\text{, para algun } t\in T_{c}^{\tau } \\ \Updownarrow \\ \lbrack \varphi_{1}(t_{1}, \dotsc, t_{n},t)\rbrack^{c}\not\in \mathcal{U}\text{, para algun }t\in T_{c}^{\tau } \\ \Updownarrow \\ \lbrack \lnot \varphi_{1}(t_{1}, \dotsc, t_{n},t)\rbrack\not\in \mathcal{U}\text{, para algun }t\in T_{c}^{\tau } \\ \Updownarrow \\ \lbrack \forall v\;\lnot \varphi_{1}(t_{1}, \dotsc, t_{n},v)\rbrack\not\in \mathcal{U} \\ \Updownarrow \\ \lbrack \forall v\;\lnot \varphi_{1}(t_{1}, \dotsc, t_{n},v)\rbrack^{c}\in \mathcal{U} \\ \Updownarrow \\ \lbrack \lnot \forall v\;\lnot \varphi_{1}(t_{1}, \dotsc, t_{n},v)\rbrack\in \mathcal{U } \\ \Updownarrow \\ \lbrack \varphi (t_{1}, \dotsc, t_{n})\rbrack\in \mathcal{U}. \end{array} $
  %
  %   Pero ahora notese que (5) en particular nos dice que para cada sentencia $ \psi \in S^{\tau }$, $\mathbf{A}_{\mathcal{U}}\models \psi $ si y solo si $ \lbrack\psi \rbrack\in \mathcal{U}.$ De esta forma llegamos a que $\mathbf{A}_{\mathcal{U }}\models \Sigma $ y $\mathbf{A}_{\mathcal{U}}\models \lnot \varphi_{0}$, lo cual contradice la suposicion de que $(\Sigma, \tau)\models \varphi_{0}. $
  %   Ahora supongamos que $\tau $ es cualquier tipo. Sean $s_{1}$ y $s_{2}$ un par de simbolos no pertenecientes a la lista
  %
  %   $\displaystyle \forall \ \ \exists \ \ \lnot \ \ \vee \ \ \wedge \ \ \rightarrow \ \ \leftrightarrow \ \ (\ \ )\ \ ,\ \equiv \ \ \mathsf{X}\ \ \mathit{0}\ \ \mathit{1}\ \ \dotsc\ \ \mathit{9}\ \ \mathbf{0}\ \ \mathbf{1}\ \ \dotsc\ \ \mathbf{9} $
  %
  %   y tales que ninguno ocurra en alguna palabra de $\mathcal{C}\cup \mathcal{F} \cup \mathcal{R}.$ Si $(\Sigma, \tau)\models \varphi $, entonces usando el Lema de Coincidencia se puede ver que $(\Sigma ,(\mathcal{C}\cup \{s_{1}s_{2}s_{1},s_{1}s_{2}s_{2}s_{1},\dotsc\},\mathcal{F},\mathcal{R} ,a))\models \varphi $, por lo cual
  %   $\displaystyle (\Sigma ,(\mathcal{C}\cup \{s_{1}s_{2}s_{1},s_{1}s_{2}s_{2}s_{1},\dotsc\}, \mathcal{F},\mathcal{R},a))\vdash \varphi . $
  %
  %   Pero por Lema 162, tenemos que $(\Sigma, \tau)\vdash \varphi .$
  % \end{proof}

  % Corollary 173. Con prueba. Corollary 83.
  \begin{corollary}
    \PN Toda teoría consistente tiene un modelo.
  \end{corollary}
  \begin{proof}
    Supongamos $(\Sigma, \tau)$ es consistente y no tiene modelos. Entonces $ (\Sigma, \tau)\models \left( \varphi
    \wedge \lnot \varphi \right) $, con lo cual por completitud $(\Sigma, \tau)\vdash \left( \varphi \wedge \lnot
    \varphi \right) $, lo cual es absurdo.
  \end{proof}

  % Corollary 174. Con prueba. Corollary 84.
  \begin{corollary}
    \PN \textbf{(Teorema de Compacidad)}
    \begin{enumerate}[(a)]
      \item Si $(\Sigma, \tau)$ es tal que $(\Sigma_{0}, \tau)$ tiene un modelo, para cada subconjunto finito
      $\Sigma_{0} \subseteq \Sigma$, entonces $(\Sigma, \tau)$ tiene un modelo.
      \item Si $(\Sigma, \tau) \models \varphi$, entonces hay un subconjunto finito $\Sigma_{0} \subseteq \Sigma$ tal
      que $(\Sigma_{0}, \tau) \models \varphi$.
    \end{enumerate}
  \end{corollary}
  \begin{proof}
    (a) Si $(\Sigma, \tau)$ fuera inconsistente habria un subconjunto finito $ \Sigma _{0}\subseteq \Sigma $ tal que la teoria $(\Sigma _{0},\tau )$ es inconsistente ($\Sigma _{0}$ puede ser formado con los axiomas de $\Sigma $ usados en una prueba que atestigue que $(\Sigma, \tau)\vdash \left( \varphi \wedge \lnot \varphi \right) $). O sea que $(\Sigma, \tau)$ es consistente por lo cual tiene un modelo.

    (b) Si $(\Sigma, \tau)\models \varphi $, entonces por completitud, $(\Sigma, \tau)\vdash \varphi $. Pero entonces hay un subconjunto finito $\Sigma _{0}\subseteq \Sigma $ tal que $(\Sigma _{0},\tau )\vdash \varphi $, es decir tal que $(\Sigma _{0},\tau )\models \varphi $ (correccion).
  \end{proof}

  % El número 175 es usado para un ejemplo.

  % Lemma 176. Con prueba. Lemma 85.
  \begin{lemma}
    \PN Dados $t_{1}, \dotsc, t_{n}, t =_{d} t(x_{1}, \dotsc, x_{n}) \in \TAU$, se tiene que $t^{\mathbf{T}^{\tau}}
    \lbrack t_{1}, \dotsc, t_{n} \rbrack = t(t_{1}, \dotsc, t_{n})$.
  \end{lemma}
  \begin{proof}
    Para cada $k\geq 0$, sea

    - Teo$_{k}$: Dados $t_{1}, \dotsc, t_{n}\in \TAU$ y $ t=_{d}t(x_{1}, \dotsc, x_{n})\in T_{k}^{\tau }$, se tiene que $t^{\mathbf{T} ^{\tau }}\lbrack t_{1}, \dotsc, t_{n}\rbrack=t(t_{1}, \dotsc, t_{n})$.
    Veamos que es cierto Teo$_{0}$. Hay dos casos

    Caso $t=_{d}t(x_{1}, \dotsc, x_{n})=c\in \mathcal{C}$.

    Entonces tenemos

    $\displaystyle \begin{array}{cll} t^{\mathbf{T}^{\tau }}\lbrack t_{1}, \dotsc, t_{n}\rbrack & = & c^{\mathbf{T}^{\tau }} \\ & = & c \\ & = & t(t_{1}, \dotsc, t_{n}) \end{array} $

    Caso $t=_{d}t(x_{1}, \dotsc, x_{n})=x_{i}$, para algun $i$.

    Entonces tenemos

    $\displaystyle \begin{array}{cll} t^{\mathbf{T}^{\tau }}\lbrack t_{1}, \dotsc, t_{n}\rbrack & = & t_{i} \\ & = & t(t_{1}, \dotsc, t_{n}) \end{array} $

    Veamos que Teo$_{k}$ implica Teo$_{k+1}$. Supongamos que vale Teo$_{k}$. Sean $t_{1}, \dotsc, t_{n}\in \TAU$ y $t=_{d}t(x_{1}, \dotsc, x_{n})\in T_{k+1}^{\tau }-T_{k}^{\tau }$. Hay $f\in \mathcal{F}_{m}$, con $m\geq 1$, y terminos $s_{1}, \dotsc, s_{m}\in T_{k}^{\tau }$ tales que $t=f(s_{1}, \dotsc, s_{m})$ . Notese que $s_{i}=_{d}s_{i}(x_{1}, \dotsc, x_{n})$, $i=1, \dotsc, m$. Tenemos entonces que
    $\displaystyle \begin{array}{lll} t^{\mathbf{T}^{\tau }}\lbrack t_{1}, \dotsc, t_{n}\rbrack & = & f(s_{1}, \dotsc, s_{m})^{\mathbf{T} ^{\tau }}\lbrack t_{1}, \dotsc, t_{n}\rbrack \\ & = & f^{\mathbf{T}^{\tau }}(s_{1}^{\mathbf{T}^{\tau }}\lbrack t_{1}, \dotsc, t_{n}\rbrack, \dotsc, s_{m}^{\mathbf{T}^{\tau }}\lbrack t_{1}, \dotsc, t_{n}\rbrack) \\ & = & f^{\mathbf{T}^{\tau }}(s_{1}(t_{1}, \dotsc, t_{n}), \dotsc, s_{m}(t_{1}, \dotsc, t_{n})) \\ & = & f(s_{1}(t_{1}, \dotsc, t_{n}), \dotsc, s_{m}(t_{1}, \dotsc, t_{n})) \\ & = & t(t_{1}, \dotsc, t_{n}) \end{array} $

    con lo cual vale Teo$_{k+1}$
  \end{proof}

  % Lemma 177. Con prueba. Lemma 86.
  \begin{lemma}
    \PN \textbf{(Universal Mapping Property)} Si $\mathbf{A}$ es cualquier $\tau$-álgebra y $F: Var \rightarrow A$, es
    una función cualquiera, entonces $F$ puede ser extendida a un homomorfismo $\bar{F}: \mathbf{T}^{\tau} \rightarrow
    \mathbf{A}$.
  \end{lemma}
  \begin{proof}
    Definamos $\bar{F}$ de la siguiente manera:

    $\displaystyle \bar{F}(t)=t^{\mathbf{A}}\lbrack(F(x_{1}),F(x_{2}),\dotsc)\rbrack $

    Es claro que $\bar{F}$ extiende a $F$. Veamos que es un homomorfismo. Dada $ c\in \mathcal{C}$, tenemos que
    $\displaystyle \begin{array}{lll} \bar{F}(c^{\mathbf{T}^{\tau }}) & = & \bar{F}(c) \\ & = & c^{\mathbf{A}}\lbrack(F(x_{1}),F(x_{2}),\dotsc)\rbrack \\ & = & c^{\mathbf{A}} \end{array} $

    Dados $f\in \mathcal{F}_{n}$, $t_{1}, \dotsc, t_{n}\in \TAU$ tenemos que
    $\displaystyle \begin{array}{lll} \bar{F}(f^{\mathbf{T}^{\tau }}(t_{1}, \dotsc, t_{n})) & = & \bar{F} (f(t_{1}, \dotsc, t_{n})) \\ & = & f(t_{1}, \dotsc, t_{n})^{\mathbf{A}}\lbrack(F(x_{1}),F(x_{2}),\dotsc)\rbrack \\ & = & f^{\mathbf{A}}(t_{1}^{\mathbf{A}}\lbrack(F(x_{1}),F(x_{2}),\dotsc)\rbrack, \dotsc, t_{n}^{ \mathbf{A}}\lbrack(F(x_{1}),F(x_{2}),\dotsc)\rbrack) \\ & = & f^{\mathbf{A}}(\bar{F}(t_{1}), \dotsc, \bar{F}(t_{n})) \end{array} $

    con lo cual hemos probado que $\bar{F}$ es un homomorfismo.
  \end{proof}

  % Lemma 178. Con prueba. Lemma 87.
  \begin{lemma}
    \PN Todas las reglas excepto las reglas anteriores son universales en el sentido que si $\varphi$ se deduce de
    $\psi_{1}, \dotsc, \psi_{k}$ por alguna de estas reglas, entonces $\left((\psi_{1} \wedge \dotsc \wedge \psi_{k})
    \rightarrow \varphi \right)$ es una sentencia universalmente válida.
  \end{lemma}
  \begin{proof}
    Veamos que la regla de reemplazo es universal. Basta con ver por induccion en $k$ que

    - Teo$_{k}$: Sean $t,s\in \TAU$, $r\in T_{k}^{\tau }$ y sea $ \mathbf{A}$ una $\tau $-algebra tal que $t^{\mathbf{A}}\lbrack\vec{a}\rbrack=s^{\mathbf{A }}\lbrack\vec{a}\rbrack$, para cada $\vec{a}\in A^{\mathbb{N}}$. Entonces $r^{\mathbf{A} }\lbrack\vec{a}\rbrack=\tilde{r}^{\mathbf{A}}\lbrack\vec{a}\rbrack$, para cada $\vec{a}\in A^{ \mathbb{N}}$, donde $\tilde{r}$ es el resultado de reemplazar algunas ocurrencias de $t$ en $r$ por $s.$
    La prueba de Teo$_{0}$ es dejada al lector. Asumamos que vale Teo$_{k}$ y probemos que vale Teo$_{k+1}$. Sean $t,s\in \TAU$, $r\in T_{k+1}^{\tau }-T_{k}^{\tau }$ y sea $\mathbf{A}$ una $\tau $-algebra tal que $t^{\mathbf{A }}\lbrack\vec{a}\rbrack=s^{\mathbf{A}}\lbrack\vec{a}\rbrack$, para cada $\vec{a}\in A^{\mathbb{N}}$. Sea $\tilde{r}$ el resultado de reemplazar algunas ocurrencias de $t$ en $r$ por $s$. El caso $t=r$ es trivial. Supongamos entonces que $t\neq r$. Supongamos $r=f(r_{1}, \dotsc, r_{n})$, con $r_{1}, \dotsc, r_{n}\in T_{k}^{\tau }$ y $ f\in \mathcal{F}_{n}$. Notese que por Lema 123 tenemos que $\tilde{r}=f(\tilde{r}_{1}, \dotsc, \tilde{r}_{n})$, donde cada $\tilde{r} _{i} $ es el resultado de reemplazar algunas ocurrencias de $t$ en $r_{i}$ por $s$. Para $\vec{a}\in A^{\mathbb{N}}$ se tiene que

    $\displaystyle \begin{array}{cclll} r^{\mathbf{A}}\lbrack\vec{a}\rbrack & = & f(r_{1}, \dotsc, r_{n})^{\mathbf{A}}\lbrack\vec{a}\rbrack & & \\ & = & f^{\mathbf{A}}(r_{1}^{\mathbf{A}}\lbrack\vec{a}\rbrack, \dotsc, r_{n}^{\mathbf{A}}\lbrack\vec{ a}\rbrack) & & \\ & = & f^{\mathbf{A}}(\tilde{r}_{1}^{\mathbf{A}}\lbrack\vec{a}\rbrack, \dotsc, \tilde{r}_{n}^{ \mathbf{A}}\lbrack\vec{a}\rbrack) & & \text{por Teo}_{k} \\ & = & f(\tilde{r}_{1}, \dotsc, \tilde{r}_{n})^{\mathbf{A}}\lbrack\vec{a}\rbrack & & \\ & = & \tilde{r}^{\mathbf{A}}\lbrack\vec{a}\rbrack & & \end{array} $

    lo cual prueba Teo$_{k+1}$
    Veamos que la regla de substitucion es universal. Supongamos $\mathbf{A} \models t\approx s$, con $t=_{d}t(x_{1}, \dotsc, x_{n})$ y $ s=_{d}s(x_{1}, \dotsc, x_{n})$. Veremos que entonces $\mathbf{A}\models t(p_{1}, \dotsc, p_{n})\approx s(p_{1}, \dotsc, p_{n}).$ Supongamos que $ p_{i}=_{d}p_{i}(x_{1}, \dotsc, x_{m})$, para cada $i=1, \dotsc, n.$ Por (a) del Lema 146, tenemos que

    $\displaystyle \begin{array}{rcl} t(p_{1}, \dotsc, p_{n}) & =& _{d}t(p_{1}, \dotsc, p_{n})(x_{1}, \dotsc, x_{m}) \\ s(p_{1}, \dotsc, p_{n}) & =& _{d}s(p_{1}, \dotsc, p_{n})(x_{1}, \dotsc, x_{m}) \end{array} $

    Sea $\vec{a}\in A^{m}$. Tenemos que
    $\displaystyle \begin{array}{rcl} t(p_{1}, \dotsc, p_{n})^{\mathbf{A}}\left\lbrack \vec{a}\right\rbrack & = & t^{\mathbf{A}} \left\lbrack p_{1}^{\mathbf{A}}\left\lbrack \vec{a}\right\rbrack , \dotsc, p_{n}^{\mathbf{A}}\left\lbrack \vec{a}\right\rbrack \right\rbrack \\ & = & s^{\mathbf{A}}\left\lbrack p_{1}^{\mathbf{A}}\left\lbrack \vec{a}\right\rbrack , \dotsc, p_{n}^{\mathbf{A}}\left\lbrack \vec{a}\right\rbrack \right\rbrack \\ & = & s(p_{1}, \dotsc, p_{n})^{\mathbf{A}}\left\lbrack \vec{a}\right\rbrack \end{array} $

    lo cual nos dice que $\mathbf{A}\models t(p_{1}, \dotsc, p_{n})\approx s(p_{1}, \dotsc, p_{n})$.
  \end{proof}

  % Theorem 179. Con prueba. Theorem 88.
  \begin{theorem}
    \PN \textbf{(Corrección)} Si $(\Sigma, \tau) \vdash_{ec} p \approx q$, entonces $(\Sigma, \tau) \models p \approx q$.
  \end{theorem}
  \begin{proof}
    Sea

    $\displaystyle p_{1}\approx q_{1}, \dotsc, p_{n}\approx q_{n} $

    una prueba ecuacional de $p\approx q$ en $(\Sigma, \tau).$ Usando el lema anterior se puede probar facilmente por induccion en $i$ que $(\Sigma, \tau)\models p_{i}\approx q_{i}$, por lo cual $(\Sigma, \tau)\models p\approx q. $
  \end{proof}

  % Theorem 180. Con prueba. Theorem 89.
  \begin{theorem}
    \PN \textbf{(Completitud) (Birkhoff)} Sea $(\Sigma, \tau)$ una teoría ecuacional. Si $(\Sigma, \tau) \models p
    \approx q$, entonces $(\Sigma, \tau) \vdash_{ec} p \approx q$.
  \end{theorem}
  % \begin{proof} % TODO error de compilación.
  %   Supongamos $(\Sigma, \tau)\models p\approx q.$ Sea $\theta $ la siguiente relacion binaria sobre $\TAU$:
  %
  %   $\displaystyle \theta =\{(t,s):(\Sigma, \tau)\vdash _{ec}t\approx s\}. $
  %
  %   Dejamos al lector probar que $\theta $ es una congruencia de $\mathbf{T} ^{\tau }$. Veamos que
  %   (*) $t^{\mathbf{T}^{\tau }/\theta }\lbrack t_{1}/\theta , \dotsc, t_{n}/\theta \rbrack=t(t_{1}, \dotsc, t_{n})/\theta $, para todo $t_{1}, \dotsc, t_{n}$, $ t=_{d}t(x_{1}, \dotsc, x_{n})$
  %   Por Corolario 142 tenemos que
  %
  %   $\displaystyle t^{\mathbf{T}^{\tau }/\theta }\lbrack t_{1}/\theta , \dotsc, t_{n}/\theta \rbrack=t^{\mathbf{T} ^{\tau }}\lbrack t_{1}, \dotsc, t_{n}\rbrack/\theta $
  %
  %   Pero por Lema 170 tenemos que $t^{\mathbf{T}^{\tau }}\lbrack t_{1}, \dotsc, t_{n}\rbrack=t(t_{1}, \dotsc, t_{n})$ por lo cual (*) es verdadera.
  %   Veamos que $\mathbf{T}^{\tau }/\theta \models \Sigma .$ Sea $t\approx s$ un elemento de $\Sigma $, con $t=_{d}t(x_{1}, \dotsc, x_{n})$ y $ s=_{d}s(x_{1}, \dotsc, x_{n}).$ Veremos que $\mathbf{T}^{\tau }/\theta \models t\approx s$, es decir veremos que
  %
  %   $\displaystyle t^{\mathbf{T}^{\tau }/\theta }\lbrack t_{1}/\theta , \dotsc, t_{n}/\theta \rbrack=s^{\mathbf{T} ^{\tau }/\theta }\lbrack t_{1}/\theta , \dotsc, t_{n}/\theta \rbrack $
  %
  %   para todo $t_{1}/\theta , \dotsc, t_{n}/\theta \in \TAU/\theta $. Notese que
  %   $\displaystyle (\Sigma, \tau)\vdash _{ec}t(t_{1}, \dotsc, t_{n})\approx s(t_{1}, \dotsc, t_{n}) $
  %
  %   por lo cual $t(t_{1}, \dotsc, t_{n})/\theta =s(t_{1}, \dotsc, t_{n})/\theta .$ Por (*) tenemos entonces
  %   $\displaystyle t^{\mathbf{T}^{\tau }/\theta }\lbrack t_{1}/\theta , \dotsc, t_{n}/\theta \rbrack=t(t_{1}, \dotsc, t_{n})/\theta =s(t_{1}, \dotsc, t_{n})/\theta =s^{\mathbf{T}^{\tau }/\theta }\lbrack t_{1}/\theta , \dotsc, t_{n}/\theta \rbrack, $
  %
  %   lo cual nos dice que $\mathbf{T}^{\tau }/\theta $ satisface la identidad $ t\approx s.$
  %   Ya que $\mathbf{T}^{\tau }/\theta \models \Sigma $, por hipotesis tenemos que $\mathbf{T}^{\tau }/\theta \models p\approx q.$ Es decir que si $ p=_{d}p(x_{1}, \dotsc, x_{n})$ y $q=_{d}q(x_{1}, \dotsc, x_{n})$ tenemos que $p^{ \mathbf{T}^{\tau }/\theta }\lbrack t_{1}/\theta , \dotsc, t_{n}/\theta \rbrack=q^{\mathbf{T} ^{\tau }/\theta }\lbrack t_{1}/\theta , \dotsc, t_{n}/\theta \rbrack$, para todo $ t_{1}, \dotsc, t_{n}\in \TAU$. En particular, tomando $t_{i}=x_{i}$, $ i=1, \dotsc, n$ tenemos que
  %
  %   $\displaystyle p^{\mathbf{T}^{\tau }/\theta }\lbrackx_{1}/\theta , \dotsc, x_{n}/\theta \rbrack=q^{\mathbf{T} ^{\tau }/\theta }\lbrackx_{1}/\theta , \dotsc, x_{n}/\theta \rbrack $
  %
  %   lo cual por (*) nos dice que $p/\theta =q/\theta $, produciendo $(\Sigma, \tau)\vdash _{ec}p\approx q$.
  % \end{proof}
