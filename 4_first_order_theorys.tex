\section{Teorias de primer orden}

  % Lemma 155. Con prueba. Lemma 65.
  \begin{lemma} \label{lemma_65}
    \PN Si $(\varphi_{1}, \varphi_{2}) \in Generaliz^{\tau}$, entonces el nombre de constante $c$ del cual habla la
    definición de $Generaliz^{\tau}$ está unívocamente determinado por el par $(\varphi_{1}, \varphi_{2})$.
  \end{lemma}
  \begin{proof}
    \PN Recordemos la definición de $Generaliz^{\tau}$.
    \begin{eqnarray*}
      Generaliz^{\tau} &=& \{(\psi, \forall v \tilde{\psi}): \psi \in S^{\tau}, \ v \ \text{no ocurre en} \ \psi \
      \text{y existe } \ c \in \mathcal{C} \ \text{tal que} \ \tilde{\psi} = \text{resultado de} \\
      && \text{reemplazar} \ \text{en} \ \psi \ \text{cada ocurrencia de} \ c \ \text{por} \ v\}
    \end{eqnarray*}

    \PN Notese que $c$ es el único nombre de constante que ocurre en $\varphi_{1}$ y no ocurre en $\varphi_{2}$.
  \end{proof}

  % Lemma 156. Con prueba. Lemma 66.
  \begin{lemma} \label{lemma_66}
    \PN Si $(\varphi_{1}, \varphi_{2}) \in Elec^{\tau}$, entonces el nombre de constante $e$ del cual habla la
    definición de $Elec^{\tau}$ está unívocamente determinado por el par $(\varphi_{1}, \varphi_{2})$.
  \end{lemma}
  \begin{proof}
    \PN Recordemos la definición de $Elec^{\tau}$.
    \begin{eqnarray*}
      Elect^{\tau} = \{(\exists v \varphi(v), \varphi(e)): \varphi =_{d} \varphi(v), \ Li(\varphi) = \{v\} \ \text{y} \
      e \in \mathcal{C} \ \text{no ocurre en} \ \varphi\}
    \end{eqnarray*}

    \PN Notese que $e$ es el único nombre de constante que ocurre en $\varphi_{1}$ y no ocurre en $\varphi_{2}$.
  \end{proof}

  % Lemma 157. Sin prueba. Lemma 67.
  \begin{lemma} \label{lemma_67}
    \PN Todas las reglas excepto las reglas de elección y generalización son universales en el sentido que si $\varphi$
    se deduce de $\psi_{1}, \dotsc, \psi_{k}$ por alguna de estas reglas, entonces $\left((\psi_{1} \wedge \dotsc \wedge
    \psi_{k}) \rightarrow \varphi \right)$ es una sentencia universalmente válida.
  \end{lemma}

  % Lemma 158. Sin prueba. Lemma 68.
  \begin{lemma} \label{lemma_68}
    \PN Sea $\pmb{\varphi} \in S^{\tau +}$, hay únicos $n \geq 1$ y $\varphi_{1}, \dotsc, \varphi_{n} \in S^{\tau}$
    tales que $\pmb{\varphi} = \varphi_{1} \dotsc \varphi_{n}$.
  \end{lemma}

  % Lemma 159. Sin prueba. Lemma 69.
  \begin{lemma} \label{lemma_69}
    \PN Sea $\mathbf{J} \in Just^{+}$, hay únicos $n \geq 1$ y $J_{1}, \dotsc, J_{n} \in Just$ tales que $\mathbf{J} =
    J_{1} \dotsc J_{n}$.
  \end{lemma}

  % Lemma 160. Sin prueba. Lemma 70.
  \begin{lemma} \label{lemma_70}
    \PN Sea $(\pmb{\varphi}, \mathbf{J})$ una prueba de $\varphi$ en $(\Sigma, \tau)$.
    \begin{enumerate}
      \item Sea $m \in \mathbb{N}$ tal que $\mathbf{J}_{i} \neq \mathrm{HIPOTESIS}\bar{m}$, para cada $i = 1, \dotsc,
      n(\pmb{\varphi})$. Supongamos que $\mathbf{J}_{i} = \mathrm{HIPOTESIS}\bar{k}$ y que $\mathbf{J}_{j} =
      \mathrm{TESIS}\bar{k} \alpha$, con $\lbrack\alpha\rbrack_{1} \notin Num$. Sea $\mathbf{\tilde{J}}$ el resultado de
      reemplazar en $\mathbf{J}$ la justificación $\mathbf{J}_{i}$ por $\mathrm{HIPOTESIS}\bar{m}$ y reemplazar la
      justificación $\mathbf{J}_{j}$ por $\mathrm{TESIS}\bar{m}\alpha$, entonces $(\pmb{\varphi}, \mathbf{\tilde{J}})$
      es una prueba de $\varphi$ en $(\Sigma, \tau)$.
      \item Sea $\mathcal{C}_{1}$ el conjunto de nombres de constante que ocurren en alguna $\pmb{\varphi}_{i}$ y que
      no pertenecen a $\mathcal{C}$. Sea $e \in \mathcal{C}_{1}-\mathcal{C}$. Sea $\tilde{e} \notin \mathcal{C} \cup
      \mathcal{C}_{1}$ tal que $(\mathcal{C} \cup (\mathcal{C}_{1}-\{e\}) \cup \{\tilde{e}\}, \mathcal{F}, \mathcal{R},
      a)$ es un tipo. Sea $\tilde{\varphi}_{i} =$ resultado de reemplazar en $\pmb{\varphi}_{i}$ cada ocurrencia de
      $e$ por $\tilde{e}$, entonces $(\pmb{\tilde{\varphi}}_{1} \dotsc \pmb{\tilde{\varphi}}_{n(\mathbf{\varphi})},
      \mathbf{J})$ es una prueba de $\varphi$ en $(\Sigma, \tau)$.
    \end{enumerate}
  \end{lemma}

  % Lemma 161. Con prueba. Lemma 71.
  \begin{lemma} \label{lemma_71}
    \PN Sea $(\Sigma, \tau)$ una teoría.
    \begin{enumerate}[(1)]
      \item Si $(\Sigma, \tau) \vdash \varphi_{1}, \dotsc, \varphi_{n}$ y $(\Sigma \cup \{\varphi_{1}, \dotsc,
      \varphi_{n}\},\tau) \vdash \varphi$ entonces $(\Sigma, \tau) \vdash \varphi$.
      \item Si $(\Sigma, \tau) \vdash \varphi_{1}, \dotsc, \varphi_{n}$ y $\varphi$ se deduce por alguna regla universal
      a partir de $\varphi_{1}, \dotsc, \varphi_{n}$, entonces $(\Sigma, \tau) \vdash \varphi$.
      \item Si $(\Sigma, \tau)$ es inconsistente, entonces $(\Sigma, \tau) \vdash \varphi$, para toda sentencia
      $\varphi$.
      \item Si $(\Sigma, \tau)$ es consistente y $(\Sigma, \tau) \vdash \varphi$, entonces $(\Sigma \cup \{\varphi\},
      \tau)$ es consistente.
      \item $(\Sigma, \tau) \vdash (\varphi \rightarrow \psi)$ si y solo si $ (\Sigma \cup \{\varphi\}, \tau) \vdash
      \psi$.
      \item Si $(\Sigma, \tau) \not \vdash \lnot \varphi$, entonces $(\Sigma \cup \{\varphi\}, \tau)$ es consistente.
    \end{enumerate}
  \end{lemma}
  \begin{proof}
    \begin{enumerate}[(1)]
      \item Haremos el caso $n = 2.$ Supongamos entonces que $(\Sigma, \tau) \vdash \varphi_{1}, \varphi_{2}$ y
        \linebreak $(\Sigma \ \cup \ \{\varphi_{1}, \varphi_{2}\}, \tau) \vdash \varphi$. Sean:
        \begin{itemize}
          \item $(\varphi_{1}^{1} \dotsc \varphi_{n_{1}}^{1}, J_{1}^{1} \dotsc J_{n_{1}}^{1})$ una prueba de
            $\varphi_{1}$ en $(\Sigma, \tau)$.
          \item $(\varphi_{1}^{2} \dotsc \varphi_{n_{2}}^{2}, J_{1}^{2} \dotsc J_{n_{2}}^{2})$ una prueba de
            $\varphi_{2}$ en $(\Sigma, \tau)$.
          \item $(\psi_{1} \dotsc \psi_{n}, J_{1} \dotsc J_{n})$ una prueba de $\varphi$ en $(\Sigma \ \cup \
            \{\varphi_{1}, \varphi_{2}\}, \tau)$.
        \end{itemize}

        \PN Notese que por el \textbf{Lemma~\ref{lemma_70}}, podemos suponer que estas tres pruebas no comparten ningún
        nombre de constante auxiliar y que tampoco comparten números asociados a hipotesis o tesis.

        \vspace{3mm}
        \PN Para cada $i = 1, \dotsc, n$, definamos $\widetilde{J_{i}}$ de la siguiente manera:
        \begin{itemize}
          \item Si $\psi_{i} = \varphi_{1}$ y $J_{i} = \mathrm{AXIOMAPROPIO}$, entonces $\widetilde{J_{i}} =
            \mathrm{EVOCACION}(\overline{n_{1}})$
          \item Si $\psi_{i} = \varphi_{2}$ y $J_{i} = \mathrm{AXIOMAPROPIO}$, entonces $\widetilde{J_{i}} =
            \mathrm{EVOCACION}(\overline{n_{1} + n_{2}})$.
          \item Si $\psi_{i} \notin \{\varphi_{1},\varphi_{2}\}$ y $J_{i} = \mathrm{AXIOMAPROPIO}$, entonces
            $\widetilde{J_{i}} = \mathrm{AXIOMAPROPIO}$.
          \item Si $J_{i} = \mathrm{AXIOMALOGICO}$, entonces $\widetilde{J_{i}}= \mathrm{AXIOMALOGICO}$
          \item Si $J_{i} = \mathrm{CONCLUSION}$, entonces $\widetilde{J_{i}} = \mathrm{CONCLUSION}$.
          \item Si $J_{i} = \mathrm{HIPOTESIS}\bar{k}$, entonces $\widetilde{J_{i}}= \mathrm{HIPOTESIS}\bar{k}$
          \item Si $J_{i} = \alpha P(\overline{l_{1}}, \dotsc, \overline{l_{k}})$, con $\alpha \in \{\varepsilon\} \
            \cup \ \{\mathrm{TESIS}\bar{k}: k \in \mathbb{N}\}$, entonces \linebreak $\widetilde{J_{i}} = \alpha
            P(\overline{l_{1} + n_{1}+n_{2}}, \dotsc, \overline{l_{k} + n_{1} + n_{2}})$
        \end{itemize}

        \PN Para cada $i = 1, \dotsc, n_{2}$, definamos $\widetilde{J_{i}^{2}}$ de la siguiente manera.
        \begin{itemize}
          \item Si $J_{i}^{2} = \mathrm{AXIOMAPROPIO}$, entonces $\widetilde{J_{i}^{2}} = \mathrm{AXIOMAPROPIO}$
          \item Si $J_{i}^{2} = \mathrm{AXIOMALOGICO}$, entonces $\widetilde{J_{i}^{2}} = \mathrm{AXIOMALOGICO}$
          \item Si $J_{i}^{2} = \mathrm{CONCLUSION}$, entonces $\widetilde{J_{i}^{2}} = \mathrm{CONCLUSION}$.
          \item Si $J_{i}^{2} = \mathrm{HIPOTESIS}\bar{k}$, entonces $\widetilde{J_{i}^{2}} = \mathrm{HIPOTESIS}\bar{k}$
          \item Si $J_{i}^{2} = \alpha P(\overline{l_{1}}, \dotsc, \overline{l_{k}})$, con $\alpha \in \{\varepsilon\}
            \ \cup \ \{\mathrm{TESIS}\bar{k}: k \in \mathbb{N}\}$, entonces \linebreak $\widetilde{J_{i}^{2}} = \alpha
            P(\overline{l_{1} + n_{1}}, \dotsc, \overline{l_{k} + n_{1}})$
        \end{itemize}

        \PN Luego,
        \[
          (\varphi_{1}^{1} \dotsc \varphi_{n_{1}}^{1} \varphi_{1}^{2} \dotsc \varphi_{n_{2}}^{2} \psi_{1} \dotsc
          \psi_{n}, J_{1}^{1} \dotsc J_{n_{1}}^{1} \widetilde{J_{1}^{2}} \dotsc \widetilde{J_{n_{2}}^{2}}
          \widetilde{J_{1}} \dotsc \widetilde{J_{n}})
        \]
        \PN es una prueba de $\varphi$ en $(\Sigma, \tau)$.

      \item Supongamos que $(\Sigma, \tau) \vdash \varphi_{1}, \dotsc, \varphi_{n}$ y que $\varphi$ se deduce por regla
        R a partir de $\varphi_{1}, \dotsc, \varphi_{n}$, con R universal. Notese que:
        \[
          \begin{array}{llll}
            1. & \varphi_{1} && \text{AXIOMAPROPIO} \\
            2. & \varphi_{2} && \text{AXIOMAPROPIO} \\
            \vdots & \vdots && \vdots \\
            n. & \varphi_{n} && \text{AXIOMAPROPIO} \\
            n+1. & \varphi && \text{R}(\bar{1}, \dotsc, \bar{n})
          \end{array}
        \]
        \PN es una prueba de $\varphi$ en $(\Sigma \ \cup \ \{\varphi_{1}, \dotsc, \varphi_{n}\}, \tau)$, lo cual por
        (1) nos dice que $(\Sigma, \tau) \vdash \varphi$.

      \item Si $(\Sigma, \tau)$ es inconsistente, entonces por definición tenemos que $(\Sigma, \tau) \vdash (\psi
        \wedge \lnot \psi)$ para alguna sentencia $\psi$. Dada una sentencia cualquiera $\varphi$ tenemos que $\varphi$
        se deduce por la regla del absurdo a partir de $(\psi \wedge \lnot \psi)$ con lo cual (2) nos dice que $(\Sigma,
        \tau) \vdash \varphi$.

      \item Supongamos $(\Sigma, \tau)$ es consistente y $(\Sigma, \tau) \vdash \varphi$. Si $(\Sigma \ \cup \
        \{\varphi\}, \tau)$ fuera inconsistente, entonces $(\Sigma \ \cup \ \{\varphi\}, \tau) \vdash (\psi \wedge \lnot
        \psi)$, para alguna sentencia $\psi$, lo cual por (1) nos diría que $(\Sigma, \tau) \vdash (\psi \wedge \lnot
        \psi)$, es decir, que $(\Sigma, \tau)$ es inconsistente.

      \item \PN \begin{tabular}{|c|} \hline $\Rightarrow$ \\\hline \end{tabular} Supongamos $(\Sigma, \tau) \vdash
        (\varphi \rightarrow \psi)$, entonces tenemos que $(\Sigma \ \cup \ \{\varphi\}, \tau) \vdash (\varphi
        \rightarrow \psi), \varphi$, lo cual por (2) nos dice que $(\Sigma \ \cup \ \{\varphi\}, \tau) \vdash \psi$.

        \PN \begin{tabular}{|c|} \hline $\Leftarrow$ \\\hline \end{tabular} Supongamos ahora que $(\Sigma \ \cup \
          \{\varphi\}, \tau) \vdash \psi$. Sea $(\varphi_{1} \dotsc \varphi_{n}, J_{1} \dotsc J_{n})$ una prueba de
          $\psi$ en $(\Sigma \cup \{\varphi\}, \tau)$. Notese que podemos suponer que $J_{n}$ es de la forma
          $P(\overline{l_{1}}, \dotsc, \overline{l_{k}})$ . Definimos $\widetilde{J_{i}} = $ $\mathrm{TESIS}\bar{m}
          P(\overline{l_{1}+1} , \dotsc, \overline{l_{k}+1})$, donde $m$ es tal que ninguna $J_{i}$ es igual a
          $\mathrm{HIPOTESIS}\bar{m}$.

        \PN Para cada $i = 1, \dotsc, n-1$, definamos $\widetilde{J_{i}}$ de la siguiente manera:
        \begin{itemize}
          \item Si $\varphi_{i} = \varphi $ y $J_{i} = \mathrm{AXIOMAPROPIO}$, entonces $\widetilde{J_{i}} =
            \mathrm{EVOCACION}(1)$
          \item Si $\varphi_{i}\neq \varphi $ y $J_{i} = \mathrm{AXIOMAPROPIO}$, entonces $\widetilde{J_{i}} =
            \mathrm{AXIOMAPROPIO}$
          \item Si $J_{i} = \mathrm{AXIOMALOGICO}$, entonces $\widetilde{J_{i}}= \mathrm{AXIOMALOGICO}$
          \item Si $J_{i} = \mathrm{CONCLUSION}$, entonces $\widetilde{J_{i}} = \mathrm{CONCLUSION}$
          \item Si $J_{i} = \mathrm{HIPOTESIS}\bar{k}$ entonces $\widetilde{J_{i}}= \mathrm{HIPOTESIS}\bar{k}$
          \item Si $J_{i} = \alpha P(\overline{l_{1}}, \dotsc, \overline{l_{k}})$, con $\alpha \in \{\varepsilon\}
            \ \cup \ \{\mathrm{TESIS}\bar{k}: k \in \mathbb{N}\}$, entonces $\widetilde{J_{i}} = \alpha
            P(\overline{l_{1}+1}, \dotsc, \overline{l_{k}+1})$
        \end{itemize}

        \PN Luego,
        \[
          (\varphi \varphi_{1} \dotsc \varphi_{n} (\varphi \rightarrow \psi), \text{HIPOTESIS}\bar{m} \widetilde{J_{1}}
          \dotsc \widetilde{J_{n}} \text{CONCLUSION})
        \]
        \PN es una prueba de $(\varphi \rightarrow \psi)$ en $(\Sigma, \tau)$.

      \item Supongamos que $(\Sigma, \tau) \not\vdash \lnot \varphi$ y supongamos que $(\Sigma \ \cup \ \{\varphi\},
        \tau) \vdash (\psi \wedge \lnot \psi)$. Por (5), tenemos que $(\Sigma, \tau) \vdash (\varphi \rightarrow (\psi
        \wedge \lnot \psi))$. La siguiente prueba atestigua que $(\Sigma, \tau) \vdash \lnot \varphi$:
        \[
          \begin{array}{llll}
            1. & \varphi \rightarrow (\psi \wedge \lnot \psi) && \text{AXIOMAPROPIO} \\
            2. & \lnot\lnot\varphi \leftrightarrow \varphi && \text{AXIOMALOGICO} \\
            3. & \lnot\lnot\varphi \leftrightarrow (\psi \wedge \lnot \psi) && \text{REEMPLAZO}(2,1) \\
            4. & \lnot\lnot\varphi \rightarrow (\psi \wedge \lnot \psi) && \text{EQUIVALENCIAELIMINACION}(3) \\
            5. & \lnot\lnot\lnot\varphi && \text{ABSURDO}(4) \\
            6. & \lnot\varphi && \text{REEMPLAZO}(2,5) \\
          \end{array}
        \]
        \PN lo cual es absurdo y por lo tanto $(\Sigma \ \cup \ \{\varphi\}, \tau)$ es consistente.
    \end{enumerate}
  \end{proof}

  % Lemma 162. Sin prueba. Theorem 72.
  \begin{theorem} \label{theorem_72}
    \PN \textbf{(Corrección)} $(\Sigma, \tau) \vdash \varphi$ implica $(\Sigma, \tau) \models \varphi$.
  \end{theorem}

  % Lemma 163. Con prueba. Corollary 73.
  \begin{corollary} \label{corollary_73}
    \PN Si $(\Sigma, \tau)$ tiene un modelo, entonces $(\Sigma, \tau)$ es consistente.
  \end{corollary}
  \begin{proof}
    \PN Supongamos $\mathbf{A}$ es un modelo de $(\Sigma, \tau)$. Si $(\Sigma, \tau)$ fuera inconsistente, tendriamos
    que hay una $\varphi \in S^{\tau}$ tal que $(\Sigma, \tau) \vdash (\varphi \wedge \lnot \varphi)$, lo cual por
    el \textbf{Theorem~\ref{theorem_72}}, obtendríamos que $\mathbf{A} \models (\varphi \wedge \lnot \varphi)$, lo cual
    es un absurdo y vino de suponer que $(\Sigma, \tau)$ era inconsistente.
  \end{proof}

  % Lemma 164. Con prueba. Lemma 74.
  \begin{lemma} \label{lemma_74}
    \PN $\dashv \vdash_{T}$ es una relación de equivalencia.
  \end{lemma}
  \begin{proof}
    \PN \newline
    \begin{itemize}
      \item \textit{Reflexiva:} La relación es reflexiva ya que $(\varphi \leftrightarrow \varphi)$ es un axioma
        lógico, y por lo tanto $((\varphi \leftrightarrow \varphi), \text{AXIOMALOGICO})$ es una prueba de $(\varphi
        \leftrightarrow \varphi)$ en $T$.
      \item \textit{Simétrica:} Supongamos que $\varphi \dashv \vdash_{T} \psi$, es decir $T \vdash (\varphi
        \leftrightarrow \psi)$. Ya que $(\varphi \leftrightarrow \psi)$ se deduce de $\psi \leftrightarrow \varphi$ por
        la regla de conmutatividad, el \textbf{Lemma~\ref{lemma_71}}, nos dice que $T \vdash (\psi \leftrightarrow
        \varphi)$, es decir, $\psi \dashv \vdash_{T} \varphi$.
      \item \textit{Transitiva:} Supongamos que $\varphi \dashv \vdash_{T} \psi$ y que $\psi \dashv \vdash_{T} \phi$, es
        decir $T \vdash (\varphi \leftrightarrow \psi)$ y $T \vdash (\psi \leftrightarrow \phi)$. Ya que $(\varphi
        \leftrightarrow \phi)$ se deduce de $\varphi \leftrightarrow \psi$ y $\psi \leftrightarrow \phi$ por
        la regla de transitividad, el \textbf{Lemma~\ref{lemma_71}}, nos dice que $T \vdash (\varphi \leftrightarrow
        \phi)$, es decir, $\varphi \dashv \vdash_{T} \phi$.
    \end{itemize}
  \end{proof}

  % Lemma 165. Con prueba. Lemma 75.
  \begin{lemma} \label{lemma_75}
    \PN Dada una teoria $T = (\Sigma, \tau)$, se tiene que:
    \begin{enumerate}[(1)]
      \item $\{\varphi \in S^{\tau}: \varphi \ \text{es un teorema de T}\} \in S^{\tau}/ \dashv \vdash_{T}$
      \item $\{\varphi \in S^{\tau}: \varphi \ \text{es refutable en T}\} \in S^{\tau}/ \dashv \vdash_{T}$
    \end{enumerate}
  \end{lemma}
  \begin{proof}
    \begin{enumerate}[(1)]
      \item $\{\varphi \in S^{\tau}: \varphi \ \text{es un teorema de T}\} \in S^{\tau}/ \dashv \vdash_{T}$: Sean
        $\varphi, \psi$ teoremas en $T$, veremos que $\varphi \dashv \vdash_{T} \psi$. Notese que
        \[
          \begin{array}{llll}
            1. & \varphi && \text{HIPOTESIS1} \\
            2. & \psi && \text{TESIS1AXIOMAPROPIO} \\
            3. & (\varphi \rightarrow \psi) && \text{CONCLUSION} \\
            4. & \psi && \text{HIPOTESIS2} \\
            5. & \varphi && \text{TESIS2AXIOMAPROPIO} \\
            6. & (\psi \rightarrow \varphi) && \text{CONCLUSION} \\
            7. & (\varphi \leftrightarrow \psi) && \text{EQUIVALENCIAINTRODUCCION}(3,6)
          \end{array}
        \]
        \PN justifica que $(\Sigma \ \cup \ \{\varphi, \psi\}, \tau) \vdash (\varphi \leftrightarrow \psi)$ lo cual por
        el \textbf{Lemma~\ref{lemma_71}} tenes que $(\Sigma, \tau) \vdash (\varphi \leftrightarrow \psi)$, obteniendo
        que $\varphi \dashv \vdash_{T} \psi$.
      \item $\{\varphi \in S^{\tau}: \varphi \ \text{es refutable en T}\} \in S^{\tau}/ \dashv \vdash_{T}$: Sean
        $\varphi, \psi$ refutables en $T$, veremos que $\varphi \dashv \vdash_{T} \psi$. Notese que
        \[
          \begin{array}{llll}
            1. & \varphi && \text{HIPOTESIS1} \\
            2. & \lnot \psi && \text{HIPOTESIS2} \\
            3. & \lnot \varphi && \text{AXIOMAPROPIO} \\
            4. & (\varphi \wedge \lnot \varphi) && \text{TESIS2CONJUNCIONINTRODUCCION}(1,3) \\
            5. & \lnot \psi \rightarrow (\varphi \wedge \lnot \varphi) && \text{CONCLUSION} \\
            6. & \psi && \text{TESIS1ABSURDO}(5) \\
            7. & (\varphi \rightarrow \psi) && \text{CONCLUSION} \\
            8. & \psi && \text{HIPOTESIS3} \\
            9. & \lnot \varphi && \text{HIPOTESIS4} \\
            10. & \lnot \psi && \text{AXIOMAPROPIO} \\
            11. & (\psi \wedge \lnot \psi) && \text{TESIS4CONJUNCIONINTRODUCCION}(8,10) \\
            12. & \lnot \varphi \rightarrow (\psi \wedge \lnot \psi) && \text{CONCLUSION} \\
            13. & \varphi && \text{TESIS}3\text{ABSURDO}(5) \\
            14. & (\psi \rightarrow \varphi) && \text{CONCLUSION} \\
            15. & (\varphi \leftrightarrow \psi) && \text{EQUIVALENCIAINTRODUCCION}(7,14)
          \end{array}
        \]
       \PN justifica que $(\Sigma \ \cup \ \{\lnot \varphi, \lnot \psi\}, \tau) \vdash (\varphi \leftrightarrow \psi)$
       lo ual por el \textbf{Lemma~\ref{lemma_71}} tenes que $(\Sigma, \tau) \vdash (\varphi \leftrightarrow \psi)$,
       obteniendo que $\varphi \dashv \vdash_{T} \psi$.
    \end{enumerate}
  \end{proof}

  % TODO: detalle
  % Lemma 166. Con prueba. Lemma 76.
  \begin{lemma} \label{lemma_76}
    \PN Sea $T = (\Sigma, \tau)$ una teoría, entonces $(S^{\tau}/\mathrm{\dashv \vdash}, \SU^{T}, \IN^{T}, 0^{T},
    1^{T})$ es un álgebra de Boole.
  \end{lemma}
  \begin{proof}
    \PN Por definición de Álgebra de Boole, debemos probar que cualesquiera sean $\varphi_{1}, \varphi_{2}, \varphi_{3}
    \in S^{\tau}$, se cumplen las siguientes igualdades:
    \begin{enumerate}[(1)]
      \item $[\varphi_{1}]_{T} \ \IN^{T} \ [\varphi_{1}]_{T} = [\varphi_{1}]_{T}$: Sea $\varphi_{1} \in S^{\tau}$ fija.
        Por la definición de la operación $\IN^{T}$ debemos probar que:
        \[
          [(\varphi_{1} \wedge \varphi_{1})]_{T} = [\varphi_{1}]_{T}
        \]
        \PN es decir, debemos probar que $T \vdash ((\varphi_{1} \wedge \varphi_{1}) \leftrightarrow \varphi_{1})$.
        \[
          \begin{array}{llll}
            1. & (\varphi_{1} \wedge \varphi_{1}) && \text{HIPOTESIS1} \\
            2. & \varphi_{1} && \text{TESIS1CONJUNCIONELIMINACION}(1) \\
            3. & (\varphi_{1} \wedge \varphi_{1}) \rightarrow \varphi_{1} && \text{CONCLUSION} \\
            4. & \varphi_{1} && \text{HIPOTESIS2} \\
            5. & (\varphi_{1} \wedge \varphi_{1}) && \text{TESIS2CONJUNCIONINTRODUCCION}(4) \\
            6. & \varphi_{1} \rightarrow (\varphi_{1} \wedge \varphi_{1}) && \text{CONCLUSION} \\
            7. & (\varphi_{1} \wedge \varphi_{1}) \leftrightarrow \varphi_{1} && \text{EQUIVALENCIAINTRODUCCION}(3,6)
          \end{array}
        \]

      \item $[\varphi_{1}]_{T} \ \SU^{T} \ [\varphi_{1}]_{T} = [\varphi_{1}]_{T}$: Sea $\varphi_{1} \in S^{\tau}$ fija.
        Por la definición de la operación $\SU^{T}$ debemos probar que:
        \[
          [(\varphi_{1} \vee \varphi_{1})]_{T} = [\varphi_{1}]_{T}
        \]
        \PN es decir, debemos probar que $T \vdash ((\varphi_{1} \vee \varphi_{1}) \leftrightarrow \varphi_{1})$.
        \[
          \begin{array}{llll}
            1. & (\varphi_{1} \vee \varphi_{1}) && \text{HIPOTESIS1} \\
            2. & \varphi_{1} \leftrightarrow \varphi_{1} && \text{AXIOMALOGICO} \\
            3. & \varphi_{1} \rightarrow \varphi_{1} && \text{EQUIVALENCIAELIMINACION}(2) \\
            4. & \varphi_{1} && \text{TESIS1DIVISIONPORCASOS}(1,3,3) \\
            5. & (\varphi_{1} \vee \varphi_{1}) \rightarrow \varphi_{1} && \text{CONCLUSION} \\
            6. & \varphi_{1} && \text{HIPOTESIS3} \\
            7. & (\varphi_{1} \vee \varphi_{1}) && \text{TESIS3DISJUNCIONINTRODUCCION}(6) \\
            8. & \varphi \rightarrow (\varphi_{1} \vee \varphi_{1}) && \text{CONCLUSION} \\
            9. & (\varphi_{1} \vee \varphi_{1}) \leftrightarrow \varphi_{1} && \text{EQUIVALENCIAINTRODUCCION}(5,8)
          \end{array}
        \]

      \item $[\varphi_{1}]_{T} \ \IN^{T} \ [\varphi_{2}]_{T} = [\varphi_{2}]_{T} \ \IN^{T} \ [\varphi_{1}]_{T}$: Sean
        $\varphi_{1}, \varphi_{2} \in S^{\tau}$ fijas. Por la definición de la operación $\IN^{T}$ debemos probar que:
        \[
          [(\varphi_{1} \wedge \varphi_{2})]_{T} = [(\varphi_{2} \wedge \varphi_{1})]_{T}
        \]
        \PN es decir, debemos probar que $T \vdash ((\varphi_{1} \wedge \varphi_{2}) \leftrightarrow (\varphi_{2} \wedge
        \varphi_{1}))$.
        \[
          \begin{array}{llll}
            1. & (\varphi_{1} \wedge \varphi_{2}) && \text{HIPOTESIS1} \\
            2. & \varphi_{1} && \text{CONJUNCIONELIMINACION}(1) \\
            3. & \varphi_{2} && \text{CONJUNCIONELIMINACION}(1) \\
            4. & (\varphi_{2} \wedge \varphi_{1}) && \text{TESIS1CONJUNCIONINTRODUCCION}(3,2) \\
            5. & (\varphi_{1} \wedge \varphi_{2}) \rightarrow (\varphi_{2} \wedge \varphi_{1}) && \text{CONCLUSION} \\
            6. & (\varphi_{2} \wedge \varphi_{1}) && \text{HIPOTESIS2} \\
            7. & \varphi_{2} && \text{CONJUNCIONELIMINACION}(6) \\
            8. & \varphi_{1} && \text{CONJUNCIONELIMINACION}(6) \\
            9. & (\varphi_{1} \wedge \varphi_{2}) && \text{TESIS1CONJUNCIONINTRODUCCION}(8,7) \\
            10. & (\varphi_{2} \wedge \varphi_{1}) \rightarrow (\varphi_{1} \wedge \varphi_{2}) && \text{CONCLUSION} \\
            11. & (\varphi_{1} \wedge \varphi_{2}) \leftrightarrow (\varphi_{2} \wedge \varphi_{1}) &&
              \text{EQUIVALENCIAINTRODUCCION}(5,10)
          \end{array}
        \]

      \item $[\varphi_{1}]_{T} \ \SU^{T} \ [\varphi_{2}]_{T} = [\varphi_{2}]_{T} \ \SU^{T} \ [\varphi_{1}]_{T}$: Sean
        $\varphi_{1}, \varphi_{2} \in S^{\tau}$ fijas. Por la definición de la operación $\SU^{T}$ debemos probar que:
        \[
          [(\varphi_{1} \vee \varphi_{2})]_{T} = [(\varphi_{2} \vee \varphi_{1})]_{T}
        \]
        \PN es decir, debemos probar que $T \vdash ((\varphi_{1} \vee \varphi_{2}) \leftrightarrow (\varphi_{2} \vee
        \varphi_{1}))$.
        \[
          \begin{array}{llll}
            1. & (\varphi_{1} \vee \varphi_{2}) && \text{HIPOTESIS1} \\
            2. & \varphi_{1} && \text{HIPOTESIS2} \\
            3. & (\varphi_{2} \vee \varphi_{1}) && \text{TESIS2DISJUNCIONINTRODUCCION}(2) \\
            4. & \varphi_{1} \rightarrow (\varphi_{2} \vee \varphi_{1}) && \text{CONCLUSION} \\
            5. & \varphi_{2} && \text{HIPOTESIS3} \\
            6. & (\varphi_{2} \vee \varphi_{1}) && \text{TESIS3DISJUNCIONINTRODUCCION}(2) \\
            7. & \varphi_{2} \rightarrow (\varphi_{2} \vee \varphi_{1}) && \text{CONCLUSION} \\
            8. & (\varphi_{2} \vee \varphi_{1}) && \text{TESIS1DIVISIONPORCASOS}(1,4,7) \\
            9. & (\varphi_{1} \vee \varphi_{2}) \rightarrow (\varphi_{2} \vee \varphi_{1}) && \text{CONCLUSION} \\
            10. & (\varphi_{2} \vee \varphi_{1}) && \text{HIPOTESIS4} \\
            11. & \varphi_{2} && \text{HIPOTESIS5} \\
            12. & (\varphi_{1} \vee \varphi_{2}) && \text{TESIS5DISJUNCIONINTRODUCCION}(11) \\
            13. & \varphi_{2} \rightarrow (\varphi_{1} \vee \varphi_{2}) && \text{CONCLUSION} \\
            14. & \varphi_{1} && \text{HIPOTESIS6} \\
            15. & (\varphi_{1} \vee \varphi_{2}) && \text{TESIS6DISJUNCIONINTRODUCCION}(14) \\
            16. & \varphi_{1} \rightarrow (\varphi_{1} \vee \varphi_{2}) && \text{CONCLUSION} \\
            17. & (\varphi_{1} \vee \varphi_{2}) && \text{TESIS4DIVISIONPORCASOS}(10,13,16) \\
            18. & (\varphi_{2} \vee \varphi_{1}) \rightarrow (\varphi_{1} \vee \varphi_{2}) && \text{CONCLUSION} \\
            19. & (\varphi_{1} \vee \varphi_{2}) \leftrightarrow (\varphi_{2} \vee \varphi_{1}) &&
              \text{EQUIVALENCIAINTRODUCCION}(9,18)
          \end{array}
        \]

      \item $[\varphi_{1}]_{T} \ \IN^{T}([\varphi_{2}]_{T} \ \IN^{T} \ [\varphi_{3}]_{T}) = ([\varphi_{1}]_{T} \ \IN^{T}
        \ [\varphi_{2}]_{T}) \ \IN^{T} \ [\varphi_{3}]_{T}$: Sean $\varphi_{1}, \varphi_{2}, \varphi_{3} \in
        S^{\tau}$ fijas. Por la definición de la operación $\IN^{T}$ debemos probar que:
        \[
          [(\varphi_{1} \wedge (\varphi_{2} \wedge \varphi_{3}))]_{T} = [((\varphi_{1} \wedge \varphi_{2}) \wedge
          \varphi_{3})]_{T}
        \]
        \PN es decir, debemos probar que $T \vdash ((\varphi_{1} \wedge (\varphi_{2} \wedge \varphi_{3}))
        \leftrightarrow ((\varphi_{1} \wedge \varphi_{2}) \wedge \varphi_{3}))$.
        \[
          \begin{array}{llll}
            1. & (\varphi_{1} \wedge (\varphi_{2} \wedge \varphi_{3})) && \text{HIPOTESIS1} \\
            2. & \varphi_{1} && \text{CONJUNCIONELIMINACION}(1) \\
            3. & (\varphi_{2} \wedge \varphi_{3}) && \text{CONJUNCIONELIMINACION}(1) \\
            4. & \varphi_{2} && \text{CONJUNCIONELIMINACION}(3) \\
            5. & \varphi_{3} && \text{CONJUNCIONELIMINACION}(3) \\
            6. & (\varphi_{1} \wedge \varphi_{2}) && \text{CONJUNCIONINTRODUCCION}(2,4) \\
            7. & ((\varphi_{1} \wedge \varphi_{2}) \wedge \varphi_{3}) && \text{TESIS1CONJUNCIONINTRODUCCION}(6,5) \\
            8. & (\varphi_{1} \wedge (\varphi_{2} \wedge \varphi_{3})) \rightarrow ((\varphi_{1} \wedge \varphi_{2})
              \wedge \varphi_{3}) && \text{CONCLUSION} \\
            9. & ((\varphi_{1} \wedge \varphi_{2}) \wedge \varphi_{3}) && \text{HIPOTESIS2} \\
            10. & (\varphi_{1} \wedge \varphi_{2}) && \text{CONJUNCIONELIMINACION}(9) \\
            11. & \varphi_{3} && \text{CONJUNCIONELIMINACION}(9) \\
            12. & \varphi_{1} && \text{CONJUNCIONELIMINACION}(10) \\
            13. & \varphi_{2} && \text{CONJUNCIONELIMINACION}(10) \\
            14. & (\varphi_{2} \wedge \varphi_{3}) && \text{CONJUNCIONINTRODUCCION}(13,11) \\
            15. & (\varphi_{1} \wedge (\varphi_{2} \wedge \varphi_{3})) && \text{TESIS1CONJUNCIONINTRODUCCION}(12,14) \\
            16. & ((\varphi_{1} \wedge \varphi_{2}) \wedge \varphi_{3}) \rightarrow (\varphi_{1} \wedge (\varphi_{2}
              \wedge \varphi_{3})) && \text{CONCLUSION} \\
            17. & (\varphi_{1} \wedge (\varphi_{2} \wedge \varphi_{3})) \leftrightarrow ((\varphi_{1} \wedge
              \varphi_{2}) \wedge \varphi_{3}) && \text{EQUIVALENCIAINTRODUCCION}(8,16) \\
          \end{array}
        \]

      \item $[\varphi_{1}]_{T} \ \SU^{T} \ ([\varphi_{2}]_{T} \ \SU^{T} \ [\varphi_{3}]_{T}) = ([\varphi_{1}]_{T} \
        \SU^{T} \ [\varphi_{2}]_{T}) \ \SU^{T} \ [\varphi_{3}]_{T}$: Sean $\varphi_{1}, \varphi_{2}, \varphi_{3} \in
        S^{\tau}$ fijas. Por la definición de la operación $\SU^{T}$ debemos probar que:
        \[
          [(\varphi_{1} \vee (\varphi_{2} \vee \varphi_{3}))]_{T} = [((\varphi_{1} \vee \varphi_{2}) \vee
          \varphi_{3})]_{T}
        \]
        \PN es decir, debemos probar que $T \vdash ((\varphi_{1} \vee (\varphi_{2} \vee \varphi_{3})) \leftrightarrow
        ((\varphi_{1} \vee \varphi_{2}) \vee \varphi_{3}))$. Notese que por el \textbf{Lemma~\ref{lemma_71}}, basta con
        probar que:
        \[
          \begin{array}{rcl}
            T & \vdash & ((\varphi_{1} \vee (\varphi_{2} \vee \varphi_{3})) \rightarrow ((\varphi_{1} \vee \varphi_{2})
              \vee \varphi_{3})) \\
            T & \vdash & (((\varphi_{1} \vee \varphi_{2}) \vee \varphi_{3}) \rightarrow (\varphi_{1} \vee (\varphi_{2}
              \vee \varphi_{3})))
          \end{array}
        \]
        \PN Prueba de $((\varphi_{1} \vee (\varphi_{2} \vee \varphi_{3})) \rightarrow ((\varphi_{1} \vee \varphi_{2})
        \vee \varphi_{3}))$
        \[
          \begin{array}{llll}
            1. & (\varphi_{1} \vee (\varphi_{2} \vee \varphi_{3})) && \text{HIPOTESIS1} \\
            2. & \varphi_{1} && \text{HIPOTESIS2} \\
            3. & (\varphi_{1} \vee \varphi_{2}) && \text{DISJUNCIONINTRODUCCION}(2) \\
            4. & ((\varphi_{1} \vee \varphi_{2}) \vee \varphi_{3}) && \text{TESIS2DISJUNCIONINTRODUCCION}(3) \\
            5. & \varphi_{1} \rightarrow ((\varphi_{1} \vee \varphi_{2}) \vee \varphi_{3}) && \text{CONCLUSION} \\
            6. & (\varphi_{2} \vee \varphi_{3}) && \text{HIPOTESIS3} \\
            7. & \varphi_{2} && \text{HIPOTESIS4} \\
            8. & (\varphi_{1} \vee \varphi_{2}) && \text{DISJUNCIONINTRODUCCION}(7) \\
            9. & ((\varphi_{1} \vee \varphi_{2}) \vee \varphi_{3}) && \text{TESIS4DISJUNCIONINTRODUCCION}(8) \\
            10. & \varphi_{2} \rightarrow ((\varphi_{1} \vee \varphi_{2}) \vee \varphi_{3}) && \text{CONCLUSION} \\
            11. & \varphi_{3} && \text{HIPOTESIS5} \\
            12. & ((\varphi_{1} \vee \varphi_{2}) \vee \varphi_{3}) && \text{TESIS5DISJUNCIONINTRODUCCION}(11) \\
            13. & \varphi_{3} \rightarrow ((\varphi_{1} \vee \varphi_{2}) \vee \varphi_{3}) && \text{CONCLUSION} \\
            14. & ((\varphi_{1} \vee \varphi_{2}) \vee \varphi_{3}) && \text{TESIS3DIVISIONPORCASOS}(6,10,13) \\
            15. & (\varphi_{2} \vee \varphi_{3}) \rightarrow ((\varphi_{1} \vee \varphi_{2}) \vee \varphi_{3}) &&
              \text{CONCLUSION} \\
            16. & ((\varphi_{1} \vee \varphi_{2}) \vee \varphi_{3}) && \text{TESIS1DIVISIONPORCASOS}(1,5,15) \\
            17. & (\varphi_{1} \vee (\varphi_{2} \vee \varphi_{3})) \rightarrow ((\varphi_{1} \vee \varphi_{2}) \vee
              \varphi_{3}) && \text{CONCLUSION}
          \end{array}
        \]

        \PN Prueba de $((\varphi_{1} \vee \varphi_{2}) \vee \varphi_{3})) \rightarrow ((\varphi_{1} \vee (\varphi_{2}
        \vee \varphi_{3}))$
        \[
          \begin{array}{llll}
            1. & ((\varphi_{1} \vee \varphi_{2}) \vee \varphi_{3}) && \text{HIPOTESIS1} \\
            2. & \varphi_{3} && \text{HIPOTESIS2} \\
            3. & (\varphi_{2} \vee \varphi_{3}) && \text{DISJUNCIONINTRODUCCION}(2) \\
            4. & (\varphi_{1} \vee (\varphi_{2} \vee \varphi_{3})) && \text{TESIS2DISJUNCIONINTRODUCCION}(3) \\
            5. & \varphi_{3} \rightarrow (\varphi_{1} \vee (\varphi_{2} \vee \varphi_{3})) && \text{CONCLUSION} \\
            6. & (\varphi_{1} \vee \varphi_{2}) && \text{HIPOTESIS3} \\
            7. & \varphi_{1} && \text{HIPOTESIS4} \\
            8. & (\varphi_{1} \vee (\varphi_{2} \vee \varphi_{3})) && \text{TESIS4DISJUNCIONINTRODUCCION}(7) \\
            9. & \varphi_{1} \rightarrow (\varphi_{1} \vee (\varphi_{2} \vee \varphi_{3})) && \text{CONCLUSION} \\
            10. & \varphi_{2} && \text{HIPOTESIS5} \\
            11. & (\varphi_{2} \vee \varphi_{3}) && \text{DISJUNCIONINTRODUCCION}(10) \\
            12. & (\varphi_{1} \vee (\varphi_{2} \vee \varphi_{3})) && \text{TESIS5DISJUNCIONINTRODUCCION}(11) \\
            13. & \varphi_{2} \rightarrow (\varphi_{1} \vee (\varphi_{2} \vee \varphi_{3})) && \text{CONCLUSION} \\
            14. & (\varphi_{1} \vee (\varphi_{2} \vee \varphi_{3})) && \text{TESIS3DIVISIONPORCASOS}(6,9,13) \\
            15. & (\varphi_{1} \vee \varphi_{2}) \rightarrow (\varphi_{1} \vee (\varphi_{2} \vee \varphi_{3})) &&
              \text{CONCLUSION} \\
            16. & (\varphi_{1} \vee (\varphi_{2}) \vee \varphi_{3})) && \text{TESIS1DIVISIONPORCASOS}(1,15,5) \\
            17. & ((\varphi_{1} \vee \varphi_{2}) \vee \varphi_{3}) \rightarrow (\varphi_{1} \vee (\varphi_{2} \vee
              \varphi_{3})) && \text{CONCLUSION}
          \end{array}
        \]

      \item $[\varphi_{1}]_{T} \ \SU^{T} \ ([\varphi_{1}]_{T} \ \IN^{T} \ [\varphi_{2}]_{T}) = [\varphi_{1}]_{T}$: Sean
        $\varphi_{1}, \varphi_{2} \in S^{\tau}$ fijas. Por la definición de la operación $\SU^{T}$ debemos probar que:
        \[
          [(\varphi_{1} \vee (\varphi_{1} \wedge \varphi_{2}))]_{T} = [\varphi_{1}]_{T}
        \]
        \PN es decir, debemos probar que $T \vdash ((\varphi_{1} \vee (\varphi_{1} \wedge \varphi_{2})) \leftrightarrow
        \varphi_{1})$.
        \[
        \begin{array}{llll}
          1. & (\varphi_{1} \vee (\varphi_{1} \wedge \varphi_{2})) && \text{HIPOTESIS1} \\
          2. & \varphi_{1} \leftrightarrow \varphi_{1} && \text{AXIOMALOGICO} \\
          3. & \varphi_{1} \rightarrow \varphi_{1} && \text{EQUIVALENCIAELIMINACION}(2) \\
          4. & (\varphi_{1} \wedge \varphi_{2}) && \text{HIPOTESIS2} \\
          5. & \varphi_{1} && \text{TESIS2CONJUNCIONELIMINACION}(4) \\
          6. & (\varphi_{1} \wedge \varphi_{2}) \rightarrow \varphi_{1} && \text{CONCLUSION} \\
          7. & \varphi_{1} && \text{TESIS1DIVISIONPORCASOS}(1,3,6) \\
          8. & (\varphi_{1} \vee (\varphi_{1} \wedge \varphi_{2})) \rightarrow \varphi_{1} && \text{CONCLUSION} \\
          9. & \varphi_{1} && \text{HIPOTESIS3} \\
          10. & (\varphi_{1} \vee (\varphi_{1} \wedge \varphi_{2} )) && \text{TESIS3DISJUNCIONINTRODUCCION}(9) \\
          11. & \varphi_{1} \rightarrow (\varphi_{1} \vee (\varphi \wedge \varphi_{2})) && \text{CONCLUSION} \\
          12. & ((\varphi_{1} \vee (\varphi_{1} \wedge \varphi_{2})) \leftrightarrow \varphi_{1}) &&
            \text{EQUIVALENCIAINTRODUCCION}(8,11)
        \end{array}
        \]

      \item $[\varphi_{1}]_{T} \ \IN^{T}([\varphi_{1}]_{T} \ \SU^{T} \ [\varphi_{2}]_{T}) = [\varphi_{1}]_{T}$: Sean
        $\varphi_{1}, \varphi_{2} \in S^{\tau}$ fijas. Por la definición de la operación $\IN^{T}$ debemos probar que:
        \[
          [(\varphi_{1} \wedge (\varphi_{1} \vee \varphi_{2}))]_{T} = [\varphi_{1}]_{T}
        \]
        \PN es decir, debemos probar que $T \vdash ((\varphi_{1} \wedge (\varphi_{1} \vee \varphi_{2})) \leftrightarrow
        \varphi_{1})$.
        \[
        \begin{array}{llll}
          1. & (\varphi_{1} \wedge (\varphi_{1} \vee \varphi_{2})) && \text{HIPOTESIS1} \\
          2. & \varphi_{1} && \text{TESIS1CONJUNCIONELIMINACION}(1) \\
          3. & (\varphi_{1} \wedge (\varphi_{1} \vee \varphi_{2})) \rightarrow \varphi_{1} && \text{CONCLUSION} \\
          4. & \varphi_{1} && \text{HIPOTESIS2} \\
          5. & (\varphi_{1} \vee \varphi_{2}) && \text{DISJUNCIONINTRODUCCION}(4) \\
          6. & \varphi_{1} \wedge (\varphi_{1} \vee \varphi_{2}) && \text{TESIS2CONJUNCIONINTRODUCCION}(4,5) \\
          7. & \varphi_{1} \rightarrow \varphi_{1} \wedge (\varphi_{1} \vee \varphi_{2}) && \text{CONCLUSION} \\
          8. & ((\varphi_{1} \wedge (\varphi_{1} \vee \varphi_{2})) \leftrightarrow \varphi_{1}) &&
            \text{EQUIVALENCIAINTRODUCCION}(3,7)
        \end{array}
        \]

      \item $0^{T} \ \SU^{T} \ [\varphi_{1}]_{T} = [\varphi_{1}]_{T}$: Ya que $(\varphi \wedge \lnot\varphi)$ es
        refutable en $T$, tenemos que el \textbf{Lemma~\ref{lemma_75}} nos dice que $0^{T} =
        \{\varphi \in S^{\tau}: \varphi$ es refutable de $T\} = [(\varphi \wedge \lnot\varphi)]_{T}$, es decir,
        que debemos probar que para cualquier $\varphi_{1} \in S^{\tau}$, se da que:
        \[
          [0]_{T} \ \IN^{T} \ [(\varphi \wedge \lnot\varphi)]_{T} = \{\varphi \in S^{\tau}: \varphi \ \text{es refutable
          en} \ T\}
        \]
       \PN Ya que $[\varphi_{1}]_{T} \ \SU^{T} \ [\forall x_{1} (x_{1} \equiv x_{1})]_{T} = [\varphi_{1} \vee \forall
       x_{1}(x_{1}\equiv x_{1})]_{T}$, debemos probar que $\varphi_{1} \vee \forall x_{1} (x_{1} \equiv x_{1})$ es un
       teorema de $T$, lo cual es atestiguado por la siguiente prueba
        \[
          \begin{array}{llll}
            1. & c \equiv c && \text{AXIOMALOGICO} \\
            2. & \forall x_{1} (x_{1} \equiv x_{1}) && \text{GENERALIZACION}(1) \\
            3. & (\varphi_{1} \vee \forall x_{1} (x_{1}\equiv x_{1})) && \text{DISJUNCIONINTRODUCCION}(2)
          \end{array}
        \]

      \item $[\varphi_{1}]_{T} \ \SU^{T} \ 1^{T} = 1^{T}$: Ya que $\forall x_{1} (x_{1} \equiv x_{1})$ es un teorema de
        $T$, atestiguado por la prueba
        \[
          \begin{array}{llll}
            1. & c \equiv c && \text{AXIOMALOGICO} \\
            2. & \forall x_{1} (x_{1} \equiv x_{1}) && \text{GENERALIZACION}(1)
          \end{array}
        \]
        \PN donde $c$ es un nombre de constante que no pertenece a $\mathcal{C}$ y tal que $(\mathcal{C} \cup \{c\},
        \mathcal{F}, \mathcal{R}, a)$ es un tipo. Tenemos que el \textbf{Lemma~\ref{lemma_75}} nos dice que $1^{T} =
        \{\varphi \in S^{\tau}: \varphi$ es un teorema de $T\} = [\forall x_{1} (x_{1} \equiv x_{1})]_{T}$, es decir,
        que debemos probar que para cualquier $\varphi_{1} \in S^{\tau}$, se da que:
        \[
          [\varphi_{1}]_{T} \ \SU^{T} \ [\forall x_{1} (x_{1} \equiv x_{1})]_{T} = \{\varphi \in S^{\tau}: \varphi \
          \text{es un teorema de} \ T\}
        \]
       \PN Ya que $[\varphi_{1}]_{T} \ \SU^{T} \ [\forall x_{1} (x_{1} \equiv x_{1})]_{T} = [\varphi_{1} \vee \forall
       x_{1}(x_{1}\equiv x_{1})]_{T}$, debemos probar que $\varphi_{1} \vee \forall x_{1} (x_{1} \equiv x_{1})$ es un
       teorema de $T$, lo cual es atestiguado por la siguiente prueba
        \[
          \begin{array}{llll}
            1. & c \equiv c && \text{AXIOMALOGICO} \\
            2. & \forall x_{1} (x_{1} \equiv x_{1}) && \text{GENERALIZACION}(1) \\
            3. & (\varphi_{1} \vee \forall x_{1} (x_{1}\equiv x_{1})) && \text{DISJUNCIONINTRODUCCION}(2)
          \end{array}
        \]

      \item $[\varphi_{1}]_{T} \ \SU^{T} \ ([\varphi_{1}]_{T})^{\mathsf{c}^{T}} = 1^{T}$: Sea $\varphi_{1} \in S^{\tau}$
        fija. Ya que $\forall x_{1} (x_{1} \equiv x_{1})$ es un teorema de $T$ y por la definición de la operación
        $\SU^{T}$ debemos probar que:
        \[
          [(\varphi_{1} \vee \lnot\varphi_{1})]_{T} = [\forall x_{1} (x_{1} \equiv x_{1})]_{T}
        \]
        \PN es decir, debemos probar que $T \vdash ((\varphi_{1} \vee \lnot\varphi_{1}) \leftrightarrow \forall x_{1}
        (x_{1} \equiv x_{1}))$.
        \[
          \begin{array}{llll}
            1. & (\varphi_{1} \vee \lnot\varphi_{1}) && \text{HIPOTESIS1} \\
            2. & c \equiv c && \text{AXIOMALOGICO} \\
            3. & \forall x_{1} (x_{1} \equiv x_{1}) && \text{TESIS1GENERALIZACION}(2) \\
            4. & (\varphi_{1} \vee \lnot\varphi_{1}) \rightarrow \forall x_{1} (x_{1} \equiv x_{1}) &&
              \text{CONCLUSION} \\
            5. & \forall x_{1} (x_{1} \equiv x_{1}) && \text{HIPOTESIS2} \\
            6. & (\varphi_{1} \vee \lnot\varphi_{1}) && \text{TESIS2AXIOMALOGICO} \\
            7. & \forall x_{1} (x_{1} \equiv x_{1}) \rightarrow (\varphi_{1} \vee \lnot\varphi_{1}) &&
              \text{CONCLUSION} \\
            8. & (\varphi_{1} \vee \lnot\varphi_{1}) \leftrightarrow \forall x_{1} (x_{1} \equiv x_{1}) &&
              \text{EQUIVALENCIAINTRODUCCION}(8,16) \\
          \end{array}
        \]
        \PN donde $c$ es un nombre de constante que no pertenece a $\mathcal{C}$ y tal que $(\mathcal{C} \cup \{c\},
        \mathcal{F}, \mathcal{R}, a)$ es un tipo.

      \item $[\varphi_{1}]_{T} \ \IN^{T} \ ([\varphi_{1}]_{T})^{\mathsf{c}^{T}} = 0^{T}$

      \item $[\varphi_{1}]_{T} \ \IN^{T}([\varphi_{2}]_{T} \ \SU^{T} \ [\varphi_{3}]_{T}) = ([\varphi_{1}]_{T} \ \IN^{T}
        \ [\varphi_{2}]_{T}) \ \SU^{T} \  ([\varphi_{1}]_{T} \ \IN^{T} \ [\varphi_{3}]_{T})$: Sean $\varphi_{1},
        \varphi_{2}, \varphi_{3} \in S^{\tau}$ fijas. Por la definición de las operaciones $\SU^{T}, \IN^{T}$ debemos
        probar que:
          \[
            [(\varphi_{1} \wedge (\varphi_{2} \vee \varphi_{3}))]_{T} = [((\varphi_{1} \wedge \varphi_{2}) \vee
            (\varphi_{1} \wedge \varphi_{3}))]_{T}
          \]
          \PN es decir, debemos probar que $T \vdash (\varphi_{1} \wedge (\varphi_{2} \vee \varphi_{3})) \leftrightarrow
          ((\varphi_{1} \wedge \varphi_{2}) \vee (\varphi_{1} \wedge \varphi_{3}))$. Notese que por el
          \textbf{Lemma~\ref{lemma_71}}, basta con probar que:
          \[
            \begin{array}{rcl}
              T & \vdash & (\varphi_{1} \wedge (\varphi_{2} \vee \varphi_{3})) \rightarrow ((\varphi_{1} \wedge
                \varphi_{2}) \vee (\varphi_{1} \wedge \varphi_{3})) \\
              T & \vdash & ((\varphi_{1} \wedge \varphi_{2}) \vee (\varphi_{1} \wedge \varphi_{3})) \rightarrow
                (\varphi_{1} \wedge (\varphi_{2} \vee \varphi_{3}))
            \end{array}
          \]
          \PN Prueba de $(\varphi_{1} \wedge (\varphi_{2} \vee \varphi_{3})) \rightarrow ((\varphi_{1} \wedge
          \varphi_{2}) \vee (\varphi_{1} \wedge \varphi_{3}))$
          \[
            \begin{array}{llll}
              1. & (\varphi_{1} \wedge (\varphi_{2} \vee \varphi_{3})) && \text{HIPOTESIS1} \\
              2. & \varphi_{1} && \text{CONJUNCIONELIMINACION}(1) \\
              3. & (\varphi_{2} \vee \varphi_{3}) && \text{CONJUNCIONELIMINACION}(1) \\
              4. & \varphi_{2} && \text{HIPOTESIS2} \\
              5. & (\varphi_{1} \wedge \varphi_{2}) && \text{CONJUNCIONINTRODUCCION}(2,4) \\
              6. & ((\varphi_{1} \wedge \varphi_{2}) \vee (\varphi_{1} \wedge \varphi_{3})) &&
                \text{TESIS2DISJUNCIONINTRODUCCION}(5) \\
              7. & \varphi_{2} \rightarrow ((\varphi_{1} \wedge \varphi_{2}) \vee (\varphi_{1} \wedge \varphi_{3})) &&
                \text{CONCLUSION} \\
              8. & \varphi_{3} && \text{HIPOTESIS3} \\
              9. & (\varphi_{1} \wedge \varphi_{3}) && \text{CONJUNCIONINTRODUCCION}(2,8) \\
              10. & ((\varphi_{1} \wedge \varphi_{2}) \vee (\varphi_{1} \wedge \varphi_{3})) &&
                \text{TESIS3DISJUNCIONINTRODUCCION}(9) \\
              11. & \varphi_{3} \rightarrow ((\varphi_{1} \wedge \varphi_{2}) \vee (\varphi_{1} \wedge \varphi_{3})) &&
                \text{CONCLUSION} \\
              12. & ((\varphi_{1} \wedge \varphi_{2}) \vee (\varphi_{1} \wedge \varphi_{3})) &&
                \text{TESIS1DIVISIONPORCASOS}(3,7,11) \\
              13. & (\varphi_{1} \wedge (\varphi_{2} \vee \varphi_{3})) \rightarrow ((\varphi_{1} \wedge \varphi_{2})
                \vee (\varphi_{1} \wedge \varphi_{3})) && \text{CONCLUSION}
            \end{array}
          \]
          \PN Prueba de $((\varphi_{1} \wedge \varphi_{2}) \vee (\varphi_{1} \wedge \varphi_{3})) \rightarrow
            (\varphi_{1} \wedge (\varphi_{2} \vee \varphi_{3}))$
          \[
          \begin{array}{llll}
            1. & ((\varphi_{1} \wedge \varphi_{2}) \vee (\varphi_{1} \wedge \varphi_{3})) && \text{HIPOTESIS1} \\
            2. & (\varphi_{1} \wedge \varphi_{2}) && \text{HIPOTESIS2} \\
            3. & \varphi_{1} && \text{CONJUNCIONELIMINACION}(2) \\
            4. & \varphi_{2} && \text{CONJUNCIONELIMINACION}(2) \\
            5. & (\varphi_{2} \vee \varphi_{3}) && \text{DISJUNCIONINTRODUCCION}(4) \\
            6. & \varphi_{1} \wedge (\varphi_{2} \vee \varphi_{3}) && \text{TESIS2CONJUNCIONINTRODUCCION}(3,5) \\
            7. & (\varphi_{1} \wedge \varphi_{2})\rightarrow (\varphi_{1} \wedge (\varphi_{2} \vee \varphi_{3})) &&
              \text{CONCLUSION} \\
            8. & (\varphi_{1} \wedge \varphi_{3}) && \text{HIPOTESIS}3 \\
            9. & \varphi_{1} && \text{CONJUNCIONELIMINACION}(8) \\
            10. & \varphi_{3} && \text{CONJUNCIONELIMINACION}(8) \\
            11. & (\varphi_{2} \vee \varphi_{3}) && \text{DISJUNCIONINTRODUCCION}(10) \\
            12. & \varphi_{1} \wedge (\varphi_{2} \vee \varphi_{3}) && \text{TESIS3CONJUNCIONINTRODUCCION}(9,11) \\
            13. & (\varphi_{1} \wedge \varphi_{3}) \rightarrow (\varphi_{1} \wedge (\varphi_{2} \vee \varphi_{3})) &&
              \text{CONCLUSION} \\
            14. & (\varphi_{1} \wedge (\varphi_{2} \vee \varphi_{3})) && \text{TESIS1DIVISIONPORCASOS}(1,7,13) \\
            15. & ((\varphi_{1} \wedge \varphi_{2}) \vee (\varphi_{1} \wedge \varphi_{3})) \rightarrow (\varphi_{1}
              \wedge (\varphi_{2} \vee \varphi_{3})) && \text{CONCLUSION}
          \end{array}
          \]
    \end{enumerate}
  \end{proof}

  % Lemma 167. Con prueba. Lemma 77.
  \begin{lemma} \label{lemma_77}
    \PN Sea T una teoría y sea $\leq^{T}$ el orden parcial asociado al álgebra de Boole $\mathcal{A}_{T}$ (es decir
    $[\varphi]_{T} \leq^{T} [\psi]_{T}$ si y solo si $[\varphi]_{T} \ \SU^{T} \ [\psi]_{T} = [\psi]_{T})$, entonces se
    tiene que:
    \[
      [\varphi]_{T} \leq^{T} [\psi]_{T} \text{si y solo si} \ T \vdash (\varphi \rightarrow \psi)
    \]
  \end{lemma}
  \begin{proof}
    \PN \begin{tabular}{|c|} \hline $\Rightarrow$ \\\hline \end{tabular} Supongamos $[\varphi]_{T} \leq^{T} [\psi]_{T}$,
    es decir, $[\varphi]_{T} \ \SU^{T} \ [\psi]_{T} = [\psi]_{T}$. Por la definición de $\SU^{T}$, tenemos que
    $[(\varphi \vee \psi)]_{T} = [\psi]_{T}$, es decir, $T \vdash ((\varphi \vee \psi) \leftrightarrow \psi)$. Luego, la
    siguiente prueba atestigua que $T \vdash (\varphi \rightarrow \psi)$
    \[
      \begin{array}{llll}
        1. & \varphi && \text{HIPOTESIS1} \\
        2. & (\varphi \vee \psi) && \text{DISJUNCIONINTRODUCCION}(1) \\
        3. & (\varphi \vee \psi) \leftrightarrow \psi && \text{AXIOMAPROPIO} \\
        4. & (\varphi \vee \psi) \rightarrow \psi && \text{EQUIVALENCIAELIMINACION}(3) \\
        5. & \psi && \text{TESIS1MODUSPONENS}(2,4) \\
        6. & \varphi \rightarrow \psi && \text{CONCLUSION}
      \end{array}
    \]

    \PN \begin{tabular}{|c|} \hline $\Leftarrow$ \\\hline \end{tabular} Supongamos $T \vdash (\varphi \rightarrow \psi)$,
    la siguiente prueba atestigua que $T \vdash ((\varphi \vee \psi) \leftrightarrow \psi)$.
    \[
      \begin{array}{llll}
        1. & (\varphi \vee \psi) && \text{HIPOTESIS1} \\
        2. & \varphi \rightarrow \psi && \text{AXIOMAPROPIO} \\
        3. & \psi \leftrightarrow \psi && \text{AXIOMALOGICO} \\
        4. & \psi \rightarrow \psi && \text{EQUIVALENCIAELIMINACION}(3) \\
        5. & \psi && \text{TESIS1DIVISIONPORCASOS}(1,2,4) \\
        6. & (\varphi \vee \psi) \rightarrow \psi && \text{CONCLUSION} \\
        7. & \psi && \text{HIPOTESIS2} \\
        8. & (\varphi \vee \psi) && \text{TESIS2DISJUNCIONINTRODUCCION} \\
        9. & \psi \rightarrow (\varphi \vee \psi) && \text{CONCLUSION} \\
        10. & (\varphi \vee \psi) \leftrightarrow \psi && \text{EQUIVALENCIAINTRODUCCION}(6,9)
      \end{array}
    \]
    \PN Esto nos dice que $[\varphi \vee \psi]_{T} = [\psi]_{T}$, que por la definición de $\SU^{T}$, tenemos que
    $[\varphi]_{T} \ \SU^{T} \ [\psi]_{T} = [\psi]_{T}$, es decir, $[\varphi_{T}] \leq^{T} [\psi]_{T}$.
  \end{proof}

  % Lemma 168. Con prueba. Lemma 78.
  \begin{lemma} \label{lemma_78}
    \PN Sean $\tau = (\mathcal{C}, \mathcal{F}, \mathcal{R}, a)$ y $\tau^{\prime} = (\mathcal{C}^{\prime},
    \mathcal{F}^{\prime}, \mathcal{R}^{\prime}, a^{\prime})$ tipos.
    \begin{enumerate}
      \item Si $\mathcal{C} \subseteq \mathcal{C}^{\prime}, \mathcal{F} \subseteq \mathcal{F}^{\prime}, \mathcal{R}
      \subseteq \mathcal{R}^{\prime}$ y $a^{\prime}\mid_{\mathcal{F} \cup \mathcal{R}} = a$, entonces $(\Sigma, \tau)
      \vdash \varphi$ implica $(\Sigma, \tau^{\prime}) \vdash \varphi$.
      \item Si $\mathcal{C} \subseteq \mathcal{C}^{\prime}, \mathcal{F} = \mathcal{F}^{\prime}, \mathcal{R} =
      \mathcal{R}^{\prime}$ y $a^{\prime} = a$, entonces $(\Sigma, \tau^{\prime}) \vdash \varphi$ implica $(\Sigma,
      \tau) \vdash \varphi$, cada vez que $\Sigma \cup \{\varphi\} \subseteq S^{\tau}$.
    \end{enumerate}
  \end{lemma}
  \begin{proof}
    \PN \newline
    \begin{enumerate}[(1)]
      \item Supongamos $(\Sigma, \tau) \vdash \varphi$. Sea $(\varphi_{1} \dotsc \varphi_{n}, J_{1} \dotsc J_{n})$ la
        prueba de $\varphi$ en $(\Sigma, \tau)$. Sea $\mathcal{C}_{1}$ el conjunto de nombres de constante que ocurren
        en alguna $\varphi_{i}$ y que no pertenecen a $\mathcal{C}$. Notese que aplicando varias veces el
        \textbf{Lemma~\ref{lemma_70}}, podemos obtener una prueba $(\tilde{\varphi}_{1} \dotsc \tilde{\varphi}_{n},
        J_{1} \dotsc J_{n})$ de $\varphi$ en $(\Sigma, \tau)$ la cual cumple que los nombres de constante que ocurren en
        alguna $\psi_{i}$ y que no pertenecen a $\mathcal{C}$ no pertenecen a $\mathcal{C}^{\prime}$. Luego,
        $(\tilde{\varphi}_{1} \dotsc \tilde{\varphi}_{n}, J_{1} \dotsc J_{n})$ es una prueba de $\varphi$ en $(\Sigma,
        \tau^{\prime})$, con lo cual $(\Sigma, \tau^{\prime}) \vdash \varphi$.

      \item Supongamos $(\Sigma, \tau^{\prime}) \vdash \varphi$. Sea $(\pmb{\varphi}, \mathbf{J})$ de $\varphi$ en
        $(\Sigma, \tau^{\prime})$. Veremos que $(\pmb{\varphi}, \mathbf{J})$ es una prueba de $\varphi$ en $(\Sigma,
        \tau)$. Recordemos la definición de prueba:
        \PN Sea $(\Sigma, \tau)$ una teoría de primer orden. Sea $\varphi$ una sentencia de tipo $\tau$. Una
        \textbf{prueba} de $\varphi$ en $(\Sigma, \tau)$ será un par adecuado $(\mathbf{\varphi}, \mathbf{J})$ de algún
        tipo $\tau_{1} = (\mathcal{C} \cup \mathcal{C}_{1}, \mathcal{F}, \mathcal{R}, a)$, con $\mathcal{C}_{1}$ finito
        y disjunto con $\mathcal{C}$, tal que
        \begin{enumerate}[(1)]
          \item Cada $\mathbf{\varphi}_{i}$ es una sentencia de tipo $\tau_{1}$
          \item $\mathbf{\varphi}_{n(\mathbf{\varphi})} = \varphi$
          \item Si $\left\langle i, j \right\rangle \in \mathcal{B}^{\mathbf{J}}$, entonces $\mathbf{J}_{j+1}$ es de la
            forma $\alpha \mathrm{CONCLUSION}$ y $\mathbf{\varphi}_{j+1} = (\mathbf{\varphi}_{i} \rightarrow
            \mathbf{\varphi}_{j})$
          \item Para cada $i = 1, \dotsc, n(\mathbf{\varphi}),$ se da una de las siguientes:
          \begin{enumerate}[(a)]
            \item $\mathbf{J}_{i} = \mathrm{HIPOTESIS}\bar{k}$ para algún $k \in \mathbf{N}$
            \item $\mathbf{J}_{i}$ es de la forma $\alpha \mathrm{CONCLUSION}$ y hay un $j$ tal que $\left\langle j, i-1
              \right\rangle \in \mathcal{B}^{\mathbf{J}}$ y $\mathbf{\varphi}_{i} = (\mathbf{\varphi}_{j} \rightarrow
              \mathbf{\varphi}_{i-1})$
            \item $\mathbf{J}_{i}$ es de la forma $\alpha \mathrm{AXIOMALOGICO}$ y $\mathbf{\varphi}_{i}$ es un axioma
              lógico de tipo $\tau_{1}$
            \item $\mathbf{J}_{i}$ es de la forma $\alpha \mathrm{AXIOMAPROPIO}$ y $\mathbf{\varphi}_{i} \in \Sigma$
            \item $\mathbf{J}_{i}$ es de la forma $\alpha \mathrm{PARTICULARIZACION} (\bar{l})$, con $l$ anterior a $i$
              y $(\mathbf{\varphi}_{l}, \mathbf{\varphi}_{i}) \in Partic^{\tau_{1}}$
            \item $\mathbf{J}_{i}$ es de la forma $\alpha \mathrm{COMMUTATIVIDAD}(\bar{l})$, con $l$ anterior a $i$ y
              $(\mathbf{\varphi}_{l}, \mathbf{\varphi}_{i}) \in Commut^{\tau_{1}}$
            \item $\mathbf{J}_{i}$ es de la forma $\alpha \mathrm{ABSURDO}(\bar{l})$, con $l$ anterior a $i$ y
              $(\mathbf{\varphi}_{l}, \mathbf{\varphi}_{i}) \in Absur^{\tau_{1}}$
            \item $\mathbf{J}_{i}$ es de la forma $\alpha \mathrm{EVOCACION}(\bar{l} )$, con $l$ anterior a $i$ y
              $(\mathbf{\varphi}_{l}, \mathbf{\varphi}_{i}) \in Evoc^{\tau_{1}}$
            \item $\mathbf{J}_{i}$ es de la forma $\alpha \mathrm{EXISTENCIA}(\bar{l })$, con $l$ anterior a $i$ y
              $(\mathbf{\varphi}_{l}, \mathbf{\varphi}_{i}) \in Exist^{\tau_{1}}$
            \item $\mathbf{J}_{i}$ es de la forma $\alpha \mathrm{CONJUNCIONELIMINACION}(\bar{l})$, con $l$ anterior a
              $i$ y $(\mathbf{\varphi}_{l}, \mathbf{\varphi}_{i}) \in ConjElim^{\tau_{1}}$
            \item $\mathbf{J}_{i}$ es de la forma $\alpha \mathrm{DISJUNCIONINTRODUCCION}(\bar{l})$, con $l$ anterior a
              $i$ y $(\mathbf{\varphi}_{l}, \mathbf{\varphi}_{i}) \in DisjInt^{\tau_{1}}$
            \item $\mathbf{J}_{i}$ es de la forma $\alpha \mathrm{EQUIVALENCIAELIMINACION}(\bar{l})$, con $l$ anterior a
              $i$ y $(\mathbf{\varphi}_{l}, \mathbf{\varphi}_{i}) \in EquivElim^{\tau_{1}}$
            \item $\mathbf{J}_{i}$ es de la forma $\alpha \mathrm{MODUSPONENS}(\overline{l_{1}}, \overline{l_{2}})$, con
              $l_{1}$ y $l_{2}$ anteriores a $i$ y $(\mathbf{\varphi}_{l_{1}}, \mathbf{\varphi}_{l_{2}},
              \mathbf{\varphi}_{i}) \in ModPon^{\tau_{1}}$
            \item $\mathbf{J}_{i}$ es de la forma $\alpha \mathrm{CONJUNCIONINTRODUCCION}(\overline{l_{1}},
              \overline{l_{2}})$, con $l_{1}$ y $ l_{2}$ anteriores a $i$ y $(\mathbf{\varphi}_{l_{1}},
              \mathbf{\varphi}_{l_{2}}, \mathbf{\varphi}_{i}) \in ConjInt^{\tau_{1}}$
            \item $\mathbf{J}_{i}$ es de la forma $\alpha \mathrm{EQUIVALENCIAINTRODUCCION}(\overline{l_{1}},
              \overline{l_{2}})$, con $l_{1}$ y $l_{2}$ anteriores a $i$ y $(\mathbf{\varphi}_{l_{1}},
              \mathbf{\varphi}_{l_{2}}, \mathbf{\varphi}_{i}) \in EquivInt^{\tau_{1}}$
            \item $\mathbf{J}_{i}$ es de la forma $\alpha \mathrm{DISJUNCIONELIMINACION}(\overline{l_{1}},
              \overline{l_{2}})$, con $l_{1}$ y $ l_{2}$ anteriores a $i$ y $(\mathbf{\varphi}_{l_{1}},
              \mathbf{\varphi}_{l_{2}}, \mathbf{\varphi}_{i}) \in DisjElim^{\tau_{1}}$
            \item $\mathbf{J}_{i}$ es de la forma $\alpha \mathrm{REEMPLAZO}(\overline{l_{1}}, \overline{l_{2}})$, con
              $l_{1}$ y $l_{2}$ anteriores a $i$ y $(\mathbf{\varphi}_{l_{1}}, \mathbf{\varphi}_{l_{2}},
              \mathbf{\varphi}_{i}) \in Reemp^{\tau_{1}}$
            \item $\mathbf{J}_{i}$ es de la forma $\alpha \mathrm{TRANSITIVIDAD}(\overline{l_{1}}, \overline{l_{2}})$,
              con $l_{1}$ y $l_{2}$ anteriores a $i$ y $(\mathbf{\varphi}_{l_{1}}, \mathbf{\varphi}_{l_{2}},
              \mathbf{\varphi}_{i}) \in Trans^{\tau_{1}}$
            \item $\mathbf{J}_{i}$ es de la forma $\alpha \mathrm{DIVISIONPORCASOS}(\overline{l_{1}}, \overline{l_{2}},
              \overline{l_{3}})$, con $l_{1},l_{2}$ y $ l_{3}$ anteriores a $i$ y y $(\mathbf{\varphi}_{l_{1}},
              \mathbf{\varphi}_{l_{2}}, \mathbf{\varphi}_{l_{3}}, \mathbf{\varphi}_{i}) \in DivPorCas^{\tau_{1}}$
            \item $\mathbf{J}_{i}$ es de la forma $\alpha \mathrm{ELECCION}(\bar{l}) $, con $l$ anterior a $i$ y
              $(\mathbf{\varphi}_{l}, \mathbf{\varphi}_{i}) \in Elec^{\tau_{1}}$ vía el nombre de constante $e$, el cual
              no pertenece a $ \mathcal{C}$ y no ocurre en $\mathbf{\varphi}_{1}, \dotsc, \mathbf{\varphi}_{i-1}$.
            \item $\mathbf{J}_{i}$ es de la forma $\alpha \mathrm{GENERALIZACION}(\bar{l})$, con $l$ anterior a $i$ y
              $(\mathbf{\varphi}_{l}, \mathbf{\varphi}_{i}) \in Generaliz^{\tau_{1}}$ vía el nombre de constante $c$ el
              cual cumple:
              \begin{enumerate}[(i)]
                \item $c \not\in \mathcal{C}$
                \item $c$ no es un nombre de constante que ocurra en $\mathbf{\varphi}$ el cual sea introducido por la
                  aplicación de la regla de elección, es decir para cada $u \in \{1, \dotsc, n(\mathbf{\varphi})\}$, si
                  $\mathbf{J}_{u}$ es de la forma $\alpha \mathrm{ELECCION}(\bar{v})$, entonces no se da que
                  $(\mathbf{\varphi}_{v}, \mathbf{\varphi}_{u}) \in Elec^{\tau_{1}}$ vía $c$.
                \item $c$ no ocurre en ninguna hipotesis de $\mathbf{\varphi}_{l}$.
                \item Ningún nombre de constante que ocurra en $\mathbf{\varphi}_{l}$ o en sus hipotesis, depende de $c$.
              \end{enumerate}
          \end{enumerate}
        \end{enumerate}

        \PN Ya que $(\mathbf{\varphi}, \mathbf{J})$ es una prueba de $\varphi$ en $(\Sigma, \tau^{\prime})$ hay un
        conjunto finito $\mathcal{C}_{1}$, disjunto con $\mathcal{C}^{\prime}$, tal que $(\mathcal{C}^{\prime} \cup
        \mathcal{C}_{1}, \mathcal{F}, \mathcal{R}, a)$ es un tipo y cada $\mathbf{\varphi}_{i}$ es una sentencia de tipo
        $(\mathcal{C}^{\prime} \cup \mathcal{C}_{1}, \mathcal{F}, \mathcal{R}, a)$. Notese que
        $\widetilde{\mathcal{C}_{1}} = \mathcal{C}_{1} \cup (\mathcal{C}^{\prime}-\mathcal{C})$ es tal que $(\mathcal{C}
        \cup \widetilde{\mathcal{C}_{1}}, \mathcal{F}, \mathcal{R}, a)$ es un tipo y cada $\mathbf{\varphi}_{i}$ es una
        sentencia de tipo $(\mathcal{C} \cup \widetilde{\mathcal{C}_{1}}, \mathcal{F}, \mathcal{R}, a)$, con lo cual
        $(\mathbf{\varphi}, \mathbf{J})$ cumple el punto (1) de la definición de prueba. Con la excepción de los puntos
        4(f) y 4(g)i para los cuales es necesario notar que $\mathcal{C} \subseteq \mathcal{C}^{\prime}$, todos los
        demás puntos se cumplen en forma directa.
    \end{enumerate}
  \end{proof}

  % Lemma 169. Sin prueba. Lemma 79.
  \begin{lemma} \label{lemma_79}
    \PN Sea $(\Sigma, \tau)$ una teoría y supongamos que $\tau$ tiene una cantidad infinita de nombres de constante que
    no ocurren en las sentencias de $\Sigma$, entonces para cada formula $\varphi =_{d} \varphi(v)$, se tiene que
    $\lbrack \forall v\varphi (v) \rbrack_{T} = \inf(\{\lbrack \varphi(t) \rbrack_{T}: t$ es un término cerrado$\})$.
  \end{lemma}

  % Lemma 170. Sin prueba. Lemma 80.
  \begin{lemma} \label{lemma_80}
    \PN \textbf{(Coincidencia)} Sean $\tau$ y $\tau^{\prime}$ dos tipos cualesquiera y sea $\tau_{\cap}$ dado por:
    \begin{itemize}
      \item $\mathcal{C}_{\cap} = \mathcal{C} \cap \mathcal{C}^{\prime}$
      \item $\mathcal{F}_{\cap} = \{f \in \mathcal{F} \cap \mathcal{F}^{\prime}: a(f) = a^{\prime}(f)\}$
      \item $\mathcal{R}_{\cap} = \{r \in \mathcal{R} \cap \mathcal{R}^{\prime}: a(r) = a^{\prime}(r)\}$
      \item $a_{\cap} = a\mid_{\mathcal{F}_{\cap} \; \cup \; \mathcal{R}_{\cap}}$
    \end{itemize}

    \PN Sean $\mathbf{A}$ y $\mathbf{A}^{\prime}$ modelos de tipo $\tau$ y $\tau^{\prime}$ respectivamente. Supongamos
    que $A = A^{\prime}$ y que $c^{\mathbf{A}} = c^{\mathbf{A}^{\prime}}$, para cada $c \in \mathcal{C}_{\cap},
    f^{\mathbf{A}} = f^{\mathbf{A}^{\prime}}$, para cada $f \in \mathcal{F}_{\cap}$ y $r^{\mathbf{A}} =
    r^{\mathbf{A}^{\prime}}$, para cada $r \in \mathcal{R}_{\cap}$, entonces:
    \begin{enumerate}[(a)]
      \item Para cada $t =_{d} t(\vec{v}) \in T^{\tau_{\cap}}$ se tiene que $t^{\mathbf{A}} \lbrack \vec{a} \rbrack =
      t^{\mathbf{A}^{\prime}} \lbrack \vec{a} \rbrack$, para cada $\vec{a} \in A^{n}$.
      \item Para cada $\varphi =_{d} \varphi (\vec{v}) \in F^{\tau_{\cap}}$ se tiene que:
      \[
        \mathbf{A} \models \varphi \lbrack \vec{a} \rbrack \ \text{si y solo si} \ \mathbf{A}^{\prime} \models \varphi
        \lbrack \vec{a} \rbrack
      \]
      \item Si $\Sigma \cup \{\varphi\} \subseteq S^{\tau_{\cap}}$, entonces:
      \[
        (\Sigma, \tau) \models \varphi \ \text{si y solo si} \ (\Sigma, \tau^{\prime}) \models \varphi
      \]
    \end{enumerate}
  \end{lemma}

  % Lemma 171. Sin prueba. Lemma 81.
  \begin{lemma} \label{lemma_81}
    \PN Sea $\tau$ un tipo. Hay una infinitupla $(\gamma_{1}, \gamma_{2}, \dotsc) \in F^{\tau_{\mathbb{N}}}$ tal que:
    \begin{enumerate}
      \item $\lvert Li(\gamma_{j}) \rvert \leq 1$, para cada $j = 1, 2, \dotsc$
      \item Si $\lvert Li(\gamma )\rvert \leq 1$, entonces $\gamma = \gamma_{j}$, para algún $j \in \mathbb{N}$
    \end{enumerate}
  \end{lemma}

  % Theorem 172. Con prueba. Theorem 82.
  \begin{theorem} \label{theorem_82}
    \PN \textbf{(Completitud) (G{\"o}del)} Sea $T = (\Sigma, \tau)$ una teoría de primer orden. Si $T \models \varphi$
    entonces $T \vdash \varphi$.
  \end{theorem}
  \begin{proof}
    Primero probaremos completitud para el caso en que $\tau $ tiene una cantidad infinita de nombres de constante que no ocurren en las sentencias de $ \Sigma $. Lo probaremos por el absurdo, es decir supongamos que $\varphi_{0} $ es tal que $(\Sigma, \tau)\models \varphi_{0}$ y $(\Sigma, \tau)\not\vdash \varphi_{0}.$ Notese que ya que $(\Sigma, \tau)\not\vdash \varphi_{0}$, tenemos que $\lbrack\lnot \varphi_{0}\rbrack\not=0^{\mathcal{A}_{(\Sigma, \tau)}}.$ Para cada $j\in \mathbb{N}$, sea $w_{j}\in Var$ tal que $ Li(\gamma _{j})\subseteq \{w_{j}\}$. Para cada $j$, declaremos $\gamma _{j}=_{d}\gamma _{j}(w_{j})$. Notese que por el Lema 163 tenemos que $\inf \{\lbrack\gamma _{j}(t)\rbrack:t\in T_{c}^{\tau }\} = \lbrack\forall w_{j}\gamma _{j}(w_{j})\rbrack$, para cada $j=1,2,\dotsc$. Por el Teorema de Rasiova y Sikorski tenemos que hay un filtro primo de $\mathcal{A}_{(\Sigma, \tau)}$ , $\mathcal{U}$ el cual cumple:

    (a) $\lbrack\lnot \varphi_{0}\rbrack\in \mathcal{U}$
    (b) para cada $j\in \mathbb{N}$, $\{\lbrack\gamma _{j}(t)\rbrack:t\in T_{c}^{\tau }\}\subseteq \mathcal{U}$ implica que $\lbrack\forall w_{j}\gamma _{j}(w_{j})\rbrack\in \mathcal{U}$
    Ya que la sucesion de las $\gamma _{i}$ cubre todas las formulas con a lo sumo una variable libre, podemos reescribir la propiedad (b) de la siguiente manera

    (b)$^{\prime}$ para cada $\varphi =_{d}\varphi (v) \in F^{\tau }$, si $\{\lbrack\varphi (t)\rbrack:t\in T_{c}^{\tau }\}\subseteq \mathcal{U}$ entonces $ \lbrack\forall v\varphi (v)\rbrack\in \mathcal{U}$
    Definamos sobre $T_{c}^{\tau }$ la siguiente relacion:

    $\displaystyle t\bowtie s\text{si y solo si }\lbrack(t\equiv s)\rbrack\in \mathcal{U}\text{.} $

    Veamos entonces que:
    (1) $\bowtie $ es de equivalencia.
    (2) Para cada $\varphi =_{d}\varphi (v_{1}, \dotsc, v_{n}) \in F^{\tau }$, $ t_{1}, \dotsc, t_{n},s_{1}, \dotsc, s_{n}\in T_{c}^{\tau }$, si $t_{1}\bowtie s_{1}$, $ t_{2}\bowtie s_{2}$, $\dotsc$, $t_{n}\bowtie s_{n}$, entonces $\lbrack\varphi (t_{1}, \dotsc, t_{n})\rbrack\in \mathcal{U}$ si y solo si $\lbrack\varphi (s_{1}, \dotsc, s_{n})\rbrack\in \mathcal{U}$.
    (3) Para cada $f\in \mathcal{F}_{n}$, $ t_{1}, \dotsc, t_{n},s_{1}, \dotsc, s_{n}\in T_{c}^{\tau }$,
    $\displaystyle t_{1}\bowtie s_{1},t_{2}\bowtie s_{2}, \dotsc, \;t_{n}\bowtie s_{n}\text{implica }f(t_{1}, \dotsc, t_{n})\bowtie f(s_{1}, \dotsc, s_{n}). $

    Probaremos (2). Notese que

    $\displaystyle (\Sigma, \tau)\vdash \left( (t_{1}\equiv s_{1})\wedge (t_{2}\equiv s_{2})\wedge \dotsc\wedge (t_{n}\equiv s_{n})\wedge \varphi (t_{1}, \dotsc, t_{n})\right) \rightarrow \varphi (s_{1}, \dotsc, s_{n}) $

    lo cual nos dice que
    $\displaystyle \lbrack (t_{1}\equiv s_{1})\rbrack \ \IN \ \lbrack(t_{2}\equiv s_{2})\rbrack \ \IN \  \dotsc \ \IN \ \lbrack(t_{n}\equiv s_{n})\rbrack \ \IN \ \lbrack\varphi (t_{1}, \dotsc, t_{n})\rbrack\leq \lbrack \varphi (s_{1}, \dotsc, s_{n})\rbrack $

    de lo cual se desprende que
    $\displaystyle \lbrack \varphi (t_{1}, \dotsc, t_{n})\rbrack\in \mathcal{U}\text{implica }\lbrack\varphi (s_{1}, \dotsc, s_{n})\rbrack\in \mathcal{U} $

    ya que $\mathcal{U}$ es un filtro. La otra implicacion es analoga.
    Para probar (3) podemos tomar $\varphi =\left( f(v_{1}, \dotsc, v_{n})\equiv f(s_{1}, \dotsc, s_{n})\right) $ y aplicar (2).

    Definamos ahora un modelo $\mathbf{A}_{\mathcal{U}}$ de tipo $\tau $ de la siguiente manera:

    - Universo de $\mathbf{A}_{\mathcal{U}}=T_{c}^{\tau }/\mathrm{\bowtie }$
    - $f^{\mathbf{A}_{\mathcal{U}}}(t_{1}/\mathrm{\bowtie }, \dotsc, t_{n}/ \mathrm{\bowtie })=f(t_{1}, \dotsc, t_{n})/\mathrm{\bowtie }$, $f\in \mathcal{F}_{n}$, $t_{1}, \dotsc, t_{n}\in T_{c}^{\tau }\;$
    - $r^{\mathbf{A}_{\mathcal{U}}} = \{(t_{1}/\mathrm{\bowtie }, \dotsc, t_{n}/ \mathrm{\bowtie }):\lbrack(t_{1}, \dotsc, t_{n})\rbrack\in \mathcal{U}\}$, $r\in \mathcal{R}_{n}.$
    Notese que la definicion de $f^{\mathbf{A}_{\mathcal{U}}}$ es inambigua por (3). Probaremos las siguientes propiedades basicas:

    (4) Para cada $t=_{d}t(v_{1}, \dotsc, v_{n}) \in \TAU$, $ t_{1}, \dotsc, t_{n}\in T_{c}^{\tau }$, tenemos que
    $\displaystyle t^{\mathbf{A}_{\mathcal{U}}}\lbrack t_{1}/\mathrm{\bowtie }, \dotsc, t_{n}/\mathrm{\bowtie }\rbrack=t(t_{1}, \dotsc, t_{n})/\mathrm{\bowtie } $

    (5) Para cada $\varphi =_{d}\varphi (v_{1}, \dotsc, v_{n}) \in F^{\tau }$, $ t_{1}, \dotsc, t_{n}\in T_{c}^{\tau }$, tenemos que
    $\displaystyle \mathbf{A}_{\mathcal{U}}\models \varphi \lbrack t_{1}/\mathrm{\bowtie } , \dotsc, t_{n}/\mathrm{\bowtie }\rbrack\text{si y solo si }\lbrack\varphi (t_{1}, \dotsc, t_{n})\rbrack\in \mathcal{U}. $

    La prueba de (4) es directa por induccion. Probaremos (5) por induccion en el $k$ tal que $\varphi \in F_{k}^{\tau }$. El caso $k=0$ es dejado al lector. Supongamos (5) vale para $\varphi \in F_{k}^{\tau }$. Sea $\varphi =_{d}\varphi (v_{1}, \dotsc, v_{n}) \in F_{k+1}^{\tau }-F_{k}^{\tau }.$ Hay varios casos:

    CASO $\varphi (v_{1}, \dotsc, v_{n})=\left( \varphi_{1}(v_{1}, \dotsc, v_{n}) \vee \varphi_{2}(v_{1}, \dotsc, v_{n})\right) .$

    Tenemos

    $\displaystyle \begin{array}{c} \mathbf{A}_{\mathcal{U}}\models \varphi \lbrack t_{1}/\mathrm{\bowtie } , \dotsc, t_{n}/\mathrm{\bowtie }\rbrack \\ \Updownarrow \\ \mathbf{A}_{\mathcal{U}}\models \varphi_{1}\lbrack t_{1}/\mathrm{\bowtie } , \dotsc, t_{n}/\mathrm{\bowtie }\rbrack\text{o }\mathbf{A}_{\mathcal{U}}\models \varphi_{2}\lbrack t_{1}/\mathrm{\bowtie }, \dotsc, t_{n}/\mathrm{\bowtie }\rbrack \\ \Updownarrow \\ \lbrack \varphi_{1}(t_{1}, \dotsc, t_{n})\rbrack\in \mathcal{U}\text{o }\lbrack\varphi_{2}(t_{1}, \dotsc, t_{n})\rbrack\in \mathcal{U} \\ \Updownarrow \\ \lbrack \varphi_{1}(t_{1}, \dotsc, t_{n})\rbrack\ \mathsf{s\ }\lbrack\varphi_{2}(t_{1}, \dotsc, t_{n})\rbrack\in \mathcal{U} \\ \Updownarrow \\ \lbrack \left( \varphi_{1}(t_{1}, \dotsc, t_{n}) \vee \varphi_{2}(t_{1}, \dotsc, t_{n})\right) \rbrack\in \mathcal{U} \\ \Updownarrow \\ \lbrack \varphi (t_{1}, \dotsc, t_{n})\rbrack\in \mathcal{U}. \end{array} $

    CASO $\varphi (v_{1}, \dotsc, v_{n})=\forall v\varphi_{1}(v_{1}, \dotsc, v_{n},v).$
    Tenemos

    $\displaystyle \begin{array}{c} \mathbf{A}_{\mathcal{U}}\models \varphi \lbrack t_{1}/\mathrm{\bowtie } , \dotsc, t_{n}/\mathrm{\bowtie }\rbrack \\ \Updownarrow \\ \mathbf{A}_{\mathcal{U}}\models \varphi_{1}\lbrack t_{1}/\mathrm{\bowtie } , \dotsc, t_{n}/\mathrm{\bowtie },t/\mathrm{\bowtie }\rbrack\text{, para todo }t\in T_{c}^{\tau } \\ \Updownarrow \\ \lbrack \varphi_{1}(t_{1}, \dotsc, t_{n},t)\rbrack\in \mathcal{U}\text{, para todo } t\in T_{c}^{\tau } \\ \Updownarrow \\ \lbrack \forall v\varphi_{1}(t_{1}, \dotsc, t_{n},v)\rbrack\in \mathcal{U} \\ \Updownarrow \\ \lbrack \varphi (t_{1}, \dotsc, t_{n})\rbrack\in \mathcal{U}. \end{array} $

    CASO $\varphi (v_{1}, \dotsc, v_{n})=\exists v\varphi_{1}(v_{1}, \dotsc, v_{n},v).$
    Tenemos

    $\displaystyle \begin{array}{c} \mathbf{A}_{\mathcal{U}}\models \varphi \lbrack t_{1}/\mathrm{\bowtie } , \dotsc, t_{n}/\mathrm{\bowtie }\rbrack \\ \Updownarrow \\ \mathbf{A}_{\mathcal{U}}\models \varphi_{1}\lbrack t_{1}/\mathrm{\bowtie } , \dotsc, t_{n}/\mathrm{\bowtie },t/\mathrm{\bowtie }\rbrack\text{, para algun }t\in T_{c}^{\tau } \\ \Updownarrow \\ \lbrack \varphi_{1}(t_{1}, \dotsc, t_{n},t)\rbrack\in \mathcal{U}\text{, para algun } t\in T_{c}^{\tau } \\ \Updownarrow \\ \lbrack \varphi_{1}(t_{1}, \dotsc, t_{n},t)\rbrack^{c}\not\in \mathcal{U}\text{, para algun }t\in T_{c}^{\tau } \\ \Updownarrow \\ \lbrack \lnot \varphi_{1}(t_{1}, \dotsc, t_{n},t)\rbrack\not\in \mathcal{U}\text{, para algun }t\in T_{c}^{\tau } \\ \Updownarrow \\ \lbrack \forall v\;\lnot \varphi_{1}(t_{1}, \dotsc, t_{n},v)\rbrack\not\in \mathcal{U} \\ \Updownarrow \\ \lbrack \forall v\;\lnot \varphi_{1}(t_{1}, \dotsc, t_{n},v)\rbrack^{c}\in \mathcal{U} \\ \Updownarrow \\ \lbrack \lnot \forall v\;\lnot \varphi_{1}(t_{1}, \dotsc, t_{n},v)\rbrack\in \mathcal{U } \\ \Updownarrow \\ \lbrack \varphi (t_{1}, \dotsc, t_{n})\rbrack\in \mathcal{U}. \end{array} $

    Pero ahora notese que (5) en particular nos dice que para cada sentencia $ \psi \in S^{\tau }$, $\mathbf{A}_{\mathcal{U}}\models \psi $ si y solo si $ \lbrack\psi \rbrack\in \mathcal{U}.$ De esta forma llegamos a que $\mathbf{A}_{\mathcal{U }}\models \Sigma $ y $\mathbf{A}_{\mathcal{U}}\models \lnot \varphi_{0}$, lo cual contradice la suposicion de que $(\Sigma, \tau)\models \varphi_{0}. $
    Ahora supongamos que $\tau $ es cualquier tipo. Sean $s_{1}$ y $s_{2}$ un par de simbolos no pertenecientes a la lista

    $\displaystyle \forall \ \ \exists \ \ \lnot \ \ \vee \ \ \wedge \ \ \rightarrow \ \ \leftrightarrow \ \ (\ \ )\ \ ,\ \equiv \ \ \mathsf{X}\ \ \mathit{0}\ \ \mathit{1}\ \ \dotsc\ \ \mathit{9}\ \ \mathbf{0}\ \ \mathbf{1}\ \ \dotsc\ \ \mathbf{9} $

    y tales que ninguno ocurra en alguna palabra de $\mathcal{C}\cup \mathcal{F} \cup \mathcal{R}.$ Si $(\Sigma, \tau)\models \varphi $, entonces usando el Lema de Coincidencia se puede ver que $(\Sigma ,(\mathcal{C}\cup \{s_{1}s_{2}s_{1},s_{1}s_{2}s_{2}s_{1},\dotsc\},\mathcal{F},\mathcal{R} ,a))\models \varphi $, por lo cual
    $\displaystyle (\Sigma ,(\mathcal{C}\cup \{s_{1}s_{2}s_{1},s_{1}s_{2}s_{2}s_{1},\dotsc\}, \mathcal{F},\mathcal{R},a))\vdash \varphi . $

    Pero por Lema 162, tenemos que $(\Sigma, \tau)\vdash \varphi .$
  \end{proof}

  % Corollary 173. Con prueba. Corollary 83.
  \begin{corollary} \label{corollary_83}
    \PN Toda teoría consistente tiene un modelo.
  \end{corollary}
  \begin{proof}
    \PN Supongamos $(\Sigma, \tau)$ es consistente y no tiene modelos. Entonces $(\Sigma, \tau) \models \left(\varphi
    \wedge \lnot \varphi \right) $, con lo cual por \textbf{Completitud}, $(\Sigma, \tau) \vdash \left(\varphi \wedge
    \lnot \varphi \right)$, lo cual es absurdo. Dicho absurdo vino de suponer que $(\Sigma, \tau)$ no tenía modelos.
  \end{proof}

  % Corollary 174. Con prueba. Corollary 84.
  \begin{corollary} \label{corollary_84}
    \PN \textbf{(Teorema de Compacidad)}
    \begin{enumerate}[(a)]
      \item Si $(\Sigma, \tau)$ es tal que $(\Sigma_{0}, \tau)$ tiene un modelo, para cada subconjunto finito
      $\Sigma_{0} \subseteq \Sigma$, entonces $(\Sigma, \tau)$ tiene un modelo.
      \item Si $(\Sigma, \tau) \models \varphi$, entonces hay un subconjunto finito $\Sigma_{0} \subseteq \Sigma$ tal
      que $(\Sigma_{0}, \tau) \models \varphi$.
    \end{enumerate}
  \end{corollary}
  \begin{proof}
    \begin{enumerate}[(a)]
      \item Si $(\Sigma, \tau)$ no tuviera un modelo, es decir, si fuera inconsistente, habría un subconjunto finito
        $\Sigma_{0} \subseteq \Sigma$ tal que la teoría $(\Sigma_{0}, \tau)$ es inconsistente, lo cual es absurdo, pues
        cada subconjunto finito de $\Sigma$ es consistente.
      \item Si $(\Sigma, \tau) \models \varphi$, entonces por \textbf{Completitud}, $(\Sigma, \tau) \vdash \varphi$, es
        decir que hay un subconjunto finito $\Sigma_{0} \subseteq \Sigma$ tal que $(\Sigma_{0}, \tau) \vdash \varphi$.
        Por lo tanto, por \textbf{Correción}, $(\Sigma_{0}, \tau) \models \varphi$.
    \end{enumerate}
  \end{proof}

  % El número 175 es usado para un ejemplo.

  % Lemma 176. Nada. Lemma 85.
  \begin{lemma}
    \PN Este lema no se evalua.
  \end{lemma}

  % Lemma 177. Nada. Lemma 86.
  \begin{lemma}
    \PN Este lema no se evalua.
  \end{lemma}

  % Lemma 178. Nada. Lemma 87.
  \begin{lemma}
    \PN Este lema no se evalua.
  \end{lemma}

  % Theorem 179. Con prueba. Theorem 88.
  \begin{theorem}
    \PN Este teorema no se evalua.
  \end{theorem}

  % Theorem 180. Con prueba. Theorem 89.
  \begin{theorem}
    \PN Este teorema no se evalua.
  \end{theorem}

  % Corollary 181. Sin prueba. Corollary 90.
  \begin{corollary}
    \PN Este corolario no se evalua.
  \end{corollary}
