\section{La aritmética de Peano}

  % Lemma 182. Con prueba. Lemma 91.
  \begin{lemma} \label{lemma_91}
    \PN $\pmb{\omega}$ es un modelo de $Arit$.
  \end{lemma}
  \begin{proof}
    Sea $\psi =_{d}\psi (v_{1},...,v_{n},v)$, f\'{o}rmula de $\tau _{A}$. Veremos que $\mathbf{\omega }\vDash Ind_{\psi }$. Sea

    $\displaystyle \varphi =((\psi (\vec{v},0)\wedge \forall v\ (\psi (\vec{v},v)\rightarrow \psi (\vec{v},v+1)))\rightarrow \forall v\ \psi (\vec{v},v)) $

    Declaremos $\varphi =_{d}\varphi (v_{1},...,v_{n})$. Notese que $\mathbf{ \omega }\vDash Ind_{\psi }$ si y solo si para cada $a_{1},...,a_{n}\in \omega $ se tiene que $\mathbf{\omega }\vDash \varphi \lbrack \vec{a}]$. Sean $a_{1},...,a_{n}\in \omega $ fijos. Probaremos que $\mathbf{\omega } \vDash \varphi \lbrack \vec{a}]$. Notar que si
    $\displaystyle \mathbf{\omega }\nvDash (\psi (\vec{v},0)\wedge \forall v\ (\psi (\vec{v} ,v)\rightarrow \psi (\vec{v},v+1)))[\vec{a}] $

    entonces $\mathbf{\omega }\vDash \varphi \lbrack \vec{a}]$ por lo cual podemos hacer solo el caso en que
    $\displaystyle \mathbf{\omega }\vDash (\psi (\vec{v},0)\wedge \forall v\ (\psi (\vec{v} ,v)\rightarrow \psi (\vec{v},v+1)))[\vec{a}] $

    Sea $S=\{a\in \omega :\mathbf{\omega }\vDash \psi (\vec{v},v)[\vec{a},a]\}$. Ya que $\mathbf{\omega }\vDash \psi (\vec{v},0)[\vec{a}]$, es facil ver usando el lema de reemplazo que $\mathbf{\omega }\vDash \psi (\vec{v},v)[ \vec{a},0]$, lo cual nos dice que $0\in S$. Ya que $\mathbf{\omega }\vDash (\forall v\ (\psi (\vec{v},v)\rightarrow \psi (\vec{v},v+1)))[\vec{a}]$, tenemos que
    (1) Para cada $a\in \omega $, si $\mathbf{\omega }\vDash \psi (\vec{v} ,v)[\vec{a},a]$, entonces $\mathbf{\omega }\vDash \psi (\vec{v},v+1)[\vec{a} ,a]$.
    Pero por el lema de reemplazo, tenemos que $\mathbf{\omega }\vDash \psi ( \vec{v},v+1)[\vec{a},a]$ sii $\mathbf{\omega }\vDash \psi (\vec{v},v)[\vec{a} ,a+1]$, lo cual nos dice que

    (2) Para cada $a\in \omega $, si $\mathbf{\omega }\vDash \psi (\vec{v} ,v)[\vec{a},a]$, entonces $\mathbf{\omega }\vDash \psi (\vec{v},v)[\vec{a} ,a+1]$.
    Ya que (2) nos dice que $a\in S$ implica $a+1\in S$, tenemos que $S=\omega $ ya que $0\in S$. Es decir que para cada $a\in \omega $, se da que $\mathbf{ \omega }\vDash \psi (\vec{v},v)[\vec{a},a]$ lo cual nos dice que $\mathbf{ \omega }\vDash \forall v\ \psi (\vec{v},v)[\vec{a}]$.

    Es rutina probar que $\mathbf{\omega }$ satisface los otros 15 axiomas de $ Arit$. $\Box$
  \end{proof}

  % Proposition 183. Sin prueba. Proposition 92.
  \begin{proposition} \label{proposition_92}
    \PN Hay un modelo de $Arit$ el cual no es isomorfo a $\pmb{\omega}$.
  \end{proposition}

  % Lemma 184. Nada. Lemma 93.
  \begin{lemma}
    \PN Este lema no se evalua.
  \end{lemma}

  % Lemma 185. Nada. Lemma 94.
  \begin{lemma}
    \PN Este lema no se evalua.
  \end{lemma}

  % Lemma 186. Nada. Lemma 95.
  \begin{lemma}
    \PN Este lema no se evalua.
  \end{lemma}

  % Lemma 187. Nada. Lemma 96.
  \begin{lemma}
    \PN Este lema no se evalua.
  \end{lemma}

  % Lemma 188. Nada. Lemma 97.
  \begin{lemma}
    \PN Este lema no se evalua.
  \end{lemma}

  % Lemma 189. Sin prueba. Lemma 98.
  \begin{lemma} \label{lemma_98}
    \PN Los conjuntos $T^{\tau_{A}^{e}}, F^{\tau_{A}^{e}}, T^{\tau_{A}}$ y $F^{\tau_{A}}$ son $\mathcal{A}$-recursivos.
  \end{lemma}

  % Lemma 190. Sin prueba. Lemma 99.
  \begin{lemma} \label{lemma_99}
    \PN Los siguientes predicados son $\mathcal{A}$-recursivos:
    \begin{enumerate}[(a)]
      \item \CL $v$ ocurre libremente en $\varphi$ a partir de $i$\CR: $\omega \times Var \times F^{\tau_{A}^{e}}
        \rightarrow \omega$
      \item \CL $v \in Li(\varphi)$\CR: $Var \times F^{\tau_{A}^{e}} \rightarrow \omega$
      \item \CL $v$ es sustituible por $t$ en $\varphi$\CR: $Var \times T^{\tau_{A}^{e}} \times F^{\tau_{A}^{e}}
        \rightarrow \omega$
    \end{enumerate}
  \end{lemma}

  % Lemma 191. Sin prueba. Lemma 100.
  \begin{lemma} \label{lemma_100}
    \PN Las funciones $\lambda svt[\downarrow_{v}^{t}(s)]$ y $\lambda \varphi vt[\downarrow_{v}^{t}(\varphi)]$ son
    $\mathcal{A}$-recursivas.
  \end{lemma}

  % Lemma 192. Sin prueba. Lemma 101.
  \begin{lemma} \label{lemma_101}
    \PN El predicado $R: F^{\tau_{A}^{e}} \times F^{\tau_{A}^{e}} \times F^{\tau_{A}^{e}} \times F^{\tau_{A}^{e}}
    \rightarrow \omega$, dado por:
    \[
      R(\varphi, \tilde{\varphi}, \psi_{1}, \psi_{2}) = \left\{\begin{array}{cccl}
                                                                \begin{array}{c}
                                                                  1 \\
                                                                  \ \\
                                                                \end{array}
                                                                &&&
                                                                \begin{array}{cl}
                                                                  \text{si } \tilde{\varphi} = &\text{resultado de
                                                                    reemplazar algunas (posiblemente 0)} \\
                                                                  &\text{ocurrencias de} \ \psi_{1} \ \text{en} \
                                                                    \varphi \ \text{por} \ \psi_{2}
                                                                \end{array} \\ 0 &&& \text{caso contrario}
                                                              \end{array}\right.
    \]
    \PN es $\mathcal{A}$-recursivo.
  \end{lemma}

  % Lemma 193. Sin prueba. Lemma 102.
  \begin{lemma} \label{lemma_102}
    \PN Sea $\Sigma$ un alfabeto finito. Sea $S \subseteq \Sigma^{\ast}$ un conjunto $\Sigma$-recursivo. El conjunto
    $S^{+}$ es $\Sigma$-recursivo.
  \end{lemma}

  % Lemma 194. Sin prueba. Lemma 103.
  \begin{lemma} \label{lemma_103}
    \PN Los conjuntos $ModPon^{\tau^{e}_{A}}, Elect^{\tau^{e}_{A}}, Reemp^{\tau^{e}_{A}}, ConjInt^{\tau^{e}_{A}},
    ConjElim^{\tau^{e}_{A}}, EquivInt^{\tau^{e}_{A}}, \linebreak DisjElim^{\tau^{e}_{A}}, DisjInt^{\tau^{e}_{A}},
    EquivElim^{\tau^{e}_{A}}, Generaliz^{\tau^{e}_{A}}, Commut^{\tau^{e}_{A}}, Trans^{\tau^{e}_{A}},
    Exist^{\tau^{e}_{A}}, Evoc^{\tau^{e}_{A}}, \linebreak Absur^{\tau^{e}_{A}}, DivPorCas^{\tau^{e}_{A}},
    Partic^{\tau^{e}_{A}}$, son $\mathcal{A}$-recursivos.
  \end{lemma}

  % Lemma 195. Sin prueba. Lemma 104.
  \begin{lemma} \label{lemma_104}
    \PN El predicado \CL$\psi$ se deduce de $\varphi$ por generalización con constante $c$, con respecto a
    $\tau^{e}_{A}$\CR: $S^{\tau_{A}^{e}} \times S^{\tau_{A}^{e}} \times Aux \rightarrow \omega$ es $\mathcal{A}$-recursivo.
  \end{lemma}

  % Lemma 196. Sin prueba. Lemma 105.
  \begin{lemma} \label{lemma_105}
    \PN El predicado \CL$\psi$ se deduce de $\varphi$ por elección con constante $e$, con respecto a $\tau^{e}_{A}$\CR:
    $S^{\tau_{A}^{e}} \times S^{\tau_{A}^{e}} \times Aux \rightarrow \omega$ es $\mathcal{A}$-PR.
  \end{lemma}

  % Lemma 197. Sin prueba. Lemma 106.
  \begin{lemma} \label{lemma_106}
    \PN $AxLog^{\tau_{A}^{e}}$ es $\mathcal{A}$-recursivo.
  \end{lemma}

  % Lemma 198. Sin prueba. Lemma 107.
  \begin{lemma} \label{lemma_107}
    \PN Las funciones:
    \begin{equation*}
      \begin{aligned}
        S^{\tau_{A}^{e}+} &\rightarrow& \omega \\
        \pmb{\varphi} &\rightarrow& n(\pmb{\varphi}) \\
      \end{aligned}
      \qquad\qquad\qquad
      \begin{aligned}
        \omega \times S^{\tau_{A}^{e}+} &\rightarrow& S^{\tau_{A}^{e}} \cup \{\varepsilon\} \\
        (i, \pmb{\varphi}) &\rightarrow& \pmb{\varphi}_{i} \\
      \end{aligned}
    \end{equation*}
    \PN son $\mathcal{A}$-recursivas.
  \end{lemma}

  % Lemma 199. Sin prueba. Lemma 108.
  \begin{lemma} \label{lemma_108}
    \PN $Just$ es $\mathcal{B}$-recursivo. Las funciones:
    \begin{equation*}
      \begin{aligned}
        Just^{+} &\rightarrow& \omega \\
        \mathbf{J} &\rightarrow& n(\mathbf{J}) \\
      \end{aligned}
      \qquad\qquad\qquad
      \begin{aligned}
        \omega \times Just^{+} &\rightarrow& Just \cup \{\varepsilon\} \\
        (i, \mathbf{J}) &\rightarrow& \mathbf{J}_{i} \\
      \end{aligned}
    \end{equation*}
    \PN son $\mathcal{B}$-recursivas.
  \end{lemma}

  % Lemma 200. Sin prueba. Lemma 109.
  \begin{lemma} \label{lemma_109}
    \PN El conjunto $\{\mathbf{J} \in Just^{+}: \mathbf{J}$ es balanceada$\}$ es $\mathcal{B}$-recursivo.
  \end{lemma}

  % Lemma 201. Sin prueba. Lemma 110.
  \begin{lemma} \label{lemma_110}
    \PN El predicado
    \[
      \begin{array}{rcl}
        \omega \times S^{\tau_{A}^{e}} \times S^{\tau_{A}^{e}+} \times Just^{+} &\rightarrow& \omega \\
        (i, \varphi, \pmb{\varphi}, \mathbf{J}) &\rightarrow& \left\{
          \begin{array}{ccl}
            1 && \text{si} \ (\pmb{\varphi}, \mathbf{J}) \ \text{es adecuado y} \ \varphi \ \text{es hipótesis de} \
              \pmb{\varphi}_{i} \ \text{en} \ (\pmb{\varphi}, \mathbf{J}) \\
            0 && \text{caso contrario}
          \end{array}\right.
      \end{array}
    \]
    \PN es $(\mathcal{A} \cup \mathcal{B})$-recursivo.
  \end{lemma}

  % Lemma 202. Sin prueba. Lemma 111.
  \begin{lemma} \label{lemma_111}
    \PN El predicado
    \[
      \begin{array}{rcl}
        \mathcal{C} \times \mathcal{C} \times S^{\tau_{A}^{e}+} \times Just^{+} &\rightarrow& \omega \\
        (e, d, \pmb{\varphi}, \mathbf{J}) &\rightarrow& \left\{
          \begin{array}{ccl}
            1 && \text{si} \ (\pmb{\varphi}, \mathbf{J}) \ \text{es adecuado y} \ e \ \text{depende de} \ d \ \text{en}
              \ (\pmb{\varphi}, \mathbf{J}) \\
            0 && \text{caso contrario}
          \end{array}\right.
      \end{array}
    \]
    \PN es $(\mathcal{A} \cup \mathcal{B})$-recursivo.
  \end{lemma}

  % Lemma 203. Sin prueba. Lemma 112.
  \begin{lemma} \label{lemma_112}
    \PN Sea $(\Sigma, \tau_{A})$ una teoría tal que $\Sigma$ es $\mathcal{A}$-recursivo (resp. $\mathcal{A}$-RE),
    entonces $Pruebas_{(\Sigma, \tau_{A})}$ es $(\mathcal{A} \cup \mathcal{B})$-recursivo (resp. $(\mathcal{A} \cup
    \mathcal{B})$-RE).
  \end{lemma}

  % Proposition 204. Sin prueba. Proposition 113.
  \begin{lemma} \label{lemma_113}
    \PN Si $(\Sigma, \tau_{A})$ es una teoría tal que $\Sigma$ es $\mathcal{A}$-RE., entonces $Teo_{(\Sigma, \tau_{A})}$
    es $\mathcal{A}$-RE.
  \end{lemma}

  % Lemma 205. Nada. Proposition 114.
  \begin{lemma}
    \PN Este lema no se evalua.
  \end{lemma}

  % Proposition 206. Sin prueba. Proposition 115.
  \begin{proposition} \label{proposition_115}
    \PN Si $h$ es $\emptyset$-recursiva, entonces $h$ es representable.
  \end{proposition}

  % Lemma 207. Sin prueba. Lemma 116.
  \begin{lemma} \label{lemma_116}
    \PN Hay un predicado $P: \omega \times \omega \rightarrow \omega$ el cual es $\emptyset$-PR y tal que el predicado
    $Q = \lambda x\left[(\exists t \in \omega) P(t, x)\right]: \omega \rightarrow \omega$ no es $\emptyset$-recursivo.
  \end{lemma}

  % Lemma 208. Nada. Lemma 117.
  \begin{lemma}
    \PN Este lema no se evalua.
  \end{lemma}

  % Lemma 209. Sin prueba. Lemma 118.
  \begin{lemma} \label{lemma_118}
    \PN Si $Verd_{\mathbf{\omega}}$ es $\mathcal{A}$-RE, entonces es $\mathcal{A}$-recursivo.
  \end{lemma}

  % Lemma 210. Con prueba. Lemma 119.
  \begin{lemma} \label{lemma_119}
    \PN $Verd_{\mathbf{\omega}}$ no es $\mathcal{A}$-RE.
  \end{lemma}
  \begin{proof}
    Por el Lema 200 hay un predicado $\emptyset $-p.r., $ P:\omega \times \omega \rightarrow \omega $ tal que el predicado $Q=\lambda x \left[ (\exists t\in \omega )P(t,x)\right] :\omega \rightarrow \omega $ no es $\emptyset $-recursivo. Notese que $Q$ tampoco es $A$-recursivo. Ya que $P $ es representable, hay una formula $\varphi =_{d}\varphi (v_{1},v_{2},v)\in F^{\tau _{A}}$ la cual cumple

    $\displaystyle \mathbf{\omega }\models \varphi \left[ t,x,k\right] \text{si y solo si } P(t,x)=k, $

    cualesquiera sean $t,x,k\in \omega .$ Sea $\psi =\varphi (v_{1},v_{2},1)$. Notese que $\psi =_{d}\psi (v_{1},v_{2})$ y que
    $\displaystyle \mathbf{\omega }\models \psi \left[ t,x\right] \text{si y solo si }P(t,x)=1 \text{,} $

    cualesquiera sean $t,x\in \omega .$ Sea $\psi_{0}=\exists v_{1}\ \psi (v_{1},v_{2})$. Notese que $\psi_{0}=_{d}\psi_{0}(v_{2})$ y que
    $\displaystyle \mathbf{\omega }\models \psi_{0}\left[ x\right] \text{si y solo si }Q(x)=1 $

    cualesquiera sea $x\in \omega $. Por el lema de reemplazo tenemos que para $ x\in \omega $,
    $\displaystyle \mathbf{\omega }\models \psi_{0}\left[ x\right] \text{si y solo si } \mathbf{\omega }\models \psi_{0}(\widehat{x}) $

    (justifique), por lo cual
    $\displaystyle \mathbf{\omega }\models \psi_{0}(\widehat{x})\text{si y solo si }Q(x)=1 $

    cualesquiera sea $x\in \omega $. Ya que $\psi_{0}(\widehat{x})$ es una sentencia,
    $\displaystyle \psi_{0}(\widehat{x})\in Verd_{\mathbf{\omega }}\text{si y solo si }Q(x)=1 $

    Sea $h:\omega \rightarrow A^{\ast }$, dada por $h(x)=\psi_{0}(\widehat{x})$ . Es facil ver que $h$ es $A$-recursiva. Ya que $Q=\chi _{Verd_{\mathbf{ \omega }}}\circ h$ y $Q$ no es $A$-recursivo, tenemos que $\chi _{Verd_{ \mathbf{\omega }}}$ no es $A$-recursiva, es decir que $Verd_{\mathbf{\omega } }$ es un conjunto no $A$-recursivo. El lema anterior nos dice entonces que es $Verd_{\mathbf{\omega }}$ no es $A$-r.e..
  \end{proof}

  % Theorem 211. Con prueba. Theorem 120.
  \begin{theorem} \label{theorem_120}
    \PN \textbf{(Incompletitud) (Godel)}. Si $\Sigma \subseteq Verd_{\mathbf{\omega}}$ es $\mathcal{A}$-RE, entonces
    $Teo_{(\Sigma, \tau_{A})} \subsetneq Verd_{\mathbf{\omega}}$.
  \end{theorem}
  \begin{proof}
    Por el Teorema de Correccion, tenemos que $Teo_{(\Sigma ,\tau _{A})}\subseteq Verd_{\mathbf{\omega }}$. Ya que $Teo_{(\Sigma ,\tau _{A})}$ es $A$-r.e y $Verd_{\mathbf{\omega }}$ no lo es, tenemos que $Teo_{(\Sigma ,\tau _{A})}\neq Verd_{\mathbf{\omega }}$. $\Box$
  \end{proof}

  % Corollary 212. Con prueba. Corollary 121.
  \begin{corollary} \label{corollary_121}
    \PN Existe $\varphi \in S^{\tau_{A}}$ tal que $Arit \nvdash \varphi$ y $Arit \nvdash \lnot \varphi$.
  \end{corollary}
  \begin{proof}
    Dejamos al lector la prueba de que el conjunto $\Sigma _{A}$ es $\mathcal{A}$ -r.e.. Una ves probado esto, podemos
    aplicar el teorema anterior a la teoria $Arit=(\Sigma _{A},\tau _{A})$, lo cual nos dice que $Teo_{Arit}\subsetneq
    Verd_{\mathbf{\omega }}$. Sea $\varphi \in Verd_{\mathbf{\omega } }-Teo_{Arit} $. O sea que $Arit\nvdash \varphi $ y
    $\varphi \in Verd_{ \mathbf{\omega }}$. Ya que $\lnot \varphi \notin Verd_{\mathbf{\omega }}$, tenemos que $\lnot
    \varphi \notin Teo_{Arit}$, es decir $Arit\nvdash \lnot \varphi$.
  \end{proof}
