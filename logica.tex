\documentclass[12pt,a4paper]{article}
\usepackage[utf8]{inputenc}[spanish]
\usepackage{amsmath}
\usepackage{amsfonts}
\usepackage{amssymb}
\usepackage{lmodern}
\usepackage{amsmath}
\usepackage{amsbsy}
\usepackage{amsthm}
\usepackage{enumerate}
\usepackage{graphicx}
\usepackage{mathtools}
\usepackage{stackrel}
\usepackage{multicol}
\usepackage{hyperref}
\usepackage{cleveref}
\usepackage[left=2cm,right=2cm,top=2cm,bottom=2cm]{geometry}
\providecommand{\abs}[1]{\lvert#1\rvert}

% Párrafo
\newcommand{\PN}{\par\noindent}
% Comillas izquierdas
\newcommand{\CL}{\textquotedblleft}
% Comillas derechas
\newcommand{\CR}{\textquotedblright}


% Conjuntos
\newcommand{\SIGMA}{\Sigma^{\ast}}
\newcommand{\TAU}{T^{\tau}}
\newcommand{\FT}{F^{\tau}}


% Supremo e ínfimo
\newcommand{\SU}{\mathsf{s}}
\newcommand{\IN}{\mathsf{i}}


% Poset
\newcommand{\POSET}{(P, \leq)}
% Poset primo
\newcommand{\POSETPRIMO}{(P^{\prime}, \leq^{\prime})}

% Reticulado
\newcommand{\RET}{(L, \SU, \IN)}
% Reticulado primo
\newcommand{\RETPRIMO}{(L^{\prime}, \SU^{\prime}, \IN^{\prime})}
% Reticulado cocientado
\newcommand{\RETCOCIENTADO}{(L/\theta, \mathsf{\tilde{s}}, \mathsf{\tilde{\imath}})}

% Reticulado acotado
\newcommand{\ACOTADO}{(L, \SU, \IN, 0, 1)}
% Reticulado acotado primo
\newcommand{\ACOTADOPRIMO}{(L^{\prime}, \SU^{\prime}, \IN^{\prime}, 0^{\prime}, 1^{\prime})}
% Reticulado acotado cocientado
\newcommand{\ACOTADOCOCIENTADO}{(L/\theta, \mathsf{\tilde{s}}, \mathsf{\tilde{\imath}}, 0/\theta, 1/\theta)}

% Reticulado complementado
\newcommand{\COMPLEMENTADO}{(L, \SU, \IN, ^{c}, 0, 1)}
% Reticulado complementado primo
\newcommand{\COMPLEMENTADOPRIMO}{(L^{\prime}, \SU^{\prime}, \IN^{\prime}, ^{c^{\prime}} 0^{\prime}, 1^{\prime})}
% Reticulado complementado cocientado
\newcommand{\COMPLEMENTADOCOCIENTADO}{(L/\theta, \mathsf{\tilde{s}}, \mathsf{\tilde{\imath}}, ^{c}/\theta, 0/\theta, 1/\theta)}

% Álgebra de Boole
\newcommand{\BOOLE}{(B, \SU, \IN, ^{c}, 0, 1)}


% Teoremas
\newtheorem{theorem}[equation]{Theorem}
\newtheorem{lemma}[equation]{Lemma}
\newtheorem{proposition}[equation]{Proposition}
\newtheorem{corollary}[equation]{Corollary}

\author{Agustín Curto, agucurto95@gmail.com}
\title{Resumen de teoremas para el final \\ de Lógica}
\date{2017}

\begin{document}
	\clearpage\maketitle
	\thispagestyle{empty}
	\tableofcontents

	\vspace{5cm}
	\PN \textbf{Nota:} Este resumen se corresponde con la materia dictada en el año 2017. El autor no se
	responsabiliza de posibles cambios que pudiesen realizarse en los temas dictados en la misma, así como tampoco de
	errores involuntarios que pudiesen existir en dicho resumen.

	\vspace{\fill}
	\begin{center}
		Por favor, mejorá este documento en github
		\includegraphics[width=1cm]{graphics/github.png} \\
		https://github.com/ResumenesFaMAF/resumenLogica
	\end{center}

	\pagebreak

	\section{Estructuras algebráicas ordenadas}

  % Lemma 91: Con prueba. Lemma 1.
  \begin{lemma} \label{lemma_1}
    \PN Sean $\POSET$ y $\POSETPRIMO$ posets. Supongamos que $F$ es un isomorfismo de $\POSET$ en $\POSETPRIMO$,
    entonces:
    \begin{enumerate}[a)]
      \item Para cada $S \subseteq P$ y cada $a \in P$, se tiene que $a$ es \textbf{cota superior} (resp.
        \textbf{inferior}) de $S$ si y solo si $F(a)$ es \textbf{cota superior} (resp. \textbf{inferior}) de $F(S)$.
      \item Para cada $S \subseteq P$, se tiene que existe $\sup (S)$ si y solo si existe $\sup (F(S))$ y en el
        caso de que existan tales elementos se tiene que $F(\sup (S)) = \sup (F(S))$.
      \item $P$ tiene $1$ (resp. $0$) si y solo si $P^{\prime}$ tiene $1$ (resp. $0$) y en tal caso tales elementos
        están conectados por $F$.
      \item Para cada $m \in P$, $m$ es \textbf{maximal} (resp. \textbf{minimal}) si y solo si $F(m)$ es
        \textbf{maximal} (resp. \textbf{minimal}).
      \item Para $a, b \in P$, tenemos que $a \prec b$ si y solo si $F(a) \prec^{\prime} F(b)$.
    \end{enumerate}
  \end{lemma}
  \begin{proof}
    \begin{enumerate}[a)]
      \item Probaremos solo el caso de la \textbf{cota superior}.
        \PN \begin{tabular}{|c|} \hline $\Rightarrow$ \\\hline \end{tabular} Supongamos que $a$ es \textbf{cota
        superior} de $S$, veamos entonces que $F(a)$ es \textbf{cota superior} de $F(S)$. Sean:
        \begin{itemize}
          \item $x \in F(S)$
          \item $s \in S$ tal que $x = F(s)$.
        \end{itemize}

        \PN Ya que $s \leq a$, tenemos que $x = F(s) \leq^{\prime} F(a)$. Luego, $F(a)$ es \textbf{cota superior}.

        \PN \begin{tabular}{|c|} \hline $\Leftarrow$ \\\hline \end{tabular} Supongamos ahora que $F(a)$ es \textbf{cota
        superior} de $F(S)$ y veamos entonces que $a$ es cota superior de $S$.

        \PN Sea $s \in S$, ya que $F(s) \leq^{\prime} F(a)$, tenemos que $s = F^{-1}(F(s)) \leq^{\prime} F^{-1}(F(a)) =
        a$. Por lo tanto, $a$ es \textbf{cota superior}.

      \item \begin{tabular}{|c|} \hline $\Rightarrow$ \\\hline \end{tabular} Supongamos existe $\sup (S)$. Veamos
        que $F(\sup (S))$ es el supremo de $F(S)$. Por el iniciso (a) $F(\sup (S))$ es cota superior de $F(S)$. Veamos
        que es la menor de las cotas superiores. Supongamos $b^{\prime}$ cota superior de $F(S)$, entonces
        $F^{-1}(b^{\prime})$ es cota superior de $S$, es decir, $\sup (S) \leq F^{-1}(b^{\prime})$, produciendo
        $F(\sup (S)) \leq^{\prime} b^{\prime}$. Por lo tanto, $F(\sup (S))$ es el supremo de $F(S)$.

        \PN \begin{tabular}{|c|} \hline $\Leftarrow$ \\\hline \end{tabular} Supongamos existe $\sup (F(S))$. Veamos
        que $F^{-1}(\sup (F(S)))$ es el supremo de $S$. Nuevamente, por el iniciso (a) $F^{-1}(\sup (F(S)))$ es cota
        superior de $S$. Veamos que es la menor de las cotas superiores. Supongamos $b$ cota superior de $S$, entonces
        $F(b)$ es cota superior de $F(S)$, es decir, $\sup (F(S)) \leq F(b)$, produciendo $F^{-1}(\sup (F(S))) \leq b$.
        Por lo tanto, $F^{-1}(\sup (F(S)))$ es el supremo de $S$.

      \item Se desprende del inciso (b) tomando $S = P$.
      \item Probaremos solo el caso \textbf{maximal}.
        \PN \begin{tabular}{|c|} \hline $\Rightarrow$ \\\hline \end{tabular} Supongamos que $m$ es maximal de
        $\POSET$. Veamos que $F(m)$ es maximal de $\POSETPRIMO$. Supongamos que $F(m)$ no es maximal
        de $\POSETPRIMO$, es decir, $F(m) <^{\prime} b^{\prime} \ \forall b^{\prime} \in P^{\prime}$.
        Dado que $F$ es isomorfismo:
        \begin{eqnarray*}
    			F^{-1}(F(m)) < F^{-1}(b^{\prime}) \\
    			m < F^{-1}(b^{\prime})
    		\end{eqnarray*}
        \PN Lo cual es un absurdo, dado que $m$ es maximal de $\POSET$. Por lo tanto, $F(m)$ es maximal de
        $\POSETPRIMO$.

        \PN \begin{tabular}{|c|} \hline $\Leftarrow$ \\\hline \end{tabular} Supongamos que $F(m)$ es maximal de
        $\POSETPRIMO$. Veamos que $m$ es maximal de $\POSET$. Supongamos que $m$ no es maximal
        de $\POSET$, es decir, $m < b \ \forall b \in P$. Dado que $F$ es isomorfismo:
        \[
    			F(m) < F(b)
    		\]
        \PN Lo cual es un absurdo, dado que $F(m)$ es maximal de $\POSETPRIMO$. Por lo tanto, $m$ es
        maximal de $\POSET$.

      \item \begin{tabular}{|c|} \hline $\Rightarrow$ \\\hline \end{tabular} Supongamos $a \prec b$, veamos que $F(a)
        \prec^{\prime} F(b)$. Debemos ver:
        \begin{enumerate}[1)]
          \item $F(a) <^{\prime} F(b)$
          \item $\nexists z^{\prime}$ tal que $F(a) < z^{\prime} < F(b)$
        \end{enumerate}

        \PN Ya que $a \prec b$, por definición tenemos: \begin{tabular}{|c|} \hline $a < b$ y $\nexists z$ tal que
        $a < z < b$ \\\hline \end{tabular} $(\star)$

        \PN Dado que la función $F$ es un isomorfismo, se cumple (1). Veamos que se cumple (2), supongamos que $\exists
        z^{\prime}$ tal que $F(a) < z^{\prime} < F(b)$. Luego, nuevamente utilizando que $F$ es isomorfismo, tenemos:
        \begin{eqnarray*}
    			F^{-1}(F(a)) < &F^{-1}(z^{\prime})& < F^{-1}(F(b)) \\
    			a < &F^{-1}(z^{\prime})& < b
    		\end{eqnarray*}
        \PN Lo cual, contradice $(\star)$, el absurdo vino de suponer que $\exists z^{\prime}$ tal que
        $F(a) < z^{\prime} < F(b)$, por lo tanto $\nexists z^{\prime}$ tal que $F(a) < z^{\prime} < F(b)$.

        \PN Finalmente, dado que se cumplen los puntos (1) y (2), se cumple también $F(a) \prec^{\prime} F(b)$.

        \PN \begin{tabular}{|c|} \hline $\Leftarrow$ \\\hline \end{tabular} Supongamos $F(a) \prec^{\prime} F(b)$,
        veamos que $a \prec b$.

		    \PN Ya que $F^{-1}: \POSETPRIMO \rightarrow \POSET$ es isomorfismo, por lo ya visto tenemos:
    		\begin{eqnarray*}
    			F^{-1}(F(a)) &\prec& F^{-1}(F(b)) \\
    			a &\prec& b
    		\end{eqnarray*}
    \end{enumerate}
  \end{proof}

  % Lemma 92: Con prueba. Lemma 2.
  \begin{lemma} \label{lemma_2}
    \PN Dado un reticulado $(L, \leq)$ y elementos $x, y, z, w \in L$, se cumplen las siguientes propiedades:
    \begin{multicols}{2}
      \begin{enumerate}[(1)]
        \item $x \leq x \ \SU \ y$
        \item $x \ \IN \ y \leq x$
        \item $x \ \SU \ x = x \ \IN \ x = x$
        \item $x \ \SU \ y = y \ \SU \ x$
        \item $x \ \IN \ y = y \ \IN \ x$
        \item $x \leq y \Leftrightarrow x \ \SU \ y = y \Leftrightarrow x \ \IN \ y = x$
        \item $x \ \SU \ (x \ \IN \ y) = x$
        \item $x \ \IN \ (x \ \SU \ y) = x$
        \item $(x \ \SU \ y) \ \SU \ z = x \ \SU \ (y \ \SU \ z)$
        \item $(x \ \IN \ y) \ \IN \ z = x \ \IN \ (y \ \IN \ z)$
        \item Si $x \leq z$ e $y \leq w$ entonces:
          \begin{itemize}
            \item $x \ \SU \ y \leq z \ \SU \ w$
            \item $x \ \IN \ y \leq z \ \IN \ w$
          \end{itemize}
        \item $(x \ \IN \ y) \ \SU \ (x \ \IN \ z) \leq x \ \IN \ (y \ \SU \ z)$
      \end{enumerate}
    \end{multicols}
  \end{lemma}
  \begin{proof}
    \PN Dado que las propiedades $(1), (2), (3), (4), (5), (6)$, son consecuencia inmediata de las definiciones de $\SU$
    e $\IN$, probaremos solo las restantes.
    \begin{enumerate}
      \begin{multicols}{2}
        \item[(7)]
          \begin{alignat*}{3}
            x \ \IN \ y &\leq& x \qquad &\text{Por } (2) \\
            (x \ \IN \ y) \ \SU \ x &=& x \qquad &\text{Por } (6) \\
            x \ \SU \ (x \ \IN \ y) &=& x \qquad &\text{Por } (3)
          \end{alignat*}

        \item[(8)]
          \begin{alignat*}{3}
            x &\leq& x \ \SU \ y \qquad \; && \text{Por } (1) \\
            x \ \IN \ (y \ \SU \ x) &=& x \qquad\qquad &&\text{Por } (6)
          \end{alignat*}
          \vspace{3mm}
      \end{multicols}

      \item[(9)] Para probar la igualdad probaremos las siguientes desigualdades:
        \begin{itemize}
          \item \begin{tabular}{|c|} \hline $(x \ \SU \ y) \ \SU \ z \leq x \ \SU \ (y \ \SU \ z)$\\\hline \end{tabular}
            \PN Notese que $x \ \SU \ (y \ \SU \ z)$ es cota superior de $\{x, y, z\}$ ya que:
            \begin{eqnarray*}
              x &\leq& x \ \SU \ (y \ \SU \ z) \\
              y &\leq& (y \ \SU \ z) \leq x \ \SU \ (y \ \SU \ z) \\
              z &\leq& (y \ \SU \ z) \leq x \ \SU \ (y \ \SU \ z)
            \end{eqnarray*}

            \PN Por otro lado, $x \ \SU \ (y \ \SU \ z)$ es cota superior de $\{x, y\}$, tenemos que $x \ \SU \ y \leq x
            \ \SU \ (y \ \SU \ z)$, por lo cual $x \ \SU \ (y \ \SU \ z)$ es cota superior del conjunto $\{x \ \SU \ y,
            z\}$, lo cual dice que $(x \ \SU \ y) \ \SU \ z \leq x \ \SU \ (y \ \SU \ z)$.

          \item \begin{tabular}{|c|} \hline $(x \ \SU \ y) \ \SU \ z \geq x \ \SU \ (y \ \SU \ z)$\\\hline \end{tabular}
            \PN Notese que $(x \ \SU \ y) \ \SU \ z$ es cota superior de $\{x, y, z\}$ ya que:
            \begin{eqnarray*}
              x &\leq& x \ \SU \ y \leq (x \ \SU \ y) \ \SU \ z \\
              y &\leq& x \ \SU \ y \leq (x \ \SU \ y) \ \SU \ z \\
              z &\leq& (x \ \SU \ y) \ \SU \ z
            \end{eqnarray*}

            \PN Por otro lado, $(x \ \SU \ y) \ \SU \ z$ es cota superior de $\{y, z\}$, tenemos que $y \ \SU \ z \leq
            (x \ \SU \ y) \ \SU \ z$, por lo cual $(x \ \SU \ y) \ \SU \ z$ es cota superior del conjunto $\{x, y \ \SU
            \ z\}$, lo cual dice que $(x \ \SU \ y) \ \SU \ z \geq x \ \SU \ (y \ \SU \ z)$.
        \end{itemize}

        \PN Por lo tanto, $(x \ \SU \ y) \ \SU \ z = x \ \SU \ (y \ \SU \ z)$

      \item[(10)] Para probar la igualdad probaremos las siguientes desigualdades:
        \begin{itemize}
          \item \begin{tabular}{|c|} \hline $(x \ \IN \ y) \ \IN \ z \leq x \ \IN \ (y \ \IN \ z)$\\\hline \end{tabular}
            \PN Notese que $x \ \IN \ (y \ \IN \ z)$ es cota inferior de $\{x, y, z\}$ ya que:
            \begin{eqnarray*}
              x \ \IN \ (y \ \IN \ z) &\leq& x \\
              (y \ \IN \ z) \leq x \ \IN \ (y \ \IN \ z) &\leq& y \\
              z &\leq& (y \ \IN \ z) \leq x \ \IN \ (y \ \IN \ z)
            \end{eqnarray*}

            \PN Por otro lado, $x \ \IN \ (y \ \IN \ z)$ es cota inferior de $\{x, y\}$, tenemos que $x \ \IN \ y \leq x
            \ \IN \ (y \ \IN \ z)$, por lo cual $x \ \IN \ (y \ \IN \ z)$ es cota inferior del conjunto $\{x \ \IN \ y,
            z\}$, lo cual dice que $(x \ \IN \ y) \ \IN \ z \leq x \ \IN \ (y \ \IN \ z)$.

          \item \begin{tabular}{|c|} \hline $(x \ \IN \ y) \ \IN \ z \geq x \ \IN \ (y \ \IN \ z)$\\\hline \end{tabular}
            \PN Notese que $(x \ \IN \ y) \ \IN \ z$ es cota inferior de $\{x, y, z\}$ ya que:
            \begin{eqnarray*}
              x &\leq& x \ \IN \ y \leq (x \ \SU \ y) \ \IN \ z \\
              y &\leq& x \ \IN \ y \leq (x \ \SU \ y) \ \IN \ z \\
              z &\leq& (x \ \IN \ y) \ \IN \ z
            \end{eqnarray*}

            \PN Por otro lado, $(x \ \IN \ y) \ \IN \ z$ es cota inferior de $\{y, z\}$, tenemos que $y \ \IN \ z \leq
            (x \ \IN \ y) \ \IN \ z$, por lo cual $(x \ \IN \ y) \ \IN \ z$ es cota inferior del conjunto $\{x, y \ \IN
            \ z\}$, lo cual dice que $(x \ \IN \ y) \ \IN \ z \geq x \ \IN \ (y \ \IN \ z)$.
        \end{itemize}

        \PN Por lo tanto, $(x \ \IN \ y) \ \IN \ z = x \ \IN \ (y \ \IN \ z)$

      \item[(11)]
        \begin{multicols}{2}
          \begin{alignat*}{3}
            x &\leq& z &\leq& z \ \SU \ w \\
            y &\leq& w &\leq& z \ \SU \ w \\
          \end{alignat*}
          \begin{alignat*}{3}
            \\
            x &\leq& z \Rightarrow x \ \IN \ y &\leq& z \\
            y &\leq& w \Rightarrow x \ \IN \ y &\leq& w
          \end{alignat*}
        \end{multicols}
        \PN Luego, $z \ \SU \ w$ es cota superior de $\{x, y\}$ y $x \ \IN \ y$ es cota inferior de $\{z, w\}$, por lo
        tanto, $x \ \SU \ y \leq z \ \SU \ w$ y $x \ \IN \ y \leq z \ \IN \ w$.

      \item[(12)]
        \begin{equation*}
          \left.
          \begin{array}{l}
            (x \ \IN \ y), (x \ \IN \ z) \leq x \\
            (x \ \IN \ y), (x \ \IN \ z) \leq y \ \SU \ z
          \end{array}
          \right \rbrace \Rightarrow (x \ \IN \ y), (x \ \IN \ z) \leq x \ \IN \ (y \ \SU \ z)
        \end{equation*}

        \[
          \therefore (x \ \IN \ y) \ \SU \ (x \ \IN \ z) \leq x \ \IN \ (y \ \SU \ z)
        \]
    \end{enumerate}
  \end{proof}

  % Lemma 93: Con prueba. Lemma 3.
  \begin{lemma} \label{lemma_3}
    \PN Sea $(L, \leq)$ un reticulado, dados elementos $x_{1}, \dotsc, x_{n} \in L$, con $n \geq 2$, se tiene
    \[
      \begin{array}{rcl}
        (\dotsc (x_{1} \ \SU \ x_{2}) \ \SU \ \dotsc) \ \SU \ x_{n} &=& \sup (\{x_{1}, \dotsc, x_{n}\}) \\
        (\dotsc (x_{1} \ \IN \ x_{2}) \ \IN \ \dotsc) \ \IN \ x_{n} &=& \inf (\{x_{1}, \dotsc, x_{n}\})
      \end{array}
    \]
  \end{lemma}
  \begin{proof}
    \PN Probaremos por inducción en $n$.

    \vspace{3mm}
    \PN \underline{Caso Base:} \begin{tabular}{|c|} \hline $n = 2$ \\\hline \end{tabular}
    \[
      \begin{array}{rcl}
        x_{1} \ \SU \ x_{2} &=& \sup(\{x_{1}, x_{2}\}) \\
        x_{1} \ \IN \ x_{2} &=& \inf(\{x_{1}, x_{2}\})
      \end{array}
    \]

    \PN Lo cual vale, dado que es la definición.

    \vspace{3mm}
		\PN \underline{Caso Inductivo:} \begin{tabular}{|c|} \hline $n > 2$ \\\hline \end{tabular}

    \PN Supongamos ahora que vale para $n$ y veamos entonces que vale para $n+1$. Sean $x_{1}, \dotsc, x_{n+1} \in L$,
    por hipótesis inductiva tenemos que:
    \begin{eqnarray*}
      (\dotsc (x_{1} \ \SU \ x_{2}) \ \SU \ \dotsc) \ \SU \ x_{n} &=& \sup(\{x_{1}, \dotsc, x_{n}\}) \ \ (\star_{1}) \\
      (\dotsc (x_{1} \ \IN \ x_{2}) \ \IN \ \dotsc) \ \IN \ x_{n} &=& \inf(\{x_{1}, \dotsc, x_{n}\}) \ \ \ (\star_{2})
    \end{eqnarray*}


    \PN Veamos entonces que:
    \begin{eqnarray*}
      ((\dotsc(x_{1} \ \SU \ x_{2}) \ \SU \ \dotsc) \ \SU \ x_{n}) \ \SU \ x_{n+1} &=& \sup(\{x_{1}, \dotsc, x_{n+1}\})
        \ \ (\dag_{1}) \\
      ((\dotsc(x_{1} \ \IN \ x_{2}) \ \IN \ \dotsc) \ \IN \ x_{n}) \ \IN \ x_{n+1} &=& \inf(\{x_{1}, \dotsc, x_{n+1}\})
        \ \ \ (\dag_{2})
    \end{eqnarray*}

    \PN Para ello debemos ver $((\dotsc(x_{1} \ \SU \ x_{2}) \ \SU \ \dotsc) \ \SU \ x_{n}) \ \SU \ x_{n+1}$ es cota
    superior de $\{x_{1}, \dotsc, x_{n+1}\}$ y que es la menor de las cotas superiores. Además, que $((\dotsc(x_{1} \
    \IN \ x_{2}) \ \IN \ \dotsc) \ \IN \ x_{n}) \ \IN \ x_{n+1}$ es cota inferior de $\{x_{1}, \dotsc, x_{n+1}\}$ y que
    es la mayor de las cotas inferiores.

    \vspace{5mm}
    \PN Es fácil ver que $((\dotsc(x_{1} \ \SU \ x_{2}) \ \SU \ \dotsc) \ \SU \ x_{n}) \ \SU \ x_{n+1}$ es cota superior
    de $ \{x_{1}, \dotsc, x_{n+1}\}$. Supongamos que $z$ es otra cota superior de $\{x_{1}, \dotsc, x_{n+1}\}$. Ya que
    $z$ es también cota superior del conjunto $\{x_{1}, \dotsc, x_{n}\}$, por $(\star_{1})$ tenemos que:
    \[
      (\dotsc (x_{1} \ \SU \ x_{2}) \ \SU \ \dotsc) \ \SU \ x_{n} \leq z
    \]

    \PN Además, dado que $x_{n+1} \leq z$, tenemos que:
    \[
      ((\dotsc (x_{1} \ \SU \ x_{2}) \ \SU \ \dotsc) \ \SU \ x_{n}) \ \SU \ x_{n+1} \leq z
    \]

    \PN Por lo tanto, vale $(\dag_{1})$.

    \vspace{5mm}
    \PN Nuevamente, es fácil ver que $((\dotsc(x_{1} \ \IN \ x_{2}) \ \IN \ \dotsc) \ \IN \ x_{n}) \ \IN \ x_{n+1}$ es
    cota inferior de $ \{x_{1}, \dotsc, x_{n+1}\}$. Supongamos que $z^{\prime}$ es otra cota inferior de $\{x_{1},
    \dotsc, x_{n+1}\}$. Ya que $z^{\prime}$ es también cota inferior del conjunto $\{x_{1}, \dotsc, x_{n}\}$, por
    $(\star_{2})$ tenemos que:
    \[
      z^{\prime} \leq (\dotsc (x_{1} \ \IN \ x_{2}) \ \IN \ \dotsc) \ \IN \ x_{n}
    \]

    \PN Además, dado que $z^{\prime} \leq x_{n+1}$, tenemos que:
    \[
      z^{\prime} \leq ((\dotsc (x_{1} \ \IN \ x_{2}) \ \IN \ \dotsc) \ \IN \ x_{n}) \ \IN \ x_{n+1}
    \]

    \PN Por lo tanto, vale $(\dag_{2})$.
  \end{proof}

  % Theorem 94: Con prueba. Theorem 4.
  \begin{theorem} \label{theorem_4}
    \PN Sea $\RET$ un reticulado, la relación binaria definida por:
    \[
      x \leq y \Leftrightarrow x \ \SU \ y = y
    \]
    \PN es un orden parcial sobre $L$ para el cual se cumple:
    \[
      \begin{array}{rcl}
        \sup (\{x, y\}) &=& x \ \SU \ y \\
        \inf (\{x, y\}) &=& x \ \IN \ y
      \end{array}
    \]
  \end{theorem}
  \begin{proof}
    \begin{itemize}
      \item \underline{Reflexiva:} Sea $x \in L$ un elemento cualquiera. Luego,
        \begin{equation*}
          \left.
          \begin{array}{l}
            x \ \SU \ x = x \\
            x \ \IN \ x = x
          \end{array}
          \right \rbrace \Rightarrow x \leq x
        \end{equation*}

      \item \underline{Antisimétrica:} Sean $x, y \in L$ elementos cualquieras. Supongamos que $x \leq y$ e $y \leq x$,
        entonces:
        \begin{equation*}
          \left.
          \begin{array}{l}
            x \leq y \Rightarrow x \ \SU \ y = y \\
            y \leq x \Rightarrow x \ \SU \ y = x
          \end{array}
          \right \rbrace \Rightarrow x = y
        \end{equation*}

      \item \underline{Transitiva:} Supongamos que $x \leq y$ e $y \leq z$, entonces:
        \[
          x \ \SU \ z = x \ \SU \ (y \ \SU \ z) = (x \ \SU \ y) \ \SU \ z = y \ \SU \ z = z
        \]

        \PN por lo cual $x \leq z$.

        \PN Veamos ahora que $\sup(\{x, y\}) = x \ \SU \ y$. Es claro que $x \ \SU \ y$ es una cota superior del
        conjunto $\{x, y\}$, veamos que es la menor. Supongamos $x, y \leq z$, entonces:
        \[
          (x \ \SU \ y) \ \SU \ z = x \ \SU \ (y \ \SU \ z) = x \ \SU \ z = z
        \]
        \PN por lo que $x\ \SU \ y \leq z$, es decir, $x\ \SU \ y$ es la menor cota superior.

        \PN Resta probar que $\inf(\{x, y\}) = x \ \IN \ y$. Nuevamente, es claro que $x \ \IN \ y$ es una cota inferior
        del conjunto $\{x, y\}$, veamos que es la mayor. Supongamos $z \leq x, y$, entonces:
        \[
          (x \ \IN \ y) \ \IN \ z = x \ \IN \ (y \ \IN \ z) = x \ \IN \ z = z
        \]
        \PN por lo que $z \leq x\ \IN \ y$, es decir, $x\ \IN \ y$ es la mayor cota inferior.
    \end{itemize}
  \end{proof}

  % Lemma 95: Con prueba. Lemma 5.
  \begin{lemma} \label{lemma_5}
    \PN Si $F: \RET \rightarrow \RETPRIMO$ es un homomorfismo biyectivo, entonces $F$ es un isomorfismo.
  \end{lemma}
  \begin{proof}
    \PN Debemos probar que $F^{-1}$ es un homomorfismo. Sean $F(x), F(y)$ dos elementos cualesquiera de $L^{\prime}$,
    tenemos que:
    \begin{multicols}{2}
      \begin{eqnarray*}
        F^{-1}(F(x) \ \SU^{\prime} \ F(y)) &=& F^{-1}(F(x \ \SU \ y)) \\
        &=& x \ \SU \ y \\
        &=& F^{-1}(F(x)) \ \SU \ F^{-1}(F(y))
      \end{eqnarray*}

      \begin{eqnarray*}
        F^{-1}(F(x) \ \IN^{\prime} \ F(y)) &=& F^{-1}(F(x \ \IN \ y)) \\
        &=& x \ \IN \ y \\
        &=& F^{-1}(F(x)) \ \IN \ F^{-1}(F(y))
      \end{eqnarray*}
    \end{multicols}

    \PN Luego, $F^{-1}$ es homomorfismo y por lo tanto $F$ es isomorfismo.
  \end{proof}

  % Lemma 96: Con prueba. Lemma 6.
  \begin{lemma} \label{lemma_6}
    \PN Sean $\RET$ y $\RETPRIMO$ reticulados y sea $F: \RET \rightarrow \RETPRIMO$ un homomorfismo, entonces $I_{F}$ es
    un subuniverso de $\RETPRIMO$.
  \end{lemma}
  \begin{proof}
    \PN Ya que $L \neq \emptyset$, tenemos que $I_{F} \neq \emptyset$. Sean $a, b \in I_{F}, \ x, y \in L$ tales que
    $F(x) = a$ y $F(y) = b$. Se tiene que:
    \begin{alignat*}{3}
      a \ \SU^{\prime} \ b &=& F(x) \ \SU^{\prime} \ F(y) &=& F(x \ \SU \ y) \in I_{F} \\
      a \ \IN^{\prime} \ b &=& F(x) \ \IN^{\prime} \ F(y) &=& F(x \ \IN \ y) \in I_{F}
    \end{alignat*}

    \PN por lo cual $I_{F}$ es cerrada bajo $\SU^{\prime}$ e $\IN^{\prime}$.
  \end{proof}

  % Lemma 97: Con prueba. Lemma 7.
  \begin{lemma} \label{lemma_7}
    \PN Sean $\RET$ y $\RETPRIMO$ reticulados y sean $(L, \leq)$ y $(L^{\prime}, \leq^{\prime})$ los posets asociados.
    Sea $F: L \rightarrow L^{\prime}$ una función, entonces $F$ es un isomorfismo de $\RET$ en $ \RETPRIMO$ si y solo si
    $F$ es un isomorfismo de $(L, \leq)$ en $(L^{\prime}, \leq^{\prime})$.
  \end{lemma}
  \begin{proof}
    \PN \begin{tabular}{|c|} \hline $\Rightarrow$ \\\hline \end{tabular} Supongamos que $F$ es un isomorfismo de
    $\RET$ en $ \RETPRIMO$.

    \PN Sean $x, y \in L$ tales que $x \leq y$. Tenemos:
    \begin{eqnarray*}
      y &=& x \ \SU \ y \\
      F(y) &=& F(x \ \SU \ y) \\
      &=& F(x) \ \SU^{\prime} \ F(y) \\
      \therefore F(x) &\leq^{\prime}& F(y)
    \end{eqnarray*}

    \PN Sean $x^{\prime}, y^{\prime} \in L^{\prime}$ tales que $x^{\prime} \leq^{\prime} y^{\prime}$. Tenemos:
    \begin{eqnarray*}
      y^{\prime} &=& x^{\prime} \ \SU^{\prime} \ y^{\prime} \\
      F^{-1}(y^{\prime}) &=& F^{-1}(x^{\prime} \ \SU^{\prime} \ y^{\prime}) \\
      &=& F^{-1}(x^{\prime}) \ \SU \ F^{-1}(y^{\prime}) \\
      \therefore F^{-1}(x) &\leq& F^{-1}(y)
    \end{eqnarray*}

    \PN Por lo tanto, $F$ es un isomorfismo de $(L, \leq)$ en $(L^{\prime}, \leq^{\prime})$.

    \PN \begin{tabular}{|c|} \hline $\Leftarrow$ \\\hline \end{tabular} Supongamos ahora que $F$ es un isomorfismo
    de $(L, \leq)$ en $(L^{\prime}, \leq^{\prime})$, entonces el \textbf{Lemma~\ref{lemma_1}} nos dice que $F$ y
    $F_{1}$ respetan la operaciones de supremo e ínfimo, por lo cual $F$ es un isomorfismo de $\RET$ y
    $\RETPRIMO$.
  \end{proof}

  % Lemma 98: Con prueba. Lemma 8.
  \begin{lemma} \label{lemma_8}
    \PN Sea $\RETCOCIENTADO$ un reticulado. El orden parcial $\tilde{\leq}$ asociado a este reticulado cumple:
    \[
      x/\theta \ \tilde{\leq} \ y/\theta \Leftrightarrow y \ \theta \ (x \ \SU \ y)
    \]
  \end{lemma}
  \begin{proof}
    \PN Veamos que $\RETCOCIENTADO$ satisface las 7 identidades de la
    definición de reticulado. Sean $x/\theta, y/\theta, z/\theta$ elementos cualesquiera de $L/\theta$.
    \begin{multicols}{2}
      \begin{enumerate}
        \item[(I1)] \begin{tabular}{|c|} \hline $x/\theta \ \mathsf{\tilde{s}} \ x/\theta = x/\theta \
          \mathsf{\tilde{\imath}} \ x/\theta = x/\theta$ \\\hline \end{tabular}
          \begin{eqnarray*}
            x/\theta \ \mathsf{\tilde{s}} \ x/\theta &=& (x \ \SU \ x)/\theta = x /\theta \\
            x/\theta \ \mathsf{\tilde{\imath}} \ x/\theta &=& (x \ \IN \ x)/\theta = x /\theta
          \end{eqnarray*}
        \item[(I2)] \begin{tabular}{|c|} \hline $x/\theta \ \mathsf{\tilde{s}} \ y/\theta = y/\theta \
          \mathsf{\tilde{s}} \ x/\theta$ \\\hline \end{tabular}
          \begin{eqnarray*}
            x/\theta \ \mathsf{\tilde{s}} \ y/\theta &=& (x \ \SU \ y)/\theta \\
            &=& (y \ \SU \ x)/\theta \\
            &=& y/\theta \ \mathsf{\tilde{s}} \ y/\theta
          \end{eqnarray*}
        \item[(I3)] \begin{tabular}{|c|} \hline $x/\theta \ \mathsf{\tilde{\imath}} \ y/\theta = y/\theta \
          \mathsf{\tilde{\imath}} \ x/\theta$ \\\hline \end{tabular}
          \begin{eqnarray*}
            x/\theta \ \mathsf{\tilde{\imath}} \ y/\theta &=& (x \ \IN \ y)/\theta \\
            &=& (y \ \IN \ x)/\theta \\
            &=& y/\theta \ \mathsf{\tilde{\imath}} \ y/\theta
          \end{eqnarray*}
        \item[(I4)] \begin{tabular}{|c|} \hline $(x/\theta \ \mathsf{\tilde{s}} \ y/\theta) \ \mathsf{\tilde{s}} \
          z/\theta = x/\theta \ \mathsf{\tilde{s}} \ (y/\theta \ \mathsf{\tilde{s}} \ z/\theta)$ \\\hline \end{tabular}
          \begin{eqnarray*}
            (x/\theta \ \mathsf{\tilde{s}} \ y/\theta) \ \mathsf{\tilde{s}} \ z/\theta &=& (x \ \SU \ y)/\theta \
              \mathsf{\tilde{s}} \ z/\theta \\
            &=& ((x \ \SU \ y) \ \SU \ z) /\theta \\
            &=& (x \ \SU \ (y \ \SU \ z)) /\theta \\
            &=& x/\theta \ \mathsf{\tilde{s}} \ (y \ \SU \ z) /\theta \\
            &=& x/\theta \ \mathsf{\tilde{s}} \ (y /\theta \ \mathsf{\tilde{s}} \ z/\theta)
          \end{eqnarray*}
        \item[(I5)] \begin{tabular}{|c|} \hline $(x/\theta \ \mathsf{\tilde{\imath}} \ y/\theta) \
          \mathsf{\tilde{\imath}} \ z/\theta = x/\theta \ \mathsf{\tilde{\imath}} \ (y/\theta \ \mathsf{\tilde{\imath}}
          \ z/\theta)$ \\\hline \end{tabular}
          \begin{eqnarray*}
            (x/\theta \ \mathsf{\tilde{\imath}} \ y/\theta) \ \mathsf{\tilde{\imath}} \ z/\theta &=& (x \ \IN \
              y)/\theta \ \mathsf{\tilde{\imath}} \ z/\theta \\
            &=& ((x \ \IN \ y) \ \IN \ z) /\theta \\
            &=& (x \ \IN \ (y \ \IN \ z)) /\theta \\
            &=& x/\theta \ \mathsf{\tilde{\imath}} \ (y \ \IN \ z) /\theta \\
            &=& x/\theta \ \mathsf{\tilde{\imath}} \ (y /\theta \ \mathsf{\tilde{\imath}} \ z/\theta)
          \end{eqnarray*}
        \item[(I6)] \begin{tabular}{|c|} \hline $x/\theta \ \mathsf{\tilde{s}} \ (x/\theta \ \mathsf{\tilde{\imath}} \
          y/\theta) = x/\theta$ \\\hline \end{tabular}
          \begin{eqnarray*}
            x/\theta \ \mathsf{\tilde{s}} \ (x/\theta \ \mathsf{\tilde{\imath}} \ y/\theta) &=& x/\theta \ \mathsf{\tilde{s}}
              \ (x \ \IN \ y)/\theta \\
            &=& (x \ \SU \ (x \ \IN \ y))/\theta \\
            &=& x/\theta
          \end{eqnarray*}
        \item[(I7)] \begin{tabular}{|c|} \hline $x/\theta \ \mathsf{\tilde{\imath}} \ (x/\theta \ \mathsf{\tilde{s}} \
          y/\theta) = x/\theta$ \\\hline \end{tabular}
          \begin{eqnarray*}
            x/\theta \ \mathsf{\tilde{\imath}} \ (x/\theta \ \mathsf{\tilde{s}} \ y/\theta) &=& x/\theta \
              \mathsf{\tilde{\imath}} \ (x \ \SU \ y)/\theta \\
            &=& (x \ \IN \ (x \ \SU \ y))/\theta \\
            &=& x/\theta
          \end{eqnarray*}
      \end{enumerate}
    \end{multicols}
  \end{proof}

  % Corollary 99: Con prueba. Corollary 9.
  \begin{corollary} \label{corrolary_9}
    \PN Sea $\RET$ un reticulado en el cual hay un elemento máximo $1$ (resp. mínimo $0$), entonces si $\theta$ es una
    congruencia sobre $\RET, 1/\theta$ (resp. $0/\theta$) es un elemento máximo (resp. mínimo) de $\RETCOCIENTADO$.
  \end{corollary}
  \begin{proof}
    \PN Ya que $1 \ \theta \ (x \ \SU \ 1)$, para cada $x \in L$, tenemos que $x/\theta \ \tilde{\leq} \ 1/\theta$, para
    cada $x \in L$.
  \end{proof}

  % Lemma 100: Con prueba. Lemma 10.
  \begin{lemma} \label{lemma_10}
    \PN Si $F: \RET \rightarrow \RETPRIMO$ es un homomorfismo de reticulados, entonces $\ker F$ es una congruencia sobre
    $\RET$.
  \end{lemma}
  \begin{proof}
    \PN Veamos primero que $\ker F$ es una relación de equivalencia.
    \begin{itemize}
      \item \underline{Reflexiva:} $(x, x) \in$ ker $F$. Trivial pues $F(x) = F(x)$.
      \item \underline{Simétrica:} Si $(x, y) \in$ ker $F \Rightarrow (y, x) \in$ ker $F$.
        \PN Si $(x, y) \in$ ker $F \Rightarrow F(x) = F(y)$. Luego, vale también $F(y) = F(x)$.
      \item \underline{Transitiva:} Si $(x, y), (y, z) \in$ ker $F \Rightarrow (x, z) \in$ ker $F$.
        \begin{equation*}
          \left.
          \begin{array}{l}
            (x, y) \in \text{ ker } F \Rightarrow F(x) = F(y) \\
            (y, z) \in \text{ ker } F \Rightarrow F(y) = F(z)
          \end{array}
          \right \rbrace \Rightarrow F(x) = F(y) = F(z)
        \end{equation*}
        \PN Por lo tanto, $(x, z) \in$ ker $F$.
      \end{itemize}

    \PN Supongamos $x \ker F(x^{\prime})$ y $y \ker F(y^{\prime})$, entonces:
    \begin{alignat*}{4}
      F(x \ \SU \ y) &=& F(x) \ \mathsf{s^{\prime}} \ F(y) &=& F(x^{\prime}) \ \mathsf{s^{\prime}} \ F(y^{\prime}) &=&
        F(x^{\prime} \ \SU \ y^{\prime}) \\
      F(x \ \IN \ y) &=& F(x) \ \mathsf{i^{\prime}} \ F(y) &=& F(x^{\prime}) \ \mathsf{i^{\prime}} \ F(y^{\prime}) &=&
        F(x^{\prime} \ \IN \ y^{\prime}) \\
    \end{alignat*}
    \PN lo cual nos dice que $(x \ \SU \ y) \ker F(x^{\prime} \ \SU \ y^{\prime})$ y $(x \ \IN \ y) \ker F(x^{\prime} \
    \IN \ y^{\prime})$.
  \end{proof}

  % Lemma 101: Con prueba. Lemma 11.
  \begin{lemma} \label{lemma_11}
    \PN Sea $\RET$ un reticulado y sea $\theta$ una congruencia sobre $\RET$, entonces $\pi_{\theta}$ es un homomorfismo
    de $\RET$ en $\RETCOCIENTADO$. Además $\ker \pi_{\theta} = \theta$.
  \end{lemma}
  \begin{proof}
    \PN Sean $x, y \in L$ elementos cualquier. Tenemos que:
      \begin{alignat*}{4}
        \pi_{\theta}(x \ \SU \ y) &=& (x \ \SU \ y)/\theta &=& x/\theta \ \mathsf{\tilde{s}} \ y/\theta &=&
          \pi_{\theta}(x) \ \mathsf{\tilde{s}} \ \pi_{\theta}(y) \\
        \pi_{\theta}(x \ \IN \ y) &=& (x \ \IN \ y)/\theta &=& x/\theta \ \mathsf{\tilde{\imath}} \ y/\theta &=&
          \pi_{\theta}(x) \ \mathsf{\tilde{\imath}} \ \pi_{\theta}(y) \\
      \end{alignat*}
    \PN por lo cual $\pi_{\theta}$ preserva las operaciones de supremo e ínfimo.
  \end{proof}

  % Lemma 102: Con prueba. Lemma 12.
  \begin{lemma} \label{lemma_12}
    \PN Si $F: \ACOTADO \rightarrow \ACOTADOPRIMO$ un homomorfismo biyectivo, entonces $F$ es un isomorfismo.
  \end{lemma}
  \begin{proof}
      \PN Debemos probar que $F^{-1}$ es un homomorfismo. Sean $F(x), F(y)$ dos elementos cualesquiera de $L^{\prime}$,
      tenemos que:
      \begin{multicols}{2}
        \begin{eqnarray*}
          F^{-1}(F(1)) &=& F^{-1}(1^{\prime}) \\
          F^{-1}(1^{\prime}) &=& 1 \\
          \\
          F^{-1}(F(x) \ \SU^{\prime} \ F(y)) &=& F^{-1}(F(x \ \SU \ y)) \\
          &=& x \ \SU \ y \\
          &=& F^{-1}(F(x)) \ \SU \ F^{-1}(F(y))
        \end{eqnarray*}

        \begin{eqnarray*}
          F^{-1}(F(0)) &=& F^{-1}(0^{\prime}) \\
          F^{-1}(0^{\prime}) &=& 0 \\
          \\
          F^{-1}(F(x) \ \IN^{\prime} \ F(y)) &=& F^{-1}(F(x \ \IN \ y)) \\
          &=& x \ \IN \ y \\
          &=& F^{-1}(F(x)) \ \IN \ F^{-1}(F(y))
        \end{eqnarray*}
      \end{multicols}

      \PN Luego, $F^{-1}$ es homomorfismo y por lo tanto $F$ es isomorfismo.
  \end{proof}

  % Lemma 103: Con prueba. Lemma 13.
  \begin{lemma} \label{lemma_13}
    \PN Si $F: \ACOTADO \rightarrow \ACOTADOPRIMO$ es un homomorfismo, entonces $I_{F}$ es un subuniverso de $\ACOTADOPRIMO$.
  \end{lemma}
  \begin{proof}
    \PN Dado que $F$ es un homomorfismo de $\RET$ en $ \RETPRIMO$ utilizando el
    \textbf{Lemma~\ref{lemma_6}} tenemos que $I_{F}$ es subuniverso de $\RETPRIMO$ lo
    cual ya que $0^{\prime}, 1^{\prime} \in I_{F}$ implica que $I_{F}$ es un subuniverso de $(L^{\prime}, \SU^{\prime},
    \IN^{\prime}, 0^{\prime}, 1^{\prime})$.
  \end{proof}

  % Lemma 104: Sin prueba. Lemma 14.
  \begin{lemma} \label{lemma_14}
    \PN Si $F: \ACOTADO \rightarrow \ACOTADOPRIMO$ es un homomorfismo de reticulados acotados, entonces $\ker F$ es una
    congruencia sobre $\ACOTADO$.
  \end{lemma}
  \begin{proof}
    \PN Dado que $F$ es un homomorfismo de $\RET$ en $ \RETPRIMO$ utilizando el
    \textbf{Lemma~\ref{lemma_}} tenemos que $ker F$ es una congruencia sobre $\RET$ lo cual por definción nos
    dice que $ker F$ es una congruencia sobre $\ACOTADO$.
  \end{proof}

  % Lemma 105: Sin prueba. Lemma 15.
  \begin{lemma} \label{lemma_15}
    \PN Sea $\ACOTADO$ un reticulado acotado y $\theta$ una congruencia sobre $\ACOTADO$, entonces:
    \begin{enumerate}[a)]
      \item $\ACOTADOCOCIENTADO$ es un reticulado acotado.
      \item $\pi_{\theta}$ es un homomorfismo de $\ACOTADO$ en $\ACOTADOCOCIENTADO$ cuyo núcleo es $\theta$.
    \end{enumerate}
  \end{lemma}

  % Lemma 106: Sin prueba. Lemma 16.
  \begin{lemma} \label{lemma_16}
    \PN Si $F:\COMPLEMENTADO \rightarrow \COMPLEMENTADOPRIMO$ un homomorfismo biyectivo, entonces $F$ es un isomorfismo.
  \end{lemma}
  \begin{proof}
  \end{proof}

  % Lemma 107: Sin prueba. Lemma 17.
  \begin{lemma} \label{lemma_17}
    \PN Si $F: \COMPLEMENTADO \rightarrow \COMPLEMENTADOPRIMO$ es un homomorfismo, entonces $I_{F}$ es un subuniverso de
    $\COMPLEMENTADOPRIMO$.
  \end{lemma}
  \begin{proof}
  \end{proof}

  % Lemma 108: Con prueba. Lemma 18.
  \begin{lemma} \label{lemma_18}
    \PN Si $F: \COMPLEMENTADO \rightarrow \COMPLEMENTADOPRIMO$ es un homomorfismo de reticulados complementados,
    entonces $\ker F$ es una congruencia sobre $\COMPLEMENTADO$.
  \end{lemma}
  \begin{proof}
    \PN Ya que $F$ es un homomorfismo de $\COMPLEMENTADO$ en $(L^{\prime}, \SU^{\prime}, \IN^{\prime},
    ^{c^{\prime}}, 0^{\prime}$, tenemos por \textbf{Lemma~\ref{lemma_14}} que $ker F$ es una congruencia sobre $(L, \SU,
    \IN, ^{c}, 0, 1)$, es decir, solo falta probar que para todos $x, y \in L$ se tiene que $x / ker F = y / ker F$
    implica $x^{c} / ker F = y^{c} / ker F$, lo cual es dejado al lector.
  \end{proof}

  % Lemma 109: Con prueba. Lemma 19.
  \begin{lemma} \label{lemma_19}
    \PN Sea $\COMPLEMENTADO$ un reticulado complementado y sea $\theta$ una congruencia sobre $\COMPLEMENTADO$.
    \begin{enumerate}[a)]
      \item $\COMPLEMENTADOCOCIENTADO$ es un reticulado complementado.
      \item $\pi_{\theta}$ es un homomorfismo de $\COMPLEMENTADO$ en $\COMPLEMENTADOCOCIENTADO$ cuyo núcleo es $\theta$.
    \end{enumerate}
  \end{lemma}
  \begin{proof}
    \begin{itemize}[a)]
      \item
    \end{itemize}
  \end{proof}

  % Lemma 110: Con prueba. Lemma 20.
  \begin{lemma} \label{lemma_20}
    \PN Sea $\RET$ un reticulado. Son equivalentes:
    \begin{enumerate}[(1)]
      \item $x \ \IN \ (y \ \SU \ z) = (x \ \IN \ y) \ \SU \ (x \ \IN \ z)$, cualesquiera sean $x, y, z \in L$
      \item $x \ \SU \ (y \ \IN \ z) = (x \ \SU \ y) \ \IN \ (x \ \SU \ z)$, cualesquiera sean $x, y, z \in L$.
    \end{enumerate}
  \end{lemma}
  \begin{proof}
    \PN \begin{tabular}{|c|} \hline $(1) \Rightarrow (2)$ \\\hline \end{tabular} Notar que:
    \begin{eqnarray*}
      (x \ \SU \ y) \ \IN \ (x \ \SU \ z) &=& ((x \ \SU \ y) \ \IN \ x) \ \SU \ ((x \ \SU \ y) \ \IN \ z) \\
      &=& (x \ \SU \ (z \ \IN \ (x \ \SU \ y)) \\
      &=& (x \ \SU \ ((z \ \IN \ x) \ \SU \ (z \ \IN \ y)) \\
      &=& (x \ \SU \ (z \ \IN \ x)) \ \SU \ (z \ \IN \ y) \\
      &=& x \ \SU \ (z \ \IN \ y) \\
      &=& x \ \SU \ (y \ \IN \ z)
    \end{eqnarray*}

    \PN \begin{tabular}{|c|} \hline $(2) \Rightarrow (1)$ \\\hline \end{tabular} Notar que:
    \begin{eqnarray*}
      (x \ \IN \ y) \ \SU \ (x \ \IN \ z) &=& ((x \ \IN \ y) \ \SU \ x) \ \IN \ ((x \ \IN \ y) \ \SU \ z) \\
      &=& (x \ \IN \ (z \ \SU \ (x \ \IN \ y)) \\
      &=& (x \ \IN \ ((z \ \SU \ x) \ \IN \ (z \ \SU \ y)) \\
      &=& (x \ \IN \ (z \ \SU \ x)) \ \IN \ (z \ \SU \ y) \\
      &=& x \ \IN \ (z \ \SU \ y) \\
      &=& x \ \IN \ (y \ \SU \ z)
    \end{eqnarray*}
  \end{proof}

  % Lemma 111: Con prueba. Lemma 21.
  \begin{lemma} \label{lemma_21}
    \PN Si $\ACOTADO$ un reticulado acotado y distributivo, entonces todo elemento tiene a lo sumo un complemento.
  \end{lemma}
  \begin{proof}
    \PN Supongamos $x \in L$ tiene complementos $y, z$. Se tiene:
    \begin{alignat*}{3}
      y \ \SU \ x &=& \ 1 &=& \ x \ \SU \ z \\
      y \ \IN \ x &=& \ 0 &=& \ x \ \IN \ z
    \end{alignat*}
    \PN por lo cual:
    \[
      y = y \ \SU \ 0 = y \ \SU \ (x \ \IN \ z) = (y \ \SU \ x) \ \IN \ (y \ \SU \ z) = 1 \ \IN \ (y \ \SU \ z) = (x \
      \SU \ z) \ \IN \ (y \ \SU \ z) = (x \ \IN \ y) \ \SU \ z = 0 \ \SU \ z = z
    \]

    \PN Por lo tanto, $y = z$.
  \end{proof}

  % Lemma 112: Con prueba. Lemma 22.
  \begin{lemma} \label{lemma_22}
    \PN Si $S \neq \emptyset$, entonces $[S)$ es un filtro. Más aún si $F$ es un filtro y $F \supseteq S$, entonces $F
    \supseteq \lbrack S)$, es decir, \lbrack S) es el menor filtro que contiene a S.
  \end{lemma}
  \begin{proof}
    \PN Recordemos:
    \[
      [S) = \{y \in L: y \geq s_{1} \ \IN \ \dotsc \ \IN \ s_{n}, \text{ para algunos } s_{1}, \dotsc, s_{n} \in S,
      n \geq 1\}
    \]

    \begin{enumerate}
      \item \begin{tabular}{|c|} \hline $[S) \neq \emptyset$: \\\hline \end{tabular} Ya que $S \subseteq \lbrack S)$,
        tenemos que $[S) \neq \emptyset$.
      \item \begin{tabular}{|c|} \hline $x, y \in [S) \Rightarrow x \ \IN \ y \in [S)$: \\\hline \end{tabular} Sean $x,
        y$ tales que:
        \begin{eqnarray*}
          y \geq s_{1} \ \IN \ s_{2} \ \IN \ \dotsc \ \IN \ s_{n}, \ \text{ i.e, } y \in [S) \\
          z \geq t_{1} \ \IN \ t_{2} \ \IN \ \dotsc \ \IN \ t_{m}, \ \text{ i.e, } z \in [S)
        \end{eqnarray*}
        \PN con $s_{1}, s_{2}, \dotsc, s_{n}, t_{1}, t_{2}, \dotsc, t_{m} \in S$, entonces:
        \[
          y \ \IN \ z \geq s_{1} \ \IN \ s_{2} \ \IN \ \dotsc \ \IN \  s_{n} \ \IN \ t_{1} \ \IN \ t_{2} \ \IN \ \dotsc
          \ \IN \ t_{m}
        \]
      \item \begin{tabular}{|c|} \hline $x \in [S)$ y $x \leq y \Rightarrow y \in [S)$: \\\hline \end{tabular} Por
        construcción, claramente $[S)$ cumple esta propiedad.
    \end{enumerate}
  \end{proof}

  % Lemma 113: Sin prueba. Lemma 23.
  \begin{lemma} \label{lemma_23}
    \PN (\textbf{Zorn}) Sea $\POSET$ un poset y supongamos que cada cadena de $\POSET$ tiene una cota superior, entonces
    existe un elemento maximal en $\POSET$.
  \end{lemma}

  % Theorem 114: Con prueba. Theorem 24.
  \begin{theorem} \label{theorem_24}
    \PN \textbf{(Teorema del Filtro Primo)} Sea $\RET$ un reticulado distributivo y $F$ un filtro. Supongamos $x_{0} \in
    L-F$, entonces hay un filtro primo $P$ tal que $x_{0} \notin P$ y $F \subseteq P$.
  \end{theorem}
  \begin{proof}
    \PN Sea:
    \[
      \mathcal{F} = \{F_{1}: F_{1} \text{ es un filtro, } x_{0} \notin F_{1} \text{ y } F \subseteq F_{1}\}
    \]

    \PN Notar que $\mathcal{F} \neq \emptyset$, por lo cual $(\mathcal{F}, \subseteq)$ es un poset.
    \PN Veamos que cada cadena en $(\mathcal{F}, \subseteq)$ tiene una cota superior. Sea $C$ una cadena.
    \begin{itemize}
      \item Si $C = \emptyset$, entonces cualquier elemento de $\mathcal{F}$ es cota de $C$.
      \item Si $C \neq \emptyset$. Sea:
        \[
          G = \{x \in L: x \in F_{1}, \text{ para algún } F_{1} \in C\}
        \]
    \end{itemize}

    \PN Veamos que $G$ es un filtro.
    \begin{enumerate}
      \item Es claro que $G \neq \emptyset$.
      \item Supongamos que $ x,y\in G$. Sean $F_{1},F_{2}\in \mathcal{F}$ tales que $x\in F_{1}$ y $y\in F_{2}$.
      \begin{itemize}
        \item Si $F_{1}\subseteq F_{2}$, entonces ya que $F_{2}$ es un filtro tenemos que $x \ \IN \ y\in F_{2}\subseteq G$.
        \item Si $F_{2}\subseteq F_{1}$ , entonces tenemos que $x \ \IN \ y \in F_{1} \subseteq G$.
      \end{itemize}
      \PN Ya que $C$ es una cadena, tenemos que siempre $x \ \IN \ y \in G$.

      \item En forma analoga se prueba la propiedad restante ... % TODO
    \end{enumerate}

    \PN Por lo tanto, tenemos que $G$ es un filtro. Además $x_{0} \notin G$, por lo que $G \in \mathcal{F}$ es cota
    superior de $C$. Por el \textbf{Lemma~\ref{lemma_23}}, $(\mathcal{F}, \subseteq)$ tiene un elemento maximal $P$.
    Veamos que $P$ es un filtro primo.

    \vspace{2mm}
    \PN Supongamos $x \ \SU \ y \in P$ y $x, y \notin P$, entonces ya que $P$ es maximal tenemos que:
    \[
      x_{0} \in \lbrack P \cup \{x\}) \cap \lbrack P \cup \{y\})
    \]
    \PN Ya que $x_{0} \in \lbrack P \cup \{x\})$, tenemos que hay elementos $p_{1}, \dotsc, p_{n} \in P$, tales que:
    \[
      x_{0} \geq p_{1} \ \IN \ \dotsc \ \IN \ p_{n} \ \IN \ x
    \]
    \PN Ya que $x_{0} \in \lbrack P \cup \{y\})$, tenemos que hay elementos $q_{1}, \dotsc, q_{m} \in P$, tales que:
    \[
      x_{0} \geq q_{1} \ \IN \ \dotsc \ \IN \ q_{m} \ \IN \ y
    \]
    \PN Denotemos:
    \[
      p = p_{1} \ \IN \ \dotsc \ \IN \ p_{n} \ \IN \ q_{1} \ \IN \  \dotsc \ \IN \ q_{m}
    \]
    \PN tenemos que:
    \begin{eqnarray*}
      x_{0} &\geq& p \ \IN \ x \\
      x_{0} &\geq& p \ \IN \ y
    \end{eqnarray*}
    \PN Se tiene que $x_{0} \geq (p \ \IN \ x) \ \SU \ (p \ \IN \ y) = p \ \IN \ (x \ \SU \ y) \in P$, lo cual es
    absurdo ya que $x_{0} \notin P$.
  \end{proof}

  % Corollary 115: Con prueba. Corollary 25.
  \begin{corollary} \label{corollary_25}
    \PN Sea $\ACOTADO$ un reticulado acotado distributivo. Si $\emptyset \neq S \subseteq L$ es tal que $s_{1} \ \IN \
    s_{2} \ \IN \ \dotsc \ \IN \ s_{n} \neq 0$, para cada $s_{1}, \dotsc, s_{n} \in S$, entonces hay un filtro primo que
    contiene a $S$.
  \end{corollary}
  \begin{proof}
    \PN Dado que $[S) \neq L$, se puede aplicar el \textbf{Theorem~\ref{theorem_24}} (Teorema del filtro primo).
  \end{proof}

  % Lemma 116: Con prueba. Lemma 26.
  \begin{lemma} \label{lemma_26}
    \PN Sea $\BOOLE$ un algebra de Boole, entonces para un filtro $F \subseteq B$ las siguientes son equivalentes:
    \begin{enumerate}[(1)]
      \item $F$ es primo
      \item $x \in F$ ó $x^{c} \in F$, para cada $x \in B$.
    \end{enumerate}
  \end{lemma}
  \begin{proof}
    \begin{tabular}{|c|} \hline $(1) \Rightarrow (2)$\\\hline \end{tabular} Ya que $x \ \SU \ x^{c} = 1 \in F$, y F es
      filtro primo, por definición de filtro primo se cumple que $x \in F$ ó $x^{c} \in F$.

    \vspace{3mm}
    \begin{tabular}{|c|} \hline $(2) \Rightarrow (1)$\\\hline \end{tabular} Supongamos que $x \ \SU \ y \in F$ y que
    $x \notin F$, entonces por (2), $x^{c} \in F$ y por lo tanto tenemos que:
    \[
      y \geq x^{c} \ \IN \ y = 0 \ \SU \ (x^{c} \ \IN \ y) = (x^{c} \ \IN \ x) \ \SU \ (x^{c} \ \IN \ y) = x^{c} \ \IN \
      (x \ \SU \ y) \in F
    \]
    \PN lo cual dice que $y \in F$.
  \end{proof}

  % Lemma 117: Con prueba. Lemma 27.
  \begin{lemma} \label{lemma_27}
    \PN Sea $\BOOLE$ un álgebra de Boole. Supongamos que $b \neq 0$ y $a = \inf A$, con $A \subseteq B$, entonces si $b
    \ \IN \ a = 0$ existe un $e \in A$ tal que $b \ \IN \ e^{c} \neq 0$.
  \end{lemma}
  \begin{proof}
    \PN Supongamos que para cada $e \in A$, tengamos que $b \ \IN \ e^{c} = 0$, entonces tenemos que para cada
    $e \in A$,
    \[
      b = b \ \IN \ (e \ \SU \ e^{c}) = (b \ \IN \ e)\ \SU \ (b \ \IN \ e^{c}) = b \ \IN \ e
    \]
    \PN lo cual nos dice que $b$ es cota inferior de $A$. Pero si $b \leq a$, entonces $b = b \ \IN \ a = 0$, es decir,
    $b = 0$, lo cual es un absurdo dado que por hipótesis sabíamos que $b \neq 0$.
  \end{proof}

  % Theorem 118: Con prueba. Theorem 28.
  \begin{theorem} \label{theorem_28}
    \PN \textbf{(Rasiova y Sikorski)} Sea $\BOOLE$ un álgebra de Boole. Sea $x \in B$, tal que $x \neq 0$. Supongamos
    que $A_{1}, A_{2}, \dotsc$ son subconjuntos de $B$ tales que existe $\inf(A_{j})$, para cada $j = 1, 2, \dotsc$,
    entonces hay un filtro primo $P$ el cual cumple:
    \begin{enumerate}[a)]
      \item $x \in P$
      \item $A_{j} \subseteq P \Rightarrow \inf(A_{j}) \in P$, para cada $j = 1, 2, \; \dotsc$
    \end{enumerate}
  \end{theorem}
  \begin{proof}
    \PN Sea $a_{j} = \inf(A_{j})$, para $j = 1, 2, \; \dotsc $ construiremos inductivamente una sucesión $b_{0},
    b_{1}, \dotsc$ de elementos de $B$ tal que:
    \begin{itemize}
      \item $b_{0} = x$
      \item $b_{0} \ \IN \ \dotsc \ \IN \ b_{n} \neq 0$, para cada $n \geq 0$
      \item $b_{j} = a_{j}$ ó $b_{j}^{c} \in A_{j}$, para cada $j \geq 1$
    \end{itemize}
    \begin{enumerate}[(1)]
      \item Definamos $b_{0} = x$
      \item Supongamos ya definimos $b_{0}, \dotsc, b_{n}$, veamos como definir $b_{n+1}$.
        \begin{itemize}
          \item Si $(b_{0} \ \IN \ \dotsc \ \IN \  b_{n}) \ \IN \ a_{n+1}\neq 0$, entonces definamos $b_{n+1} = a_{n+1}$.
          \item Si $(b_{0} \ \IN \ \dotsc \ \IN \ b_{n}) \ \IN \ a_{n+1}=0$, entonces por el
            \textbf{Lemma~\ref{lemma_27}}, tenemos que hay un $e \in A_{n+1}$ tal que $(b_{0} \ \IN \ \dotsc \ \IN \
            b_{n}) \ \IN \ e^{c}\neq 0$, lo cual nos permite definir $b_{n+1} = e^{c}$.
        \end{itemize}
    \end{enumerate}

    \PN Dado que el conjunto $S = \{b_{0}, b_{1}, \dotsc\}$ satisface la hipótesis del
    \textbf{Corollary~\ref{corollary_25}}, por lo tanto hay un filtro primo $P$ tal que $\{b_{0}, b_{1}, \dotsc\}
    \subseteq P$, el cual satisface las propiedades (a) y (b) dado que así lo construimos.
  \end{proof}

	\section{Términos y fórmulas}

  % Lemma 119. Con prueba. Lemma 29.
  \begin{lemma} \label{lemma_29}
    \PN Supongamos $t \in T_{k}^{\tau}$, con $k \geq 1$, entonces ya sea $t \in Var \cup \mathcal{C}$ ó $t = f(t_{1},
    \dotsc, t_{n})$, con $f \in \mathcal{F}_{n}, n \geq 1, \; t_{1}, \dotsc, t_{n} \in T_{k-1}^{\tau}$.
  \end{lemma}
  \begin{proof}
    \PN Probaremos este teorema por inducción en $k$.

    \vspace{3mm}
    \PN \underline{Caso Base:} \begin{tabular}{|c|} \hline $k = 1$ \\\hline \end{tabular} Es directo, ya que por
    definición:
    \[
      T_{1}^{\tau} = Var \cup \mathcal{C} \cup \{f(t_{1}, t_{2}, \dotsc t_{n}): f \in \mathcal{F}_n, n \geq 1, t_{1},
      t_{2}, \dotsc t_{n} \in T_{0}^{\tau}\}
    \]

		\PN \underline{Caso Inductivo:} \begin{tabular}{|c|} \hline $k > 1$ \\\hline \end{tabular} Sea $t \in
    T_{k+1}^{\tau}$. Por definición de $ T_{k+1}^{\tau}$ tenemos que:
    \begin{itemize}
      \item $t \in T_{k}^{\tau}$ ó
      \item $t = f(t_{1}, \dotsc, t_{n})$ con $f \in \mathcal{F}_{n}, n \geq 1$ y $t_{1}, \dotsc, t_{n}\in T_{k}^{\tau}$.
    \end{itemize}

    \PN Si se da que $t \in T_{k}^{\tau}$, entonces podemos aplicar hipótesis inductiva y usar que $T_{k-1}^{\tau}
    \subseteq T_{k}^{\tau}$.
  \end{proof}

  % Lemma 120. Con prueba. Lemma 30.
  \begin{lemma} \label{lemma_30}
    \PN Sea $b \in Bal$. Se tiene:
    \begin{enumerate}[(1)]
      \item $\lvert b \rvert_{(} - \lvert b \rvert_{)} = 0$
      \item Si $x$ es tramo inicial propio de $b$, entonces $\lvert x \rvert_{(} - \lvert x \rvert_{)} > 0$
      \item Si $x$ es tramo final propio de $b$, entonces $\lvert x \rvert_{(} - \lvert x \rvert_{)} < 0$
    \end{enumerate}
  \end{lemma}
  \begin{proof}
    \PN Probaremos por inducción en $k$, que valen (1), (2) y (3) para cada $b \in Bal_{k}$.

    \vspace{3mm}
    \PN \underline{Caso Base:} \begin{tabular}{|c|} \hline $k = 1$ \\\hline \end{tabular}
    \begin{enumerate}[(1)]
      \item $Bal_{1} = \{()\}$. Luego, $\lvert b \rvert_{(} = \lvert b \rvert_{)} = 1$. Por lo tanto, $\lvert b
       \rvert_{(} - \lvert b \rvert_{)} = 0$.
      \item Supongamos x tramo inicial propio de b. Luego $x = ($, es decir, $\lvert x \rvert_{(} = 1$ y $\lvert x
      \rvert_{)} = 0$. Por lo tanto, $\lvert x \rvert_{(} - \lvert x \rvert_{)} > 0$.
      \item Supongamos x tramo final propio de b. Luego $x = )$, es decir, $\lvert x \rvert_{(} = 0$ y $\lvert x
      \rvert_{)} = 1$. Por lo tanto, $\lvert x \rvert_{(} - \lvert x \rvert_{)} < 0$.
    \end{enumerate}

  	\PN \underline{Caso Inductivo:} \begin{tabular}{|c|} \hline $k > 1$ \\\hline \end{tabular} Supongamos $b \in
    Bal_{k+1}$. Si $b \in Bal_{k}$, se aplica directamente HI para cualquiera de los casos. Supongamos entonces que
    $b = (b_{1} \dotsc b_{n})$, con $b_{1}, \dotsc, b_{n} \in Bal_{k}, n \geq 1$.

    \begin{enumerate}[(1)]
      \item Por HI, $b_{1}, \dotsc, b_{n}$ satifacen $\lvert b \rvert_{(} - \lvert b \rvert_{)} = 0$. Luego, $(b_{1},
        \dotsc, b_{n})$ también satisface, es decir, al agregar un paréntesis de cada tipo, el balanceo se mantiene.

      \item Sea $x$ un tramo inicial propio de $b$. Notese que $x$ es de la forma $x = (b_{1} \dotsc b_{i} y$ con
        $0 \leq i \leq n-1$ y $y$ un tramo inicial de $b_{i+1}$, pero entonces:
        \[
          \lvert x \rvert_{(} - \lvert x \rvert_{)} = 1 + \left(\sum_{j=1}^{i} \lvert b_{j} \rvert_{(} - \lvert b_{j}
          \rvert_{)}\right) + \lvert y \rvert_{(} - \lvert y \rvert_{)}
        \]
      tenemos que por HI, se da que $\lvert x \rvert_{(} - \lvert x \rvert_{)} > 0$.

      \item Sea $x$ un tramo final propio de $b$. Notese que $x$ es de la forma $x = y b_{1} \dotsc b_{i})$ con
        $1 \leq i \leq n$ y $y$ un tramo final de $b_{i+1}$, pero entonces:
        \[
          \lvert x \rvert_{(} - \lvert x \rvert_{)} = \lvert y \rvert_{(} - \lvert y \rvert_{)} + \left(\sum_{j=1}^{i}
          \lvert b_{j} \rvert_{(} - \lvert b_{j} \rvert_{)}\right) + 1
        \]
      tenemos que por HI, se da que $\lvert x \rvert_{(} - \lvert x \rvert_{)} < 0$.
    \end{enumerate}
  \end{proof}

  % Lemma 121. Sin prueba. Lemma 31.
  \begin{lemma} \label{lemma_31}
    \PN $del(xy) = del(x)del(y) \ \forall x, y \in \SIGMA$.
  \end{lemma}

  % Lemma 122. Sin prueba. Lemma 32.
  \begin{lemma} \label{lemma_32}
    \PN Supongamos que $\Sigma$ es tal que $\TAU \subseteq \SIGMA$, entonces $del(t) \in Bal$, para cada $t \in \TAU -
    (Var \cup \mathcal{C})$.
  \end{lemma}

  % Lemma 123. Con prueba. Lemma 33.
  \begin{lemma} \label{lemma_33}
    \PN Sean $s, t \in \TAU$ y supongamos que hay palabras $x,y,z$, con $y \neq \varepsilon$ tales que $s = xy$ y
    $t = yz$, entonces $x = z = \varepsilon$ ó $s, t \in \mathcal{C}$. En particular, si un término es tramo inicial o
    final de otro término, entonces dichos términos son iguales.
  \end{lemma}
  \begin{proof}
    \begin{itemize}
      \item \underline{Supongamos $s \in \mathcal{C}$}: Ya que $y \neq \varepsilon$ tenemos que $t$ debe comenzar con un
        símbolo que ocurre en un nombre de constante, lo cual dice que $t$ no puede ser ni una variable ni de la forma
        $g(t_{1}, \dotsc, t_{m})$, es decir $t \in \mathcal{C}$.

      \item \underline{Supongamos $s \in Var$}: Si sucediese que $x \neq \varepsilon$, $t$ comenzaría con alguno de los
        siguientes símbolos:
        \[
          \mathit{0} \ \mathit{1} \ \dotsc \ \mathit{9} \ \mathbf{0} \ \mathbf{1} \ \dotsc \ \mathbf{9}
        \]
        \PN lo cual es absurdo. Luego, $x = \varepsilon$ y por lo tanto $t$ debe comenzar con $\mathsf{X}$, pero esto
        dice que $t \in Var$ de lo que sigue que $z = \varepsilon$.

      \item \underline{Supongamos que $s$ es de la forma $f(s_{1}, \dotsc, s_{n})$}: Ya que $)$ debe ocurrir en $t$,
        tenemos que $t$ es de la forma $g(t_{1}, \dotsc, t_{m})$, osea que $del(s), del(t) \in Bal$. Ya que $)$ ocurre
        en $y$, $del(y) \neq \varepsilon$. Tenemos también que:
        \begin{eqnarray*}
          del(s) &=& del(x)del(y) \\
          del(t) &=& del(y)del(z)
        \end{eqnarray*}

        \PN Utilizando el \textbf{Lemma~\ref{lemma_30}} obtenemos que:
        \begin{eqnarray*}
          1) \ \lvert del(y) \rvert_{(} - \lvert del(y) \rvert_{)} \leq 0 \\
          2) \ \lvert del(y) \rvert_{(} - \lvert del(y) \rvert_{)} \geq 0
        \end{eqnarray*}
        \PN por lo cual:
        \[
          \lvert del(y) \rvert_{(} - \lvert del(y) \rvert_{)} = 0
        \]
        \PN pero entonces ya que $del(y)$ es tramo final de $del(s)$, el \textbf{Lemma~\ref{lemma_30}} nos dice que
        $del(x) = \varepsilon$. De la misma manera, obtenemos que $del(z) = \varepsilon$. Ya que que $t$ termina con $)$
        tenemos que $z = \varepsilon$, osea que $f(s_{1}, \dotsc, s_{n}) = xg(t_{1}, \dotsc, t_{m})$ con $del(x) =
        \varepsilon$, de lo que se desprende que $f=xg$ ya que $($ no ocurre en $x$. Finalmente, de la definición de
        tipo se desprende que $x = \varepsilon$.
    \end{itemize}
  \end{proof}

  % Theorem 124. Con prueba. Lemma 34.
  \begin{theorem} \label{lemma_34}
    \PN \textbf{(Lectura única de terminos)}. Dado $t \in \TAU$ se da una de las siguientes:
    \begin{enumerate}[(1)]
      \item $t \in Var \cup \mathcal{C}$
      \item Hay únicos $n \geq 1, \ f \in \mathcal{F}_{n}, \ t_{1}, \dotsc, t_{n} \in \TAU$ tales que $t = f(t_{1},
        \dotsc, t_{n})$.
    \end{enumerate}
  \end{theorem}
  \begin{proof}
    \PN En virtud del \textbf{Lemma~\ref{lemma_29}} solo nos resta probar la unicidad de $t_{1}, \dotsc, t_{n}$ en el
    punto (2). Supongamos que:
    \[
      t = f(t_{1}, \dotsc, t_{n}) = g(s_{1}, \dotsc, s_{m})
    \]
    \PN con $n, m \geq 1, \ f \in \mathcal{F}_{n}, g \in \mathcal{F}_{m}, t_{1}, \dotsc, t_{n}, s_{1}, \dotsc, s_{m} \in
    \TAU$. Notese que $f = g$, es decir, $n = m = a(f)$. Notemos que:
    \begin{itemize}
      \item $t_{1}$ es tramo inicial de $s_{1}$ ó
      \item $s_{1}$ es tramo inicial de $t_{1}$
    \end{itemize}

    \PN En general:
    \begin{itemize}
      \item $t_{i}$ es tramo inicial de $s_{i}$ ó
      \item $s_{i}$ es tramo inicial de $t_{i}$, para $1 \leq i \leq n$.
    \end{itemize}


    \PN lo cual por el \textbf{Lemma~\ref{lemma_33}} nos dice que $t_{i} = s_{i}$ con $1 \leq i \leq n$.
  \end{proof}

  % Lemma 125. Con prueba. Lemma 35.
  \begin{lemma} \label{lemma_35}
    \PN Sean $r, s, t \in \TAU$.
    \begin{enumerate}[(a)]
      \item Si $s \neq t = f(t_{1}, \dotsc, t_{n})$ y $s$ ocurre en $t$, entonces dicha ocurrencia sucede dentro de
        algún $t_{j}$, $j = 1, \dotsc, n$.
      \item Si $r, s$ ocurren en $t$, entonces dichas ocurrencias son disjuntas o una ocurre dentro de otra. En
        particular, las distintas ocurrencias de $r$ en $t$ son disjuntas.
      \item Si $t^{\prime}$ es el resultado de reemplazar una ocurrencia de $s$ en $t$ por $r$, entonces $t^{\prime} \in
        \TAU$.
    \end{enumerate}
  \end{lemma}
  \begin{proof}
    \begin{enumerate}[(a)]
      \item Supongamos la ocurrencia de $s$ comienza en algún $t_{j}$, entonces el \textbf{Lemma~\ref{lemma_33}} nos
      conduce a que dicha ocurrencia debera estar contenida en $t_{j}$. Veamos que la ocurrencia de $s$ no puede ser a
      partir de un $i \in \{1, \dotsc, \lvert f \rvert\}$. Supongamos lo contrario. Tenemos entonces que $s$ debe ser
      de la forma $g(s_{1}, \dotsc,s_{m})$ ya que no puede estar en $Var \cup \mathcal{C}$. Notese que $i\neq 1$ ya que
      en caso contrario $s$ seria un tramo inicial propio de $t$. Pero entonces $g$ debe ser un tramo final propio de $f$,
      lo cual es absurdo. Ya que $s$ no puede comenzar con parentesis o coma, hemos contemplado todos los posibles casos
      de comienzo de la ocurrencia de $s$ en $t$.

      \item Por induccion, usando (a).
      \item Por induccion, usando (a).
    \end{enumerate}
  \end{proof}

  % Lemma 126. Con prueba. Lemma 36.
  \begin{lemma} \label{lemma_36}
    \PN Supongamos $\varphi \in F_{k}^{\tau}$, con $k \geq 1$, entonces $\varphi$ es de alguna de las siguientes formas:
    \begin{itemize}
      \item $\varphi = (t \equiv s)$, con $t, s \in \TAU$.
      \item $\varphi = r(t_{1}, \dotsc, t_{n})$, con $r \in \mathcal{R}_{n}, t_{1}, \dotsc, t_{n} \in \TAU$.
      \item $\varphi = (\varphi_{1} \eta \varphi_{2})$, con $\eta \in \{\wedge, \vee, \rightarrow, \leftrightarrow\}, \
        \varphi_{1}, \varphi_{2} \in F_{k-1}^{\tau}$.
      \item $\varphi = \lnot \varphi_{1}$, con $\varphi_{1} \in F_{k-1}^{\tau}$.
      \item $\varphi = Qv\varphi_{1}$, con $Q \in \{\forall, \exists\}, \ v \in Var$ y $\varphi_{1} \in F_{k-1}^{\tau}$.
    \end{itemize}
    \PN Llamaremos ($\star$) a la lista anterior.
  \end{lemma}
  \begin{proof}
    \PN Probaremos este teorema por inducción en $k$, utilizando la definición del conjunto $\FT$.

    \vspace{3mm}
    \PN \underline{Caso Base:}
    \[
      \varphi \ \in \ \{(t \equiv s): t, s \in \TAU\} \ \cup \ \{r(t_{1}, \dotsc, t_{n}): r \in \mathcal{R}_{n}, n \geq
      1, t_{1}, \dotsc, t_{n} \in \TAU\}
    \]
    \PN por lo que $\varphi$ es de alguna de las siguientes formas:
    \begin{itemize}
      \item $\varphi = (t \equiv s)$, con $t, s \in \TAU$.
      \item $\varphi = r(t_{1}, \dotsc, t_{n})$, con $r \in \mathcal{R}_{n}, t_{1}, \dotsc, t_{n} \in \TAU$.
    \end{itemize}

    \vspace{3mm}
		\PN \underline{Caso Inductivo:} Supongamos que si $\varphi \in F_{k-1}^{\tau}$ entonces $\varphi$ es de alguna de
    las formas de ($\star$). Probaremos que si $\varphi \in F_{k}^{\tau}$ entonces $\varphi$ también es de alguna de las
    formas de la lista ($\star$).
    \begin{eqnarray*}
      \varphi \ \in \ F_{k-1}^{\tau} &\cup& \{\lnot \varphi: \varphi \in F_{k-1}^{\tau}\} \ \cup \ \{(\varphi \eta \psi):
        \varphi, \psi \in F_{k-1}^{\tau}, \eta \in \{\vee, \wedge, \rightarrow, \leftrightarrow\}\} \\
      &\cup& \ \{Qv\varphi: \varphi \in F_{k-1}^{\tau}, v \in Var, Q \in \{\forall, \exists\}\}
    \end{eqnarray*}

    \PN Luego, si $\varphi \in F_{k-1}^{\tau}$ aplicando HI obtenemos que $\varphi$ es de alguna de las formas de la
    lista anterior. Caso contrario, se dá alguna de las siguientes:
    \begin{itemize}
      \item $\varphi = (\varphi_{1} \eta \varphi_{2})$, con $\varphi_{1}, \varphi_{2} \in F_{k-1}^{\tau}, \eta \in
      \{\wedge, \vee, \rightarrow, \leftrightarrow\}$.
      \item $\varphi = \lnot \varphi_{1}$, con $\varphi_{1} \in F_{k-1}^{\tau}$.
      \item $\varphi = Qv\varphi_{1}$, con $Q \in \{\forall, \exists\}, \ v \in Var$ y $\varphi_{1} \in F_{k-1}^{\tau}$.
    \end{itemize}
  \end{proof}

  % Lemma 127. Con prueba. Lemma 37.
  \begin{lemma} \label{lemma_37}
    \PN Sea $\tau$ un tipo.
    \begin{enumerate}[a)]
      \item Supongamos que $\Sigma$ es tal que $\FT \subseteq \SIGMA$, entonces $del(\varphi) \in Bal$, para cada
        $\varphi \in \FT$.
      \item Sea $\varphi \in F_{k}^{\tau}$, con $k \geq 0$, existen $x \in (\{\lnot\} \cup \{Qv: Q \in \{\forall,
        \exists\}, v \in Var\})^{\ast}$ y $\varphi_{1} \in \FT$ tales que $\varphi = x \varphi_{1}$ y
        $\varphi_{1}$ es de la forma $(\psi_{1} \eta \psi_{2})$ o atómica. En particular toda fórmula termina con el
        símbolo $)$.
    \end{enumerate}
  \end{lemma}
  \begin{proof}
    \begin{enumerate}[a)]
      \item HACER!!!!
      \item Induccion en $k$. El caso $k=0$ es trivial. Supongamos (b) vale para cada $\varphi \in F_{k}^{\tau }$ y sea $\varphi \in F_{k+1}^{\tau }$. Hay varios casos de los cuales haremos solo dos
        \begin{itemize}
          \item CASO $\varphi =(\varphi_{1}\eta \varphi_{2})$, con $\varphi_{1},\varphi_{2}\in F_{k}^{\tau }$ y $\eta \in \{\vee ,\wedge ,\rightarrow ,\leftrightarrow \}$.

      Podemos tomar $x=\varepsilon $ y $\varphi_{1}=\varphi $.

          \item CASO $\varphi =Qx_{i}\psi $, con $\psi \in F_{k}^{\tau }$, $i\geq 1$ y $Q\in \{\forall ,\exists \}$.

      Por HI hay $\bar{x}\in (\{\lnot \}\cup \{Qv:Q\in \{\forall ,\exists \}$ y $v\in Var\})^{\ast }$ y $\psi_{1}\in \FT$ tales que $ \psi =x\psi_{1}$ y $\psi_{1}$ es de la forma $(\gamma_{1}\eta \gamma_{2}) $ o atomica. Entonces es claro que $x=Qx_{i}\bar{x}$ y $\varphi_{1}=\psi_{1}$ cumplen (b). $\Box$
      \end{itemize}
    \end{enumerate}
  \end{proof}

  % Lemma 128. Con prueba. Lemma 38.
  \begin{lemma} \label{lemma_38}
    \PN Ninguna fórmula es tramo final propio de una fórmula atómica, es decir, si $\varphi = x \psi$, con $\varphi \in
    F_{0}^{\tau}$ y $\psi \in \FT$, entonces $x = \varepsilon$.
  \end{lemma}
  \begin{proof}
    \begin{itemize}
      \item Si $\varphi$ es de la forma $(t\equiv s)$, entonces $\lvert del(y)\rvert_{(}-\lvert del(y)\rvert_{)}< 0$ para cada tramo final propio $y$ de $\varphi $, lo cual termina el caso ya que $del(\psi )$ es balanceada.
      \item Supongamos entonces $\varphi =r(t_{1}, \dotsc, t_{n})$. Notese que $\psi $ no puede ser tramo final de $t_{1}, \dotsc, t_{n})$ ya que $del(\psi )$ es balanceada y $\lvert del(y)\rvert_{(}-\lvert del(y)\rvert_{)}< 0$ para cada tramo final $y$ de $t_{1}, \dotsc, t_{n})$. Es decir que $\psi =y(t_{1}, \dotsc, t_{n})$, para algun tramo final $y$ de $r$. Ya que en $\psi $ no ocurren cuantificadores ni nexos ni el simbolo $\equiv $ el Lema 124 nos dice $\psi =\tilde{r}(s_{1}, \dotsc,s_{m})$, con $ \tilde{r}\in \mathcal{R}_{m}$, $m\geq 1$ y $s_{1}, \dotsc,s_{m}\in \TAU$. Ahora es facil usando un argumento paresido al usado en la prueba del Teorema 122 concluir que $m=n$, $s_{i}=t_{i}$, $ i=1, \dotsc,n$ y $\tilde{r}$ es tramo final de $r$. Por (3) de la definicion de tipo tenemos que $\tilde{r}=r$ lo cual nos dice que $\varphi =\psi $ y $ x=\varepsilon $ $\Box$
    \end{itemize}
  \end{proof}

  % Lemma 129. Con prueba. Lemma 39.
  \begin{lemma} \label{lemma_39}
    \PN Si $\varphi = x \psi$, con $\varphi, \psi \in \FT$ y $x$ sin paréntesis, entonces $x \in (\{\lnot\} \cup
    \{Qv: Q \in \{\forall, \exists\}$ y $v \in Var\})^{\ast}$.
  \end{lemma}
  \begin{proof}
    \PN Probaremos por inducción en k, tal que $\varphi \in F_{k}^{\tau}$.

    \vspace{3mm}
    \PN \underline{Caso Base:}
    El caso $k=0$ es probado en el lema anterior.

    \vspace{3mm}
		\PN \underline{Caso Inductivo:} Asumamos que el resultado vale cuando $ \varphi \in F_{k}^{\tau }$ y veamos que vale cuando $\varphi \in F_{k+1}^{\tau }$. Mas aun supongamos $\varphi \in F_{k+1}^{\tau }-F_{k}^{\tau }$. Primero haremos el caso en que $\varphi =Qv\varphi_{1}$, con $Q\in \{\forall ,\exists \},\;v\in Var$ y $\varphi_{1}\in F_{k}^{\tau }$ . Supongamos $x\neq \varepsilon $. Ya que $\psi $ no comienza con simbolos de $v$, tenemos que $\psi $ debe ser tramo final de $\varphi_{1}$ lo cual nos dice que hay una palabra $x_{1}$ tal que $x=Qvx_{1}$ y $\varphi_{1}=x_{1}\psi $. Por HI tenemos que $x_{1}\in (\{\lnot \}\cup \{Qv:Q\in \{\forall ,\exists \}$ y $v\in Var\})^{\ast }$ con lo cual $x\in (\{\lnot \}\cup \{Qv:Q\in \{\forall ,\exists \}$ y $v\in Var\})^{\ast }$. El caso en el que $\varphi =\lnot \varphi_{1}$ con $\varphi_{1}\in F_{k}^{\tau }$, es similar. Note que no hay mas casos posibles ya que $\varphi $ no puede comenzar con $($ porque en $x$ no ocurren parentesis por hipotesis $\Box$
  \end{proof}

  % Proposition 130. Con prueba. Lemma 40.
  \begin{proposition} \label{lemma_40}
    \PN Si $\varphi, \psi \in \FT$ y $x, y, z$ son tales que $\varphi = xy, \psi = yz$ y $y \neq \varepsilon$, entonces
    $z = \varepsilon$ y $x \in (\{\lnot\} \cup \{Qv: Q \in \{\forall, \exists\}$ y $v \in Var\})^{\ast}$. En particular
    ningún tramo inicial propio de una fórmula es una fórmula.
  \end{proposition}
  \begin{proof}
    Ya que $\varphi $ termina con $)$ tenemos que $del(y)\neq \varepsilon .$ Ya que $del(\varphi ),del(\psi )\in Bal$ y ademas

    \begin{eqnarray*}
      del(\varphi) &=& del(x)del(y) \\
      del(\psi) &=& del(y)del(z)
    \end{eqnarray*}
    \PN tenemos que $del(y)$ es tramo inicial y final de palabras balanceadas, lo cual nos dice que
    $\displaystyle \lvert del(y)\rvert_{(}-\lvert del(y)\rvert_{)}=0 $

    Pero esto por (3) del Lema 118 nos dice que $ del(x)=\varepsilon $. Similarmente obtenemos que $del(z)=\varepsilon $. Pero $\psi $ termina con $)$ lo cual nos dice que $z=\varepsilon $. Es decir que $ \varphi =x\psi $. Por el lema anterior tenemos que $x\in (\{\lnot \}\cup \{Qv:Q\in \{\forall ,\exists \}$ y $v\in Var\})^{\ast }$ $\Box$
  \end{proof}

  % Theorem 131. Con prueba. Lemma 41.
  \begin{theorem} \label{lemma_41}
    \PN \textbf{(Lectura única de fórmulas)} Dada $\varphi \in \FT$ se da una y solo una de las siguientes:
    \begin{enumerate}[(1)]
      \item $\varphi = (t \equiv s)$, con $t, s \in \TAU$
      \item $\varphi = r(t_{1}, \dotsc, t_{n})$, con $r \in \mathcal{R}_{n}, t_{1}, \dotsc, t_{n} \in \TAU$
      \item $\varphi = (\varphi_{1} \eta \varphi_{2})$, con $\eta \in \{\wedge, \vee, \rightarrow, \leftrightarrow\}, \
        \varphi_{1}, \varphi_{2} \in \FT$
      \item $\varphi = \lnot \varphi_{1}$, con $\varphi_{1} \in \FT$
      \item $\varphi = Qv \varphi_{1}$, con $Q \in \{\forall, \exists\}, \ \varphi_{1} \in \FT$ y $v \in Var$.
    \end{enumerate}

    \PN Más aún, en todos los puntos tales descomposiciones son únicas.
  \end{theorem}
  \begin{proof}
    \begin{enumerate}[(1)]
      \item
      \item
      \item
      \item
      \item
    \end{enumerate}

    Si una formula $\varphi $ satisface (1), entonces $\varphi $ no puede contener simbolos del alfabeto $\{\wedge ,\vee,
    \rightarrow ,\leftrightarrow \}$ lo cual garantiza que $\varphi $ no puede satisfacer (3). Ademas $ \varphi $ no
    puede satisfacer (2) o (4) o (5) ya que $\varphi $ comienza con $($. En forma analoga se puede terminar de ver que
    las propiedades (1), $\dotsc$,(5) son excluyentes.

    La unicidad en las descomposiciones de (4) y (5) es obvia. La de (3) se desprende facilmente del lema anterior y la de los puntos (1) y (2) del lema analogo para terminos. $\Box$
  \end{proof}

  % Lemma 132. Con prueba. Lemma 42.
  \begin{lemma} \label{lemma_42}
    \PN Sea $\tau$ un tipo.
    \begin{enumerate}[(a)]
      \item Las fórmulas atómicas no tienen subfórmulas propias.
      \item Si $\varphi$ ocurre propiamente en $(\psi \eta \varphi)$, entonces tal ocurrencia es en $\psi$ ó en
        $\varphi$.
      \item Si $\varphi$ ocurre propiamente en $\lnot \psi$, entonces tal ocurrencia es en $\psi$.
      \item Si $\varphi$ ocurre propiamente en $Qx_{k} \psi$, entonces tal ocurrencia es en $\psi$.
      \item Si $\varphi_{1}, \varphi_{2}$ ocurren en $\varphi$, entonces dichas ocurrencias son disjuntas o una contiene
        a la otra.
      \item Si $\lambda^{\prime}$ es el resultado de reemplazar alguna ocurrencia de $\varphi$ en $\lambda$ por $\psi$,
        entonces $\lambda^{\prime} \in \FT$.
    \end{enumerate}
  \end{lemma}
  \begin{proof}
    Ejercicio.
  \end{proof}

	\section{Estructuras}

  % % Lemma 131. Con prueba. Lemma 43.
  % \begin{lemma}
  %   \PN Sea $\mathbf{A}$ una estructura de tipo $\tau$ y sea $t\in \TAU$. Supongamos que $\vec{a}, \vec{b}$ son
  %   asignaciones tales que $a_{i} = b_{i},$ cada vez que $x_{i}$ ocurra en $t$, entonces $t^{\mathbf{A}}[\vec{a}] =
  %   t^{\mathbf{A}}[\vec{b}]$.
  % \end{lemma}
  % \begin{proof}
  %   Sea
  %
  %   - Teo$_{k}$: El lema vale para $t\in T_{k}^{\tau }$.
  %   Teo$_{0}$ es facil de probar. Veamos Teo$_{k}\Rightarrow $Teo$_{k+1}$. Supongamos $t\in T_{k+1}^{\tau }-T_{k}^{\tau }$ y sean $\vec{a},\vec{b}$ asignaciones tales que $a_{i}=b_{i},$ cada vez que $x_{i}$ ocurra en $t$. Notese que $t=f(t_{1}, \dotsc, t_{n})$, con $f\in \mathcal{F}_{n},\;n\geq 1$ y $ t_{1}, \dotsc, t_{n}\in \TAU$. Notese que para cada $j=1, \dotsc, n$, tenemos que $a_{i}=b_{i},$ cada vez que $x_{i}$ ocurra en $t_{j}$, lo cual por Teo$_{k}$ nos dice que
  %
  %   $\displaystyle t_{j}^{\mathbf{A}}[\vec{a}]=t_{j}^{\mathbf{A}}[\vec{b}]\text{, }j=1, \dotsc, n $
  %
  %   Se tiene entonces que
  %   $\displaystyle \begin{array}{ccl} t^{\mathbf{A}}[\vec{a}] & = & i(f)(t_{1}^{\mathbf{A}}[\vec{a}], \dotsc, t_{n}^{ \mathbf{A}}[\vec{a}])\text{ (por def de }t^{\mathbf{A}}[\vec{a}]\text{)} \\ & = & i(f)(t_{1}^{\mathbf{A}}[\vec{b}], \dotsc, t_{n}^{\mathbf{A}}[\vec{b}]) \\ & = & t^{\mathbf{A}}[\vec{b}]\text{ (por def de }t^{\mathbf{A}}[\vec{a}] \text{)} \end{array} $
  %
  %   $\Box$
  % \end{proof}
  %
  % % Lemma 132. Con prueba. Lemma 44.
  % \begin{lemma}
  %   \begin{enumerate}[(a)]
  %     \item $Li((t \equiv s)) = \{v \in Var: v$ ocurre en $t$ ó $v$ ocurre en $s\}$
  %     \item $Li(r(t_{1}, \dotsc, ,t_{n})) = \{v \in Var: v$ ocurre en algún $t_{i}\}$
  %     \item $Li(\lnot \varphi) = Li(\varphi)$
  %     \item $Li((\varphi \eta \psi)) = Li(\varphi) \cup Li(\psi)$
  %     \item $Li(Qx_{j}\varphi) = Li(\varphi)-\{x_{j}\}$
  %   \end{enumerate}
  % \end{lemma}
  % \begin{proof}
  %   (a) y (b) son triviales de las definiciones, teniendo en cuenta que si una variable $v$ ocurre en $(t\equiv s)$ (resp. en $r(t_{1}, \dotsc, t_{n})$) entonces $v$ ocurre en $t$ o $v$ ocurre en $s$ (resp.$v$ ocurre en algun $ t_{i}$)
  %
  %   (d) Supongamos $v\in Li((\varphi \eta \psi ))$, entonces hay un $i$ tal que $ v$ ocurre libremente en $(\varphi \eta \psi )$ a partir de $i$. Por definicion tenemos que ya sea $v$ ocurre libremente en $\varphi $ a partir de $i-1$ o $v$ ocurre libremente en $\psi $ a partir de $i-\left\vert (\varphi \eta \right\vert $, con lo cual $v\in Li(\varphi )\cup Li(\psi )$
  %
  %   Supongamos ahora que $v\in Li(\varphi )\cup Li(\psi )$. S.p.d.g. supongamos $ v\in Li(\psi )$. Por definicion tenemos que hay un $i$ tal que $v$ ocurre libremente en $\psi $ a partir de $i$. Pero notese que esto nos dice por definicion que $v$ ocurre libremente en $(\varphi \eta \psi )$ a partir de $ i+\left\vert (\varphi \eta \right\vert $ con lo cual $v\in Li((\varphi \eta \psi ))$.
  %
  %   (c) es similar a (d)
  %
  %   (e) Supongamos $v\in Li(Qx_{j}\varphi )$, entonces hay un $i$ tal que $v$ ocurre libremente en $Qx_{j}\varphi $ a partir de $i$. Por definicion tenemos que $v\neq x_{j}$ y $v$ ocurre libremente en $\varphi $ a partir de $ i-\left\vert Qx_{j}\right\vert $, con lo cual $v\in Li(\varphi )-\{x_{j}\}$
  %
  %   Supongamos ahora que $v\in Li(\varphi )-\{x_{j}\}$. Por definicion tenemos que hay un $i$ tal que $v$ ocurre libremente en $\varphi $ a partir de $i$. Ya que $v\neq x_{j}$ esto nos dice por definicion que $v$ ocurre libremente en $Qx_{j}\varphi $ a partir de $i+\left\vert Qx_{j}\right\vert $, con lo cual $v\in Li(Qx_{j}\varphi )$. $\Box$
  % \end{proof}
  %
  % % Lemma 133. Con prueba. Lemma 45.
  % \begin{lemma}
  %   \PN Supongamos que $\vec{a}, \vec{b}$ son asignaciones tales que si $x_{i} \in Li(\varphi)$, entonces $a_{i} =
  %   b_{i}$, entonces $\mathbf{A} \models \varphi \lbrack \vec{a}] \Leftrightarrow \mathbf{A} \models \varphi \lbrack
  %   \vec{b}]$.
  % \end{lemma}
  % \begin{proof}
  %   Probaremos por induccion en $k$ que el lema vale para cada $\varphi \in F_{k}^{\tau }.$ El caso $k=0$ se desprende del Lema 131. Veamos que Teo$_{k}$ implica Teo$_{k+1}.$ Sea $\varphi \in F_{k+1}^{\tau }-F_{k}^{\tau }.$ Hay varios casos:
  %
  %   CASO $\varphi =(\varphi_{1}\wedge \varphi_{2})$.
  %
  %   Ya que $Li(\varphi_{i})\subseteq Li(\varphi )$, $i=1,2$, Teo$_{k}$ nos dice que $\mathbf{A} \models \varphi_{i}[\vec{a}]$ sii $\mathbf{A} \models \varphi_{i}[\vec{b}]$, para $i=1,2$. Se tiene entonces que
  %
  %   $\displaystyle \begin{array}{l} \mathbf{A} \models \varphi \lbrack \vec{a}] \\ \ \ \Updownarrow \text{ (por (3) en la def de }\mathbf{A} \models \varphi \lbrack \vec{a}]\text{)} \\ \mathbf{A} \models \varphi_{1}[\vec{a}]\text{ y }\mathbf{A} \models \varphi_{2}[\vec{a}] \\ \ \ \Updownarrow \text{ (por Teo}_{k}\text{)} \\ \mathbf{A} \models \varphi_{1}[\vec{b}]\text{ y }\mathbf{A} \models \varphi_{2}[\vec{b}] \\ \ \ \Updownarrow \text{(por (3) en la def de }\mathbf{A} \models \varphi \lbrack \vec{a}]\text{)} \\ \mathbf{A} \models \varphi \lbrack \vec{b}] \end{array} $
  %
  %   CASO $\varphi =(\varphi_{1}\wedge \varphi_{2})$.
  %
  %   Es completamente similar al anterior.
  %
  %   CASO $\varphi =\lnot \varphi_{1}.$
  %
  %   Es completamente similar al anterior.
  %
  %   CASO $\varphi =\forall x_{j}\varphi_{1}.$
  %
  %   Supongamos $\mathbf{A} \models \varphi \lbrack \vec{a}]$. Entonces por (8) en la def de $\mathbf{A} \models \varphi \lbrack \vec{a}]$ se tiene que $\mathbf{A} \models \varphi_{1}[\downarrow _{j}^{a}(\vec{a})]$, para todo $a\in A$. Notese que $\downarrow _{j}^{a}(\vec{a})$ y $\downarrow _{j}^{a}(\vec{b})$ coinciden en toda $x_{i}$ de $x_{i}\in Li(\varphi_{1})\subseteq Li(\varphi_{1})\cup \{x_{j}\}$, con lo cual por Teo$_{k}$ se tiene que $\mathbf{A} \models \varphi_{1}[\downarrow _{j}^{a}(\vec{b})]$, para todo $a\in A$, lo cual por (8) en la def de $\mathbf{A} \models \varphi \lbrack \vec{a}]$ nos dice que $\mathbf{A} \models \varphi \lbrack \vec{b}]$. La prueba de que $\mathbf{A} \models \varphi \lbrack \vec{b}]$ implica que $ \mathbf{A} \models \varphi \lbrack \vec{a}]$ es similar.
  %
  %   CASO $\varphi =\exists x_{j}\varphi_{1}$.
  %
  %   Es similar al anterior. $\Box$
  % \end{proof}
  %
  % % Corollary 134. Sin prueba. Lemma 46.
  % \begin{corollary}
  %   \PN Si $\varphi$ es una sentencia, entonces $\mathbf{A} \models \varphi \lbrack \vec{a}] \Leftrightarrow \mathbf{A}
  %   \models \varphi \lbrack \vec{b}]$, cualesquiera sean las asignaciones $\vec{a}, \vec{b}$.
  % \end{corollary}
  %
  % % Lemma 135. Con prueba. Lemma 47.
  % \begin{lemma}
  %   \begin{enumerate}[(a)]
  %     \item Si $Li(\varphi) \cup Li(\psi) \subseteq \{x_{i_{1}}, \dotsc, x_{i_{n}}\}$, entonces $\varphi \thicksim \psi$
  %     si y solo si la sentencia $\forall x_{i_{1}} \dotsc \forall x_{i_{n}}(\varphi \leftrightarrow \psi)$ es
  %     universalmente válida.
  %     \item Si $\varphi_{i} \thicksim \psi_{i}, i = 1, 2$, entonces $\lnot \varphi_{1} \thicksim \lnot \psi_{1},
  %     (\varphi_{1} \eta \varphi_{2}) \thicksim (\psi_{1} \eta \psi_{2})$ y $Qv\varphi_{1} \thicksim Qv \psi_{1}$.
  %     \item Si $\varphi \thicksim \psi$ y $\alpha^{\prime}$ es el resultado de reemplazar en una fórmula $\alpha$
  %     algunas (posiblemente $0$) ocurrencias de $\varphi$ por $\psi$, entonces $\alpha \thicksim \alpha^{\prime}$.
  %   \end{enumerate}
  % \end{lemma}
  % \begin{proof}
  %   Tenemos que
  %
  %   $\displaystyle \begin{array}{l} \varphi \thicksim \psi \\ \ \ \Updownarrow \text{ (por (6) de la def de}\models \text{)} \\ \mathbf{A} \models (\varphi \leftrightarrow \psi )[\vec{a}]\text{, para todo } \mathbf{A}\text{ y toda }\vec{a}\in A^{\mathbf{N}} \\ \ \ \Updownarrow \\ \mathbf{A} \models (\varphi \leftrightarrow \psi )[\downarrow _{i_{n}}^{a}(\vec{a })]\text{, para todo }\mathbf{A}\text{, }a\in A\text{ y toda }\vec{a}\in A^{ \mathbf{N}} \\ \ \ \Updownarrow (\text{por (8) de la def de}\models ) \\ \mathbf{A} \models \forall x_{i_{n}}(\varphi \leftrightarrow \psi )[\vec{a}] \text{, para todo }\mathbf{A}\text{ y toda }\vec{a}\in A^{\mathbf{N}} \\ \ \ \Updownarrow \\ \mathbf{A} \models \forall x_{i_{n}}(\varphi \leftrightarrow \psi )[\downarrow _{i_{n-1}}^{a}(\vec{a})]\text{, para todo }\mathbf{A}\text{, }a\in A\text{ y toda }\vec{a}\in A^{\mathbf{N}} \\ \ \ \Updownarrow \text{ (por (8) de la def de}\models \text{)} \\ \mathbf{A} \models \forall x_{i_{n-1}}\forall x_{i_{n}}(\varphi \leftrightarrow \psi )[\vec{a}]\text{, para todo }\mathbf{A}\text{ y toda }\vec{a}\in A^{ \mathbf{N}} \\ \ \ \Updownarrow \\ \ \ \ \ \vdots \\ \ \ \Updownarrow \\ \mathbf{A} \models \forall x_{i_{1}}...\forall x_{i_{n}}(\varphi \leftrightarrow \psi )[\vec{a}]\text{, para todo }\mathbf{A}\text{ y toda }\vec{a}\in A^{ \mathbf{N}} \\ \ \ \Updownarrow \\ \forall x_{i_{1}}...\forall x_{i_{n}}(\varphi \leftrightarrow \psi )\text{ es universalmente valida} \end{array} $
  %
  %   (b) Es dejado al lector.
  %
  %   (c) Por induccion en el $k$ tal que $\alpha \in F_{k}^{\tau }$. $\Box$
  % \end{proof}
  %
  % % Lemma 136. Con prueba. Lemma 48.
  % \begin{lemma}
  %   \PN Sea $F: \mathbf{A} \rightarrow \mathbf{B}$ un homomorfismo, entonces:
  %   \[
  %     F(t^{\mathbf{A}}[(a_{1}, a_{2}, \dotsc)] = t^{\mathbf{B}}[F(a_{1}), F(a_{2}), \dotsc)]
  %   \]
  %   \PN para cada $t \in \TAU, (a_{1}, a_{2}, \dotsc) \in A^{\mathbf{N}}$.
  % \end{lemma}
  % \begin{proof}
  %   Sea
  %
  %   - Teo$_{k}$: Si $F:\mathbf{A} \rightarrow \mathbf{B}$ es un homomorfismo, entonces
  %   $\displaystyle F(t^{\mathbf{A}}[(a_{1},a_{2},...)]=t^{\mathbf{B}}[F(a_{1}),F(a_{2}),...)] $
  %
  %   para cada $t\in T_{k}^{\tau }$, $(a_{1},a_{2},...)\in A^{\mathbf{N}}$.
  %   Teo$_{0}$ es trivial. Veamos que Teo$_{k}$ implica Teo$_{k+1}$. Supongamos que vale Teo$_{k}$ y supongamos $F:\mathbf{A} \rightarrow \mathbf{B}$ es un homomorfismo, $t\in T_{k+1}^{\tau }-T_{k}^{\tau }$ y $\vec{a} =(a_{1},a_{2},...)\in A^{\mathbf{N}}$. Denotemos $(F(a_{1}),F(a_{2}),...)$ con $F(\vec{a})$. Por Lema 117, $t=f(t_{1}, \dotsc, t_{n})$, con $n\geq 1 $,$\;f\in \mathcal{F}_{n}$ y $t_{1}, \dotsc, t_{n}\in T_{k}^{\tau }$. Tenemos entonces
  %
  %   $\displaystyle \begin{array}{ccl} F(t^{\mathbf{A}}[\vec{a}]) & = & F(f(t_{1}, \dotsc, t_{n})^{\mathbf{A}}[\vec{a}]) \\ & = & F(f^{\mathbf{A}}(t_{1}^{\mathbf{A}}[\vec{a}], \dotsc, t_{n}^{\mathbf{A}}[ \vec{a}])) \\ & = & f^{\mathbf{B}}(F(t_{1}^{\mathbf{A}}[\vec{a}]), \dotsc, F(t_{n}^{\mathbf{A}}[ \vec{a}])) \\ & = & f^{\mathbf{B}}(t_{1}^{\mathbf{B}}[F(\vec{a})], \dotsc, t_{n}^{\mathbf{B}}[F( \vec{a})])) \\ & = & f(t_{1}, \dotsc, t_{n})^{\mathbf{B}}[F(\vec{a})] \\ & = & t^{\mathbf{B}}[F(\vec{a})] \end{array} $
  %
  %   $\Box$
  % \end{proof}
  %
  % % Lemma 137. Sin prueba. Lemma 49.
  % \begin{lemma}
  %   \PN Supongamos que $F: \mathbf{A} \rightarrow \mathbf{B}$ es un isomorfismo. Sea $\varphi \in F^{\tau}$, entonces:
  %   \[
  %     \mathbf{A} \models \varphi \lbrack (a_{1}, a_{2}, \dotsc)] \Leftrightarrow \mathbf{B} \models \varphi \lbrack
  %     (F(a_{1}), F(a_{2}), \dotsc)]
  %   \]
  %   \PN para cada $(a_{1}, a_{2}, \dotsc) \in A^{\mathbf{N}}$. En particular $\mathbf{A}$ y $\mathbf{B}$ satisfacen las
  %   mismas sentencias de tipo $\tau$.
  % \end{lemma}
  %
  % % Lemma 138. Con prueba. Lemma 50.
  % \begin{lemma}
  %   \PN Si $F: \mathbf{A} \rightarrow \mathbf{B}$ es un homomorfismo biyectivo, entonces $F$ es un isomorfismo.
  % \end{lemma}
  % \begin{proof}
  %   Solo falta probar que $F^{-1}$ es un homomorfismo. Supongamos que $c\in \mathcal{C}$. Ya que $F(c^{\mathbf{A}})=c^{\mathbf{B}}$, tenemos que $ F^{-1}(c^{\mathbf{B}})=c^{\mathbf{A}}$, por lo cual $F^{-1}$ cumple (1) de la definicion de homomorfismo. Supongamos ahora que $f\in \mathcal{F}_{n}$ y sean $b_{1}, \dotsc, b_{n}\in B$. Sean $a_{1}, \dotsc, a_{n}\in A$ tales que $ F(a_{i})=b_{i}$, $i=1, \dotsc, n$. Tenemos que
  %
  %   $\displaystyle \begin{array}{ccl} F^{-1}(f^{\mathbf{B}}(b_{1}, \dotsc, b_{n})) & = & F^{-1}(f^{\mathbf{B} }(F(a_{1}), \dotsc, F(a_{n}))) \\ & = & F^{-1}(F(f^{\mathbf{A}}(a_{1}, \dotsc, a_{n})) \\ & = & f^{\mathbf{A}}(a_{1}, \dotsc, a_{n}) \\ & = & f^{\mathbf{A}}(F^{-1}(b_{1}), \dotsc, F^{-1}(b_{n})) \end{array} $
  %
  %   por lo cual $F^{-1}$ satisface (2) de la definicion de homomorfismo $\Box$
  % \end{proof}
  %
  % % Lemma 139. Con prueba. Lemma 51.
  % \begin{lemma}
  %   \PN Si $F: \mathbf{A} \rightarrow \mathbf{B}$ es un homomorfismo, entonces $I_{F}$ es un subuniverso de
  %   $\mathbf{B}$.
  % \end{lemma}
  % \begin{proof}
  %   Ya que $A\neq \varnothing ,$ tenemos que $I_{F}\neq \varnothing .$ Es claro que $ c^{\mathbf{B}}=F(c^{\mathbf{A}})\in I_{F},$ para cada $c\in \mathcal{C}$. Sea $f\in \mathcal{F}_{n}$ y sean $b_{1}, \dotsc, b_{n}\in I_{F}$ Sean $ a_{1}, \dotsc, a_{n}$ tales que $F(a_{i})=b_{i},$ $i=1, \dotsc, n$. Tenemos que
  %
  %   $\displaystyle f^{\mathbf{B}}(b_{1}, \dotsc, b_{n})=f^{\mathbf{B}}(F(a_{1}), \dotsc, F(a_{n}))=F(f^{ \mathbf{A}}(a_{1}, \dotsc, a_{n}))\in I_{F} $
  %
  %   por lo cual $I_{F}$ es cerrada bajo $f^{\mathbf{B}}$.
  % \end{proof}
  %
  % % Lemma 140. Con prueba. Lemma 52.
  % \begin{lemma}
  %   \PN Si $F: \mathbf{A} \rightarrow \mathbf{B}$ es un homomorfismo, entonces $\ker F$ es una congruencia sobre
  %   $\mathbf{A}$.
  % \end{lemma}
  % \begin{proof}
  %   Sea $f\in \mathcal{F}_{n}$. Supongamos que $a_{1}, \dotsc, a_{n},b_{1}, \dotsc, b_{n} \in A$ son tales que $a_{1}\ker Fb_{1}, \dotsc, a_{n}\ker Fb_{n}$. Tenemos entonces que
  %
  %   $\displaystyle \begin{array}{ccl} F(f^{\mathbf{A}}(a_{1}, \dotsc, a_{n})) & = & f^{\mathbf{B} }(F(a_{1}), \dotsc, F(a_{n})) \\ & = & f^{\mathbf{B}}(F(b_{1}), \dotsc, F(b_{n})) \\ & = & F(f^{\mathbf{B}}(b_{1}, \dotsc, b_{n})) \end{array} $
  %
  %   lo cual nos dice que $f^{\mathbf{A}}(a_{1}, \dotsc, a_{n})\ker Ff^{\mathbf{B} }(b_{1}, \dotsc, b_{n})$ $\Box$
  % \end{proof}
  %
  % % Lemma 141. Con prueba. Lemma 53.
  % \begin{lemma}
  %   \PN $\pi_{\theta}: \mathbf{A} \rightarrow \mathbf{A}/\theta$ es un homomorfismo cuyo núcleo es $\theta$.
  % \end{lemma}
  % \begin{proof}
  %   Sea $c\in \mathcal{C}$. Tenemos que
  %
  %   $\displaystyle \pi _{\theta }(c^{\mathbf{A}})=c^{\mathbf{A}}/\theta =c^{\mathbf{A}/\theta } $
  %
  %   Sea $f\in \mathcal{F}_{n}$, con $n\geq 1$ y sean $a_{1}, \dotsc, a_{n}\in A$. Tenemos que
  %   $\displaystyle \begin{array}{ccl} \pi _{\theta }(f^{\mathbf{A}}(a_{1}, \dotsc, a_{n})) & = & f^{\mathbf{A} }(a_{1}, \dotsc, a_{n})/\theta \\ & = & f^{\mathbf{A}/\theta }(a_{1}/\theta , \dotsc, a_{n}/\theta ) \\ & = & f^{\mathbf{A}/\theta }(\pi _{\theta }(a_{1}), \dotsc, \pi _{\theta }(a_{n})) \end{array} $
  %
  %   con lo cual $\pi _{\theta }$ es un homomorfismo. Es trivial que $\ker \pi _{\theta }=\theta $ $\Box$
  % \end{proof}
  %
  % % Corollary 142. Con prueba. Lemma 54.
  % \begin{corollary}
  %   \PN Para cada $t \in \TAU, \vec{a} \in A^{\mathbf{N}}$, se tiene que $t^{\mathbf{A}/\theta}[(a_{1}/\theta,
  %   a_{2}/\theta, \dotsc)] = t^{\mathbf{A}}[(a_{1}, a_{2}, \dotsc)]/\theta$.
  % \end{corollary}
  % \begin{proof}
  %   Ya que $\pi _{\theta }$ es un homomorfismo, se puede aplicar el Lema 136. $\Box$
  % \end{proof}
  %
  % % Theorem 143. Con prueba. Lemma 55.
  % \begin{theorem}
  %   \PN Sea $F: \mathbf{A} \rightarrow \mathbf{B}$ un homomorfismo sobreyectivo, entonces:
  %   \begin{eqnarray*}
  %     A/\ker F &\rightarrow& B \\
  %     a/\ker F &\rightarrow& F(a)
  %   \end{eqnarray*}
  %   \PN define sin ambiguedad una función $\bar{F}$ la cual es un isomorfismo de $\mathbf{A}/\ker F$ en $\mathbf{B}$.
  % \end{theorem}
  % \begin{proof}
  %   Notese que la definicion de $\bar{F}$ es inambigua ya que si $a/\ker F=a^{\prime }/\ker F$, entonces $F(a)=F(a^{\prime }).$ Ya que $F$ es sobre, tenemos que $\bar{F}$ lo es. Supongamos que $\bar{F}(a/\ker F)=\bar{F} (a^{\prime }/\ker F).$ Claramente entonces tenemos que $F(a)=F(a^{\prime })$ , lo cual nos dice que $a/\ker F=a^{\prime }/\ker F$. Esto prueba que $\bar{F }$ es inyectiva. Para ver que $\bar{F}$ es un isomorfismo, por el Lema 138, basta con ver que $\bar{F}$ es un homomorfismo. Sea $c\in \mathcal{C}$. Tenemos que
  %
  %   $\displaystyle \bar{F}(c^{\mathbf{A}/\ker F})=\bar{F}(c^{\mathbf{A}}/\ker F)=F(c^{\mathbf{A} })=c^{\mathbf{B}} $
  %
  %   Sea $f\in \mathcal{F}_{n}$. Sean $a_{1}, \dotsc, a_{n}\in A$. Tenemos que
  %   $\displaystyle \begin{array}{ccl} \bar{F}(f^{\mathbf{A}/\ker F}(a_{1}/\ker F, \dotsc, a_{n}/\ker F)) & = & \bar{F} (f^{\mathbf{A}}(a_{1}, \dotsc, a_{n})/\ker F) \\ & = & F(f^{\mathbf{A}}(a_{1}, \dotsc, a_{n})) \\ & = & f^{\mathbf{B}}(F(a_{1}), \dotsc, F(a_{n})) \\ & = & f^{\mathbf{B}}(\bar{F}(a_{1}/\ker F), \dotsc, \bar{F}(a_{n}/\ker F)) \end{array} $
  %
  %   con lo cual $\bar{F}$ cunple (2) de la definicion de homomorfismo
  % \end{proof}
  %
  % % Lemma 144. Con prueba. Lemma 56.
  % \begin{lemma}
  %   \PN Los mapeos $\pi_{1}: A \times B \rightarrow A$ y $\pi_{2}: A \times B \rightarrow A$ son homomorfismos.
  % \end{lemma}
  % \begin{proof}
  %   Veamos que $\pi _{1}$ es un homomorfismo. Primero notese que si $c\in \mathcal{C}$, entonces
  %
  %   $\displaystyle \pi _{1}(c^{\mathbf{A}\times \mathbf{B}})=\pi _{1}((c^{\mathbf{A}},c^{ \mathbf{B}}))=c^{\mathbf{A}} $
  %
  %   Sea $f\in \mathcal{F}_{n}$, con $n\geq 1$ y sean $ (a_{1},b_{1}), \dotsc, (a_{n},b_{n})\in A\times B$. Tenemos que
  %   $\displaystyle \begin{array}{ccl} \pi _{1}(f^{\mathbf{A}\times \mathbf{B}}((a_{1},b_{1}), \dotsc, (a_{n},b_{n})) & = & \pi _{1}((f^{\mathbf{A}}(a_{1}, \dotsc, a_{n}),f^{\mathbf{B}}(b_{1}, \dotsc, b_{n})) \\ & = & f^{\mathbf{A}}(a_{1}, \dotsc, a_{n}) \\ & = & f^{\mathbf{A}}(\pi _{1}(a_{1},b_{1}), \dotsc, \pi _{1}(a_{n},b_{n})) \end{array} $
  %
  %   con lo cual hemos probado que $\pi _{1}$ cumple (2) de la definicion de homomorfismo
  % \end{proof}
  %
  % % Lemma 145. Con prueba. Lemma 57.
  % \begin{lemma}
  %   \PN Para cada $t \in \TAU, ((a_{1}, b_{1}), (a_{2}, b_{2}), \dotsc) \in (A \times B)^{\mathbf{N}}$, se tiene que
  %   $t^{\mathbf{A} \times \mathbf{B}}[((a_{1}, b_{1}), (a_{2}, b_{2}), \dotsc)] = (t^{\mathbf{A}}[(a_{1}, a_{2},
  %   \dotsc)], t^{\mathbf{B}}[(b_{1}, b_{2}, \dotsc)])$.
  % \end{lemma}
  %
  % % Lemma 146. Con prueba. Lemma 58.
  % \begin{lemma}
  %   \PN Sean $w_{1}, \dotsc, w_{k}$ variables, todas distintas. Sean $v_{1}, \dotsc, v_{n}$ variables, todas distintas.
  %   Supongamos $t =_{d} t(w_{1}, \dotsc, w_{k}), s_{1} =_{d} s_{1}(v_{1}, \dotsc, v_{n}), \dotsc, s_{k} =_{d} s_{k}
  %   (v_{1}, \dotsc, v_{n})$, entonces:
  %   \begin{enumerate}[(a)]
  %     \item $t(s_{1}, \dotsc, s_{k}) =_{d} t(s_{1}, \dotsc, s_{k})(v_{1}, \dotsc, v_{n})$
  %     \item Para cada estructura $\mathbf{A}$ y $a_{1}, \dotsc, a_{n} \in A$, se tiene que:
  %       \[
  %         t(s_{1}, \dotsc, s_{k})^{\mathbf{A}}[a_{1}, \dotsc, a_{n}] = t^{\mathbf{A}}[s_{1}^{\mathbf{A}}[a_{1}, \dotsc,
  %         a_{n}], \dotsc, s_{k}^{\mathbf{A}}[a_{1}, \dotsc, a_{n}]]
  %       \]
  %   \end{enumerate}
  % \end{lemma}
  % \begin{proof}
  %   Probaremos que valen (a) y (b), por induccion en el $l$ tal que $t\in T_{l}^{\tau }.$ El caso $l=0$ es dejado al lector. Supongamos entonces que valen (a) y (b) siempre que $t\in T_{l}^{\tau }$ y veamos que entonces valen (a) y (b) cuando $t\in T_{l+1}^{\tau }-T_{l}^{\tau }$. Hay $f\in \mathcal{F} _{m}$ y $t_{1}, \dotsc, t_{m}\in T_{l}^{\tau }$ tales que $ t_{1}=_{d}t_{1}(w_{1}, \dotsc, w_{k}), \dotsc, t_{m}=_{d}t_{m}(w_{1}, \dotsc, w_{k})$ y $ t=f(t_{1}, \dotsc, t_{m})$. Notese que por (a) de la HI tenemos que
  %
  %   $\displaystyle t_{i}(s_{1}, \dotsc, s_{k})=_{d}t_{i}(s_{1}, \dotsc, s_{k})(v_{1}, \dotsc, v_{n})\text{, } i=1, \dotsc, m $
  %
  %   lo cual ya que
  %   $\displaystyle t(s_{1}, \dotsc, s_{k})=f(t_{1}(s_{1}, \dotsc, s_{k}), \dotsc, t_{m}(s_{1}, \dotsc, s_{k})) $
  %
  %   nos dice que
  %   $\displaystyle t(s_{1}, \dotsc, s_{k})=_{d}t(s_{1}, \dotsc, s_{k})(v_{1}, \dotsc, v_{n}) $
  %
  %   obteniendo asi (a). Para probar (b) notemos que por (b) de la hipotesis inductiva
  %   $\displaystyle t_{j}(s_{1}, \dotsc, s_{k})^{\mathbf{A}}[\vec{a}]=t_{j}^{\mathbf{A}}[s_{1}^{ \mathbf{A}}[\vec{a}], \dotsc, s_{k}^{\mathbf{A}}[\vec{a}]],j=1, \dotsc, m $
  %
  %   lo cual nos dice que
  %   $\displaystyle \begin{array}{ccl} t(s_{1}, \dotsc, s_{k})^{\mathbf{A}}[\vec{a}] & = & f(t_{1}(s_{1}, \dotsc, s_{k}), \dotsc, t_{m}(s_{1}, \dotsc, s_{k}))^{\mathbf{A}}[\vec{a}] \\ & = & f^{\mathbf{A}}(t_{1}(s_{1}, \dotsc, s_{k})^{\mathbf{A}}[\vec{a} ], \dotsc, t_{m}(s_{1}, \dotsc, s_{k})^{\mathbf{A}}[\vec{a}]) \\ & = & f^{\mathbf{A}}(t_{1}^{\mathbf{A}}[s_{1}^{\mathbf{A}}[\vec{a} ], \dotsc, s_{k}^{\mathbf{A}}[\vec{a}]], \dotsc, t_{m}^{\mathbf{A}}[s_{1}^{\mathbf{A}}[ \vec{a}], \dotsc, s_{k}^{\mathbf{A}}[\vec{a}]]) \\ & = & t^{\mathbf{A}}[s_{1}^{\mathbf{A}}[\vec{a}], \dotsc, s_{k}^{\mathbf{A}}[\vec{ a}]] \end{array} $
  % \end{proof}
  %
  % % Lemma 147. Con prueba. Lemma 59.
  % \begin{lemma}
  %   \PN Si $Qv$ ocurre en $\varphi$ a partir de $i$, entonces hay una única fórmula $\psi$ tal que $Qv\psi$ ocurre en
  %   $\varphi$ a partir de $i$.
  % \end{lemma}
  % \begin{proof}
  %   Por induccion en el $k$ tal que $\varphi \in F^{\tau }$.
  % \end{proof}
  % 
  % % Lemma 148. Con prueba. Lemma 60.
  % \begin{lemma}
  %   \PN Sean $w_{1}, \dotsc, w_{k}$ variables, todas distintas. Sean $v_{1}, \dotsc, v_{n}$ variables, todas distintas.
  %   Supongamos $\varphi =_{d} \varphi(w_{1}, \dotsc, w_{k}), t_{1} =_{d} t_{1}(v_{1}, \dotsc, v_{n}), \dotsc, t_{k}
  %   =_{d}t_{k}(v_{1}, \dotsc, v_{n})$ son tales que cada $w_{j}$ es sustituible por $t_{j}$ en $\varphi$, entonces:
  %   \begin{enumerate}[(a)]
  %     \item $\varphi(t_{1}, \dotsc, t_{k}) =_{d} \varphi(t_{1}, \dotsc, t_{k})(v_{1}, \dotsc, v_{n})$
  %     \item Para cada estructura $\mathbf{A}$ y $\vec{a} \in A^{n}$ se tiene:
  %       \[
  %         \mathbf{A} \models \varphi(t_{1}, \dotsc, t_{k})[\vec{a}] \Leftrightarrow \mathbf{A} \models \varphi \lbrack
  %         t_{1}^{\mathbf{A}}[\vec{a}], \dotsc, t_{k}^{ \mathbf{A}}[\vec{a}]]
  %       \]
  %   \end{enumerate}
  % \end{lemma}
  % \begin{proof}
  %   Probaremos que se dan (a) y (b), por induccion en el $l$ tal que $\varphi \in F_{l}^{\tau }.$ El caso $l=0$ es una consecuencia directa del Lema 146. Supongamos (a) y (b) valen para cada $\varphi \in F_{l}^{\tau } $ y sea $\varphi \in F_{l+1}^{\tau }-F_{l}^{\tau }.$ Notese que se puede suponer que cada $v_{i}$ ocurre en algun $t_{i},$ y que cada $w_{i}\in Li(\varphi )$, ya que para cada $\varphi ,$ el caso general se desprende del caso con estas restricciones. Hay varios casos
  %
  %   CASO $\varphi =\forall w\varphi_{1},$ con $w\not\in \{w_{1}, \dotsc, w_{k}\}$ y $ \varphi_{1}=_{d}\varphi_{1}(w_{1}, \dotsc, w_{k},w)$
  %
  %   Notese que cada $w_{j}\in Li(\varphi_{1})$. Ademas notese que $ w\not\in \{v_{1}, \dotsc, v_{n}\}$ ya que de lo contrario $w$ ocurriria en algun $ t_{j}$, y entonces $w_{j}$ no seria sustituible por $t_{j}$ en $\varphi $. Sean
  %
  %   $\displaystyle \begin{array}{ccc} \tilde{t}_{1} & = & t_{1} \\ & \vdots & \\ \tilde{t}_{k} & = & t_{k} \\ \tilde{t}_{k+1} & = & w \end{array} $
  %
  %   Notese que
  %   $\displaystyle \tilde{t}_{j}=_{d}\tilde{t}_{j}(v_{1}, \dotsc, v_{n},w) $
  %
  %   Por (a) de la hipotesis inductiva tenemos que
  %   $\displaystyle Li(\varphi_{1}(t_{1}, \dotsc, t_{k},w))=Li(\varphi_{1}(\tilde{t}_{1}, \dotsc, \tilde{ t}_{k},\tilde{t}_{k+1}))\subseteq \{v_{1}, \dotsc, v_{n},w\} $
  %
  %   y por lo tanto
  %   $\displaystyle Li(\varphi(t_{1}, \dotsc, t_{k}))\subseteq \{v_{1}, \dotsc, v_{n}\} $
  %
  %   lo cual prueba (a). Finalmente notese que
  %   $\displaystyle \begin{array}{c} \mathbf{A} \models \varphi(t_{1}, \dotsc, t_{k})\mathbf{[}\vec{a}] \\ \Updownarrow \\ \mathbf{A} \models \varphi_{1}(\tilde{t}_{1}, \dotsc, \tilde{t}_{k},\tilde{t} _{k+1})[\vec{a},a],\text{ para todo }a\in A \\ \Updownarrow \\ \mathbf{A} \models \varphi_{1}[\tilde{t}_{1}^{\mathbf{A}}[\vec{a},a], \dotsc,  \tilde{t}_{k}^{\mathbf{A}}[\vec{a},a],\tilde{t}_{k+1}^{\mathbf{A}}[\vec{a} ,a]],\text{ para todo }a\in A \\ \Updownarrow \\ \mathbf{A} \models \varphi_{1}[t_{1}^{\mathbf{A}}[\vec{a}], \dotsc, t_{k}^{ \mathbf{A}}[\vec{a}],a],\text{ para todo }a\in A \\ \Updownarrow \\ \mathbf{A} \models \varphi \lbrack t_{1}^{\mathbf{A}}[\vec{a}], \dotsc, t_{k}^{ \mathbf{A}}[\vec{a}]] \end{array} $
  %
  %   lo cual pueba (b). El caso del cuantificador $\exists $ es analogo y los casos de nexos logicos son directos.
  % \end{proof}

	\section{Teorias de primer orden}

  % Lemma 155. Con prueba. Lemma 65.
  \begin{lemma} \label{lemma_65}
    \PN Si $(\varphi_{1}, \varphi_{2}) \in Generaliz^{\tau}$, entonces el nombre de constante $c$ del cual habla la
    definición de $Generaliz^{\tau}$ está unívocamente determinado por el par $(\varphi_{1}, \varphi_{2})$.
  \end{lemma}
  \begin{proof}
    \PN Recordemos la definición de $Generaliz^{\tau}$.
    \begin{eqnarray*}
      Generaliz^{\tau} &=& \{(\psi, \forall v \tilde{\psi}): \psi \in S^{\tau}, \ v \ \text{no ocurre en} \ \psi \
      \text{y existe } \ c \in \mathcal{C} \ \text{tal que} \ \tilde{\psi} = \text{resultado de} \\
      && \text{reemplazar} \ \text{en} \ \psi \ \text{cada ocurrencia de} \ c \ \text{por} \ v\}
    \end{eqnarray*}

    \PN Notese que $c$ es el único nombre de constante que ocurre en $\varphi_{1}$ y no ocurre en $\varphi_{2}$.
  \end{proof}

  % Lemma 156. Con prueba. Lemma 66.
  \begin{lemma} \label{lemma_66}
    \PN Si $(\varphi_{1}, \varphi_{2}) \in Elec^{\tau}$, entonces el nombre de constante $e$ del cual habla la
    definición de $Elec^{\tau}$ está unívocamente determinado por el par $(\varphi_{1}, \varphi_{2})$.
  \end{lemma}
  \begin{proof}
    \PN Recordemos la definición de $Elec^{\tau}$.
    \begin{eqnarray*}
      Elect^{\tau} = \{(\exists v \varphi(v), \varphi(e)): \varphi =_{d} \varphi(v), \ Li(\varphi) = \{v\} \ \text{y} \
      e \in \mathcal{C} \ \text{no ocurre en} \ \varphi\}
    \end{eqnarray*}

    \PN Notese que $e$ es el único nombre de constante que ocurre en $\varphi_{1}$ y no ocurre en $\varphi_{2}$.
  \end{proof}

  % Lemma 157. Sin prueba. Lemma 67.
  \begin{lemma} \label{lemma_67}
    \PN Todas las reglas excepto las reglas de elección y generalización son universales en el sentido que si $\varphi$
    se deduce de $\psi_{1}, \dotsc, \psi_{k}$ por alguna de estas reglas, entonces $\left((\psi_{1} \wedge \dotsc \wedge
    \psi_{k}) \rightarrow \varphi \right)$ es una sentencia universalmente válida.
  \end{lemma}

  % Lemma 158. Sin prueba. Lemma 68.
  \begin{lemma} \label{lemma_68}
    \PN Sea $\pmb{\varphi} \in S^{\tau +}$, hay únicos $n \geq 1$ y $\varphi_{1}, \dotsc, \varphi_{n} \in S^{\tau}$
    tales que $\pmb{\varphi} = \varphi_{1} \dotsc \varphi_{n}$.
  \end{lemma}

  % Lemma 159. Sin prueba. Lemma 69.
  \begin{lemma} \label{lemma_69}
    \PN Sea $\mathbf{J} \in Just^{+}$, hay únicos $n \geq 1$ y $J_{1}, \dotsc, J_{n} \in Just$ tales que $\mathbf{J} =
    J_{1} \dotsc J_{n}$.
  \end{lemma}

  % Lemma 160. Sin prueba. Lemma 70.
  \begin{lemma} \label{lemma_70}
    \PN Sea $(\pmb{\varphi}, \mathbf{J})$ una prueba de $\varphi$ en $(\Sigma, \tau)$.
    \begin{enumerate}
      \item Sea $m \in \mathbb{N}$ tal que $\mathbf{J}_{i} \neq \mathrm{HIPOTESIS}\bar{m}$, para cada $i = 1, \dotsc,
      n(\pmb{\varphi})$. Supongamos que $\mathbf{J}_{i} = \mathrm{HIPOTESIS}\bar{k}$ y que $\mathbf{J}_{j} =
      \mathrm{TESIS}\bar{k} \alpha$, con $\lbrack\alpha\rbrack_{1} \notin Num$. Sea $\mathbf{\tilde{J}}$ el resultado de
      reemplazar en $\mathbf{J}$ la justificación $\mathbf{J}_{i}$ por $\mathrm{HIPOTESIS}\bar{m}$ y reemplazar la
      justificación $\mathbf{J}_{j}$ por $\mathrm{TESIS}\bar{m}\alpha$, entonces $(\pmb{\varphi}, \mathbf{\tilde{J}})$
      es una prueba de $\varphi$ en $(\Sigma, \tau)$.
      \item Sea $\mathcal{C}_{1}$ el conjunto de nombres de constante que ocurren en alguna $\pmb{\varphi}_{i}$ y que
      no pertenecen a $\mathcal{C}$. Sea $e \in \mathcal{C}_{1}-\mathcal{C}$. Sea $\tilde{e} \notin \mathcal{C} \cup
      \mathcal{C}_{1}$ tal que $(\mathcal{C} \cup (\mathcal{C}_{1}-\{e\}) \cup \{\tilde{e}\}, \mathcal{F}, \mathcal{R},
      a)$ es un tipo. Sea $\tilde{\varphi}_{i} =$ resultado de reemplazar en $\pmb{\varphi}_{i}$ cada ocurrencia de
      $e$ por $\tilde{e}$, entonces $(\pmb{\tilde{\varphi}}_{1} \dotsc \pmb{\tilde{\varphi}}_{n(\mathbf{\varphi})},
      \mathbf{J})$ es una prueba de $\varphi$ en $(\Sigma, \tau)$.
    \end{enumerate}
  \end{lemma}

  TODO: detalle
  % Lemma 161. Con prueba. Lemma 71.
  \begin{lemma} \label{lemma_71}
    \PN Sea $(\Sigma, \tau)$ una teoría.
    \begin{enumerate}[(1)]
      \item Si $(\Sigma, \tau) \vdash \varphi_{1}, \dotsc, \varphi_{n}$ y $(\Sigma \cup \{\varphi_{1}, \dotsc,
      \varphi_{n}\},\tau) \vdash \varphi$ entonces $(\Sigma, \tau) \vdash \varphi$.
      \item Si $(\Sigma, \tau) \vdash \varphi_{1}, \dotsc, \varphi_{n}$ y $\varphi$ se deduce por alguna regla universal
      a partir de $\varphi_{1}, \dotsc, \varphi_{n}$, entonces $(\Sigma, \tau) \vdash \varphi$.
      \item Si $(\Sigma, \tau)$ es inconsistente, entonces $(\Sigma, \tau) \vdash \varphi$, para toda sentencia
      $\varphi$.
      \item Si $(\Sigma, \tau)$ es consistente y $(\Sigma, \tau) \vdash \varphi$, entonces $(\Sigma \cup \{\varphi\},
      \tau)$ es consistente.
      \item $(\Sigma, \tau) \vdash (\varphi \rightarrow \psi)$ si y solo si $ (\Sigma \cup \{\varphi\}, \tau) \vdash
      \psi$.
      \item Si $(\Sigma, \tau) \not \vdash \lnot \varphi$, entonces $(\Sigma \cup \{\varphi\}, \tau)$ es consistente.
    \end{enumerate}
  \end{lemma}
  \begin{proof}
    \begin{enumerate}[(1)]
      \item Haremos el caso $n = 2.$ Supongamos entonces que $(\Sigma, \tau) \vdash \varphi_{1}, \varphi_{2}$ y
        \linebreak $(\Sigma \ \cup \ \{\varphi_{1}, \varphi_{2}\}, \tau) \vdash \varphi$. Sean:
        \begin{itemize}
          \item $(\varphi_{1}^{1} \dotsc \varphi_{n_{1}}^{1}, J_{1}^{1} \dotsc J_{n_{1}}^{1})$ una prueba de
            $\varphi_{1}$ en $(\Sigma, \tau)$.
          \item $(\varphi_{1}^{2} \dotsc \varphi_{n_{2}}^{2}, J_{1}^{2} \dotsc J_{n_{2}}^{2})$ una prueba de
            $\varphi_{2}$ en $(\Sigma, \tau)$.
          \item $(\psi_{1} \dotsc \psi_{n}, J_{1} \dotsc J_{n})$ una prueba de $\varphi$ en $(\Sigma \ \cup \
            \{\varphi_{1}, \varphi_{2}\}, \tau)$.
        \end{itemize}

        \PN Notese que por el \textbf{Lemma~\ref{lemma_70}}, podemos suponer que estas tres pruebas no comparten ningún
        nombre de constante auxiliar y que tampoco comparten números asociados a hipotesis o tesis.

        \vspace{3mm}
        \PN Para cada $i = 1, \dotsc, n$, definamos $\widetilde{J_{i}}$ de la siguiente manera:
        \begin{itemize}
          \item Si $\psi_{i} = \varphi_{1}$ y $J_{i} = \mathrm{AXIOMAPROPIO}$, entonces $\widetilde{J_{i}} =
            \mathrm{EVOCACION}(\overline{n_{1}})$
          \item Si $\psi_{i} = \varphi_{2}$ y $J_{i} = \mathrm{AXIOMAPROPIO}$, entonces $\widetilde{J_{i}} =
            \mathrm{EVOCACION}(\overline{n_{1} + n_{2}})$.
          \item Si $\psi_{i} \notin \{\varphi_{1},\varphi_{2}\}$ y $J_{i} = \mathrm{AXIOMAPROPIO}$, entonces
            $\widetilde{J_{i}} = \mathrm{AXIOMAPROPIO}$.
          \item Si $J_{i} = \mathrm{AXIOMALOGICO}$, entonces $\widetilde{J_{i}}= \mathrm{AXIOMALOGICO}$
          \item Si $J_{i} = \mathrm{CONCLUSION}$, entonces $\widetilde{J_{i}} = \mathrm{CONCLUSION}$.
          \item Si $J_{i} = \mathrm{HIPOTESIS}\bar{k}$, entonces $\widetilde{J_{i}}= \mathrm{HIPOTESIS}\bar{k}$
          \item Si $J_{i} = \alpha P(\overline{l_{1}}, \dotsc, \overline{l_{k}})$, con $\alpha \in \{\varepsilon\} \
            \cup \ \{\mathrm{TESIS}\bar{k}: k \in \mathbb{N}\}$, entonces \linebreak $\widetilde{J_{i}} = \alpha
            P(\overline{l_{1} + n_{1}+n_{2}}, \dotsc, \overline{l_{k} + n_{1} + n_{2}})$
        \end{itemize}

        \PN Para cada $i = 1, \dotsc, n_{2}$, definamos $\widetilde{J_{i}^{2}}$ de la siguiente manera.
        \begin{itemize}
          \item Si $J_{i}^{2} = \mathrm{AXIOMAPROPIO}$, entonces $\widetilde{J_{i}^{2}} = \mathrm{AXIOMAPROPIO}$
          \item Si $J_{i}^{2} = \mathrm{AXIOMALOGICO}$, entonces $\widetilde{J_{i}^{2}} = \mathrm{AXIOMALOGICO}$
          \item Si $J_{i}^{2} = \mathrm{CONCLUSION}$, entonces $\widetilde{J_{i}^{2}} = \mathrm{CONCLUSION}$.
          \item Si $J_{i}^{2} = \mathrm{HIPOTESIS}\bar{k}$, entonces $\widetilde{J_{i}^{2}} = \mathrm{HIPOTESIS}\bar{k}$
          \item Si $J_{i}^{2} = \alpha P(\overline{l_{1}}, \dotsc, \overline{l_{k}})$, con $\alpha \in \{\varepsilon\}
            \ \cup \ \{\mathrm{TESIS}\bar{k}: k \in \mathbb{N}\}$, entonces \linebreak $\widetilde{J_{i}^{2}} = \alpha
            P(\overline{l_{1} + n_{1}}, \dotsc, \overline{l_{k} + n_{1}})$
        \end{itemize}

        \PN Luego,
        \[
          (\varphi_{1}^{1} \dotsc \varphi_{n_{1}}^{1} \varphi_{1}^{2} \dotsc \varphi_{n_{2}}^{2} \psi_{1} \dotsc
          \psi_{n}, J_{1}^{1} \dotsc J_{n_{1}}^{1} \widetilde{J_{1}^{2}} \dotsc \widetilde{J_{n_{2}}^{2}}
          \widetilde{J_{1}} \dotsc \widetilde{J_{n}})
        \]
        \PN es una prueba de $\varphi$ en $(\Sigma, \tau)$.

      \item Supongamos que $(\Sigma, \tau) \vdash \varphi_{1}, \dotsc, \varphi_{n}$ y que $\varphi$ se deduce por regla
        R a partir de $\varphi_{1}, \dotsc, \varphi_{n}$, con R universal. Notese que:
        \[
          \begin{array}{llll}
            1. & \varphi_{1} && \text{AXIOMAPROPIO} \\
            2. & \varphi_{2} && \text{AXIOMAPROPIO} \\
            \vdots & \vdots && \vdots \\
            n. & \varphi_{n} && \text{AXIOMAPROPIO} \\
            n+1. & \varphi && \text{R}(\bar{1}, \dotsc, \bar{n})
          \end{array}
        \]
        \PN es una prueba de $\varphi$ en $(\Sigma \ \cup \ \{\varphi_{1}, \dotsc, \varphi_{n}\}, \tau)$, lo cual por
        (1) nos dice que $(\Sigma, \tau) \vdash \varphi$.

      \item Si $(\Sigma, \tau)$ es inconsistente, entonces por definición tenemos que $(\Sigma, \tau) \vdash (\psi
        \wedge \lnot \psi)$ para alguna sentencia $\psi$. Dada una sentencia cualquiera $\varphi$ tenemos que $\varphi$
        se deduce por la regla del absurdo a partir de $(\psi \wedge \lnot \psi)$ con lo cual (2) nos dice que $(\Sigma,
        \tau) \vdash \varphi$.

      \item Supongamos $(\Sigma, \tau)$ es consistente y $(\Sigma, \tau) \vdash \varphi$. Si $(\Sigma \ \cup \
        \{\varphi\}, \tau)$ fuera inconsistente, entonces $(\Sigma \ \cup \ \{\varphi\}, \tau) \vdash (\psi \wedge \lnot
        \psi)$, para alguna sentencia $\psi$, lo cual por (1) nos diría que $(\Sigma, \tau) \vdash (\psi \wedge \lnot
        \psi)$, es decir, que $(\Sigma, \tau)$ es inconsistente.

      \item \PN \begin{tabular}{|c|} \hline $\Rightarrow$ \\\hline \end{tabular} Supongamos $(\Sigma, \tau) \vdash
        (\varphi \rightarrow \psi)$, entonces tenemos que $(\Sigma \ \cup \ \{\varphi\}, \tau) \vdash (\varphi
        \rightarrow \psi), \varphi$, lo cual por (2) nos dice que $(\Sigma \ \cup \ \{\varphi\}, \tau) \vdash \psi$.

        \PN \begin{tabular}{|c|} \hline $\Leftarrow$ \\\hline \end{tabular} Supongamos ahora que $(\Sigma \ \cup \
          \{\varphi\}, \tau) \vdash \psi$. Sea $(\varphi_{1} \dotsc \varphi_{n}, J_{1} \dotsc J_{n})$ una prueba de
          $\psi$ en $(\Sigma \cup \{\varphi\}, \tau)$. Notese que podemos suponer que $J_{n}$ es de la forma
          $P(\overline{l_{1}}, \dotsc, \overline{l_{k}})$ . Definimos $\widetilde{J_{i}} = $ $\mathrm{TESIS}\bar{m}
          P(\overline{l_{1}+1} , \dotsc, \overline{l_{k}+1})$, donde $m$ es tal que ninguna $J_{i}$ es igual a
          $\mathrm{HIPOTESIS}\bar{m}$.

        \PN Para cada $i = 1, \dotsc, n-1$, definamos $\widetilde{J_{i}}$ de la siguiente manera:
        \begin{itemize}
          \item Si $\varphi_{i} = \varphi $ y $J_{i} = \mathrm{AXIOMAPROPIO}$, entonces $\widetilde{J_{i}} =
            \mathrm{EVOCACION}(1)$
          \item Si $\varphi_{i}\neq \varphi $ y $J_{i} = \mathrm{AXIOMAPROPIO}$, entonces $\widetilde{J_{i}} =
            \mathrm{AXIOMAPROPIO}$
          \item Si $J_{i} = \mathrm{AXIOMALOGICO}$, entonces $\widetilde{J_{i}}= \mathrm{AXIOMALOGICO}$
          \item Si $J_{i} = \mathrm{CONCLUSION}$, entonces $\widetilde{J_{i}} = \mathrm{CONCLUSION}$
          \item Si $J_{i} = \mathrm{HIPOTESIS}\bar{k}$ entonces $\widetilde{J_{i}}= \mathrm{HIPOTESIS}\bar{k}$
          \item Si $J_{i} = \alpha P(\overline{l_{1}}, \dotsc, \overline{l_{k}})$, con $\alpha \in \{\varepsilon\}
            \ \cup \ \{\mathrm{TESIS}\bar{k}: k \in \mathbb{N}\}$, entonces $\widetilde{J_{i}} = \alpha
            P(\overline{l_{1}+1}, \dotsc, \overline{l_{k}+1})$
        \end{itemize}

        \PN Luego,
        \[
          (\varphi \varphi_{1} \dotsc \varphi_{n} (\varphi \rightarrow \psi), \text{HIPOTESIS}\bar{m} \widetilde{J_{1}}
          \dotsc \widetilde{J_{n}} \text{CONCLUSION})
        \]
        \PN es una prueba de $(\varphi \rightarrow \psi)$ en $(\Sigma, \tau)$.

      \item Supongamos que $(\Sigma, \tau) \not\vdash \lnot \varphi$ y supongamos que $(\Sigma \ \cup \ \{\varphi\},
        \tau) \vdash (\psi \wedge \lnot \psi)$. Por (5), tenemos que $(\Sigma, \tau) \vdash (\varphi \rightarrow (\psi
        \wedge \lnot \psi))$. La siguiente prueba atestigua que $(\Sigma, \tau) \vdash \lnot \varphi$:
        \[
          \begin{array}{llll}
            1. & \varphi \rightarrow (\psi \wedge \lnot \psi) && \text{AXIOMAPROPIO} \\
            2. & \lnot\lnot\varphi \leftrightarrow \varphi && \text{AXIOMALOGICO} \\
            3. & \lnot\lnot\varphi \leftrightarrow (\psi \wedge \lnot \psi) && \text{REEMPLAZO}(2,1) \\
            4. & \lnot\lnot\varphi \rightarrow (\psi \wedge \lnot \psi) && \text{EQUIVALENCIAELIMINACION}(3) \\
            5. & \lnot\lnot\lnot\varphi && \text{ABSURDO}(4) \\
            6. & \lnot\varphi && \text{REEMPLAZO}(2,5) \\
          \end{array}
        \]
        \PN lo cual es absurdo y por lo tanto $(\Sigma \ \cup \ \{\varphi\}, \tau)$ es consistente.
    \end{enumerate}
  \end{proof}

  % Lemma 162. Sin prueba. Theorem 72.
  \begin{theorem} \label{theorem_72}
    \PN \textbf{(Corrección)} $(\Sigma, \tau) \vdash \varphi$ implica $(\Sigma, \tau) \models \varphi$.
  \end{theorem}

  % Lemma 163. Con prueba. Corollary 73.
  \begin{corollary} \label{corollary_73}
    \PN Si $(\Sigma, \tau)$ tiene un modelo, entonces $(\Sigma, \tau)$ es consistente.
  \end{corollary}
  \begin{proof}
    \PN Supongamos $\mathbf{A}$ es un modelo de $(\Sigma, \tau)$. Si $(\Sigma, \tau)$ fuera inconsistente, tendriamos
    que hay una $\varphi \in S^{\tau}$ tal que $(\Sigma, \tau) \vdash (\varphi \wedge \lnot \varphi)$, lo cual por
    el \textbf{Theorem~\ref{theorem_72}}, obtendríamos que $\mathbf{A} \models (\varphi \wedge \lnot \varphi)$, lo cual
    es un absurdo y vino de suponer que $(\Sigma, \tau)$ era inconsistente.
  \end{proof}

  % Lemma 164. Con prueba. Lemma 74.
  \begin{lemma} \label{lemma_74}
    \PN $\dashv \vdash_{T}$ es una relación de equivalencia.
  \end{lemma}
  \begin{proof}
    \PN \newline
    \begin{itemize}
      \item \textit{Reflexiva:} La relación es reflexiva ya que $(\varphi \leftrightarrow \varphi)$ es un axioma
        lógico, y por lo tanto $((\varphi \leftrightarrow \varphi), \text{AXIOMALOGICO})$ es una prueba de $(\varphi
        \leftrightarrow \varphi)$ en $T$.
      \item \textit{Simétrica:} Supongamos que $\varphi \dashv \vdash_{T} \psi$, es decir $T \vdash (\varphi
        \leftrightarrow \psi)$. Ya que $(\varphi \leftrightarrow \psi)$ se deduce de $\psi \leftrightarrow \varphi$ por
        la regla de conmutatividad, el \textbf{Lemma~\ref{lemma_71}}, nos dice que $T \vdash (\psi \leftrightarrow
        \varphi)$, es decir, $\psi \dashv \vdash_{T} \varphi$.
      \item \textit{Transitiva:} Supongamos que $\varphi \dashv \vdash_{T} \psi$ y que $\psi \dashv \vdash_{T} \phi$, es
        decir $T \vdash (\varphi \leftrightarrow \psi)$ y $T \vdash (\psi \leftrightarrow \phi)$. Ya que $(\varphi
        \leftrightarrow \phi)$ se deduce de $\varphi \leftrightarrow \psi$ y $\psi \leftrightarrow \phi$ por
        la regla de transitividad, el \textbf{Lemma~\ref{lemma_71}}, nos dice que $T \vdash (\varphi \leftrightarrow
        \phi)$, es decir, $\varphi \dashv \vdash_{T} \phi$.
    \end{itemize}
  \end{proof}

  % Lemma 165. Con prueba. Lemma 75.
  \begin{lemma} \label{lemma_75}
    \PN Dada una teoria $T = (\Sigma, \tau)$, se tiene que:
    \begin{enumerate}[(1)]
      \item $\{\varphi \in S^{\tau}: \varphi \ \text{es un teorema de T}\} \in S^{\tau}/ \dashv \vdash_{T}$
      \item $\{\varphi \in S^{\tau}: \varphi \ \text{es refutable en T}\} \in S^{\tau}/ \dashv \vdash_{T}$
    \end{enumerate}
  \end{lemma}
  \begin{proof}
    \begin{enumerate}[(1)]
      \item $\{\varphi \in S^{\tau}: \varphi \ \text{es un teorema de T}\} \in S^{\tau}/ \dashv \vdash_{T}$: Sean
        $\varphi, \psi$ teoremas en $T$, veremos que $\varphi \dashv \vdash_{T} \psi$. Notese que
        \[
          \begin{array}{llll}
            1. & \varphi && \text{HIPOTESIS1} \\
            2. & \psi && \text{TESIS1AXIOMAPROPIO} \\
            3. & (\varphi \rightarrow \psi) && \text{CONCLUSION} \\
            4. & \psi && \text{HIPOTESIS2} \\
            5. & \varphi && \text{TESIS2AXIOMAPROPIO} \\
            6. & (\psi \rightarrow \varphi) && \text{CONCLUSION} \\
            7. & (\varphi \leftrightarrow \psi) && \text{EQUIVALENCIAINTRODUCCION}(3,6)
          \end{array}
        \]
        \PN justifica que $(\Sigma \ \cup \ \{\varphi, \psi\}, \tau) \vdash (\varphi \leftrightarrow \psi)$ lo cual por
        el \textbf{Lemma~\ref{lemma_71}} tenes que $(\Sigma, \tau) \vdash (\varphi \leftrightarrow \psi)$, obteniendo
        que $\varphi \dashv \vdash_{T} \psi$.
      \item $\{\varphi \in S^{\tau}: \varphi \ \text{es refutable en T}\} \in S^{\tau}/ \dashv \vdash_{T}$: Sean
        $\varphi, \psi$ refutables en $T$, veremos que $\varphi \dashv \vdash_{T} \psi$. Notese que
        \[
          \begin{array}{llll}
            1. & \varphi && \text{HIPOTESIS1} \\
            2. & \lnot \psi && \text{HIPOTESIS2} \\
            3. & \lnot \varphi && \text{AXIOMAPROPIO} \\
            4. & (\varphi \wedge \lnot \varphi) && \text{TESIS2CONJUNCIONINTRODUCCION}(1,3) \\
            5. & \lnot \psi \rightarrow (\varphi \wedge \lnot \varphi) && \text{CONCLUSION} \\
            6. & \psi && \text{TESIS1ABSURDO}(5) \\
            7. & (\varphi \rightarrow \psi) && \text{CONCLUSION} \\
            8. & \psi && \text{HIPOTESIS3} \\
            9. & \lnot \varphi && \text{HIPOTESIS4} \\
            10. & \lnot \psi && \text{AXIOMAPROPIO} \\
            11. & (\psi \wedge \lnot \psi) && \text{TESIS4CONJUNCIONINTRODUCCION}(8,10) \\
            12. & \lnot \varphi \rightarrow (\psi \wedge \lnot \psi) && \text{CONCLUSION} \\
            13. & \varphi && \text{TESIS}3\text{ABSURDO}(5) \\
            14. & (\psi \rightarrow \varphi) && \text{CONCLUSION} \\
            15. & (\varphi \leftrightarrow \psi) && \text{EQUIVALENCIAINTRODUCCION}(7,14)
          \end{array}
        \]
       \PN justifica que $(\Sigma \ \cup \ \{\lnot \varphi, \lnot \psi\}, \tau) \vdash (\varphi \leftrightarrow \psi)$
       lo ual por el \textbf{Lemma~\ref{lemma_71}} tenes que $(\Sigma, \tau) \vdash (\varphi \leftrightarrow \psi)$,
       obteniendo que $\varphi \dashv \vdash_{T} \psi$.
    \end{enumerate}
  \end{proof}

  % Lemma 166. Con prueba. Lemma 76.
  \begin{lemma} \label{lemma_76}
    \PN Sea $T = (\Sigma, \tau)$ una teoría, entonces $(S^{\tau}/\mathrm{\dashv \vdash}, \SU^{T}, \IN^{T}, 0^{T},
    1^{T})$ es un álgebra de Boole.
  \end{lemma}
  \begin{proof}
    \PN Por definición de Álgebra de Boole, debemos probar que cualesquiera sean $\varphi_{1}, \varphi_{2}, \varphi_{3}
    \in S^{\tau}$, se cumplen las siguientes igualdades:
    \begin{enumerate}[(1)]
      % \item $[\varphi_{1}]_{T} \ \IN^{T} \ [\varphi_{1}]_{T} = [\varphi_{1}]_{T}$: Sea $\varphi_{1} \in S^{\tau}$ fija.
      %   Por la definición de la operación $\IN^{T}$ debemos probar que:
      %   \[
      %     [(\varphi_{1} \wedge \varphi_{1})]_{T} = [\varphi_{1}]_{T}
      %   \]
      %   \PN es decir, debemos probar que $T \vdash ((\varphi_{1} \wedge \varphi_{1}) \leftrightarrow \varphi_{1})$.
      %   \[
      %     \begin{array}{llll}
      %       1. & (\varphi_{1} \wedge \varphi_{1}) && \text{HIPOTESIS1} \\
      %       2. & \varphi_{1} && \text{TESIS1CONJUNCIONELIMINACION}(1) \\
      %       3. & (\varphi_{1} \wedge \varphi_{1}) \rightarrow \varphi_{1} && \text{CONCLUSION} \\
      %       4. & \varphi_{1} && \text{HIPOTESIS2} \\
      %       5. & (\varphi_{1} \wedge \varphi_{1}) && \text{TESIS2CONJUNCIONINTRODUCCION}(4) \\
      %       6. & \varphi_{1} \rightarrow (\varphi_{1} \wedge \varphi_{1}) && \text{CONCLUSION} \\
      %       7. & (\varphi_{1} \wedge \varphi_{1}) \leftrightarrow \varphi_{1} && \text{EQUIVALENCIAINTRODUCCION}(3,6)
      %     \end{array}
      %   \]
      %
      % \item $[\varphi_{1}]_{T} \ \SU^{T} \ [\varphi_{1}]_{T} = [\varphi_{1}]_{T}$: Sea $\varphi_{1} \in S^{\tau}$ fija.
      %   Por la definición de la operación $\SU^{T}$ debemos probar que:
      %   \[
      %     [(\varphi_{1} \vee \varphi_{1})]_{T} = [\varphi_{1}]_{T}
      %   \]
      %   \PN es decir, debemos probar que $T \vdash ((\varphi_{1} \vee \varphi_{1}) \leftrightarrow \varphi_{1})$.
      %   \[
      %     \begin{array}{llll}
      %       1. & (\varphi_{1} \vee \varphi_{1}) && \text{HIPOTESIS1} \\
      %       2. & \varphi_{1} \leftrightarrow \varphi_{1} && \text{AXIOMALOGICO} \\
      %       3. & \varphi_{1} \rightarrow \varphi_{1} && \text{EQUIVALENCIAELIMINACION}(2) \\
      %       4. & \varphi_{1} && \text{TESIS1DIVISIONPORCASOS}(1,3,3) \\
      %       5. & (\varphi_{1} \vee \varphi_{1}) \rightarrow \varphi_{1} && \text{CONCLUSION} \\
      %       6. & \varphi_{1} && \text{HIPOTESIS3} \\
      %       7. & (\varphi_{1} \vee \varphi_{1}) && \text{TESIS3DISJUNCIONINTRODUCCION}(6) \\
      %       8. & \varphi \rightarrow (\varphi_{1} \vee \varphi_{1}) && \text{CONCLUSION} \\
      %       9. & (\varphi_{1} \vee \varphi_{1}) \leftrightarrow \varphi_{1} && \text{EQUIVALENCIAINTRODUCCION}(5,8)
      %     \end{array}
      %   \]
      %
      % \item $[\varphi_{1}]_{T} \ \IN^{T} \ [\varphi_{2}]_{T} = [\varphi_{2}]_{T} \ \IN^{T} \ [\varphi_{1}]_{T}$: Sean
      %   $\varphi_{1}, \varphi_{2} \in S^{\tau}$ fijas. Por la definición de la operación $\IN^{T}$ debemos probar que:
      %   \[
      %     [(\varphi_{1} \wedge \varphi_{2})]_{T} = [(\varphi_{2} \wedge \varphi_{1})]_{T}
      %   \]
      %   \PN es decir, debemos probar que $T \vdash ((\varphi_{1} \wedge \varphi_{2}) \leftrightarrow (\varphi_{2} \wedge
      %   \varphi_{1}))$.
      %   \[
      %     \begin{array}{llll}
      %       1. & (\varphi_{1} \wedge \varphi_{2}) && \text{HIPOTESIS1} \\
      %       2. & \varphi_{1} && \text{CONJUNCIONELIMINACION}(1) \\
      %       3. & \varphi_{2} && \text{CONJUNCIONELIMINACION}(1) \\
      %       4. & (\varphi_{2} \wedge \varphi_{1}) && \text{TESIS1CONJUNCIONINTRODUCCION}(3,2) \\
      %       5. & (\varphi_{1} \wedge \varphi_{2}) \rightarrow (\varphi_{2} \wedge \varphi_{1}) && \text{CONCLUSION} \\
      %       6. & (\varphi_{2} \wedge \varphi_{1}) && \text{HIPOTESIS2} \\
      %       7. & \varphi_{2} && \text{CONJUNCIONELIMINACION}(6) \\
      %       8. & \varphi_{1} && \text{CONJUNCIONELIMINACION}(6) \\
      %       9. & (\varphi_{1} \wedge \varphi_{2}) && \text{TESIS1CONJUNCIONINTRODUCCION}(8,7) \\
      %       10. & (\varphi_{2} \wedge \varphi_{1}) \rightarrow (\varphi_{1} \wedge \varphi_{2}) && \text{CONCLUSION} \\
      %       11. & (\varphi_{1} \wedge \varphi_{2}) \leftrightarrow (\varphi_{2} \wedge \varphi_{1}) &&
      %         \text{EQUIVALENCIAINTRODUCCION}(5,10)
      %     \end{array}
      %   \]
      %
      % \item $[\varphi_{1}]_{T} \ \SU^{T} \ [\varphi_{2}]_{T} = [\varphi_{2}]_{T} \ \SU^{T} \ [\varphi_{1}]_{T}$: Sean
      %   $\varphi_{1}, \varphi_{2} \in S^{\tau}$ fijas. Por la definición de la operación $\SU^{T}$ debemos probar que:
      %   \[
      %     [(\varphi_{1} \vee \varphi_{2})]_{T} = [(\varphi_{2} \vee \varphi_{1})]_{T}
      %   \]
      %   \PN es decir, debemos probar que $T \vdash ((\varphi_{1} \vee \varphi_{2}) \leftrightarrow (\varphi_{2} \vee
      %   \varphi_{1}))$.
      %   \[
      %     \begin{array}{llll}
      %       1. & (\varphi_{1} \vee \varphi_{2}) && \text{HIPOTESIS1} \\
      %       2. & \varphi_{1} && \text{HIPOTESIS2} \\
      %       3. & (\varphi_{2} \vee \varphi_{1}) && \text{TESIS2DISJUNCIONINTRODUCCION}(2) \\
      %       4. & \varphi_{1} \rightarrow (\varphi_{2} \vee \varphi_{1}) && \text{CONCLUSION} \\
      %       5. & \varphi_{2} && \text{HIPOTESIS3} \\
      %       6. & (\varphi_{2} \vee \varphi_{1}) && \text{TESIS3DISJUNCIONINTRODUCCION}(2) \\
      %       7. & \varphi_{2} \rightarrow (\varphi_{2} \vee \varphi_{1}) && \text{CONCLUSION} \\
      %       8. & (\varphi_{2} \vee \varphi_{1}) && \text{TESIS1DIVISIONPORCASOS}(1,4,7) \\
      %       9. & (\varphi_{1} \vee \varphi_{2}) \rightarrow (\varphi_{2} \vee \varphi_{1}) && \text{CONCLUSION} \\
      %       10. & (\varphi_{2} \vee \varphi_{1}) && \text{HIPOTESIS4} \\
      %       11. & \varphi_{2} && \text{HIPOTESIS5} \\
      %       12. & (\varphi_{1} \vee \varphi_{2}) && \text{TESIS5DISJUNCIONINTRODUCCION}(11) \\
      %       13. & \varphi_{2} \rightarrow (\varphi_{1} \vee \varphi_{2}) && \text{CONCLUSION} \\
      %       14. & \varphi_{1} && \text{HIPOTESIS6} \\
      %       15. & (\varphi_{1} \vee \varphi_{2}) && \text{TESIS6DISJUNCIONINTRODUCCION}(14) \\
      %       16. & \varphi_{1} \rightarrow (\varphi_{1} \vee \varphi_{2}) && \text{CONCLUSION} \\
      %       17. & (\varphi_{1} \vee \varphi_{2}) && \text{TESIS4DIVISIONPORCASOS}(10,13,16) \\
      %       18. & (\varphi_{2} \vee \varphi_{1}) \rightarrow (\varphi_{1} \vee \varphi_{2}) && \text{CONCLUSION} \\
      %       19. & (\varphi_{1} \vee \varphi_{2}) \leftrightarrow (\varphi_{2} \vee \varphi_{1}) &&
      %         \text{EQUIVALENCIAINTRODUCCION}(9,18)
      %     \end{array}
      %   \]
      %
      % \item $[\varphi_{1}]_{T} \ \IN^{T}([\varphi_{2}]_{T} \ \IN^{T} \ [\varphi_{3}]_{T}) = ([\varphi_{1}]_{T} \ \IN^{T}
      %   \ [\varphi_{2}]_{T}) \ \IN^{T} \ [\varphi_{3}]_{T}$: Sean $\varphi_{1}, \varphi_{2}, \varphi_{3} \in
      %   S^{\tau}$ fijas. Por la definición de la operación $\IN^{T}$ debemos probar que:
      %   \[
      %     [(\varphi_{1} \wedge (\varphi_{2} \wedge \varphi_{3}))]_{T} = [((\varphi_{1} \wedge \varphi_{2}) \wedge
      %     \varphi_{3})]_{T}
      %   \]
      %   \PN es decir, debemos probar que $T \vdash ((\varphi_{1} \wedge (\varphi_{2} \wedge \varphi_{3}))
      %   \leftrightarrow ((\varphi_{1} \wedge \varphi_{2}) \wedge \varphi_{3}))$.
      %   \[
      %     \begin{array}{llll}
      %       1. & (\varphi_{1} \wedge (\varphi_{2} \wedge \varphi_{3})) && \text{HIPOTESIS1} \\
      %       2. & \varphi_{1} && \text{CONJUNCIONELIMINACION}(1) \\
      %       3. & (\varphi_{2} \wedge \varphi_{3}) && \text{CONJUNCIONELIMINACION}(1) \\
      %       4. & \varphi_{2} && \text{CONJUNCIONELIMINACION}(3) \\
      %       5. & \varphi_{3} && \text{CONJUNCIONELIMINACION}(3) \\
      %       6. & (\varphi_{1} \wedge \varphi_{2}) && \text{CONJUNCIONINTRODUCCION}(2,4) \\
      %       7. & ((\varphi_{1} \wedge \varphi_{2}) \wedge \varphi_{3}) && \text{TESIS1CONJUNCIONINTRODUCCION}(6,5) \\
      %       8. & (\varphi_{1} \wedge (\varphi_{2} \wedge \varphi_{3})) \rightarrow ((\varphi_{1} \wedge \varphi_{2})
      %         \wedge \varphi_{3}) && \text{CONCLUSION} \\
      %       9. & ((\varphi_{1} \wedge \varphi_{2}) \wedge \varphi_{3}) && \text{HIPOTESIS2} \\
      %       10. & (\varphi_{1} \wedge \varphi_{2}) && \text{CONJUNCIONELIMINACION}(9) \\
      %       11. & \varphi_{3} && \text{CONJUNCIONELIMINACION}(9) \\
      %       12. & \varphi_{1} && \text{CONJUNCIONELIMINACION}(10) \\
      %       13. & \varphi_{2} && \text{CONJUNCIONELIMINACION}(10) \\
      %       14. & (\varphi_{2} \wedge \varphi_{3}) && \text{CONJUNCIONINTRODUCCION}(13,11) \\
      %       15. & (\varphi_{1} \wedge (\varphi_{2} \wedge \varphi_{3})) && \text{TESIS1CONJUNCIONINTRODUCCION}(12,14) \\
      %       16. & ((\varphi_{1} \wedge \varphi_{2}) \wedge \varphi_{3}) \rightarrow (\varphi_{1} \wedge (\varphi_{2}
      %         \wedge \varphi_{3})) && \text{CONCLUSION} \\
      %       17. & (\varphi_{1} \wedge (\varphi_{2} \wedge \varphi_{3})) \leftrightarrow ((\varphi_{1} \wedge
      %         \varphi_{2}) \wedge \varphi_{3}) && \text{EQUIVALENCIAINTRODUCCION}(8,16) \\
      %     \end{array}
      %   \]
      %
      % \item $[\varphi_{1}]_{T} \ \SU^{T} \ ([\varphi_{2}]_{T} \ \SU^{T} \ [\varphi_{3}]_{T}) = ([\varphi_{1}]_{T} \
      %   \SU^{T} \ [\varphi_{2}]_{T}) \ \SU^{T} \ [\varphi_{3}]_{T}$: Sean $\varphi_{1}, \varphi_{2}, \varphi_{3} \in
      %   S^{\tau}$ fijas. Por la definición de la operación $\SU^{T}$ debemos probar que:
      %   \[
      %     [(\varphi_{1} \vee (\varphi_{2} \vee \varphi_{3}))]_{T} = [((\varphi_{1} \vee \varphi_{2}) \vee
      %     \varphi_{3})]_{T}
      %   \]
      %   \PN es decir, debemos probar que $T \vdash ((\varphi_{1} \vee (\varphi_{2} \vee \varphi_{3})) \leftrightarrow
      %   ((\varphi_{1} \vee \varphi_{2}) \vee \varphi_{3}))$. Notese que por el \textbf{Lemma~\ref{lemma_71}}, basta con
      %   probar que:
      %   \[
      %     \begin{array}{rcl}
      %       T & \vdash & ((\varphi_{1} \vee (\varphi_{2} \vee \varphi_{3})) \rightarrow ((\varphi_{1} \vee \varphi_{2})
      %         \vee \varphi_{3})) \\
      %       T & \vdash & (((\varphi_{1} \vee \varphi_{2}) \vee \varphi_{3}) \rightarrow (\varphi_{1} \vee (\varphi_{2}
      %         \vee \varphi_{3})))
      %     \end{array}
      %   \]
      %   \PN Prueba de $((\varphi_{1} \vee (\varphi_{2} \vee \varphi_{3})) \rightarrow ((\varphi_{1} \vee \varphi_{2})
      %   \vee \varphi_{3}))$
      %   \[
      %     \begin{array}{llll}
      %       1. & (\varphi_{1} \vee (\varphi_{2} \vee \varphi_{3})) && \text{HIPOTESIS1} \\
      %       2. & \varphi_{1} && \text{HIPOTESIS2} \\
      %       3. & (\varphi_{1} \vee \varphi_{2}) && \text{DISJUNCIONINTRODUCCION}(2) \\
      %       4. & ((\varphi_{1} \vee \varphi_{2}) \vee \varphi_{3}) && \text{TESIS2DISJUNCIONINTRODUCCION}(3) \\
      %       5. & \varphi_{1} \rightarrow ((\varphi_{1} \vee \varphi_{2}) \vee \varphi_{3}) && \text{CONCLUSION} \\
      %       6. & (\varphi_{2} \vee \varphi_{3}) && \text{HIPOTESIS3} \\
      %       7. & \varphi_{2} && \text{HIPOTESIS4} \\
      %       8. & (\varphi_{1} \vee \varphi_{2}) && \text{DISJUNCIONINTRODUCCION}(7) \\
      %       9. & ((\varphi_{1} \vee \varphi_{2}) \vee \varphi_{3}) && \text{TESIS4DISJUNCIONINTRODUCCION}(8) \\
      %       10. & \varphi_{2} \rightarrow ((\varphi_{1} \vee \varphi_{2}) \vee \varphi_{3}) && \text{CONCLUSION} \\
      %       11. & \varphi_{3} && \text{HIPOTESIS5} \\
      %       12. & ((\varphi_{1} \vee \varphi_{2}) \vee \varphi_{3}) && \text{TESIS5DISJUNCIONINTRODUCCION}(11) \\
      %       13. & \varphi_{3} \rightarrow ((\varphi_{1} \vee \varphi_{2}) \vee \varphi_{3}) && \text{CONCLUSION} \\
      %       14. & ((\varphi_{1} \vee \varphi_{2}) \vee \varphi_{3}) && \text{TESIS3DIVISIONPORCASOS}(6,10,13) \\
      %       15. & (\varphi_{2} \vee \varphi_{3}) \rightarrow ((\varphi_{1} \vee \varphi_{2}) \vee \varphi_{3}) &&
      %         \text{CONCLUSION} \\
      %       16. & ((\varphi_{1} \vee \varphi_{2}) \vee \varphi_{3}) && \text{TESIS1DIVISIONPORCASOS}(1,5,15) \\
      %       17. & (\varphi_{1} \vee (\varphi_{2} \vee \varphi_{3})) \rightarrow ((\varphi_{1} \vee \varphi_{2}) \vee
      %         \varphi_{3}) && \text{CONCLUSION}
      %     \end{array}
      %   \]
      %
      %   \PN Prueba de $((\varphi_{1} \vee \varphi_{2}) \vee \varphi_{3})) \rightarrow ((\varphi_{1} \vee (\varphi_{2}
      %   \vee \varphi_{3}))$
      %   \[
      %     \begin{array}{llll}
      %       1. & ((\varphi_{1} \vee \varphi_{2}) \vee \varphi_{3}) && \text{HIPOTESIS1} \\
      %       2. & \varphi_{3} && \text{HIPOTESIS2} \\
      %       3. & (\varphi_{2} \vee \varphi_{3}) && \text{DISJUNCIONINTRODUCCION}(2) \\
      %       4. & (\varphi_{1} \vee (\varphi_{2} \vee \varphi_{3})) && \text{TESIS2DISJUNCIONINTRODUCCION}(3) \\
      %       5. & \varphi_{3} \rightarrow (\varphi_{1} \vee (\varphi_{2} \vee \varphi_{3})) && \text{CONCLUSION} \\
      %       6. & (\varphi_{1} \vee \varphi_{2}) && \text{HIPOTESIS3} \\
      %       7. & \varphi_{1} && \text{HIPOTESIS4} \\
      %       8. & (\varphi_{1} \vee (\varphi_{2} \vee \varphi_{3})) && \text{TESIS4DISJUNCIONINTRODUCCION}(7) \\
      %       9. & \varphi_{1} \rightarrow (\varphi_{1} \vee (\varphi_{2} \vee \varphi_{3})) && \text{CONCLUSION} \\
      %       10. & \varphi_{2} && \text{HIPOTESIS5} \\
      %       11. & (\varphi_{2} \vee \varphi_{3}) && \text{DISJUNCIONINTRODUCCION}(10) \\
      %       12. & (\varphi_{1} \vee (\varphi_{2} \vee \varphi_{3})) && \text{TESIS5DISJUNCIONINTRODUCCION}(11) \\
      %       13. & \varphi_{2} \rightarrow (\varphi_{1} \vee (\varphi_{2} \vee \varphi_{3})) && \text{CONCLUSION} \\
      %       14. & (\varphi_{1} \vee (\varphi_{2} \vee \varphi_{3})) && \text{TESIS3DIVISIONPORCASOS}(6,9,13) \\
      %       15. & (\varphi_{1} \vee \varphi_{2}) \rightarrow (\varphi_{1} \vee (\varphi_{2} \vee \varphi_{3})) &&
      %         \text{CONCLUSION} \\
      %       16. & (\varphi_{1} \vee (\varphi_{2}) \vee \varphi_{3})) && \text{TESIS1DIVISIONPORCASOS}(1,15,5) \\
      %       17. & ((\varphi_{1} \vee \varphi_{2}) \vee \varphi_{3}) \rightarrow (\varphi_{1} \vee (\varphi_{2} \vee
      %         \varphi_{3})) && \text{CONCLUSION}
      %     \end{array}
      %   \]
      %
      % \item $[\varphi_{1}]_{T} \ \SU^{T} \ ([\varphi_{1}]_{T} \ \IN^{T} \ [\varphi_{2}]_{T}) = [\varphi_{1}]_{T}$: Sean
      %   $\varphi_{1}, \varphi_{2} \in S^{\tau}$ fijas. Por la definición de la operación $\SU^{T}$ debemos probar que:
      %   \[
      %     [(\varphi_{1} \vee (\varphi_{1} \wedge \varphi_{2}))]_{T} = [\varphi_{1}]_{T}
      %   \]
      %   \PN es decir, debemos probar que $T \vdash ((\varphi_{1} \vee (\varphi_{1} \wedge \varphi_{2})) \leftrightarrow
      %   \varphi_{1})$.
      %   \[
      %   \begin{array}{llll}
      %     1. & (\varphi_{1} \vee (\varphi_{1} \wedge \varphi_{2})) && \text{HIPOTESIS1} \\
      %     2. & \varphi_{1} \leftrightarrow \varphi_{1} && \text{AXIOMALOGICO} \\
      %     3. & \varphi_{1} \rightarrow \varphi_{1} && \text{EQUIVALENCIAELIMINACION}(2) \\
      %     4. & (\varphi_{1} \wedge \varphi_{2}) && \text{HIPOTESIS2} \\
      %     5. & \varphi_{1} && \text{TESIS2CONJUNCIONELIMINACION}(4) \\
      %     6. & (\varphi_{1} \wedge \varphi_{2}) \rightarrow \varphi_{1} && \text{CONCLUSION} \\
      %     7. & \varphi_{1} && \text{TESIS1DIVISIONPORCASOS}(1,3,6) \\
      %     8. & (\varphi_{1} \vee (\varphi_{1} \wedge \varphi_{2})) \rightarrow \varphi_{1} && \text{CONCLUSION} \\
      %     9. & \varphi_{1} && \text{HIPOTESIS3} \\
      %     10. & (\varphi_{1} \vee (\varphi_{1} \wedge \varphi_{2} )) && \text{TESIS3DISJUNCIONINTRODUCCION}(9) \\
      %     11. & \varphi_{1} \rightarrow (\varphi_{1} \vee (\varphi \wedge \varphi_{2})) && \text{CONCLUSION} \\
      %     12. & ((\varphi_{1} \vee (\varphi_{1} \wedge \varphi_{2})) \leftrightarrow \varphi_{1}) &&
      %       \text{EQUIVALENCIAINTRODUCCION}(8,11)
      %   \end{array}
      %   \]
      %
      % \item $[\varphi_{1}]_{T} \ \IN^{T}([\varphi_{1}]_{T} \ \SU^{T} \ [\varphi_{2}]_{T}) = [\varphi_{1}]_{T}$: Sean
      %   $\varphi_{1}, \varphi_{2} \in S^{\tau}$ fijas. Por la definición de la operación $\IN^{T}$ debemos probar que:
      %   \[
      %     [(\varphi_{1} \wedge (\varphi_{1} \vee \varphi_{2}))]_{T} = [\varphi_{1}]_{T}
      %   \]
      %   \PN es decir, debemos probar que $T \vdash ((\varphi_{1} \wedge (\varphi_{1} \vee \varphi_{2})) \leftrightarrow
      %   \varphi_{1})$.
      %   \[
      %   \begin{array}{llll}
      %     1. & (\varphi_{1} \wedge (\varphi_{1} \vee \varphi_{2})) && \text{HIPOTESIS1} \\
      %     2. & \varphi_{1} && \text{TESIS1CONJUNCIONELIMINACION}(1) \\
      %     3. & (\varphi_{1} \wedge (\varphi_{1} \vee \varphi_{2})) \rightarrow \varphi_{1} && \text{CONCLUSION} \\
      %     4. & \varphi_{1} && \text{HIPOTESIS2} \\
      %     5. & (\varphi_{1} \vee \varphi_{2}) && \text{DISJUNCIONINTRODUCCION}(4) \\
      %     6. & \varphi_{1} \wedge (\varphi_{1} \vee \varphi_{2}) && \text{TESIS2CONJUNCIONINTRODUCCION}(4,5) \\
      %     7. & \varphi_{1} \rightarrow \varphi_{1} \wedge (\varphi_{1} \vee \varphi_{2}) && \text{CONCLUSION} \\
      %     8. & ((\varphi_{1} \wedge (\varphi_{1} \vee \varphi_{2})) \leftrightarrow \varphi_{1}) &&
      %       \text{EQUIVALENCIAINTRODUCCION}(3,7)
      %   \end{array}
      %   \]
      %
      \item $0^{T} \ \SU^{T} \ [\varphi_{1}]_{T} = [\varphi_{1}]_{T}$: Ya que $(\varphi \wedge \lnot\varphi)$ es
        refutable en $T$, tenemos que el \textbf{Lemma~\ref{lemma_75}} nos dice que $0^{T} =
        \{\varphi \in S^{\tau}: \varphi$ es refutable de $T\} = [(\varphi \wedge \lnot\varphi)]_{T}$, es decir,
        que debemos probar que para cualquier $\varphi_{1} \in S^{\tau}$, se da que:
        \[
          [0]_{T} \ \IN^{T} \ [(\varphi \wedge \lnot\varphi)]_{T} = \{\varphi \in S^{\tau}: \varphi \ \text{es refutable
          en} \ T\}
        \]
       \PN Ya que $[\varphi_{1}]_{T} \ \SU^{T} \ [\forall x_{1} (x_{1} \equiv x_{1})]_{T} = [\varphi_{1} \vee \forall
       x_{1}(x_{1}\equiv x_{1})]_{T}$, debemos probar que $\varphi_{1} \vee \forall x_{1} (x_{1} \equiv x_{1})$ es un
       teorema de $T$, lo cual es atestiguado por la siguiente prueba
        \[
          \begin{array}{llll}
            1. & c \equiv c && \text{AXIOMALOGICO} \\
            2. & \forall x_{1} (x_{1} \equiv x_{1}) && \text{GENERALIZACION}(1) \\
            3. & (\varphi_{1} \vee \forall x_{1} (x_{1}\equiv x_{1})) && \text{DISJUNCIONINTRODUCCION}(2)
          \end{array}
        \]

      % \item $[\varphi_{1}]_{T} \ \SU^{T} \ 1^{T} = 1^{T}$: Ya que $\forall x_{1} (x_{1} \equiv x_{1})$ es un teorema de
      %   $T$, atestiguado por la prueba
      %   \[
      %     \begin{array}{llll}
      %       1. & c \equiv c && \text{AXIOMALOGICO} \\
      %       2. & \forall x_{1} (x_{1} \equiv x_{1}) && \text{GENERALIZACION}(1)
      %     \end{array}
      %   \]
      %   \PN donde $c$ es un nombre de constante que no pertenece a $\mathcal{C}$ y tal que $(\mathcal{C} \cup \{c\},
      %   \mathcal{F}, \mathcal{R}, a)$ es un tipo. Tenemos que el \textbf{Lemma~\ref{lemma_75}} nos dice que $1^{T} =
      %   \{\varphi \in S^{\tau}: \varphi$ es un teorema de $T\} = [\forall x_{1} (x_{1} \equiv x_{1})]_{T}$, es decir,
      %   que debemos probar que para cualquier $\varphi_{1} \in S^{\tau}$, se da que:
      %   \[
      %     [\varphi_{1}]_{T} \ \SU^{T} \ [\forall x_{1} (x_{1} \equiv x_{1})]_{T} = \{\varphi \in S^{\tau}: \varphi \
      %     \text{es un teorema de} \ T\}
      %   \]
      %  \PN Ya que $[\varphi_{1}]_{T} \ \SU^{T} \ [\forall x_{1} (x_{1} \equiv x_{1})]_{T} = [\varphi_{1} \vee \forall
      %  x_{1}(x_{1}\equiv x_{1})]_{T}$, debemos probar que $\varphi_{1} \vee \forall x_{1} (x_{1} \equiv x_{1})$ es un
      %  teorema de $T$, lo cual es atestiguado por la siguiente prueba
      %   \[
      %     \begin{array}{llll}
      %       1. & c \equiv c && \text{AXIOMALOGICO} \\
      %       2. & \forall x_{1} (x_{1} \equiv x_{1}) && \text{GENERALIZACION}(1) \\
      %       3. & (\varphi_{1} \vee \forall x_{1} (x_{1}\equiv x_{1})) && \text{DISJUNCIONINTRODUCCION}(2)
      %     \end{array}
      %   \]
      %
      % \item $[\varphi_{1}]_{T} \ \SU^{T} \ ([\varphi_{1}]_{T})^{\mathsf{c}^{T}} = 1^{T}$: Sea $\varphi_{1} \in S^{\tau}$
      %   fija. Ya que $\forall x_{1} (x_{1} \equiv x_{1})$ es un teorema de $T$ y por la definición de la operación
      %   $\SU^{T}$ debemos probar que:
      %   \[
      %     [(\varphi_{1} \vee \lnot\varphi_{1})]_{T} = [\forall x_{1} (x_{1} \equiv x_{1})]_{T}
      %   \]
      %   \PN es decir, debemos probar que $T \vdash ((\varphi_{1} \vee \lnot\varphi_{1}) \leftrightarrow \forall x_{1}
      %   (x_{1} \equiv x_{1}))$.
      %   \[
      %     \begin{array}{llll}
      %       1. & (\varphi_{1} \vee \lnot\varphi_{1}) && \text{HIPOTESIS1} \\
      %       2. & c \equiv c && \text{AXIOMALOGICO} \\
      %       3. & \forall x_{1} (x_{1} \equiv x_{1}) && \text{TESIS1GENERALIZACION}(2) \\
      %       4. & (\varphi_{1} \vee \lnot\varphi_{1}) \rightarrow \forall x_{1} (x_{1} \equiv x_{1}) &&
      %         \text{CONCLUSION} \\
      %       5. & \forall x_{1} (x_{1} \equiv x_{1}) && \text{HIPOTESIS2} \\
      %       6. & (\varphi_{1} \vee \lnot\varphi_{1}) && \text{TESIS2AXIOMALOGICO} \\
      %       7. & \forall x_{1} (x_{1} \equiv x_{1}) \rightarrow (\varphi_{1} \vee \lnot\varphi_{1}) &&
      %         \text{CONCLUSION} \\
      %       8. & (\varphi_{1} \vee \lnot\varphi_{1}) \leftrightarrow \forall x_{1} (x_{1} \equiv x_{1}) &&
      %         \text{EQUIVALENCIAINTRODUCCION}(8,16) \\
      %     \end{array}
      %   \]
      %   \PN donde $c$ es un nombre de constante que no pertenece a $\mathcal{C}$ y tal que $(\mathcal{C} \cup \{c\},
      %   \mathcal{F}, \mathcal{R}, a)$ es un tipo.

      \item $[\varphi_{1}]_{T} \ \IN^{T} \ ([\varphi_{1}]_{T})^{\mathsf{c}^{T}} = 0^{T}$

      % \item $[\varphi_{1}]_{T} \ \IN^{T}([\varphi_{2}]_{T} \ \SU^{T} \ [\varphi_{3}]_{T}) = ([\varphi_{1}]_{T} \ \IN^{T}
      %   \ [\varphi_{2}]_{T}) \ \SU^{T} \  ([\varphi_{1}]_{T} \ \IN^{T} \ [\varphi_{3}]_{T})$: Sean $\varphi_{1},
      %   \varphi_{2}, \varphi_{3} \in S^{\tau}$ fijas. Por la definición de las operaciones $\SU^{T}, \IN^{T}$ debemos
      %   probar que:
      %     \[
      %       [(\varphi_{1} \wedge (\varphi_{2} \vee \varphi_{3}))]_{T} = [((\varphi_{1} \wedge \varphi_{2}) \vee
      %       (\varphi_{1} \wedge \varphi_{3}))]_{T}
      %     \]
      %     \PN es decir, debemos probar que $T \vdash (\varphi_{1} \wedge (\varphi_{2} \vee \varphi_{3})) \leftrightarrow
      %     ((\varphi_{1} \wedge \varphi_{2}) \vee (\varphi_{1} \wedge \varphi_{3}))$. Notese que por el
      %     \textbf{Lemma~\ref{lemma_71}}, basta con probar que:
      %     \[
      %       \begin{array}{rcl}
      %         T & \vdash & (\varphi_{1} \wedge (\varphi_{2} \vee \varphi_{3})) \rightarrow ((\varphi_{1} \wedge
      %           \varphi_{2}) \vee (\varphi_{1} \wedge \varphi_{3})) \\
      %         T & \vdash & ((\varphi_{1} \wedge \varphi_{2}) \vee (\varphi_{1} \wedge \varphi_{3})) \rightarrow
      %           (\varphi_{1} \wedge (\varphi_{2} \vee \varphi_{3}))
      %       \end{array}
      %     \]
      %     \PN Prueba de $(\varphi_{1} \wedge (\varphi_{2} \vee \varphi_{3})) \rightarrow ((\varphi_{1} \wedge
      %     \varphi_{2}) \vee (\varphi_{1} \wedge \varphi_{3}))$
      %     \[
      %       \begin{array}{llll}
      %         1. & (\varphi_{1} \wedge (\varphi_{2} \vee \varphi_{3})) && \text{HIPOTESIS1} \\
      %         2. & \varphi_{1} && \text{CONJUNCIONELIMINACION}(1) \\
      %         3. & (\varphi_{2} \vee \varphi_{3}) && \text{CONJUNCIONELIMINACION}(1) \\
      %         4. & \varphi_{2} && \text{HIPOTESIS2} \\
      %         5. & (\varphi_{1} \wedge \varphi_{2}) && \text{CONJUNCIONINTRODUCCION}(2,4) \\
      %         6. & ((\varphi_{1} \wedge \varphi_{2}) \vee (\varphi_{1} \wedge \varphi_{3})) &&
      %           \text{TESIS2DISJUNCIONINTRODUCCION}(5) \\
      %         7. & \varphi_{2} \rightarrow ((\varphi_{1} \wedge \varphi_{2}) \vee (\varphi_{1} \wedge \varphi_{3})) &&
      %           \text{CONCLUSION} \\
      %         8. & \varphi_{3} && \text{HIPOTESIS3} \\
      %         9. & (\varphi_{1} \wedge \varphi_{3}) && \text{CONJUNCIONINTRODUCCION}(2,8) \\
      %         10. & ((\varphi_{1} \wedge \varphi_{2}) \vee (\varphi_{1} \wedge \varphi_{3})) &&
      %           \text{TESIS3DISJUNCIONINTRODUCCION}(9) \\
      %         11. & \varphi_{3} \rightarrow ((\varphi_{1} \wedge \varphi_{2}) \vee (\varphi_{1} \wedge \varphi_{3})) &&
      %           \text{CONCLUSION} \\
      %         12. & ((\varphi_{1} \wedge \varphi_{2}) \vee (\varphi_{1} \wedge \varphi_{3})) &&
      %           \text{TESIS1DIVISIONPORCASOS}(3,7,11) \\
      %         13. & (\varphi_{1} \wedge (\varphi_{2} \vee \varphi_{3})) \rightarrow ((\varphi_{1} \wedge \varphi_{2})
      %           \vee (\varphi_{1} \wedge \varphi_{3})) && \text{CONCLUSION}
      %       \end{array}
      %     \]
      %     \PN Prueba de $((\varphi_{1} \wedge \varphi_{2}) \vee (\varphi_{1} \wedge \varphi_{3})) \rightarrow
      %       (\varphi_{1} \wedge (\varphi_{2} \vee \varphi_{3}))$
      %     \[
      %     \begin{array}{llll}
      %       1. & ((\varphi_{1} \wedge \varphi_{2}) \vee (\varphi_{1} \wedge \varphi_{3})) && \text{HIPOTESIS1} \\
      %       2. & (\varphi_{1} \wedge \varphi_{2}) && \text{HIPOTESIS2} \\
      %       3. & \varphi_{1} && \text{CONJUNCIONELIMINACION}(2) \\
      %       4. & \varphi_{2} && \text{CONJUNCIONELIMINACION}(2) \\
      %       5. & (\varphi_{2} \vee \varphi_{3}) && \text{DISJUNCIONINTRODUCCION}(4) \\
      %       6. & \varphi_{1} \wedge (\varphi_{2} \vee \varphi_{3}) && \text{TESIS2CONJUNCIONINTRODUCCION}(3,5) \\
      %       7. & (\varphi_{1} \wedge \varphi_{2})\rightarrow (\varphi_{1} \wedge (\varphi_{2} \vee \varphi_{3})) &&
      %         \text{CONCLUSION} \\
      %       8. & (\varphi_{1} \wedge \varphi_{3}) && \text{HIPOTESIS}3 \\
      %       9. & \varphi_{1} && \text{CONJUNCIONELIMINACION}(8) \\
      %       10. & \varphi_{3} && \text{CONJUNCIONELIMINACION}(8) \\
      %       11. & (\varphi_{2} \vee \varphi_{3}) && \text{DISJUNCIONINTRODUCCION}(10) \\
      %       12. & \varphi_{1} \wedge (\varphi_{2} \vee \varphi_{3}) && \text{TESIS3CONJUNCIONINTRODUCCION}(9,11) \\
      %       13. & (\varphi_{1} \wedge \varphi_{3}) \rightarrow (\varphi_{1} \wedge (\varphi_{2} \vee \varphi_{3})) &&
      %         \text{CONCLUSION} \\
      %       14. & (\varphi_{1} \wedge (\varphi_{2} \vee \varphi_{3})) && \text{TESIS1DIVISIONPORCASOS}(1,7,13) \\
      %       15. & ((\varphi_{1} \wedge \varphi_{2}) \vee (\varphi_{1} \wedge \varphi_{3})) \rightarrow (\varphi_{1}
      %         \wedge (\varphi_{2} \vee \varphi_{3})) && \text{CONCLUSION}
      %     \end{array}
      %     \]
    \end{enumerate}
  \end{proof}

  % Lemma 167. Con prueba. Lemma 77.
  \begin{lemma} \label{lemma_77}
    \PN Sea T una teoría y sea $\leq^{T}$ el orden parcial asociado al álgebra de Boole $\mathcal{A}_{T}$ (es decir
    $[\varphi]_{T} \leq^{T} [\psi]_{T}$ si y solo si $[\varphi]_{T} \ \SU^{T} \ [\psi]_{T} = [\psi]_{T})$, entonces se
    tiene que:
    \[
      [\varphi]_{T} \leq^{T} [\psi]_{T} \text{si y solo si} \ T \vdash (\varphi \rightarrow \psi)
    \]
  \end{lemma}
  \begin{proof}
    \PN \begin{tabular}{|c|} \hline $\Rightarrow$ \\\hline \end{tabular} Supongamos $[\varphi]_{T} \leq^{T} [\psi]_{T}$,
    es decir, $[\varphi]_{T} \ \SU^{T} \ [\psi]_{T} = [\psi]_{T}$. Por la definición de $\SU^{T}$, tenemos que
    $[(\varphi \vee \psi)]_{T} = [\psi]_{T}$, es decir, $T \vdash ((\varphi \vee \psi) \leftrightarrow \psi)$. Luego, la
    siguiente prueba atestigua que $T \vdash (\varphi \rightarrow \psi)$
    \[
      \begin{array}{llll}
        1. & \varphi && \text{HIPOTESIS1} \\
        2. & (\varphi \vee \psi) && \text{DISJUNCIONINTRODUCCION}(1) \\
        3. & (\varphi \vee \psi) \leftrightarrow \psi && \text{AXIOMAPROPIO} \\
        4. & (\varphi \vee \psi) \rightarrow \psi && \text{EQUIVALENCIAELIMINACION}(3) \\
        5. & \psi && \text{TESIS1MODUSPONENS}(2,4) \\
        6. & \varphi \rightarrow \psi && \text{CONCLUSION}
      \end{array}
    \]

    \PN \begin{tabular}{|c|} \hline $\Leftarrow$ \\\hline \end{tabular} Supongamos $T \vdash (\varphi \rightarrow \psi)$,
    la siguiente prueba atestigua que $T \vdash ((\varphi \vee \psi) \leftrightarrow \psi)$.
    \[
      \begin{array}{llll}
        1. & (\varphi \vee \psi) && \text{HIPOTESIS1} \\
        2. & \varphi \rightarrow \psi && \text{AXIOMAPROPIO} \\
        3. & \psi \leftrightarrow \psi && \text{AXIOMALOGICO} \\
        4. & \psi \rightarrow \psi && \text{EQUIVALENCIAELIMINACION}(3) \\
        5. & \psi && \text{TESIS1DIVISIONPORCASOS}(1,2,4) \\
        6. & (\varphi \vee \psi) \rightarrow \psi && \text{CONCLUSION} \\
        7. & \psi && \text{HIPOTESIS2} \\
        8. & (\varphi \vee \psi) && \text{TESIS2DISJUNCIONINTRODUCCION} \\
        9. & \psi \rightarrow (\varphi \vee \psi) && \text{CONCLUSION} \\
        10. & (\varphi \vee \psi) \leftrightarrow \psi && \text{EQUIVALENCIAINTRODUCCION}(6,9)
      \end{array}
    \]
    \PN Esto nos dice que $[\varphi \vee \psi]_{T} = [\psi]_{T}$, que por la definición de $\SU^{T}$, tenemos que
    $[\varphi]_{T} \ \SU^{T} \ [\psi]_{T} = [\psi]_{T}$, es decir, $[\varphi_{T}] \leq^{T} [\psi]_{T}$.
  \end{proof}

  % Lemma 168. Con prueba. Lemma 78.
  \begin{lemma} \label{lemma_78}
    \PN Sean $\tau = (\mathcal{C}, \mathcal{F}, \mathcal{R}, a)$ y $\tau^{\prime} = (\mathcal{C}^{\prime},
    \mathcal{F}^{\prime}, \mathcal{R}^{\prime}, a^{\prime})$ tipos.
    \begin{enumerate}
      \item Si $\mathcal{C} \subseteq \mathcal{C}^{\prime}, \mathcal{F} \subseteq \mathcal{F}^{\prime}, \mathcal{R}
      \subseteq \mathcal{R}^{\prime}$ y $a^{\prime}\mid_{\mathcal{F} \cup \mathcal{R}} = a$, entonces $(\Sigma, \tau)
      \vdash \varphi$ implica $(\Sigma, \tau^{\prime}) \vdash \varphi$.
      \item Si $\mathcal{C} \subseteq \mathcal{C}^{\prime}, \mathcal{F} = \mathcal{F}^{\prime}, \mathcal{R} =
      \mathcal{R}^{\prime}$ y $a^{\prime} = a$, entonces $(\Sigma, \tau^{\prime}) \vdash \varphi$ implica $(\Sigma,
      \tau) \vdash \varphi$, cada vez que $\Sigma \cup \{\varphi\} \subseteq S^{\tau}$.
    \end{enumerate}
  \end{lemma}
  \begin{proof}
    \PN \newline
    \begin{enumerate}[(1)]
      \item Supongamos $(\Sigma, \tau) \vdash \varphi$. Sea $(\varphi_{1} \dotsc \varphi_{n}, J_{1} \dotsc J_{n})$ la
        prueba de $\varphi$ en $(\Sigma, \tau)$. Sea $\mathcal{C}_{1}$ el conjunto de nombres de constante que ocurren
        en alguna $\varphi_{i}$ y que no pertenecen a $\mathcal{C}$. Notese que aplicando varias veces el
        \textbf{Lemma~\ref{lemma_70}}, podemos obtener una prueba $(\tilde{\varphi}_{1} \dotsc \tilde{\varphi}_{n},
        J_{1} \dotsc J_{n})$ de $\varphi$ en $(\Sigma, \tau)$ la cual cumple que los nombres de constante que ocurren en
        alguna $\psi_{i}$ y que no pertenecen a $\mathcal{C}$ no pertenecen a $\mathcal{C}^{\prime}$. Luego,
        $(\tilde{\varphi}_{1} \dotsc \tilde{\varphi}_{n}, J_{1} \dotsc J_{n})$ es una prueba de $\varphi$ en $(\Sigma,
        \tau^{\prime})$, con lo cual $(\Sigma, \tau^{\prime}) \vdash \varphi$.

      \item Supongamos $(\Sigma, \tau^{\prime}) \vdash \varphi$. Sea $(\pmb{\varphi}, \mathbf{J})$ de $\varphi$ en
        $(\Sigma, \tau^{\prime})$. Veremos que $(\pmb{\varphi}, \mathbf{J})$ es una prueba de $\varphi$ en $(\Sigma,
        \tau)$. Recordemos la definición de prueba:
        \PN Sea $(\Sigma, \tau)$ una teoría de primer orden. Sea $\varphi$ una sentencia de tipo $\tau$. Una
        \textbf{prueba} de $\varphi$ en $(\Sigma, \tau)$ será un par adecuado $(\mathbf{\varphi}, \mathbf{J})$ de algún
        tipo $\tau_{1} = (\mathcal{C} \cup \mathcal{C}_{1}, \mathcal{F}, \mathcal{R}, a)$, con $\mathcal{C}_{1}$ finito
        y disjunto con $\mathcal{C}$, tal que
        \begin{enumerate}[(1)]
          \item Cada $\mathbf{\varphi}_{i}$ es una sentencia de tipo $\tau_{1}$
          \item $\mathbf{\varphi}_{n(\mathbf{\varphi})} = \varphi$
          \item Si $\left\langle i, j \right\rangle \in \mathcal{B}^{\mathbf{J}}$, entonces $\mathbf{J}_{j+1}$ es de la
            forma $\alpha \mathrm{CONCLUSION}$ y $\mathbf{\varphi}_{j+1} = (\mathbf{\varphi}_{i} \rightarrow
            \mathbf{\varphi}_{j})$
          \item Para cada $i = 1, \dotsc, n(\mathbf{\varphi}),$ se da una de las siguientes:
          \begin{enumerate}[(a)]
            \item $\mathbf{J}_{i} = \mathrm{HIPOTESIS}\bar{k}$ para algún $k \in \mathbf{N}$
            \item $\mathbf{J}_{i}$ es de la forma $\alpha \mathrm{CONCLUSION}$ y hay un $j$ tal que $\left\langle j, i-1
              \right\rangle \in \mathcal{B}^{\mathbf{J}}$ y $\mathbf{\varphi}_{i} = (\mathbf{\varphi}_{j} \rightarrow
              \mathbf{\varphi}_{i-1})$
            \item $\mathbf{J}_{i}$ es de la forma $\alpha \mathrm{AXIOMALOGICO}$ y $\mathbf{\varphi}_{i}$ es un axioma
              lógico de tipo $\tau_{1}$
            \item $\mathbf{J}_{i}$ es de la forma $\alpha \mathrm{AXIOMAPROPIO}$ y $\mathbf{\varphi}_{i} \in \Sigma$
            \item $\mathbf{J}_{i}$ es de la forma $\alpha \mathrm{PARTICULARIZACION} (\bar{l})$, con $l$ anterior a $i$
              y $(\mathbf{\varphi}_{l}, \mathbf{\varphi}_{i}) \in Partic^{\tau_{1}}$
            \item $\mathbf{J}_{i}$ es de la forma $\alpha \mathrm{COMMUTATIVIDAD}(\bar{l})$, con $l$ anterior a $i$ y
              $(\mathbf{\varphi}_{l}, \mathbf{\varphi}_{i}) \in Commut^{\tau_{1}}$
            \item $\mathbf{J}_{i}$ es de la forma $\alpha \mathrm{ABSURDO}(\bar{l})$, con $l$ anterior a $i$ y
              $(\mathbf{\varphi}_{l}, \mathbf{\varphi}_{i}) \in Absur^{\tau_{1}}$
            \item $\mathbf{J}_{i}$ es de la forma $\alpha \mathrm{EVOCACION}(\bar{l} )$, con $l$ anterior a $i$ y
              $(\mathbf{\varphi}_{l}, \mathbf{\varphi}_{i}) \in Evoc^{\tau_{1}}$
            \item $\mathbf{J}_{i}$ es de la forma $\alpha \mathrm{EXISTENCIA}(\bar{l })$, con $l$ anterior a $i$ y
              $(\mathbf{\varphi}_{l}, \mathbf{\varphi}_{i}) \in Exist^{\tau_{1}}$
            \item $\mathbf{J}_{i}$ es de la forma $\alpha \mathrm{CONJUNCIONELIMINACION}(\bar{l})$, con $l$ anterior a
              $i$ y $(\mathbf{\varphi}_{l}, \mathbf{\varphi}_{i}) \in ConjElim^{\tau_{1}}$
            \item $\mathbf{J}_{i}$ es de la forma $\alpha \mathrm{DISJUNCIONINTRODUCCION}(\bar{l})$, con $l$ anterior a
              $i$ y $(\mathbf{\varphi}_{l}, \mathbf{\varphi}_{i}) \in DisjInt^{\tau_{1}}$
            \item $\mathbf{J}_{i}$ es de la forma $\alpha \mathrm{EQUIVALENCIAELIMINACION}(\bar{l})$, con $l$ anterior a
              $i$ y $(\mathbf{\varphi}_{l}, \mathbf{\varphi}_{i}) \in EquivElim^{\tau_{1}}$
            \item $\mathbf{J}_{i}$ es de la forma $\alpha \mathrm{MODUSPONENS}(\overline{l_{1}}, \overline{l_{2}})$, con
              $l_{1}$ y $l_{2}$ anteriores a $i$ y $(\mathbf{\varphi}_{l_{1}}, \mathbf{\varphi}_{l_{2}},
              \mathbf{\varphi}_{i}) \in ModPon^{\tau_{1}}$
            \item $\mathbf{J}_{i}$ es de la forma $\alpha \mathrm{CONJUNCIONINTRODUCCION}(\overline{l_{1}},
              \overline{l_{2}})$, con $l_{1}$ y $ l_{2}$ anteriores a $i$ y $(\mathbf{\varphi}_{l_{1}},
              \mathbf{\varphi}_{l_{2}}, \mathbf{\varphi}_{i}) \in ConjInt^{\tau_{1}}$
            \item $\mathbf{J}_{i}$ es de la forma $\alpha \mathrm{EQUIVALENCIAINTRODUCCION}(\overline{l_{1}},
              \overline{l_{2}})$, con $l_{1}$ y $l_{2}$ anteriores a $i$ y $(\mathbf{\varphi}_{l_{1}},
              \mathbf{\varphi}_{l_{2}}, \mathbf{\varphi}_{i}) \in EquivInt^{\tau_{1}}$
            \item $\mathbf{J}_{i}$ es de la forma $\alpha \mathrm{DISJUNCIONELIMINACION}(\overline{l_{1}},
              \overline{l_{2}})$, con $l_{1}$ y $ l_{2}$ anteriores a $i$ y $(\mathbf{\varphi}_{l_{1}},
              \mathbf{\varphi}_{l_{2}}, \mathbf{\varphi}_{i}) \in DisjElim^{\tau_{1}}$
            \item $\mathbf{J}_{i}$ es de la forma $\alpha \mathrm{REEMPLAZO}(\overline{l_{1}}, \overline{l_{2}})$, con
              $l_{1}$ y $l_{2}$ anteriores a $i$ y $(\mathbf{\varphi}_{l_{1}}, \mathbf{\varphi}_{l_{2}},
              \mathbf{\varphi}_{i}) \in Reemp^{\tau_{1}}$
            \item $\mathbf{J}_{i}$ es de la forma $\alpha \mathrm{TRANSITIVIDAD}(\overline{l_{1}}, \overline{l_{2}})$,
              con $l_{1}$ y $l_{2}$ anteriores a $i$ y $(\mathbf{\varphi}_{l_{1}}, \mathbf{\varphi}_{l_{2}},
              \mathbf{\varphi}_{i}) \in Trans^{\tau_{1}}$
            \item $\mathbf{J}_{i}$ es de la forma $\alpha \mathrm{DIVISIONPORCASOS}(\overline{l_{1}}, \overline{l_{2}},
              \overline{l_{3}})$, con $l_{1},l_{2}$ y $ l_{3}$ anteriores a $i$ y y $(\mathbf{\varphi}_{l_{1}},
              \mathbf{\varphi}_{l_{2}}, \mathbf{\varphi}_{l_{3}}, \mathbf{\varphi}_{i}) \in DivPorCas^{\tau_{1}}$
            \item $\mathbf{J}_{i}$ es de la forma $\alpha \mathrm{ELECCION}(\bar{l}) $, con $l$ anterior a $i$ y
              $(\mathbf{\varphi}_{l}, \mathbf{\varphi}_{i}) \in Elec^{\tau_{1}}$ vía el nombre de constante $e$, el cual
              no pertenece a $ \mathcal{C}$ y no ocurre en $\mathbf{\varphi}_{1}, \dotsc, \mathbf{\varphi}_{i-1}$.
            \item $\mathbf{J}_{i}$ es de la forma $\alpha \mathrm{GENERALIZACION}(\bar{l})$, con $l$ anterior a $i$ y
              $(\mathbf{\varphi}_{l}, \mathbf{\varphi}_{i}) \in Generaliz^{\tau_{1}}$ vía el nombre de constante $c$ el
              cual cumple:
              \begin{enumerate}[(i)]
                \item $c \not\in \mathcal{C}$
                \item $c$ no es un nombre de constante que ocurra en $\mathbf{\varphi}$ el cual sea introducido por la
                  aplicación de la regla de elección, es decir para cada $u \in \{1, \dotsc, n(\mathbf{\varphi})\}$, si
                  $\mathbf{J}_{u}$ es de la forma $\alpha \mathrm{ELECCION}(\bar{v})$, entonces no se da que
                  $(\mathbf{\varphi}_{v}, \mathbf{\varphi}_{u}) \in Elec^{\tau_{1}}$ vía $c$.
                \item $c$ no ocurre en ninguna hipotesis de $\mathbf{\varphi}_{l}$.
                \item Ningún nombre de constante que ocurra en $\mathbf{\varphi}_{l}$ o en sus hipotesis, depende de $c$.
              \end{enumerate}
          \end{enumerate}
        \end{enumerate}

        \PN Ya que $(\mathbf{\varphi}, \mathbf{J})$ es una prueba de $\varphi$ en $(\Sigma, \tau^{\prime})$ hay un
        conjunto finito $\mathcal{C}_{1}$, disjunto con $\mathcal{C}^{\prime}$, tal que $(\mathcal{C}^{\prime} \cup
        \mathcal{C}_{1}, \mathcal{F}, \mathcal{R}, a)$ es un tipo y cada $\mathbf{\varphi}_{i}$ es una sentencia de tipo
        $(\mathcal{C}^{\prime} \cup \mathcal{C}_{1}, \mathcal{F}, \mathcal{R}, a)$. Notese que
        $\widetilde{\mathcal{C}_{1}} = \mathcal{C}_{1} \cup (\mathcal{C}^{\prime}-\mathcal{C})$ es tal que $(\mathcal{C}
        \cup \widetilde{\mathcal{C}_{1}}, \mathcal{F}, \mathcal{R}, a)$ es un tipo y cada $\mathbf{\varphi}_{i}$ es una
        sentencia de tipo $(\mathcal{C} \cup \widetilde{\mathcal{C}_{1}}, \mathcal{F}, \mathcal{R}, a)$, con lo cual
        $(\mathbf{\varphi}, \mathbf{J})$ cumple el punto (1) de la definición de prueba. Con la excepción de los puntos
        4(f) y 4(g)i para los cuales es necesario notar que $\mathcal{C} \subseteq \mathcal{C}^{\prime}$, todos los
        demás puntos se cumplen en forma directa.
    \end{enumerate}
  \end{proof}

  % Lemma 169. Sin prueba. Lemma 79.
  \begin{lemma} \label{lemma_79}
    \PN Sea $(\Sigma, \tau)$ una teoría y supongamos que $\tau$ tiene una cantidad infinita de nombres de constante que
    no ocurren en las sentencias de $\Sigma$, entonces para cada formula $\varphi =_{d} \varphi(v)$, se tiene que
    $\lbrack \forall v\varphi (v) \rbrack_{T} = \inf(\{\lbrack \varphi(t) \rbrack_{T}: t$ es un término cerrado$\})$.
  \end{lemma}

  % Lemma 170. Sin prueba. Lemma 80.
  \begin{lemma} \label{lemma_80}
    \PN \textbf{(Coincidencia)} Sean $\tau$ y $\tau^{\prime}$ dos tipos cualesquiera y sea $\tau_{\cap}$ dado por:
    \begin{itemize}
      \item $\mathcal{C}_{\cap} = \mathcal{C} \cap \mathcal{C}^{\prime}$
      \item $\mathcal{F}_{\cap} = \{f \in \mathcal{F} \cap \mathcal{F}^{\prime}: a(f) = a^{\prime}(f)\}$
      \item $\mathcal{R}_{\cap} = \{r \in \mathcal{R} \cap \mathcal{R}^{\prime}: a(r) = a^{\prime}(r)\}$
      \item $a_{\cap} = a\mid_{\mathcal{F}_{\cap} \; \cup \; \mathcal{R}_{\cap}}$
    \end{itemize}

    \PN Sean $\mathbf{A}$ y $\mathbf{A}^{\prime}$ modelos de tipo $\tau$ y $\tau^{\prime}$ respectivamente. Supongamos
    que $A = A^{\prime}$ y que $c^{\mathbf{A}} = c^{\mathbf{A}^{\prime}}$, para cada $c \in \mathcal{C}_{\cap},
    f^{\mathbf{A}} = f^{\mathbf{A}^{\prime}}$, para cada $f \in \mathcal{F}_{\cap}$ y $r^{\mathbf{A}} =
    r^{\mathbf{A}^{\prime}}$, para cada $r \in \mathcal{R}_{\cap}$, entonces:
    \begin{enumerate}[(a)]
      \item Para cada $t =_{d} t(\vec{v}) \in T^{\tau_{\cap}}$ se tiene que $t^{\mathbf{A}} \lbrack \vec{a} \rbrack =
      t^{\mathbf{A}^{\prime}} \lbrack \vec{a} \rbrack$, para cada $\vec{a} \in A^{n}$.
      \item Para cada $\varphi =_{d} \varphi (\vec{v}) \in F^{\tau_{\cap}}$ se tiene que:
      \[
        \mathbf{A} \models \varphi \lbrack \vec{a} \rbrack \ \text{si y solo si} \ \mathbf{A}^{\prime} \models \varphi
        \lbrack \vec{a} \rbrack
      \]
      \item Si $\Sigma \cup \{\varphi\} \subseteq S^{\tau_{\cap}}$, entonces:
      \[
        (\Sigma, \tau) \models \varphi \ \text{si y solo si} \ (\Sigma, \tau^{\prime}) \models \varphi
      \]
    \end{enumerate}
  \end{lemma}

  % Lemma 171. Sin prueba. Lemma 81.
  \begin{lemma} \label{lemma_81}
    \PN Sea $\tau$ un tipo. Hay una infinitupla $(\gamma_{1}, \gamma_{2}, \dotsc) \in F^{\tau_{\mathbb{N}}}$ tal que:
    \begin{enumerate}
      \item $\lvert Li(\gamma_{j}) \rvert \leq 1$, para cada $j = 1, 2, \dotsc$
      \item Si $\lvert Li(\gamma )\rvert \leq 1$, entonces $\gamma = \gamma_{j}$, para algún $j \in \mathbb{N}$
    \end{enumerate}
  \end{lemma}

  % Theorem 172. Con prueba. Theorem 82.
  \begin{theorem} \label{theorem_82}
    \PN \textbf{(Completitud) (G{\"o}del)} Sea $T = (\Sigma, \tau)$ una teoría de primer orden. Si $T \models \varphi$
    entonces $T \vdash \varphi$.
  \end{theorem}
  \begin{proof} % TODO error de compilación.
    Primero probaremos completitud para el caso en que $\tau $ tiene una cantidad infinita de nombres de constante que no ocurren en las sentencias de $ \Sigma $. Lo probaremos por el absurdo, es decir supongamos que $\varphi_{0} $ es tal que $(\Sigma, \tau)\models \varphi_{0}$ y $(\Sigma, \tau)\not\vdash \varphi_{0}.$ Notese que ya que $(\Sigma, \tau)\not\vdash \varphi_{0}$, tenemos que $\lbrack\lnot \varphi_{0}\rbrack\not=0^{\mathcal{A}_{(\Sigma, \tau)}}.$ Para cada $j\in \mathbb{N}$, sea $w_{j}\in Var$ tal que $ Li(\gamma _{j})\subseteq \{w_{j}\}$. Para cada $j$, declaremos $\gamma _{j}=_{d}\gamma _{j}(w_{j})$. Notese que por el Lema 163 tenemos que $\inf \{\lbrack\gamma _{j}(t)\rbrack:t\in T_{c}^{\tau }\} = \lbrack\forall w_{j}\gamma _{j}(w_{j})\rbrack$, para cada $j=1,2,\dotsc$. Por el Teorema de Rasiova y Sikorski tenemos que hay un filtro primo de $\mathcal{A}_{(\Sigma, \tau)}$ , $\mathcal{U}$ el cual cumple:

    (a) $\lbrack\lnot \varphi_{0}\rbrack\in \mathcal{U}$
    (b) para cada $j\in \mathbb{N}$, $\{\lbrack\gamma _{j}(t)\rbrack:t\in T_{c}^{\tau }\}\subseteq \mathcal{U}$ implica que $\lbrack\forall w_{j}\gamma _{j}(w_{j})\rbrack\in \mathcal{U}$
    Ya que la sucesion de las $\gamma _{i}$ cubre todas las formulas con a lo sumo una variable libre, podemos reescribir la propiedad (b) de la siguiente manera

    (b)$^{\prime}$ para cada $\varphi =_{d}\varphi (v) \in F^{\tau }$, si $\{\lbrack\varphi (t)\rbrack:t\in T_{c}^{\tau }\}\subseteq \mathcal{U}$ entonces $ \lbrack\forall v\varphi (v)\rbrack\in \mathcal{U}$
    Definamos sobre $T_{c}^{\tau }$ la siguiente relacion:

    $\displaystyle t\bowtie s\text{si y solo si }\lbrack(t\equiv s)\rbrack\in \mathcal{U}\text{.} $

    Veamos entonces que:
    (1) $\bowtie $ es de equivalencia.
    (2) Para cada $\varphi =_{d}\varphi (v_{1}, \dotsc, v_{n}) \in F^{\tau }$, $ t_{1}, \dotsc, t_{n},s_{1}, \dotsc, s_{n}\in T_{c}^{\tau }$, si $t_{1}\bowtie s_{1}$, $ t_{2}\bowtie s_{2}$, $\dotsc$, $t_{n}\bowtie s_{n}$, entonces $\lbrack\varphi (t_{1}, \dotsc, t_{n})\rbrack\in \mathcal{U}$ si y solo si $\lbrack\varphi (s_{1}, \dotsc, s_{n})\rbrack\in \mathcal{U}$.
    (3) Para cada $f\in \mathcal{F}_{n}$, $ t_{1}, \dotsc, t_{n},s_{1}, \dotsc, s_{n}\in T_{c}^{\tau }$,
    $\displaystyle t_{1}\bowtie s_{1},t_{2}\bowtie s_{2}, \dotsc, \;t_{n}\bowtie s_{n}\text{implica }f(t_{1}, \dotsc, t_{n})\bowtie f(s_{1}, \dotsc, s_{n}). $

    Probaremos (2). Notese que

    $\displaystyle (\Sigma, \tau)\vdash \left( (t_{1}\equiv s_{1})\wedge (t_{2}\equiv s_{2})\wedge \dotsc\wedge (t_{n}\equiv s_{n})\wedge \varphi (t_{1}, \dotsc, t_{n})\right) \rightarrow \varphi (s_{1}, \dotsc, s_{n}) $

    lo cual nos dice que
    $\displaystyle \lbrack (t_{1}\equiv s_{1})\rbrack \ \IN \ \lbrack(t_{2}\equiv s_{2})\rbrack \ \IN \  \dotsc \ \IN \ \lbrack(t_{n}\equiv s_{n})\rbrack \ \IN \ \lbrack\varphi (t_{1}, \dotsc, t_{n})\rbrack\leq \lbrack \varphi (s_{1}, \dotsc, s_{n})\rbrack $

    de lo cual se desprende que
    $\displaystyle \lbrack \varphi (t_{1}, \dotsc, t_{n})\rbrack\in \mathcal{U}\text{implica }\lbrack\varphi (s_{1}, \dotsc, s_{n})\rbrack\in \mathcal{U} $

    ya que $\mathcal{U}$ es un filtro. La otra implicacion es analoga.
    Para probar (3) podemos tomar $\varphi =\left( f(v_{1}, \dotsc, v_{n})\equiv f(s_{1}, \dotsc, s_{n})\right) $ y aplicar (2).

    Definamos ahora un modelo $\mathbf{A}_{\mathcal{U}}$ de tipo $\tau $ de la siguiente manera:

    - Universo de $\mathbf{A}_{\mathcal{U}}=T_{c}^{\tau }/\mathrm{\bowtie }$
    - $f^{\mathbf{A}_{\mathcal{U}}}(t_{1}/\mathrm{\bowtie }, \dotsc, t_{n}/ \mathrm{\bowtie })=f(t_{1}, \dotsc, t_{n})/\mathrm{\bowtie }$, $f\in \mathcal{F}_{n}$, $t_{1}, \dotsc, t_{n}\in T_{c}^{\tau }\;$
    - $r^{\mathbf{A}_{\mathcal{U}}} = \{(t_{1}/\mathrm{\bowtie }, \dotsc, t_{n}/ \mathrm{\bowtie }):\lbrack(t_{1}, \dotsc, t_{n})\rbrack\in \mathcal{U}\}$, $r\in \mathcal{R}_{n}.$
    Notese que la definicion de $f^{\mathbf{A}_{\mathcal{U}}}$ es inambigua por (3). Probaremos las siguientes propiedades basicas:

    (4) Para cada $t=_{d}t(v_{1}, \dotsc, v_{n}) \in \TAU$, $ t_{1}, \dotsc, t_{n}\in T_{c}^{\tau }$, tenemos que
    $\displaystyle t^{\mathbf{A}_{\mathcal{U}}}\lbrack t_{1}/\mathrm{\bowtie }, \dotsc, t_{n}/\mathrm{\bowtie }\rbrack=t(t_{1}, \dotsc, t_{n})/\mathrm{\bowtie } $

    (5) Para cada $\varphi =_{d}\varphi (v_{1}, \dotsc, v_{n}) \in F^{\tau }$, $ t_{1}, \dotsc, t_{n}\in T_{c}^{\tau }$, tenemos que
    $\displaystyle \mathbf{A}_{\mathcal{U}}\models \varphi \lbrack t_{1}/\mathrm{\bowtie } , \dotsc, t_{n}/\mathrm{\bowtie }\rbrack\text{si y solo si }\lbrack\varphi (t_{1}, \dotsc, t_{n})\rbrack\in \mathcal{U}. $

    La prueba de (4) es directa por induccion. Probaremos (5) por induccion en el $k$ tal que $\varphi \in F_{k}^{\tau }$. El caso $k=0$ es dejado al lector. Supongamos (5) vale para $\varphi \in F_{k}^{\tau }$. Sea $\varphi =_{d}\varphi (v_{1}, \dotsc, v_{n}) \in F_{k+1}^{\tau }-F_{k}^{\tau }.$ Hay varios casos:

    CASO $\varphi (v_{1}, \dotsc, v_{n})=\left( \varphi_{1}(v_{1}, \dotsc, v_{n}) \vee \varphi_{2}(v_{1}, \dotsc, v_{n})\right) .$

    Tenemos

    $\displaystyle \begin{array}{c} \mathbf{A}_{\mathcal{U}}\models \varphi \lbrack t_{1}/\mathrm{\bowtie } , \dotsc, t_{n}/\mathrm{\bowtie }\rbrack \\ \Updownarrow \\ \mathbf{A}_{\mathcal{U}}\models \varphi_{1}\lbrack t_{1}/\mathrm{\bowtie } , \dotsc, t_{n}/\mathrm{\bowtie }\rbrack\text{o }\mathbf{A}_{\mathcal{U}}\models \varphi_{2}\lbrack t_{1}/\mathrm{\bowtie }, \dotsc, t_{n}/\mathrm{\bowtie }\rbrack \\ \Updownarrow \\ \lbrack \varphi_{1}(t_{1}, \dotsc, t_{n})\rbrack\in \mathcal{U}\text{o }\lbrack\varphi_{2}(t_{1}, \dotsc, t_{n})\rbrack\in \mathcal{U} \\ \Updownarrow \\ \lbrack \varphi_{1}(t_{1}, \dotsc, t_{n})\rbrack\ \mathsf{s\ }\lbrack\varphi_{2}(t_{1}, \dotsc, t_{n})\rbrack\in \mathcal{U} \\ \Updownarrow \\ \lbrack \left( \varphi_{1}(t_{1}, \dotsc, t_{n}) \vee \varphi_{2}(t_{1}, \dotsc, t_{n})\right) \rbrack\in \mathcal{U} \\ \Updownarrow \\ \lbrack \varphi (t_{1}, \dotsc, t_{n})\rbrack\in \mathcal{U}. \end{array} $

    CASO $\varphi (v_{1}, \dotsc, v_{n})=\forall v\varphi_{1}(v_{1}, \dotsc, v_{n},v).$
    Tenemos

    $\displaystyle \begin{array}{c} \mathbf{A}_{\mathcal{U}}\models \varphi \lbrack t_{1}/\mathrm{\bowtie } , \dotsc, t_{n}/\mathrm{\bowtie }\rbrack \\ \Updownarrow \\ \mathbf{A}_{\mathcal{U}}\models \varphi_{1}\lbrack t_{1}/\mathrm{\bowtie } , \dotsc, t_{n}/\mathrm{\bowtie },t/\mathrm{\bowtie }\rbrack\text{, para todo }t\in T_{c}^{\tau } \\ \Updownarrow \\ \lbrack \varphi_{1}(t_{1}, \dotsc, t_{n},t)\rbrack\in \mathcal{U}\text{, para todo } t\in T_{c}^{\tau } \\ \Updownarrow \\ \lbrack \forall v\varphi_{1}(t_{1}, \dotsc, t_{n},v)\rbrack\in \mathcal{U} \\ \Updownarrow \\ \lbrack \varphi (t_{1}, \dotsc, t_{n})\rbrack\in \mathcal{U}. \end{array} $

    CASO $\varphi (v_{1}, \dotsc, v_{n})=\exists v\varphi_{1}(v_{1}, \dotsc, v_{n},v).$
    Tenemos

    $\displaystyle \begin{array}{c} \mathbf{A}_{\mathcal{U}}\models \varphi \lbrack t_{1}/\mathrm{\bowtie } , \dotsc, t_{n}/\mathrm{\bowtie }\rbrack \\ \Updownarrow \\ \mathbf{A}_{\mathcal{U}}\models \varphi_{1}\lbrack t_{1}/\mathrm{\bowtie } , \dotsc, t_{n}/\mathrm{\bowtie },t/\mathrm{\bowtie }\rbrack\text{, para algun }t\in T_{c}^{\tau } \\ \Updownarrow \\ \lbrack \varphi_{1}(t_{1}, \dotsc, t_{n},t)\rbrack\in \mathcal{U}\text{, para algun } t\in T_{c}^{\tau } \\ \Updownarrow \\ \lbrack \varphi_{1}(t_{1}, \dotsc, t_{n},t)\rbrack^{c}\not\in \mathcal{U}\text{, para algun }t\in T_{c}^{\tau } \\ \Updownarrow \\ \lbrack \lnot \varphi_{1}(t_{1}, \dotsc, t_{n},t)\rbrack\not\in \mathcal{U}\text{, para algun }t\in T_{c}^{\tau } \\ \Updownarrow \\ \lbrack \forall v\;\lnot \varphi_{1}(t_{1}, \dotsc, t_{n},v)\rbrack\not\in \mathcal{U} \\ \Updownarrow \\ \lbrack \forall v\;\lnot \varphi_{1}(t_{1}, \dotsc, t_{n},v)\rbrack^{c}\in \mathcal{U} \\ \Updownarrow \\ \lbrack \lnot \forall v\;\lnot \varphi_{1}(t_{1}, \dotsc, t_{n},v)\rbrack\in \mathcal{U } \\ \Updownarrow \\ \lbrack \varphi (t_{1}, \dotsc, t_{n})\rbrack\in \mathcal{U}. \end{array} $

    Pero ahora notese que (5) en particular nos dice que para cada sentencia $ \psi \in S^{\tau }$, $\mathbf{A}_{\mathcal{U}}\models \psi $ si y solo si $ \lbrack\psi \rbrack\in \mathcal{U}.$ De esta forma llegamos a que $\mathbf{A}_{\mathcal{U }}\models \Sigma $ y $\mathbf{A}_{\mathcal{U}}\models \lnot \varphi_{0}$, lo cual contradice la suposicion de que $(\Sigma, \tau)\models \varphi_{0}. $
    Ahora supongamos que $\tau $ es cualquier tipo. Sean $s_{1}$ y $s_{2}$ un par de simbolos no pertenecientes a la lista

    $\displaystyle \forall \ \ \exists \ \ \lnot \ \ \vee \ \ \wedge \ \ \rightarrow \ \ \leftrightarrow \ \ (\ \ )\ \ ,\ \equiv \ \ \mathsf{X}\ \ \mathit{0}\ \ \mathit{1}\ \ \dotsc\ \ \mathit{9}\ \ \mathbf{0}\ \ \mathbf{1}\ \ \dotsc\ \ \mathbf{9} $

    y tales que ninguno ocurra en alguna palabra de $\mathcal{C}\cup \mathcal{F} \cup \mathcal{R}.$ Si $(\Sigma, \tau)\models \varphi $, entonces usando el Lema de Coincidencia se puede ver que $(\Sigma ,(\mathcal{C}\cup \{s_{1}s_{2}s_{1},s_{1}s_{2}s_{2}s_{1},\dotsc\},\mathcal{F},\mathcal{R} ,a))\models \varphi $, por lo cual
    $\displaystyle (\Sigma ,(\mathcal{C}\cup \{s_{1}s_{2}s_{1},s_{1}s_{2}s_{2}s_{1},\dotsc\}, \mathcal{F},\mathcal{R},a))\vdash \varphi . $

    Pero por Lema 162, tenemos que $(\Sigma, \tau)\vdash \varphi .$
  \end{proof}

  % Corollary 173. Con prueba. Corollary 83.
  \begin{corollary} \label{corollary_83}
    \PN Toda teoría consistente tiene un modelo.
  \end{corollary}
  \begin{proof}
    \PN Supongamos $(\Sigma, \tau)$ es consistente y no tiene modelos. Entonces $(\Sigma, \tau) \models \left(\varphi
    \wedge \lnot \varphi \right) $, con lo cual por \textbf{Completitud}, $(\Sigma, \tau) \vdash \left(\varphi \wedge
    \lnot \varphi \right)$, lo cual es absurdo. Dicho absurdo vino de suponer que $(\Sigma, \tau)$ no tenía modelos.
  \end{proof}

  % Corollary 174. Con prueba. Corollary 84.
  \begin{corollary} \label{corollary_84}
    \PN \textbf{(Teorema de Compacidad)}
    \begin{enumerate}[(a)]
      \item Si $(\Sigma, \tau)$ es tal que $(\Sigma_{0}, \tau)$ tiene un modelo, para cada subconjunto finito
      $\Sigma_{0} \subseteq \Sigma$, entonces $(\Sigma, \tau)$ tiene un modelo.
      \item Si $(\Sigma, \tau) \models \varphi$, entonces hay un subconjunto finito $\Sigma_{0} \subseteq \Sigma$ tal
      que $(\Sigma_{0}, \tau) \models \varphi$.
    \end{enumerate}
  \end{corollary}
  \begin{proof}
    \begin{enumerate}[(a)]
      \item Si $(\Sigma, \tau)$ no tuviera un modelo, es decir, si fuera inconsistente, habría un subconjunto finito
        $\Sigma_{0} \subseteq \Sigma$ tal que la teoría $(\Sigma_{0}, \tau)$ es inconsistente, lo cual es absurdo, pues
        cada subconjunto finito de $\Sigma$ es consistente.
      \item Si $(\Sigma, \tau) \models \varphi$, entonces por \textbf{Completitud}, $(\Sigma, \tau) \vdash \varphi$, es
        decir que hay un subconjunto finito $\Sigma_{0} \subseteq \Sigma$ tal que $(\Sigma_{0}, \tau) \vdash \varphi$.
        Por lo tanto, por \textbf{Correción}, $(\Sigma_{0}, \tau) \models \varphi$.
    \end{enumerate}
  \end{proof}

  % El número 175 es usado para un ejemplo.

  % Lemma 176. Nada. Lemma 85.
  \begin{lemma}
    \PN Este lema no se evalua.
  \end{lemma}

  % Lemma 177. Nada. Lemma 86.
  \begin{lemma}
    \PN Este lema no se evalua.
  \end{lemma}

  % Lemma 178. Nada. Lemma 87.
  \begin{lemma}
    \PN Este lema no se evalua.
  \end{lemma}

  % Theorem 179. Con prueba. Theorem 88.
  \begin{theorem}
    \PN Este teorema no se evalua.
  \end{theorem}

  % Theorem 180. Con prueba. Theorem 89.
  \begin{theorem}
    \PN Este teorema no se evalua.
  \end{theorem}

  % Corollary 181. Sin prueba. Corollary 90.
  \begin{corollary}
    \PN Este corolario no se evalua.
  \end{corollary}

	\section{La aritmética de Peano}

\end{document}
