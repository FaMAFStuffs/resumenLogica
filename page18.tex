6.2.1. Subreticulados.

Sea \((L,\mathsf{s},\mathsf{i})\) un reticulado. Un conjunto no vacio \( S\subseteq L\) sera llamado subuniverso de \((L,\mathsf{s},\mathsf{i} ) \) si es cerrado bajo las operaciones \(\mathsf{s}\) e \(\mathsf{i}\) (i.e. \(x \mathsf{\;s\;}y\), \(x\;\mathsf{i}\ y\in S\), para todo \(x,y\in S)\). Notese que en tal caso la estructura \((S,\mathsf{s}\mathrm{\mid }_{S\times S},\mathsf{i} \mathrm{\mid }_{S\times S})\) es un reticulado. Diremos que \((S,\mathsf{s} \mathrm{\mid }_{S\times S},\mathsf{i}\mathrm{\mid }_{S\times S})\) es subreticulado de \((L,\mathsf{s},\mathsf{i})\).

6.2.2. Homomorfismos de reticulados.

Sean \((L,\mathsf{s},\mathsf{i})\) y \((L^{\prime },\mathsf{s}^{\prime }, \mathsf{i}^{\prime })\) reticulados. Una funcion \(F:L\rightarrow L^{\prime }\) sera llamada un homomorfismo de \((L,\mathsf{s},\mathsf{i})\) en \((L^{\prime },\mathsf{s}^{\prime },\mathsf{i}^{\prime })\) si para todo \( x,y\in L\) se cumple que

\(\displaystyle \begin{array}{rcl} F(x\mathsf{\;s\;}y) & =& F(x)\;\mathsf{s}^{\prime }\ F(y) \\ F(x\mathsf{\;i\;}y) & =& F(x)\;\mathsf{i}^{\prime }\ F(y). \end{array} \)

Un homomorfismo de \((L,\mathsf{s},\mathsf{i})\) en \((L^{\prime }, \mathsf{s}^{\prime },\mathsf{i}^{\prime })\) sera llamado isomorfismo de \((L,\mathsf{s},\mathsf{i})\) en \((L^{\prime },\mathsf{s}^{\prime }\), \(\mathsf{i}^{\prime }\) \()\) cuando sea biyectivo y su inversa sea tambien un homomorfismo. Escribiremos \((L,\mathsf{s},\mathsf{i})\cong (L^{\prime }, \mathsf{s}^{\prime },\mathsf{i}^{\prime })\) cuando exista un isomorfismo de \( (L,\mathsf{s},\mathsf{i})\) en \((L^{\prime },\mathsf{s}^{\prime },\mathsf{i} ^{\prime })\). Escribiremos \(F:(L,\mathsf{s},\mathsf{i})\rightarrow (L^{\prime },\mathsf{s}^{\prime },\mathsf{i}^{\prime })\) cuando \(F\) sea un homomorfismo de \((L,\mathsf{s},\mathsf{i})\) en \((L^{\prime }, \mathsf{s}^{\prime },\mathsf{i}^{\prime })\).
Lema 93 Si \(F:(L,\mathsf{s},\mathsf{i})\rightarrow (L^{\prime },\mathsf{s}^{\prime },\mathsf{i}^{\prime })\) es un homomorfismo biyectivo, entonces \(F\) es un isomorfismo
Prueba: Solo falta ver que \(F^{-1}\) es un homomorfismo. Sean \(F(x),F(y)\) dos elementos cualesquiera de \(L^{\prime }\). Tenemos que

\(\displaystyle F^{-1}(F(x)\;\mathsf{s}^{\prime }\ F(y))=F^{-1}(F(x\mathsf{\;s\;}y))=x \mathsf{\;s\;}y=F^{-1}(F(x))\;\mathsf{s}\ F^{-1}(F(y)) \)

\(\Box\)
Lema 94 Sean \((L,\mathsf{s},\mathsf{i})\) y \((L^{\prime },\mathsf{s}^{\prime }, \mathsf{i}^{\prime })\) reticulados y sea \(F:(L,\mathsf{s},\mathsf{i} )\rightarrow (L^{\prime },\mathsf{s}^{\prime },\mathsf{i}^{\prime })\) un homomorfismo. Entonces \(I_{F}\) es un subuniverso de \((L^{\prime },\mathsf{s} ^{\prime },\mathsf{i}^{\prime })\).
Prueba: Ya que \(L\) es no vacio tenemos que \(I_{F}\) tambien es no vacio. Sean \(a,b\in I_{F}\). Sean \(x,y\in L\) tales que \(F(x)=a\) y \(F(y)=b\). Se tiene que

\(\displaystyle \begin{array}{rcl} a\;\mathsf{s}^{\prime }\ b & =& F(x)\;\mathsf{s}^{\prime }\ F(y)=F(x\mathsf{ \;s\;}y)\in I_{F} \\ a\;\mathsf{i}^{\prime }\ b & =& F(x)\;\mathsf{i}^{\prime }\ F(y)=F(x\mathsf{ \;i\;}y)\in I_{F} \end{array} \)

por lo cual \(I_{F}\) es cerrada bajo \(\mathsf{s}^{\prime }\) e \(\mathsf{i} ^{\prime }\). \(\Box\)
Lema 95 Sean \((L,\mathsf{s},\mathsf{i})\) y \((L^{\prime },\mathsf{s}^{\prime }, \mathsf{i}^{\prime })\) reticulados y sean \((L\leq )\) y \((L^{\prime },\leq ^{\prime })\) los posets asociados. Sea \(F:L\rightarrow L^{\prime }\) una funcion. Entonces \(F\) es un isomorfismo de \((L,\mathsf{s},\mathsf{i})\) en \( (L^{\prime },\mathsf{s}^{\prime },\mathsf{i}^{\prime })\) si y solo si \(F\) es un isomorfismo de \((L,\leq )\) en \((L^{\prime },\leq ^{\prime })\).
Prueba: Supongamos \(F\) es un isomorfismo de \((L,\mathsf{s},\mathsf{i})\) en \( (L^{\prime },\mathsf{s}^{\prime },\mathsf{i}^{\prime })\). Sean \(x,y\in L\), tales que \(x\leq y\). Tenemos que \(y=x\mathsf{\;s\;}y\) por lo cual \(F(y)=F(x \mathsf{\;s\;}y)=F(x)\mathsf{\;s^{\prime }\;}F(y)\), produciendo \(F(x)\leq ^{\prime }F(y)\). En forma similar se puede ver que \(F^{-1}\) es tambien un homomorfismo de \((L^{\prime },\leq ^{\prime })\) en \((L,\leq )\). Si \(F\) es un isomorfismo de \((L,\leq )\) en \((L^{\prime },\leq ^{\prime })\), entonces el Lema 89 nos dice que \(F\) y \(F^{-1}\) respetan las operaciones de supremo e infimo por lo cual \(F\) es un isomorfismo de \((L,\mathsf{s},\mathsf{ i})\) en \((L^{\prime },\mathsf{s}^{\prime },\mathsf{i}^{\prime })\). \(\Box\)

6.2.3. Congruencias de reticulados.

Sea \((L,\mathsf{s},\mathsf{i})\) un reticulado. Una congruencia sobre \((L,\mathsf{s},\mathsf{i})\) sera una relacion de equivalencia \(\theta \) la cual cumpla:

(1) \(x\theta x^{\prime }\) y \(y\theta y^{\prime }\) implica \((x\mathsf{ \;s\;}y)\theta (x^{\prime }\mathsf{\;s\;}y^{\prime })\) y \((x\mathsf{\;i\;} y)\theta (x^{\prime }\mathsf{\;i\;}y^{\prime })\)
Gracias a esta condicion podemos definir sobre \(L/\theta \) dos operaciones binarias \(\mathsf{\tilde{s}}\) e \(\mathsf{\tilde{\imath}}\), de la siguiente manera:

\(\displaystyle \begin{array}{rcl} x/\theta \mathsf{\;\tilde{s}\;}y/\theta & =& (x\mathsf{\;s\;}y)/\theta \\ x/\theta \mathsf{\;\tilde{\imath}\;}y/\theta & =& (x\mathsf{\;i\;}y)/\theta \end{array} \)

Lema 96 \((L/\theta ,\mathsf{\tilde{s}},\mathsf{\tilde{\imath}})\) es un reticulado. El orden parcial \(\tilde{\leq}\) asociado a este reticulado cumple
\(\displaystyle x/\theta \tilde{\leq}y/\theta \text{ sii }y\theta (x\mathsf{\;s\;}y) \)
Prueba: Veamos que la estructura \((L/\theta ,\mathsf{\tilde{s}},\mathsf{\tilde{\imath }})\) cumple (I4). Sean \(x/\theta \), \(y/\theta \), \(z/\theta \) elementos cualesquiera de \(L/\theta \). Tenemos que

\(\displaystyle \begin{array}{ccl} (x/\theta \mathsf{\;\tilde{s}\;}y/\theta )\;\mathsf{\tilde{s}}\;z/\theta & = & (x\mathsf{\;s\;}y)/\theta \;\mathsf{\tilde{s}}\;z/\theta \\ & = & ((x\mathsf{\;s\;}y)\;\mathsf{s}\;z)/\theta \\ & = & (x\mathsf{\;s\;}(y\;\mathsf{s}\;z))/\theta \\ & = & x/\theta \;\mathsf{\tilde{s}}\;(y\;\mathsf{s}\;z)/\theta \\ & = & x/\theta \mathsf{\;\tilde{s}\;}(y/\theta \;\mathsf{\tilde{s}} \;z/\theta ) \end{array} \)

En forma similar se puede ver que la estructura \((L/\theta ,\mathsf{\tilde{s} },\mathsf{\tilde{\imath}})\) cumple el resto de las identidades que definen reticulado.
Por definicion, \(x/\theta \tilde{\leq}y/\theta \) sii \(y/\theta =x/\theta \mathsf{\;\tilde{s}\;}y/\theta \), por lo cual \(x/\theta \tilde{\leq}y/\theta \) sii \(y/\theta =(x\mathsf{\;s\;}y)/\theta \). \(\Box\)

Corolario 97 Sea \((L,\mathsf{s},\mathsf{i})\) un reticulado en el cual hay un elemento maximo \(1\) (resp. minimo \(0\)). Entonces si \(\theta \) es una congruencia sobre \((L,\mathsf{s},\mathsf{i})\), \(1/\theta \) (resp. \(0/\theta \)) es un elemento maximo (resp. minimo) de \((L/\theta ,\mathsf{\tilde{s}},\mathsf{ \tilde{\imath}})\).
Prueba: Ya que \(1\theta (x\mathsf{\;s\;}1)\), para cada \(x\in L\), tenemos que \( x/\theta \tilde{\leq}1/\theta \), para cada \(x\in L\). \(\Box\)

Dada una funcion \(F:A\rightarrow B\), llamaremos nucleo de \(F\) a la relacion binaria

\(\displaystyle \{(a,b)\in A^{2}:F(a)=F(b)\} \)

Con \(\ker F\) denotaremos al nucleo de \(F\). Es facil de chequear que \(\ker F\) es una relacion de equivalencia. Cuando \(F\) es un homomorfismo de algebras podemos decir aun mas.
Lema 98 Si \(F:(L,\mathsf{s},\mathsf{i})\rightarrow (L^{\prime },\mathsf{s}^{\prime }, \mathsf{i}^{\prime })\) es un homomorfismo de reticulados, entonces \(\ker F\) es una congruencia sobre \((L,\mathsf{s},\mathsf{i})\).
Prueba: Dejamos al lector ver que \(\ker F\) es una relacion de equivalencia. Supongamos \(x\ker Fx^{\prime }\) y \(y\ker Fy^{\prime }\). Entonces

\(\displaystyle F(x\mathsf{\;s\;}y)=F(x)\mathsf{\;s^{\prime }\;}F(y)=F(x^{\prime })\mathsf{ \;s^{\prime }\;}F(y^{\prime })=F(x^{\prime }\mathsf{\;s\;}y^{\prime }) \)

lo cual nos dice que \((x\mathsf{\;s\;}y)\ker F(x^{\prime }\mathsf{\;s\;} y^{\prime })\). En forma similar tenemos que \((x\mathsf{\;i\;}y)\ker F(x^{\prime }\mathsf{\;i\;}y^{\prime })\). \(\Box\)
Si \(R\) es una relacion de equivalencia sobre un conjunto \(A\), entonces con \( \pi _{R}\) denotaremos la funcion

\(\displaystyle \begin{array}{ccc} A & \rightarrow & A/R \\ a & \rightarrow & a/R \end{array} \)

A \(\pi _{R}\) la llamaremos la proyeccion canonica. Ya vimos que el nucleo de un homomorfismo es una congruencia. El siguiente lema muestra que toda congruencia es el nucleo de un homomorfismo.
Lema 99 Sea \((L,\mathsf{s},\mathsf{i})\) un reticulado y sea \(\theta \) una congruencia sobre \((L,\mathsf{s},\mathsf{i})\). Entonces \(\pi _{\theta }\) es un homomorfismo de \((L,\mathsf{s},\mathsf{i})\) en \((L/\theta ,\mathsf{\tilde{ s}},\mathsf{\tilde{\imath}})\). Ademas \(\ker \pi _{\theta }=\theta \).
Prueba: Sean \(x,y\in L\). Tenemos que

\(\displaystyle \pi _{\theta }(x\mathsf{\;s\;}y)=(x\mathsf{\;s\;}y)/\theta =x/\theta \mathsf{ \;\tilde{s}\;}y/\theta =\pi _{\theta }(x)\mathsf{\;\tilde{s}\;}\pi _{\theta }(y) \)

por lo cual \(\pi _{\theta }\) preserva la operacion supremo. Para la operacion infimo es similar. \(\Box\)
6.3. Reticulados acotados

Una \(5\)-upla \((L,\mathsf{s},\mathsf{i},0,1)\), donde \(L\) es un conjunto no vacio, \(\mathsf{s}\) e\(\;\mathsf{i\;}\)son operaciones binarias sobre \(L\) y \(0,1\in L\), sera llamada un reticulado acotado si \((L, \mathsf{s},\mathsf{i})\) es un reticulado y ademas se cumplen las siguientes identidades

(I8) \(0\mathsf{\;s\;}x=x\), para cada \(x\in L\)
(I9) \(x\mathsf{\;s\;}1=1\), para cada \(x\in L\).
Notese que si \((P,\leq )\) es un poset el cual es un reticulado y en el cual hay un maximo \(1\) y un minimo \(0\), entonces \((P,\sup ,\inf ,0,1)\) es un reticulado acotado. Ademas en virtud del Teorema 92 todo reticulado acotado se obtiene de esta forma.

6.3.1. Subreticulados acotados.

Sea \((L,\mathsf{s},\mathsf{i},0,1)\) un reticulado acotado. Un conjunto no vacio \(S\subseteq L\) sera llamado subuniverso de \((L,\mathsf{s}, \mathsf{i},0,1)\) si \(0,1\in S\) y \(S\) es cerrado bajo las operaciones \( \mathsf{s}\) e \(\mathsf{i}\). Notese que en tal caso la estructura \((S,\mathsf{ s}\mathrm{\mid }_{S\times S},\mathsf{i}\mathrm{\mid }_{S\times S},0,1)\) es un reticulado acotado. Diremos que \((S,\mathsf{s}\mathrm{\mid }_{S\times S}, \mathsf{i}\mathrm{\mid }_{S\times S},0,1)\) es subreticulado acotado de \((L,\mathsf{s},\mathsf{i},0,1)\).

6.3.2. Homomorfismos de reticulados acotados.

Sean \((L,\mathsf{s},\mathsf{i},0,1)\) y \((L^{\prime },\mathsf{s}^{\prime }, \mathsf{i}^{\prime },0^{\prime },1^{\prime })\) reticulados acotados. Una funcion \(F:L\rightarrow L^{\prime }\) sera llamada un homomorfismo de \((L,\mathsf{s},\mathsf{i},0,1)\) en \((L^{\prime },\mathsf{s} ^{\prime },\mathsf{i}^{\prime },0^{\prime },1^{\prime })\) si para todo \( x,y\in L\) se cumple que

\(\displaystyle \begin{array}{rcl} F(x\mathsf{\;s\;}y) & =& F(x)\;\mathsf{s}^{\prime }\ F(y) \\ F(x\mathsf{\;i\;}y) & =& F(x)\;\mathsf{i}^{\prime }\ F(y) \\ F(0) & =& 0^{\prime } \\ F(1) & =& 1^{\prime } \end{array} \)

Un homomorfismo de \((L,\mathsf{s},\mathsf{i},0,1)\) en \((L^{\prime },\mathsf{s}^{\prime },\mathsf{i}^{\prime },0^{\prime },1^{\prime })\) sera llamado isomorfismo cuando sea biyectivo y su inversa sea tambien un homomorfismo. Escribiremos \((L,\mathsf{s},\mathsf{i},0,1)\cong (L^{\prime },\mathsf{s}^{\prime },\mathsf{i}^{\prime },0^{\prime },1^{\prime })\) cuando exista un isomorfismo de \((L,\mathsf{s},\mathsf{i},0,1)\) en \( (L^{\prime },\mathsf{s}^{\prime },\mathsf{i}^{\prime },0^{\prime },1^{\prime })\). Escribiremos \(F:(L,\mathsf{s},\mathsf{i},0,1)\rightarrow (L^{\prime }, \mathsf{s}^{\prime },\mathsf{i}^{\prime },0^{\prime },1^{\prime })\) cuando \( F \) sea un homomorfismo de \((L,\mathsf{s},\mathsf{i,}0,1)\) en \( (L^{\prime },\mathsf{s}^{\prime },\mathsf{i}^{\prime },0^{\prime },1^{\prime })\).
Lema 100 Si \(F:(L,\mathsf{s},\mathsf{i},0,1)\rightarrow (L^{\prime },\mathsf{s} ^{\prime },\mathsf{i}^{\prime },0^{\prime },1^{\prime })\) un homomorfismo biyectivo, entonces \(F\) es un isomorfismo
Prueba: Similar a la prueba del Lemma 93. \(\Box\)

Lema 101 Si \(F:(L,\mathsf{s},\mathsf{i},0,1)\rightarrow (L^{\prime },\mathsf{s} ^{\prime },\mathsf{i}^{\prime },0^{\prime },1^{\prime })\) es un homomorfismo, entonces \(I_{F}\) es un subuniverso de \((L^{\prime },\mathsf{s} ^{\prime },\mathsf{i}^{\prime },0^{\prime },1^{\prime })\).
Prueba: Ya que \(F\) es un homomorfismo de \((L,\mathsf{s},\mathsf{i})\) en \( (L^{\prime },\mathsf{s}^{\prime },\mathsf{i}^{\prime })\) tenemos que \(I_{F}\) es subuniverso de \((L^{\prime },\mathsf{s}^{\prime },\mathsf{i}^{\prime })\) lo cual ya que \(0^{\prime },1^{\prime }\in I_{F}\) implica que \(I_{F}\) es un subuniverso de \((L^{\prime },\mathsf{s}^{\prime },\mathsf{i}^{\prime },0^{\prime },1^{\prime })\). \(\Box\)

6.3.3. Congruencias de reticulados acotados.

Sea \((L,\mathsf{s},\mathsf{i},0,1)\) un reticulado acotado. Una congruencia sobre \((L,\mathsf{s},\mathsf{i},0,1)\) sera una relacion de equivalencia \(\theta \) la cual sea una congruencia sobre \((L,\mathsf{s}, \mathsf{i})\). Tenemos definidas sobre \(L/\theta \) dos operaciones binarias \( \mathsf{\tilde{s}}\) e \(\mathsf{\tilde{\imath}}\), de la siguiente manera:

\(\displaystyle \begin{array}{rcl} x/\theta \mathsf{\tilde{s}}y/\theta & =& (x\mathsf{\;s\;}y)/\theta \\ x/\theta \mathsf{\tilde{\imath}}y/\theta & =& (x\mathsf{\;i\;}y)/\theta \end{array} \)

En forma analoga a lo hecho para reticulados, podemos probar facilmente los siguientes resultados.
Lema 102 Si \(F:(L,\mathsf{s},\mathsf{i},0,1)\rightarrow (L^{\prime },\mathsf{s} ^{\prime },\mathsf{i}^{\prime },0^{\prime },1^{\prime })\) es un homomorfismo de reticulados acotados, entonces \(\ker F\) es una congruencia sobre \((L, \mathsf{s},\mathsf{i},0,1)\).
Lema 103 Sea \((L,\mathsf{s},\mathsf{i},0,1)\) un reticulado acotado y \(\theta \) una congruencia sobre \((L,\mathsf{s},\mathsf{i},0,1)\).
(a) \((L/\theta ,\mathsf{\tilde{s}},\mathsf{\tilde{\imath}},0/\theta ,1/\theta )\) es un reticulado acotado.
(b) \(\pi _{\theta }\) es un homomorfismo de \((L,\mathsf{s},\mathsf{i} ,0,1)\) en \((L/\theta ,\mathsf{\tilde{s}},\mathsf{\tilde{\imath}},0/\theta ,1/\theta )\) cuyo nucleo es \(\theta \).
6.4. Reticulados complementados

Sea \((L,\mathsf{s},\mathsf{i},0,1)\) un reticulado acotado. Dado \( a\in L\), diremos que \(a\) es complementado cuando exista un elemento \(b\in B\) (llamado complemento de a) tal que:

\(\displaystyle \begin{array}{rcl} a\;\mathsf{s\;}b & =& 1 \\ a\;\mathsf{i\;}b & =& 0 \end{array} \)

Notese que dicho elemento \(b\) puede no ser unico, es decir \(a\) puede tener varios complementos. Una \(6\)-upla \((L,\mathsf{s},\mathsf{i},^{c},0,1)\), donde \(L\) es un conjunto no vacio, \(\mathsf{s}\) e \(\mathsf{i}\) son operaciones binarias sobre \(L\), \(^{c}\) es una operacion unaria sobre \(L\) y \( 0,1\in L\), sera llamada un reticulado complementado si \((L,\mathsf{s },\mathsf{i},0,1)\) es un reticulado acotado y ademas
(I10) \(x\mathsf{\;s\;}x^{c}=1\), para cada \(x\in L\)
(I11) \(x\mathsf{\;i\;}x^{c}=0\), para cada \(x\in L\)
Dado un reticulado acotado \((L,\mathsf{s},\mathsf{i},0,1)\) puede haber mas de una operacion unaria \(g\) tal que \((L,\mathsf{s},\mathsf{i} ,g,0,1)\) resulte un reticulado complementado. Intente dar un ejemplo de esta situacion con \(\left\vert L\right\vert =5\).

Notese que si tenemos un poset \((P,\leq )\) y una funcion \(g:P\rightarrow P\), tales que \((P,\leq )\) es un reticulado en el cual hay un maximo \(1\) y un minimo \(0\) y la funcion \(g\) cumple

\(\displaystyle \begin{array}{rcl} \sup \{x,g(x)\} & =& 1 \\ \inf \{x,g(x)\} & =& 0, \end{array} \)

entonces \((P,\sup ,\inf ,g,0,1)\) es un reticulado complementado. Ademas en virtud del Teorema 92 todo reticulado complementado se obtiene de esta forma.
6.4.1. Subreticulados complementados.

Sea \((L,\mathsf{s},\mathsf{i},^{c},0,1)\) un reticulado complementado. Un conjunto no vacio \(S\subseteq L\) sera llamado subuniverso de \((L, \mathsf{s},\mathsf{i},^{c},0,1)\) si \(S\) es un subuniverso de \((L,\mathsf{s}, \mathsf{i},0,1)\) y ademas es cerrado bajo la operacion \(^{c}\). Notese que en tal caso la estructura \((S,\mathsf{s}\mathrm{\mid }_{S\times S},\mathsf{i} \mathrm{\mid }_{S\times S},^{c}\mathrm{\mid }_{S},0,1)\) es un reticulado complementado. Diremos que \((S,\mathsf{s}\mathrm{\mid }_{S\times S},\mathsf{i }\mathrm{\mid }_{S\times S},^{c}\mathrm{\mid }_{S},0,1)\) es subreticulado complementado de \((L,\mathsf{s},\mathsf{i},^{c},0,1)\).

6.4.2. Homomorfismos de reticulados complementados.

Sean \((L,\mathsf{s},\mathsf{i},^{c},0,1)\) y \((L^{\prime },\mathsf{s}^{\prime },\mathsf{i}^{\prime },^{c^{\prime }},0^{\prime },1^{\prime })\) reticulados complementados. Una funcion \(F:L\rightarrow L^{\prime }\) sera llamada un homomorfismo de \((L,\mathsf{s},\mathsf{i},^{c},0,1)\) en \( (L^{\prime },\mathsf{s}^{\prime },\mathsf{i}^{\prime },^{c^{\prime }},0^{\prime },1^{\prime })\) si para todo \(x,y\in L\) se cumple que

\(\displaystyle \begin{array}{rcl} F(x\mathsf{\;s\;}y) & =& F(x)\;\mathsf{s}^{\prime }\ F(y) \\ F(x\mathsf{\;i\;}y) & =& F(x)\;\mathsf{i}^{\prime }\ F(y) \\ F(x^{c}) & =& F(x)^{c^{\prime }} \\ F(0) & =& 0^{\prime } \\ F(1) & =& 1^{\prime } \end{array} \)

Un homomorfismo de \((L,\mathsf{s},\mathsf{i},^{c},0,1)\) en \( (L^{\prime },\mathsf{s}^{\prime },\mathsf{i}^{\prime },^{c^{\prime }},0^{\prime },1^{\prime })\) sera llamado isomorfismo cuando sea biyectivo y su inversa sea un homomorfismo. Como es usual usaremos el simbolo \(\cong \) para denotar la relacion de isomorfismo. Escribiremos \(F:(L, \mathsf{s},\mathsf{i},^{c},0,1)\rightarrow (L^{\prime },\mathsf{s}^{\prime }\) , \(\mathsf{i}^{\prime },^{c^{\prime }},0^{\prime },1^{\prime })\) cuando \(F\) sea un homomorfismo de \((L,\mathsf{s},\mathsf{i},^{c},0,1)\) en \( (L^{\prime },\mathsf{s}^{\prime },\mathsf{i}^{\prime },^{c^{\prime }},0^{\prime },1^{\prime })\). Dejamos al lector la prueba de los siguientes lemas.
Lema 104 Si \(F:(L,\mathsf{s},\mathsf{i},^{c},0,1)\rightarrow (L^{\prime },\mathsf{s} ^{\prime },\mathsf{i}^{\prime },^{c^{\prime }},0^{\prime },1^{\prime })\) un homomorfismo biyectivo, entonces \(F\) es un isomorfismo
Lema 105 Si \(F:(L,\mathsf{s},\mathsf{i},^{c},0,1)\rightarrow (L^{\prime },\mathsf{s} ^{\prime }\),\(\mathsf{i}^{\prime },^{c^{\prime }},0^{\prime },1^{\prime })\) es un homomorfismo, entonces \(I_{F}\) es un subuniverso de \((L^{\prime }, \mathsf{s}^{\prime }\),\(\mathsf{i}^{\prime },^{c^{\prime }},0^{\prime },1^{\prime })\).
6.4.3. Congruencias de reticulados complementados.

Sea \((L,\mathsf{s},\mathsf{i},^{c},0,1)\) un reticulado complementado. Una congruencia sobre \((L,\mathsf{s},\mathsf{i},^{c},0,1)\) sera una relacion de equivalencia sobre \(L\) la cual cumpla:

(1) \(\theta \) es una congruencia sobre \((L,\mathsf{s},\mathsf{i},0,1)\)
(2) \(x/\theta =y/\theta \) implica \(x^{c}/\theta =y^{c}/\theta \)
Las condiciones anteriores nos permiten definir sobre \(L/\theta \) dos operaciones binarias \(\mathsf{\tilde{s}}\) e \(\mathsf{\tilde{\imath}}\), y una operacion unaria \(^{\tilde{c}}\) de la siguiente manera:

\(\displaystyle \begin{array}{rcl} x/\theta \mathsf{\tilde{s}\;}y/\theta & =& (x\mathsf{\;s\;}y)/\theta \\ x/\theta \mathsf{\;\tilde{\imath}\;}y/\theta & =& (x\mathsf{\;i\;}y)/\theta \\ (x/\theta )^{\tilde{c}} & =& x^{c}/\theta \end{array} \)

Tal como era de esperar tenemos entonces
Lema 106 Si \(F:(L,\mathsf{s},\mathsf{i},^{c},0,1)\rightarrow (L^{\prime },\mathsf{s} ^{\prime },\mathsf{i}^{\prime },^{c^{\prime }},0^{\prime },1^{\prime })\) es un homomorfismo de reticulados complementados, entonces \(\ker F\) es una congruencia sobre \((L,\mathsf{s},\mathsf{i},^{c},0,1)\)
Lema 107 Sea \((L,\mathsf{s},\mathsf{i},^{c},0,1)\) un reticulado complementado y sea \( \theta \) una congruencia sobre \((L,\mathsf{s},\mathsf{i},^{c},0,1)\).
(a) \((L/\theta ,\mathsf{\tilde{s}},\mathsf{\tilde{\imath}},^{\tilde{c} },0/\theta ,1/\theta )\) es un reticulado complementado.
(b) \(\pi _{\theta }\) es un homomorfismo de \((L,\mathsf{s},\mathsf{i} ,^{c},0,1)\) en \((L/\theta ,\mathsf{\tilde{s}},\mathsf{\tilde{\imath}},^{ \tilde{c}},0/\theta ,1/\theta )\) cuyo nucleo es \(\theta \).
