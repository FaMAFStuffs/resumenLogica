6.5. El teorema del filtro primo

En esta seccion probaremos un teorema que tiene importantes aplicaciones en logica. Primero un lema que nos conducira a una definicion basica.

Lema 108 Sea \((L,\mathsf{s},\mathsf{i})\) un reticulado. Son equivalentes:
(1) \(x\mathsf{\;i\;}(y\;\mathsf{s}\;z)=(x\mathsf{\;i\;}y)\;\mathsf{s} \;(x\mathsf{\;i\;}z)\), cualesquiera sean \(x,y,z\in L\)
(2) \(x\;\mathsf{s}\;(y\mathsf{\;i\;}z)=(x\mathsf{\;s\;}y)\mathsf{\;i\; }(x\;\mathsf{s}\;z)\), cualesquiera sean \(x,y,z\in L\).
Prueba: (1)\(\Rightarrow \)(2). Notese que

\(\displaystyle \begin{array}{rcl} (x\mathsf{\;s\;}y)\mathsf{\;i\;}(x\;\mathsf{s}\;z) & =& ((x\mathsf{\;s\;}y) \mathsf{\;i\;}x)\;\mathsf{s}\;((x\mathsf{\;s\;}y)\mathsf{\;i\;}z) \\ & =& (x\;\mathsf{s}\;(z\mathsf{\;i\;}(x\mathsf{\;s\;}y)) \\ & =& (x\;\mathsf{s}\;((z\;\mathsf{i\;}x)\mathsf{\;s\;}(z\;\mathsf{i\;}y)) \\ & =& (x\;\mathsf{s}\;(z\;\mathsf{i\;}x))\mathsf{\;s\;}(z\;\mathsf{i\;}y) \\ & =& x\mathsf{\;s\;}(z\;\mathsf{i\;}y) \\ & =& x\mathsf{\ s\ }(y\ \mathsf{i\ }z) \end{array} \)

(2)\(\Rightarrow \)(1) es similar. \(\Box\)
Un reticulado se llamara distributivo cuando cumpla alguna de las dos propiedades equivalentes del lema anterior.

Lema 109 Si \((L,\mathsf{s},\mathsf{i},0,1)\) un reticulado acotado y distributivo, entonces todo elemento tiene a lo sumo un complemento.
Prueba: Supongamos \(x\in L\) tiene complementos \(y,z\). Se tiene

\(\displaystyle y=y\;\mathsf{i\;}1=y\;\mathsf{i\;}(x\;\mathsf{s\;}z)=(y\;\mathsf{i\;}x)\; \mathsf{s\;}(y\;\mathsf{i\;}z)=0\;\mathsf{s\;}(y\;\mathsf{i\;}z)=y\;\mathsf{ i\;}z, \)

por lo cual \(y\leq z\). En forma analoga se muestra que \(z\leq y\). \(\Box\)
Ejercicio: Use la prueba del lema anterior para hacer un algoritmo el cual tome de entrada un reticulado acotado \((L,\mathsf{s},\mathsf{i},0,1)\) y elementos \(x,y,z\in L\) tales que \(y\neq z\) son complementos de \(x\), y de como salida elementos \(a,b,c\) tales que \(a\mathsf{\;i\;}(b\;\mathsf{s} \;c)\neq (a\mathsf{\;i\;}b)\;\mathsf{s}\;(a\mathsf{\;i\;}c)\)

Un filtro de un reticulado \((L,\mathsf{s},\mathsf{i})\) sera un subconjunto \(F\subseteq L\) tal que:

(1) \(F\neq \varnothing \)
(2) \(x,y\in F\Rightarrow x\;\mathsf{i\;}y\in F\)
(3) \(x\in F\) y \(x\leq y\Rightarrow y\in F\).
Dado un conjunto \(S\subseteq L\), denotemos con \([S)\) el siguiente conjunto

\(\displaystyle \{y\in L:y\geq s_{1}\;\mathsf{i\;}...\;\mathsf{i\;}s_{n}\text{, para algunos }s_{1},...,s_{n}\in S\text{, }n\geq 1\} \)

Lema 110 Si \(S\) es no vacio, entonces \([S)\) es un filtro. Mas aun si \(F\) es un filtro y \(F\supseteq S\), entonces \(F\supseteq \lbrack S)\).
Prueba: Ya que \(S\subseteq \lbrack S)\), tenemos que \([S)\neq \varnothing \). Claramente \([S)\) cumple la propiedad (3). Veamos cumple la (2). Si \(y\geq s_{1}\; \mathsf{i\;}s_{2}\;\mathsf{i\;}...\;\mathsf{i\;}s_{n}\) y \(z\geq t_{1}\; \mathsf{i\;}t_{2}\;\mathsf{i\;}\)...\(\;\mathsf{i\;}t_{m}\), con \( s_{1},s_{2},...,s_{n}\), \(t_{1},t_{2},...,t_{m}\in S\), entonces

\(\displaystyle y\;\mathsf{i\;}z\geq s_{1}\;\mathsf{i\;}s_{2}\;\mathsf{i\;}...\;\mathsf{i\;} s_{n}\;\mathsf{i\;}t_{1}\;\mathsf{i\;}t_{2}\;\mathsf{i\;}...\;\mathsf{i\;} t_{m}, \)

lo cual prueba (2). \(\Box\)
Llamaremos a \([S)\) el filtro generado por \(S\). Sea \( (P,\leq )\) un poset. Un subconjunto \(C\subseteq P\) sera llamado una cadena si para cada \(x,y\in C\), se tiene que \(x\leq y\) o \(y\leq x\). El siguiente resultado es una herramienta fundamental en el algebra moderna.

Lema 111 (Zorn) Sea \((P,\leq )\) un poset y supongamos cada cadena de \(P\) tiene una cota superior. Entonces hay un elemento maximal en \(P\).
Un filtro \(F\) de un reticulado \((L,\mathsf{s},\mathsf{i})\) sera llamado primo cuando se cumplan:

(1) \(F\neq L\)
(2) \(x\;\mathsf{s\;}y\in F\Rightarrow x\in F\) o \(y\in F\).
Teorema 112 (Teorema del Filtro Primo) Sea \((L,\mathsf{s},\mathsf{i})\) un reticulado distributivo y \(F\) un filtro. Supongamos \(x_{0}\in L-F\). Entonces hay un filtro primo \(P\) tal que \(x_{0}\notin P\) y \(F\subseteq P\).
Prueba: Sea

\(\displaystyle \mathcal{F}=\{F_{1}:F_{1}\text{ es un filtro, }x_{0}\notin F_{1}\text{ y } F\subseteq F_{1}\}. \)

Notese que \(\mathcal{F}\neq \varnothing \), por lo cual \((\mathcal{F},\subseteq )\) es un poset. Veamos que cada cadena en \((\mathcal{F},\subseteq )\) tiene una cota superior. Sea \(C\) una cadena. Si \(C=\varnothing \), entonces cualquier elemento de \(\mathcal{F}\) es cota de \(C\). Supongamos entonces \(C\neq \varnothing \). Sea
\(\displaystyle G=\{x\in L:x\in F_{1},\text{para algun }F_{1}\in C\}. \)

Veamos que \(G\) es un filtro. Es claro que \(G\) es no vacio. Supongamos que \( x,y\in G\). Sean \(F_{1},F_{2}\in \mathcal{F}\) tales que \(x\in F_{1}\) y \(y\in F_{2}\). Si \(F_{1}\subseteq F_{2}\), entonces ya que \(F_{2}\) es un filtro tenemos que \(x\;\mathsf{i\;}y\in F_{2}\subseteq G\). Si \(F_{2}\subseteq F_{1}\) , entonces tenemos que \(x\;\mathsf{i\;}y\in F_{1}\subseteq G\). Ya que \(C\) es una cadena, tenemos que siempre \(x\;\mathsf{i\;}y\in G\). En forma analoga se prueba la propiedad restante por lo cual tenemos que \(G\) es un filtro. Ademas \(x_{0}\notin G\), por lo que \(G\in \mathcal{F}\) es cota superior de \(C\) . Por el lema de Zorn, \((\mathcal{F},\subseteq )\) tiene un elemento maximal \( P\). Veamos que \(P\) es un filtro primo. Supongamos \(x\;\mathsf{s\;}y\in P\) y \( x,y\notin P\). Entonces ya que \(P\) es maximal tenemos que
\(\displaystyle x_{0}\in \lbrack P\cup \{x\})\cap \lbrack P\cup \{y\}) \)

Ya que \(x_{0}\in \lbrack P\cup \{x\})\), tenemos que hay elementos \( p_{1},...,p_{n}\in P\), tales que
\(\displaystyle x_{0}\geq p_{1}\;\mathsf{i\;}...\;\mathsf{i\;}p_{n}\;\mathsf{i\;}x \)

Ya que \(x_{0}\in \lbrack P\cup \{y\})\), tenemos que hay elementos \( q_{1},...,q_{m}\in P\), tales que
\(\displaystyle x_{0}\geq q_{1}\;\mathsf{i\;}...\;\mathsf{i\;}q_{m}\;\mathsf{i\;}y \)

Si llamamos \(p\) al siguiente elemento de \(P\)
\(\displaystyle p_{1}\;\mathsf{i\;}...\;\mathsf{i\;}p_{n}\;\mathsf{i\;}q_{1}\;\mathsf{i\;} ...\;\mathsf{i\;}q_{m} \)

tenemos que
\(\displaystyle \begin{array}{rcl} x_{0} & \geq & p\;\mathsf{i\;}x \\ x_{0} & \geq & p\;\mathsf{i\;}y \end{array} \)

Se tiene que \(x_{0}\geq (p\;\mathsf{i\;}x)\;\mathsf{s\;}(p\;\mathsf{i\;} y)=p\;\mathsf{i\;}(x\;\mathsf{s\;}y)\in P\), lo cual es absurdo ya que \( x_{0}\notin P\). \(\Box\)
Corolario 113 Sea \((L,\mathsf{s},\mathsf{i},0,1)\) un reticulado acotado distributivo. Si \( \varnothing \neq S\subseteq L\) es tal que \(s_{1}\;\mathsf{i\;}s_{2}\;\mathsf{ i\;}...\;\mathsf{i\;}s_{n}\neq 0\), para cada \(s_{1},...,s_{n}\in S\), entonces hay un filtro primo que contiene a \(S\).
Prueba: Notese que \([S)\neq L\) por lo cual se puede aplicar el Teorema del filtro primo. \(\Box\)

Un Algebra de Boole sera un reticulado complementado y distributivo.

Lema 114 Sea \((B,\mathsf{s},\mathsf{i},^{c},0,1)\) un algebra de Boole. Entonces para un filtro \(F\subseteq B\) las siguientes son equivalentes:
(1) \(F\) es primo
(2) \(x\in F\) o \(x^{c}\in F\), para cada \(x\in B\).
Prueba: (1)\(\Rightarrow \)(2). Ya que \(x\;\mathsf{s\;}x^{c}=1\in F\), (2) se cumple si \(F\) es primo.

(2)\(\Rightarrow \)(1). Supongamos que \(x\;\mathsf{s\;}y\in F\) y que \(x\not\in F\). Entonces por (2), \(x^{c}\in F\) y por lo tanto tenemos que

\(\displaystyle y\geq x^{c}\;\mathsf{i\;}y=(x^{c}\;\mathsf{i\;}x)\;\mathsf{s\;}(x^{c}\; \mathsf{i\;}y)=x^{c}\;\mathsf{i\;}(x\;\mathsf{s\;}y)\in F, \)

lo cual dice que \(y\in F\). \(\Box\)
Lema 115 Sea \((B,\mathsf{s},\mathsf{i},^{c},0,1)\) un algebra de Boole. Supongamos que \(b\neq 0\) y \(a=\inf A\), con \(A\subseteq B\). Entonces si \(b\;\mathsf{i\;}a=0\) , existe un \(e\in A\) tal que \(b\;\mathsf{i\;}e^{c}\neq 0\).
Prueba: Supongamos que para cada \(e\in A\), tengamos que \(b\;\mathsf{i\;}e^{c}=0\). Entonces tenemos que para cada \(e\in A\),

\(\displaystyle b=b\;\mathsf{i\;}(e\;\mathsf{s\;}e^{c})=(b\;\mathsf{i\;}e)\;\mathsf{s\;}(b\; \mathsf{i\;}e^{c})=b\;\mathsf{i\;}e, \)

lo cual nos dice que \(b\) es cota inferior de \(A\). Pero entonces \(b\leq a\), por lo cual \(b=b\;\mathsf{i\;}a=0\), contradiciendo la hipotesis. \(\Box\)
Teorema 116 (Rasiova y Sikorski) Sea \((B,\mathsf{s},\mathsf{i},^{c},0,1)\) un algebra de Boole. Sea \(x\in B\), \(x\neq 0\). Supongamos que \(A_{1},A_{2},...\) son subconjuntos de \(B\) tales que existe \(\inf (A_{j})\), para cada \(j=1,2....\) Entonces hay un filtro primo \(P\) el cual cumple:
(a) \(x\in P\)
(b) \(P\supseteq A_{j}\Rightarrow P\ni \inf (A_{j})\), para cada \( j=1,2,....\)
Prueba: Sean \(a_{j}=\inf (A_{j})\), \(j=1,2,...\). Construiremos inductivamente una sucesion \(b_{0},b_{1},...\) de elementos de \(B\) tal que:

(1) \(b_{0}=x\)
(2) \(b_{0}\;\mathsf{i\;}\)...\(\;\mathsf{i\;}b_{n}\neq 0\), para cada \( n\geq 0\)
(3) \(b_{j}=a_{j}\) o \(b_{j}^{c}\in A_{j}\), para cada \(j\geq 1\).
Definamos \(b_{0}=x\). Supongamos ya definimos \(b_{0},...,b_{n}\), veamos como definir \(b_{n+1}\). Si \((b_{0}\;\mathsf{i\;}...\;\mathsf{i\;} b_{n})\;\mathsf{i\;}a_{n+1}\neq 0\), entonces definamos \(b_{n+1}=a_{n+1}\). Si \((b_{0}\;\mathsf{i\;}...\;\mathsf{i\;}b_{n})\;\mathsf{i\;}a_{n+1}=0\), entonces por el lema anterior, tenemos que hay un \(e\in A_{n+1}\) tal que \( (b_{0}\;\mathsf{i\;}...\;\mathsf{i\;}b_{n})\;\mathsf{i\;}e^{c}\neq 0\), lo cual nos permite definir \(b_{n+1}=e^{c}\).

Usando (2) se puede probar que el conjunto \(S=\{b_{0},b_{1},...\}\) satisface la hipotesis del primer corolario del Teorema del filtro primo, por lo cual hay un filtro primo \(P\) tal que \(\{b_{0},b_{1},...\}\subseteq P\). Es facil chequear que \(P\) satisface las propiedades (a) y (b). \(\Box\)

7. Terminos y formulas

Sea \(Var\) el siguiente conjunto de palabras del alfabeto \(\{\mathsf{X}, \mathit{0},\mathit{1},...,\mathit{9},\mathbf{0},\mathbf{1},...,\mathbf{9}\}\):

\(\displaystyle Var=\{\mathsf{X}\mathbf{1},...,\mathsf{X}\mathbf{9},\mathsf{X}\mathit{1} \mathbf{0},\mathsf{X}\mathit{1}\mathbf{1},...,\mathsf{X}\mathit{1}\mathbf{9}, \mathsf{X}\mathit{2}\mathbf{0},\mathsf{X}\mathit{2}\mathbf{1},...\} \)

Es decir el elemento \(n\)-esimo de \(Var\) es la palabra de la forma \(\mathsf{X} \alpha \) donde \(\alpha \) es el resultado de reemplazar en la palabra que denota \(n\) en notacion decimal, el ultimo numeral por su correspondiente numeral bold y los otros por sus correspondientes italicos. A los elementos de \(Var\) los llamaremos variables. Llamaremos \(x_{i}\) al \(i\)-esimo elemento de \(Var\), para cada \(i\in \mathbf{N}\)
Por un tipo (de primer orden) entenderemos una 4-upla \( \tau =(\mathcal{C},\mathcal{F},\mathcal{R},a)\) tal que:

(1) Hay alfabetos finitos \(\Sigma _{1}\), \(\Sigma _{2}\) y \(\Sigma _{3}\) tales:
\(\mathcal{C}\subseteq \Sigma _{1}^{+}\), \(\mathcal{F}\subseteq \Sigma _{2}^{+}\) y \(\mathcal{R}\subseteq \Sigma _{3}^{+}\)
\(\Sigma _{1}\), \(\Sigma _{2}\) y \(\Sigma _{3}\) son disjuntos de a pares.
\(\Sigma _{1}\cup \Sigma _{2}\cup \Sigma _{3}\) no contiene ningun simbolo de la lista
\(\forall \ \exists \;\lnot \;\vee \;\wedge \;\rightarrow \;\leftrightarrow \;(\;)\;,\;\equiv \mathsf{X\;}\mathit{0}\;\mathit{1\;}...\;\mathit{9}\; \mathbf{0}\;\mathbf{1}\ ...\;\mathbf{9}\)

(2) \(a:\mathcal{F}\cup \mathcal{R}\rightarrow \mathbf{N}\) es una funcion que a cada \(p\in \mathcal{F}\cup \mathcal{R}\) le asocia un numero natural \(a(p)\), llamado la aridad de \(p\).
(3) Ninguna palabra de \(\mathcal{C}\) (resp. \(\mathcal{F}\), \(\mathcal{R }\)) es subpalabra propia de otra palabra de \(\mathcal{C}\) (resp. \(\mathcal{F} \), \(\mathcal{R}\)).
A los elementos de \(\mathcal{C}\) (resp. \(\mathcal{F}\), \(\mathcal{R} \)) los llamaremos nombres de constante (resp. nombres de funcion, nombres de relacion) de tipo \(\tau \). Dado \(n\geq 1\), sean

\(\displaystyle \begin{array}{rcl} \mathcal{F}_{n} & =& \{f\in \mathcal{F}:a(f)=n\} \\ \mathcal{R}_{n} & =& \{r\in \mathcal{R}:a(r)=n\} \end{array} \)

Dado un tipo \(\tau \), definamos los conjuntos de palabras \(T_{k}^{\tau }\), con \(k\geq 0\), de la siguiente manera:
\(\displaystyle \begin{array}{rcl} T_{0}^{\tau } & =& Var\cup \mathcal{C} \\ T_{k+1}^{\tau } & =& T_{k}^{\tau }\cup \{f(t_{1},...,t_{n}):f\in \mathcal{F} _{n},n\geq 1,t_{1},...,t_{n}\in T_{k}^{\tau }\}. \end{array} \)

Sea
\(\displaystyle T^{\tau }=\bigcup_{k\geq 0}T_{k}^{\tau } \)

Los elementos de \(T^{\tau }\) seran llamados terminos de tipo \(\tau \) .
El siguiente lema es la herramienta basica para probar propiedades de los terminos.

Lema 117 Supongamos \(t\in T_{k}^{\tau }\), con \(k\geq 1\). Entonces ya sea \(t\in Var\cup \mathcal{C}\) o \(t=f(t_{1},...,t_{n})\), con \(f\in \mathcal{F }_{n}\), \(n\geq 1,\;t_{1},...,t_{n}\in T_{k-1}^{\tau }\).
Prueba: Por induccion en \(k\).

CASO \(k=1\): Es directo ya que por definicion

\(\displaystyle T_{1}^{\tau }=Var\cup \mathcal{C}\cup \{f(t_{1},...,t_{n}):f\in \mathcal{F} _{n},n\geq 1,t_{1},...,t_{n}\in T_{0}^{\tau }\}. \)

CASO \(k\Rightarrow k+1\): Sea \(t\in T_{k+1}^{\tau }\). Por definicion de \( T_{k+1}^{\tau }\) tenemos que \(t\in T_{k}^{\tau }\) o \(t=f(t_{1},...,t_{n})\) con \(f\in \mathcal{F}_{n}\), \(n\geq 1\) y \(t_{1},...,t_{n}\in T_{k}^{\tau }\). Si se da que \(t\in T_{k}^{\tau }\), entonces podemos aplicar hipotesis inductiva y usar que \(T_{k-1}^{\tau }\subseteq T_{k}^{\tau }\). Esto completa el caso. \(\Box\)

Algunos ejemplos de propiedades de los terminos las cuales se pueden probar facilmente usando el lema anterior son

- Si \(t\in T^{\tau }\) es tal que en \(t\) ocurre el simbolo \()\), entonces \(t=f(t_{1},...,t_{n})\) con \(f\in \mathcal{F}_{n}\), \(n\geq 1\) y \( t_{1},...,t_{n}\in T^{\tau }\).
- Ningun termino comienza con un simbolo del alfabeto \(\{\mathit{0}, \mathit{1},...,\mathit{9}\}\)
- Si \(t\in T^{\tau }\) comienza con \(\mathsf{X}\) entonces \(t\in Var\)
- Si \(t\in T^{\tau }\) y \(\left[ t\right] _{i}=)\), con \(i< \left\vert t\right\vert \), entonces \(\left[ t\right] _{i+1}=\) \(,\) o \(\left[ t\right] _{i+1}=\) \()\)
Definamos conjuntos \(Bal_{k}\), con \(k\geq 1\) de la siguiente manera:

\(\displaystyle \begin{array}{rcl} Bal_{1} & =& \{()\} \\ Bal_{k+1} & =& Bal_{k}\cup \{(b_{1}...b_{n}):b_{1},...,b_{n}\in Bal_{k},n\geq 1\}. \end{array} \)

Sea
\(\displaystyle Bal=\bigcup_{k\geq 1}Bal_{k} \)

Lema 118 Sea \(b\in Bal\). Se tiene:
(1) \(\left\vert b\right\vert _{(}-\left\vert b\right\vert _{)}=0\)
(2) Si \(x\) es tramo inicial propio de \(b\), entonces \(\left\vert x\right\vert _{(}-\left\vert x\right\vert _{)} >0\)
(3) Si \(x\) es tramo final propio de \(b\), entonces \(\left\vert x\right\vert _{(}-\left\vert x\right\vert _{)}< 0\)
Prueba: Probaremos por induccion en \(k\), que valen (1), (2) y (3) para cada \(b\in Bal_{k}\). El caso \(k=1\) es trivial. Supongamos \(b\in Bal_{k+1}\). Si \(b\in Bal_{k}\), se aplica directamente HI. Supongamos entonces que \( b=(b_{1}...b_{n})\), con \(b_{1},...,b_{n}\in Bal_{k}\), \(n\geq 1\). Por HI, \( b_{1},...,b_{n}\) cumplen (1) por lo cual \(b\) cumple (1). Veamos que \(b\) cumple (2). Sea \(x\) un tramo inicial propio de \(b\). Notese que \(x\) es de la forma \(x=(b_{1}...b_{i}x_{1}\) con \(0\leq i\leq n-1\) y \(x_{1}\) un tramo inicial de \(b_{i+1}\) (en el caso \(i=0\) interpretamos \(b_{1}...b_{i}= \varepsilon )\). Pero entonces ya que

\(\displaystyle \left\vert x\right\vert _{(}-\left\vert x\right\vert _{)}=1+\left( \sum_{j=1}^{i}\left\vert b_{j}\right\vert _{(}-\left\vert b_{j}\right\vert _{)}\right) +\left\vert x_{1}\right\vert _{(}-\left\vert x_{1}\right\vert _{)} \)

tenemos que por HI, se da que \(\left\vert x\right\vert _{(}-\left\vert x\right\vert _{)} >0\). En forma analoga se puede ver que \(b\) cumple (3). \(\Box\)
« Previous
1
2
3
4
5
6
7
8
9
10
11
12
13
14
15
16
17
18
19
20
21
22
23
24
25
26
27
28
29
30
» Next
×
Lenguaje \(\mathcal{S}^{\Sigma }\)

Entorno para trabajar con el lenguaje \(\mathcal{S}^{\Sigma }\) creado por Gabriel Cerceau:

Descargar!
Close
