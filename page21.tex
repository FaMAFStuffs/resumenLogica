Gran Logico

Gran Lógico
Lenguaje \(\mathcal{S}^{\Sigma }\)
Apunte
Contacto
Login
« Previous
1
2
3
4
5
6
7
8
9
10
11
12
13
14
15
16
17
18
19
20
21
22
23
24
25
26
27
28
29
30
» Next
8. Estructuras

Sea \(A\) un conjunto. Por una operacion \(n\)-aria sobre \(A\) entenderemos una funcion cuyo dominio es \(A^{n}\) y cuya imagen esta contenida en \(A.\) Por una relacion \(n\)-aria sobre \(A\) entenderemos un subconjunto de \(A^{n}\).

Sea \(\tau \) un tipo. Una estructura o modelo de tipo \(\tau \) sera un par \(\mathbf{A}=(A,i)\) tal que:

(1) \(A\) es un conjunto no vacio llamado el universo de \( \mathbf{A}.\)
(2) \(i\) es una funcion con dominio \(\mathcal{C}\cup \mathcal{F}\cup \mathcal{R},\) tal que:
(a) para cada \(c\in \mathcal{C}\), \(i(c)\) es un elemento de \(A\)
(b) para cada \(f\in \mathcal{F}_{n}\), \(i(f)\) es una operacion \(n\) -aria sobre \(A\)
(c) para cada \(r\in \mathcal{R}_{n}\), \(i(r)\) es una relacion \(n\)-aria sobre \(A.\)
Sea \(\mathbf{A}=(A,i)\) una estructura de tipo \(\tau \). Una asignacion de \(\mathbf{A}\) sera un elemento de \(A^{\mathbf{N}}=\{\) sucesiones infinitas de elementos de \(A\}\). Si \(\vec{a}=(a_{1},a_{2},...)\) es una asignacion, entonces diremos que \(a_{j}\) es el valor que \( \vec{a}\) le asigna a la variable \(x_{j}\).

Dada una estructura \(\mathbf{A}\) de tipo \(\tau \), un termino \(t\in T^{\tau }\) y una asignacion \(\vec{a}=(a_{1},a_{2},...)\in A^{\mathbf{N}}\) definamos recursivamente \(t^{\mathbf{A}}[\vec{a}]\) de la siguiente manera

(1) Si \(t=x_{i}\in Var\), entonces \(t^{\mathbf{A}}[\vec{a}]=a_{i}\)
(2) Si \(t=c\in \mathcal{C}\), entonces \(t^{\mathbf{A}}[\vec{a}]=i(c)\)
(3) Si \(t=f(t_{1},...,t_{n})\), con \(f\in \mathcal{F}_{n},\;n\geq 1\) y \(t_{1},...,t_{n}\in T^{\tau }\), entonces \(t^{\mathbf{A}}[\vec{a} ]=i(f)(t_{1}^{\mathbf{A}}[\vec{a}],...,t_{n}^{\mathbf{A}}[\vec{a}])\)
El elemento \(t^{\mathbf{A}}[\vec{a}]\) sera llamado el valor de \(t\) en la estructura \(\mathbf{A}\) para la asignacion \(\vec{a}\).

Lema 131 Sea \(\mathbf{A}\) una estructura de tipo \(\tau \) y sea \(t\in T^{\tau }\). Supongamos que \(\vec{a},\vec{b}\) son asignaciones tales que \(a_{i}=b_{i},\) cada vez que \(x_{i}\) ocurra en \(t\). Entonces \(t^{ \mathbf{A}}[\vec{a}]=t^{\mathbf{A}}[\vec{b}]\).
Prueba: Sea

- Teo\(_{k}\): El lema vale para \(t\in T_{k}^{\tau }\).
Teo\(_{0}\) es facil de probar. Veamos Teo\(_{k}\Rightarrow \)Teo\(_{k+1}\). Supongamos \(t\in T_{k+1}^{\tau }-T_{k}^{\tau }\) y sean \(\vec{a},\vec{b}\) asignaciones tales que \(a_{i}=b_{i},\) cada vez que \(x_{i}\) ocurra en \(t\). Notese que \(t=f(t_{1},...,t_{n})\), con \(f\in \mathcal{F}_{n},\;n\geq 1\) y \( t_{1},...,t_{n}\in T^{\tau }\). Notese que para cada \(j=1,...,n\), tenemos que \(a_{i}=b_{i},\) cada vez que \(x_{i}\) ocurra en \(t_{j}\), lo cual por Teo\(_{k}\) nos dice que

\(\displaystyle t_{j}^{\mathbf{A}}[\vec{a}]=t_{j}^{\mathbf{A}}[\vec{b}]\text{, }j=1,...,n \)

Se tiene entonces que
\(\displaystyle \begin{array}{ccl} t^{\mathbf{A}}[\vec{a}] & = & i(f)(t_{1}^{\mathbf{A}}[\vec{a}],...,t_{n}^{ \mathbf{A}}[\vec{a}])\text{ (por def de }t^{\mathbf{A}}[\vec{a}]\text{)} \\ & = & i(f)(t_{1}^{\mathbf{A}}[\vec{b}],...,t_{n}^{\mathbf{A}}[\vec{b}]) \\ & = & t^{\mathbf{A}}[\vec{b}]\text{ (por def de }t^{\mathbf{A}}[\vec{a}] \text{)} \end{array} \)

\(\Box\)
Dada una asignacion \(\vec{a}\in A^{\mathbf{N}}\) y \(a\in A,\) con \(\downarrow _{i}^{a}(\vec{a})\) denotaremos la asignacion que resulta de reemplazar en \( \vec{a}\) el \(i\)-esimo elemento por \(a\).

A continuacion definiremos recursivamente la relacion \(\mathbf{A}\models \varphi \lbrack \vec{a}]\), donde \(\mathbf{A}\) es una estructura de tipo \( \tau \), \(\vec{a}\) es una asignacion y \(\varphi \in F^{\tau }\). Escribiremos \( \mathbf{A}\not\models \varphi \lbrack \vec{a}]\) para expresar que no se da \( \mathbf{A}\models \varphi \lbrack \vec{a}]\).

(1) Si \(\varphi =(t\equiv s),\) entonces
- \(\mathbf{A}\models \varphi \lbrack \vec{a}]\) si y solo si \(t^{A}[ \vec{a}]=s^{A}[\vec{a}]\)
(2) Si \(\varphi =r(t_{1},...,t_{m})\), entonces
- \(\mathbf{A}\models \varphi \lbrack \vec{a}]\) si y solo si \( (t_{1}^{A}[\vec{a}],...,t_{m}^{A}[\vec{a}])\in i(r)\)
(3) Si \(\varphi =(\varphi _{1}\wedge \varphi _{2}),\) entonces
- \(\mathbf{A}\models \varphi \lbrack \vec{a}]\) si y solo si \(\mathbf{A }\models \varphi _{1}[\vec{a}]\) y \(\mathbf{A}\models \varphi _{2}[\vec{a}]\)
(4) Si \(\varphi =(\varphi _{1}\vee \varphi _{2}),\) entonces
- \(\mathbf{A}\models \varphi \lbrack \vec{a}]\) si y solo si \(\mathbf{A }\models \varphi _{1}[\vec{a}]\) o \(\mathbf{A}\models \varphi _{2}[\vec{a}]\)
(5) Si \(\varphi =(\varphi _{1}\rightarrow \varphi _{2}),\) entonces
- \(\mathbf{A}\models \varphi \lbrack \vec{a}]\) si y solo si \(\mathbf{A }\models \varphi _{2}[\vec{a}]\) o \(\mathbf{A}\not\models \varphi _{1}[\vec{a} ]\)
(6) Si \(\varphi =(\varphi _{1}\leftrightarrow \varphi _{2}),\) entonces
- \(\mathbf{A}\models \varphi \lbrack \vec{a}]\) si y solo si ya sea se dan \(\mathbf{A}\models \varphi _{1}[\vec{a}]\) y \(\mathbf{A}\models \varphi _{2}[\vec{a}]\) o se dan \(\mathbf{A}\not\models \varphi _{1}[\vec{a}]\) y \( \mathbf{A}\not\models \varphi _{2}[\vec{a}]\)
(7) Si \(\varphi =\lnot \varphi _{1},\) entonces
- \(\mathbf{A}\models \varphi \lbrack \vec{a}]\) si y solo si \(\mathbf{A }\not\models \varphi _{1}[\vec{a}]\)
(8) Si \(\varphi =\forall x_{i}\varphi _{1}\), entonces
- \(\mathbf{A}\models \varphi \lbrack \vec{a}]\) si y solo si para cada \(a\in A\), se da que \(\mathbf{A}\models \varphi _{1}[\downarrow _{i}^{a}(\vec{ a})]\)
(9) Si \(\varphi =\exists x_{i}\varphi _{1}\), entonces
- \(\mathbf{A}\models \varphi \lbrack \vec{a}]\) si y solo si hay un \( a\in A\) tal que \(\mathbf{A}\models \varphi _{1}[\downarrow _{i}^{a}(\vec{a} )] \)
Cuando se de \(\mathbf{A}\models \varphi \lbrack \vec{a}]\) diremos que la estructura \(\mathbf{A}\) satisface \(\varphi \) en la asignacion \(\vec{a}\) y en tal caso diremos que \(\varphi \) es verdadera en \(\mathbf{A}\) para la asignacion \(\vec{a}\) . Cuando no se de \(\mathbf{A}\models \varphi \lbrack \vec{a}]\) diremos que la estructura \(\mathbf{A}\) no satisface \(\varphi \) en la asignacion \(\vec{a}\) y en tal caso diremos que \(\varphi \) es falsa en \(\mathbf{A}\) para la asignacion \(\vec{a}\). Tambien hablaremos del valor de verdad de \(\varphi \) en \( \mathbf{A}\) para la asignacion \(\vec{a}\) el cual sera igual a \(1\) si se da \(\mathbf{A}\models \varphi \lbrack \vec{a}]\) y \(0\) en caso contrario.

8.1. Variables libres

Definamos recursivamente el predicado

\(\displaystyle "v\mathit{\ ocurre\ libremente\ en\ }\varphi \mathit{\ a\ partir\ de\ } i":\omega \times Var\times F^{\tau }\rightarrow \omega \)

de la siguiente manera:
(1) Si \(\varphi \) es atomica, entonces \(v\) ocurre libremente en \( \varphi \) a partir de \(i\) sii \(v\) ocurre en \(\varphi \) a partir de \(i\)
(2) Si \(\varphi =(\varphi _{1}\eta \varphi _{2})\), entonces \(v\) ocurre libremente en \(\varphi \) a partir de \(i\) sii se da alguna de las siguientes
(a) \(v\) ocurre libremente en \(\varphi _{1}\) a partir de \(i-1\)
(b) \(v\) ocurre libremente en \(\varphi _{2}\) a partir de \(i-\left\vert (\varphi _{1}\eta \right\vert \)
(3) Si \(\varphi =\lnot \varphi _{1}\), entonces \(v\) ocurre libremente en \(\varphi \) a partir de \(i\) sii \(v\) ocurre libremente en \(\varphi _{1}\) a partir de \(i-1\)
(4) Si \(\varphi =Qw\varphi _{1}\), entonces \(v\) ocurre libremente en \( \varphi \) a partir de \(i\) sii \(v\neq w\) y \(v\) ocurre libremente en \(\varphi _{1}\) a partir de \(i-\left\vert Qw\right\vert \)
Diremos que \("v\) ocurre acotadamente en \(\varphi \) a partir de \(i"\) cuando \(v\) ocurre en \(\varphi \) a partir de \(i\) y \(v\) no ocurre libremente en \(\varphi \) a partir de \(i\). Dada una formula \(\varphi \), sea

\(\displaystyle Li(\varphi )=\{v\in Var:\text{hay un }i\text{ tal que }v\text{ ocurre libremente en }\varphi \text{ a partir de }i\}\text{.} \)

Los elementos de \(Li(\varphi )\) seran llamados variables libres de \( \varphi \). Una sentencia sera una formula \(\varphi \) tal que \( Li(\varphi )=\varnothing \). Usaremos \(S^{\tau }\) para denotar el conjunto de las sentencias de tipo \(\tau \).

Lema 132
(a) \(Li((t\equiv s))=\{v\in Var:v\) ocurre en \(t\) o \(v\) ocurre en \( s\}. \)
(b) \(Li(r(t_{1},...,t_{n}))=\{v\in Var:v\) ocurre en algun \(t_{i}\}.\)
(c) \(Li(\lnot \varphi )=Li(\varphi )\)
(d) \(Li((\varphi \eta \psi ))=Li(\varphi )\cup Li(\psi ).\)
(e) \(Li(Qx_{j}\varphi )=Li(\varphi )-\{x_{j}\}.\)
Prueba: (a) y (b) son triviales de las definiciones, teniendo en cuenta que si una variable \(v\) ocurre en \((t\equiv s)\) (resp. en \(r(t_{1},...,t_{n})\)) entonces \(v\) ocurre en \(t\) o \(v\) ocurre en \(s\) (resp.\(v\) ocurre en algun \( t_{i}\))

(d) Supongamos \(v\in Li((\varphi \eta \psi ))\), entonces hay un \(i\) tal que \( v\) ocurre libremente en \((\varphi \eta \psi )\) a partir de \(i\). Por definicion tenemos que ya sea \(v\) ocurre libremente en \(\varphi \) a partir de \(i-1\) o \(v\) ocurre libremente en \(\psi \) a partir de \(i-\left\vert (\varphi \eta \right\vert \), con lo cual \(v\in Li(\varphi )\cup Li(\psi )\)

Supongamos ahora que \(v\in Li(\varphi )\cup Li(\psi )\). S.p.d.g. supongamos \( v\in Li(\psi )\). Por definicion tenemos que hay un \(i\) tal que \(v\) ocurre libremente en \(\psi \) a partir de \(i\). Pero notese que esto nos dice por definicion que \(v\) ocurre libremente en \((\varphi \eta \psi )\) a partir de \( i+\left\vert (\varphi \eta \right\vert \) con lo cual \(v\in Li((\varphi \eta \psi ))\).

(c) es similar a (d)

(e) Supongamos \(v\in Li(Qx_{j}\varphi )\), entonces hay un \(i\) tal que \(v\) ocurre libremente en \(Qx_{j}\varphi \) a partir de \(i\). Por definicion tenemos que \(v\neq x_{j}\) y \(v\) ocurre libremente en \(\varphi \) a partir de \( i-\left\vert Qx_{j}\right\vert \), con lo cual \(v\in Li(\varphi )-\{x_{j}\}\)

Supongamos ahora que \(v\in Li(\varphi )-\{x_{j}\}\). Por definicion tenemos que hay un \(i\) tal que \(v\) ocurre libremente en \(\varphi \) a partir de \(i\). Ya que \(v\neq x_{j}\) esto nos dice por definicion que \(v\) ocurre libremente en \(Qx_{j}\varphi \) a partir de \(i+\left\vert Qx_{j}\right\vert \), con lo cual \(v\in Li(Qx_{j}\varphi )\). \(\Box\)

Lema 133 Supongamos que \(\vec{a},\vec{b}\) son asignaciones tales que si \(x_{i}\in Li(\varphi ),\) entonces \(a_{i}=b_{i}.\) Entonces \( \mathbf{A}\models \varphi \lbrack \vec{a}]\) sii \(\mathbf{A}\models \varphi \lbrack \vec{b}]\)
Prueba: Probaremos por induccion en \(k\) que el lema vale para cada \(\varphi \in F_{k}^{\tau }.\) El caso \(k=0\) se desprende del Lema 131. Veamos que Teo\(_{k}\) implica Teo\(_{k+1}.\) Sea \(\varphi \in F_{k+1}^{\tau }-F_{k}^{\tau }.\) Hay varios casos:

CASO \(\varphi =(\varphi _{1}\wedge \varphi _{2})\).

Ya que \(Li(\varphi _{i})\subseteq Li(\varphi )\), \(i=1,2\), Teo\(_{k}\) nos dice que \(\mathbf{A}\models \varphi _{i}[\vec{a}]\) sii \(\mathbf{A} \models \varphi _{i}[\vec{b}]\), para \(i=1,2\). Se tiene entonces que

\(\displaystyle \begin{array}{l} \mathbf{A}\models \varphi \lbrack \vec{a}] \\ \ \ \Updownarrow \text{ (por (3) en la def de }\mathbf{A}\models \varphi \lbrack \vec{a}]\text{)} \\ \mathbf{A}\models \varphi _{1}[\vec{a}]\text{ y }\mathbf{A}\models \varphi _{2}[\vec{a}] \\ \ \ \Updownarrow \text{ (por Teo}_{k}\text{)} \\ \mathbf{A}\models \varphi _{1}[\vec{b}]\text{ y }\mathbf{A}\models \varphi _{2}[\vec{b}] \\ \ \ \Updownarrow \text{(por (3) en la def de }\mathbf{A}\models \varphi \lbrack \vec{a}]\text{)} \\ \mathbf{A}\models \varphi \lbrack \vec{b}] \end{array} \)

CASO \(\varphi =(\varphi _{1}\wedge \varphi _{2})\).

Es completamente similar al anterior.

CASO \(\varphi =\lnot \varphi _{1}.\)

Es completamente similar al anterior.

CASO \(\varphi =\forall x_{j}\varphi _{1}.\)

Supongamos \(\mathbf{A}\models \varphi \lbrack \vec{a}]\). Entonces por (8) en la def de \(\mathbf{A}\models \varphi \lbrack \vec{a}]\) se tiene que \(\mathbf{A}\models \varphi _{1}[\downarrow _{j}^{a}(\vec{a})]\), para todo \(a\in A\). Notese que \(\downarrow _{j}^{a}(\vec{a})\) y \(\downarrow _{j}^{a}(\vec{b})\) coinciden en toda \(x_{i}\) de \(x_{i}\in Li(\varphi _{1})\subseteq Li(\varphi _{1})\cup \{x_{j}\}\), con lo cual por Teo\(_{k}\) se tiene que \(\mathbf{A}\models \varphi _{1}[\downarrow _{j}^{a}(\vec{b})]\), para todo \(a\in A\), lo cual por (8) en la def de \(\mathbf{A}\models \varphi \lbrack \vec{a}]\) nos dice que \(\mathbf{A}\models \varphi \lbrack \vec{b}]\). La prueba de que \(\mathbf{A}\models \varphi \lbrack \vec{b}]\) implica que \( \mathbf{A}\models \varphi \lbrack \vec{a}]\) es similar.

CASO \(\varphi =\exists x_{j}\varphi _{1}\).

Es similar al anterior. \(\Box\)

Corolario 134 Si \(\varphi \) es una sentencia, entonces \(\mathbf{A}\models \varphi \lbrack \vec{a}]\) sii \(\mathbf{A}\models \varphi \lbrack \vec{b}]\), cualesquiera sean las asignaciones \(\vec{a},\vec{b}\).
En virtud del corolario anterior tenemos que el valor de verdad de una sentencia \(\varphi \) en una estructura dada \(\mathbf{A}\) para una asignacion \(\vec{a}\) no depende de \(\vec{a}\), es decir este valor es ya sea \(1\) para todas las asignaciones o \(0\) para todas las asignaciones. En el primer caso diremos que \(\varphi \) es verdadera en \(\mathbf{A}\) y en el segundo caso diremos que \(\varphi \) es falsa en \(\mathbf{A}\)

Una sentencia de tipo \(\tau \) sera llamada universalmente valida si es verdadera en cada modelo de tipo \(\tau \).

8.2. Equivalencia de formulas

Dadas \(\varphi ,\psi \in F^{\tau }\) diremos que \(\varphi \) y \(\psi \) son equivalentes cuando se de la siguiente condicion

- \(\mathbf{A}\models \varphi \lbrack \vec{a}]\) si y solo si \(\mathbf{A }\models \psi \lbrack \vec{a}]\), para cada modelo de tipo \(\tau \), \(\mathbf{A }\) y cada \(\vec{a}\in A^{\mathbf{N}}\)
Escribiremos \(\varphi \thicksim \psi \) cuando \(\varphi \) y \(\psi \) sean equivalentes. Notese que \(\thicksim \) es una relacion de equivalencia.

Lema 135
(a) Si \(Li(\varphi )\cup Li(\psi )\subseteq \{x_{i_{1}},...,x_{i_{n}}\},\) entonces \(\varphi \thicksim \psi \) si y solo si la sentencia \(\forall x_{i_{1}}...\forall x_{i_{n}}(\varphi \leftrightarrow \psi )\) es universalmente valida.
(b) Si \(\varphi _{i}\thicksim \psi _{i},\) \(i=1,2,\) entonces \(\lnot \varphi _{1}\thicksim \lnot \psi _{1},\) \((\varphi _{1}\eta \varphi _{2})\thicksim (\psi _{1}\eta \psi _{2})\) y \(Qv\varphi _{1}\thicksim Qv\psi _{1}.\)
(c) Si \(\varphi \thicksim \psi \) y \(\alpha ^{\prime }\) es el resultado de reemplazar en una formula \(\alpha \) algunas (posiblemente \(0\)) ocurrencias de \(\varphi \) por \(\psi \), entonces \(\alpha \thicksim \alpha ^{\prime }.\)
Prueba: (a) Tenemos que

\(\displaystyle \begin{array}{l} \varphi \thicksim \psi \\ \ \ \Updownarrow \text{ (por (6) de la def de}\models \text{)} \\ \mathbf{A}\models (\varphi \leftrightarrow \psi )[\vec{a}]\text{, para todo } \mathbf{A}\text{ y toda }\vec{a}\in A^{\mathbf{N}} \\ \ \ \Updownarrow \\ \mathbf{A}\models (\varphi \leftrightarrow \psi )[\downarrow _{i_{n}}^{a}(\vec{a })]\text{, para todo }\mathbf{A}\text{, }a\in A\text{ y toda }\vec{a}\in A^{ \mathbf{N}} \\ \ \ \Updownarrow (\text{por (8) de la def de}\models ) \\ \mathbf{A}\models \forall x_{i_{n}}(\varphi \leftrightarrow \psi )[\vec{a}] \text{, para todo }\mathbf{A}\text{ y toda }\vec{a}\in A^{\mathbf{N}} \\ \ \ \Updownarrow \\ \mathbf{A}\models \forall x_{i_{n}}(\varphi \leftrightarrow \psi )[\downarrow _{i_{n-1}}^{a}(\vec{a})]\text{, para todo }\mathbf{A}\text{, }a\in A\text{ y toda }\vec{a}\in A^{\mathbf{N}} \\ \ \ \Updownarrow \text{ (por (8) de la def de}\models \text{)} \\ \mathbf{A}\models \forall x_{i_{n-1}}\forall x_{i_{n}}(\varphi \leftrightarrow \psi )[\vec{a}]\text{, para todo }\mathbf{A}\text{ y toda }\vec{a}\in A^{ \mathbf{N}} \\ \ \ \Updownarrow \\ \ \ \ \ \vdots \\ \ \ \Updownarrow \\ \mathbf{A}\models \forall x_{i_{1}}...\forall x_{i_{n}}(\varphi \leftrightarrow \psi )[\vec{a}]\text{, para todo }\mathbf{A}\text{ y toda }\vec{a}\in A^{ \mathbf{N}} \\ \ \ \Updownarrow \\ \forall x_{i_{1}}...\forall x_{i_{n}}(\varphi \leftrightarrow \psi )\text{ es universalmente valida} \end{array} \)

(b) Es dejado al lector.

(c) Por induccion en el \(k\) tal que \(\alpha \in F_{k}^{\tau }\). \(\Box\)

« Previous
1
2
3
4
5
6
7
8
9
10
11
12
13
14
15
16
17
18
19
20
21
22
23
24
25
26
27
28
29
30
» Next
×
Lenguaje \(\mathcal{S}^{\Sigma }\)

Entorno para trabajar con el lenguaje \(\mathcal{S}^{\Sigma }\) creado por Gabriel Cerceau:

Descargar!
Close
