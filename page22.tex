Gran Logico

Gran Lógico
Lenguaje \(\mathcal{S}^{\Sigma }\)
Apunte
Contacto
Login
« Previous
1
2
3
4
5
6
7
8
9
10
11
12
13
14
15
16
17
18
19
20
21
22
23
24
25
26
27
28
29
30
» Next
8.3. Homomorfismos

Dado un modelo de tipo \(\tau \), \(\mathbf{A}=(A,i)\), para cada \(s\in \mathcal{ C}\cup \mathcal{F}\cup \mathcal{R}\), usaremos \(s^{\mathbf{A}}\) para denotar a \(i(s)\). Sean \(\mathbf{A}\) y \(\mathbf{B}\) modelos de tipo \(\tau \). Una funcion \(F:A\rightarrow B\) sera un homomorfismo de \(\mathbf{A}\) en \(\mathbf{B}\) si se cumplen las siguientes

(1) \(F(c^{\mathbf{A}})=c^{\mathbf{B}},\;\) para todo \(c\in \mathcal{C} , \)
(2) \(F(f^{\mathbf{A}}(a_{1},...,a_{n}))=f^{\mathbf{B} }(F(a_{1}),...,F(a_{n})),\) para cada \(f\in \mathcal{F}_{n},\) \( a_{1},...,a_{n}\in A\).
(3) \((a_{1},...,a_{n})\in r^{\mathbf{A}}\) implica \( (F(a_{1}),...,F(a_{n}))\in r^{\mathbf{B}},\) para todo \(r\in \mathcal{R}_{n},\) \(a_{1},...,a_{n}\in A.\)
Un isomorfismo de \(\mathbf{A}\) en \(\mathbf{B}\) sera un un homomorfismo de \(\mathbf{A}\) en \(\mathbf{B}\) el cual sea biyectivo y cuya inversa sea un homomorfismo. Diremos que los modelos \( \mathbf{A}\) y \(\mathbf{B}\) son isomorfos (en simbolos: \(\mathbf{A} \cong \mathbf{B}\)), cuando haya un isomorfismo \(F:A\rightarrow B.\)

Ejercicio: Pruebe que la relacion \(\cong \) es reflexiva, transitiva y simetrica.

Lema 136 Sea \(F:\mathbf{A}\rightarrow \mathbf{B}\) un homomorfismo. Entonces
\(\displaystyle F(t^{\mathbf{A}}[(a_{1},a_{2},...)]=t^{\mathbf{B}}[F(a_{1}),F(a_{2}),...)] \)
para cada \(t\in T^{\tau }\), \((a_{1},a_{2},...)\in A^{\mathbf{N}}\).
Prueba: Sea

- Teo\(_{k}\): Si \(F:\mathbf{A}\rightarrow \mathbf{B}\) es un homomorfismo, entonces
\(\displaystyle F(t^{\mathbf{A}}[(a_{1},a_{2},...)]=t^{\mathbf{B}}[F(a_{1}),F(a_{2}),...)] \)

para cada \(t\in T_{k}^{\tau }\), \((a_{1},a_{2},...)\in A^{\mathbf{N}}\).
Teo\(_{0}\) es trivial. Veamos que Teo\(_{k}\) implica Teo\(_{k+1}\). Supongamos que vale Teo\(_{k}\) y supongamos \(F:\mathbf{A}\rightarrow \mathbf{B}\) es un homomorfismo, \(t\in T_{k+1}^{\tau }-T_{k}^{\tau }\) y \(\vec{a} =(a_{1},a_{2},...)\in A^{\mathbf{N}}\). Denotemos \((F(a_{1}),F(a_{2}),...)\) con \(F(\vec{a})\). Por Lema 117, \(t=f(t_{1},...,t_{n})\), con \(n\geq 1 \),\(\;f\in \mathcal{F}_{n}\) y \(t_{1},...,t_{n}\in T_{k}^{\tau }\). Tenemos entonces

\(\displaystyle \begin{array}{ccl} F(t^{\mathbf{A}}[\vec{a}]) & = & F(f(t_{1},...,t_{n})^{\mathbf{A}}[\vec{a}]) \\ & = & F(f^{\mathbf{A}}(t_{1}^{\mathbf{A}}[\vec{a}],...,t_{n}^{\mathbf{A}}[ \vec{a}])) \\ & = & f^{\mathbf{B}}(F(t_{1}^{\mathbf{A}}[\vec{a}]),...,F(t_{n}^{\mathbf{A}}[ \vec{a}])) \\ & = & f^{\mathbf{B}}(t_{1}^{\mathbf{B}}[F(\vec{a})],...,t_{n}^{\mathbf{B}}[F( \vec{a})])) \\ & = & f(t_{1},...,t_{n})^{\mathbf{B}}[F(\vec{a})] \\ & = & t^{\mathbf{B}}[F(\vec{a})] \end{array} \)

\(\Box\)
Lema 137 Supongamos que \(F:\mathbf{A}\rightarrow \mathbf{B}\) es un isomorfismo\(.\) Sea \(\varphi \in F^{\tau }.\) Entonces
\(\displaystyle \mathbf{A}\models \varphi \lbrack (a_{1},a_{2},...)]\text{ }sii\text{ } \mathbf{B}\models \varphi \lbrack (F(a_{1}),F(a_{2}),...)] \)
para cada \((a_{1},a_{2},...)\in A^{\mathbf{N}}\). En particular \(\mathbf{A}\) y \(\mathbf{B}\) satisfacen las mismas sentencias de tipo \(\tau \).
8.4. Algebras

Un tipo \(\tau \) sera llamado algebraico si no contiene nombres de relacion. Un modelo de un tipo algebraico \(\tau \) sera llamado una \(\tau \)- algebra. Ejemplos clasicos de \(\tau \)-algebras son los grupos (\( \tau =(\{e\},\{.^{2}\},\varnothing ,a)\)), los reticulados, los reticulados acotados, las algebras de Boole, etc.

Una propiedad particular de los homomorfismos de \(\tau \)-algebras es la siguiente

Lema 138 Si \(F:\mathbf{A}\rightarrow \mathbf{B}\) es un homomorfismo biyectivo, entonces \(F\) es un isomorfismo
Prueba: Solo falta probar que \(F^{-1}\) es un homomorfismo. Supongamos que \(c\in \mathcal{C}\). Ya que \(F(c^{\mathbf{A}})=c^{\mathbf{B}}\), tenemos que \( F^{-1}(c^{\mathbf{B}})=c^{\mathbf{A}}\), por lo cual \(F^{-1}\) cumple (1) de la definicion de homomorfismo. Supongamos ahora que \(f\in \mathcal{F}_{n}\) y sean \(b_{1},...,b_{n}\in B\). Sean \(a_{1},...,a_{n}\in A\) tales que \( F(a_{i})=b_{i}\), \(i=1,...,n\). Tenemos que

\(\displaystyle \begin{array}{ccl} F^{-1}(f^{\mathbf{B}}(b_{1},...,b_{n})) & = & F^{-1}(f^{\mathbf{B} }(F(a_{1}),...,F(a_{n}))) \\ & = & F^{-1}(F(f^{\mathbf{A}}(a_{1},...,a_{n})) \\ & = & f^{\mathbf{A}}(a_{1},...,a_{n}) \\ & = & f^{\mathbf{A}}(F^{-1}(b_{1}),...,F^{-1}(b_{n})) \end{array} \)

por lo cual \(F^{-1}\) satisface (2) de la definicion de homomorfismo \(\Box\)
8.4.1. Subalgebras.

Si \(\mathbf{A}\) es una \(\tau \)-algebra, entonces un subuniverso de \( \mathbf{A}\) sera un subconjunto no vacio \(B\subseteq A\) tal que:

(1) \(c^{\mathbf{A}}\in B,\) para cada \(c\in \mathcal{C}\)
(2) \(f^{\mathbf{A}}(a_{1},...,a_{n})\in B,\) para cada \( (a_{1},...,a_{n})\in B^{n},\) \(f\in \mathcal{F}_{n}\)
Dado un subuniverso de \(\mathbf{A}\) podemos definir en forma natural una \(\tau \)-algebra \(\mathbf{B}\) de la siguiente manera:

(1) Universo de \(\mathbf{B}=B\)
(2) \(c^{\mathbf{B}}=c^{\mathbf{A}},\) para cada \(c\in \mathcal{C}\)
(3) \(f^{\mathbf{B}}=f^{\mathbf{A}}\mid _{B^{n}},\) para cada \(f\in \mathcal{F}_{n}\).
En tal caso diremos que \(\mathbf{B}\) es subalgebra de \( \mathbf{A}.\)

Lema 139 Si \(F:\mathbf{A}\rightarrow \mathbf{B}\) es un homomorfismo, entonces \(I_{F}\) es un subuniverso de \(\mathbf{B}\)
Prueba: Ya que \(A\neq \varnothing ,\) tenemos que \(I_{F}\neq \varnothing .\) Es claro que \( c^{\mathbf{B}}=F(c^{\mathbf{A}})\in I_{F},\) para cada \(c\in \mathcal{C}\). Sea \(f\in \mathcal{F}_{n}\) y sean \(b_{1},...,b_{n}\in I_{F}\) Sean \( a_{1},...,a_{n}\) tales que \(F(a_{i})=b_{i},\) \(i=1,...,n\). Tenemos que

\(\displaystyle f^{\mathbf{B}}(b_{1},...,b_{n})=f^{\mathbf{B}}(F(a_{1}),...,F(a_{n}))=F(f^{ \mathbf{A}}(a_{1},...,a_{n}))\in I_{F} \)

por lo cual \(I_{F}\) es cerrada bajo \(f^{\mathbf{B}}\). \(\Box\)
8.4.2. Congruencias.

Sea \(\mathbf{A}\) una \(\tau \)-algebra. Una congruencia sobre \(\mathbf{A}\) es una relacion de equivalencia \(\theta \) sobre \(A\) la cual cumple que

\(\displaystyle a_{1}\theta b_{1},...,a_{n}\theta b_{n}\text{ implica }f^{\mathbf{A} }(a_{1},...,a_{n})\theta f^{\mathbf{A}}(b_{1},...,b_{n}) \)

cualesquiera sean \(a_{1},...,a_{n},b_{1},...,b_{n}\in A\) y\(\;f\in \mathcal{F} _{n}\).
Dada una congruencia \(\theta \) sobre \(\mathbf{A}\) se puede formar una nueva algebra \(\mathbf{A}/\theta \) de la siguiente manera:

(1) Universo de \(\mathbf{A}/\theta =A/\theta =\{a/\theta :a\in A\}=\{\) clases de equivalencia de \(\theta \}\)
(2) \(f^{\mathbf{A}/\theta }(a_{1}/\theta ,...,a_{n}/\theta )=f^{ \mathbf{A}}(a_{1},...,a_{n})/\theta ,\) para cada \(f\in \mathcal{F}_{n}.\)
(3) \(c^{\mathbf{A}/\theta }=c^{\mathbf{A}}/\theta ,\) para cada \(c\in \mathcal{C}\)
\(\mathbf{A}/\theta \) sera llamada el algebra cociente de \( \mathbf{A}\) por \(\theta \).

Lema 140 Si \(F:\mathbf{A}\rightarrow \mathbf{B}\) es un homomorfismo, entonces \(\ker F\) es una congruencia sobre \(\mathbf{A}\)
Prueba: Sea \(f\in \mathcal{F}_{n}\). Supongamos que \(a_{1},...,a_{n},b_{1},...,b_{n} \in A\) son tales que \(a_{1}\ker Fb_{1},...,a_{n}\ker Fb_{n}\). Tenemos entonces que

\(\displaystyle \begin{array}{ccl} F(f^{\mathbf{A}}(a_{1},...,a_{n})) & = & f^{\mathbf{B} }(F(a_{1}),...,F(a_{n})) \\ & = & f^{\mathbf{B}}(F(b_{1}),...,F(b_{n})) \\ & = & F(f^{\mathbf{B}}(b_{1},...,b_{n})) \end{array} \)

lo cual nos dice que \(f^{\mathbf{A}}(a_{1},...,a_{n})\ker Ff^{\mathbf{B} }(b_{1},...,b_{n})\) \(\Box\)
Al mapeo

\(\displaystyle \begin{array}{lll} A & \rightarrow & A/\theta \\ a & \rightarrow & a/\theta \end{array} \)

lo llamaremos la proyeccion canonica y lo denotaremos con \(\pi _{\theta }\).
Lema 141 \(\pi _{\theta }:\mathbf{A}\rightarrow \mathbf{A}/\theta \) es un homomorfismo cuyo nucleo es \(\theta \)
Prueba: Sea \(c\in \mathcal{C}\). Tenemos que

\(\displaystyle \pi _{\theta }(c^{\mathbf{A}})=c^{\mathbf{A}}/\theta =c^{\mathbf{A}/\theta } \)

Sea \(f\in \mathcal{F}_{n}\), con \(n\geq 1\) y sean \(a_{1},...,a_{n}\in A\). Tenemos que
\(\displaystyle \begin{array}{ccl} \pi _{\theta }(f^{\mathbf{A}}(a_{1},...,a_{n})) & = & f^{\mathbf{A} }(a_{1},...,a_{n})/\theta \\ & = & f^{\mathbf{A}/\theta }(a_{1}/\theta ,...,a_{n}/\theta ) \\ & = & f^{\mathbf{A}/\theta }(\pi _{\theta }(a_{1}),...,\pi _{\theta }(a_{n})) \end{array} \)

con lo cual \(\pi _{\theta }\) es un homomorfismo. Es trivial que \(\ker \pi _{\theta }=\theta \) \(\Box\)
Corolario 142 Para cada \(t\in T^{\tau }\), \(\vec{a}\in A^{\mathbf{N} },\) se tiene que \(t^{\mathbf{A}/\theta }[(a_{1}/\theta ,a_{2}/\theta ,...)]=t^{\mathbf{A}}[(a_{1},a_{2},...)]/\theta .\)
Prueba: Ya que \(\pi _{\theta }\) es un homomorfismo, se puede aplicar el Lema 136. \(\Box\)

Teorema 143 Sea \(F:\mathbf{A}\rightarrow \mathbf{B}\) un homomorfismo sobreyectivo. Entonces
\(\displaystyle \begin{array}{lll} A/\ker F & \rightarrow & B \\ a/\ker F & \rightarrow & F(a) \end{array} \)
define sin ambiguedad una funcion \(\bar{F}\) la cual es un isomorfismo de \( \mathbf{A}/\ker F\) en \(\mathbf{B}\)
Prueba: Notese que la definicion de \(\bar{F}\) es inambigua ya que si \(a/\ker F=a^{\prime }/\ker F\), entonces \(F(a)=F(a^{\prime }).\) Ya que \(F\) es sobre, tenemos que \(\bar{F}\) lo es. Supongamos que \(\bar{F}(a/\ker F)=\bar{F} (a^{\prime }/\ker F).\) Claramente entonces tenemos que \(F(a)=F(a^{\prime })\) , lo cual nos dice que \(a/\ker F=a^{\prime }/\ker F\). Esto prueba que \(\bar{F }\) es inyectiva. Para ver que \(\bar{F}\) es un isomorfismo, por el Lema 138, basta con ver que \(\bar{F}\) es un homomorfismo. Sea \(c\in \mathcal{C}\). Tenemos que

\(\displaystyle \bar{F}(c^{\mathbf{A}/\ker F})=\bar{F}(c^{\mathbf{A}}/\ker F)=F(c^{\mathbf{A} })=c^{\mathbf{B}} \)

Sea \(f\in \mathcal{F}_{n}\). Sean \(a_{1},...,a_{n}\in A\). Tenemos que
\(\displaystyle \begin{array}{ccl} \bar{F}(f^{\mathbf{A}/\ker F}(a_{1}/\ker F,...,a_{n}/\ker F)) & = & \bar{F} (f^{\mathbf{A}}(a_{1},...,a_{n})/\ker F) \\ & = & F(f^{\mathbf{A}}(a_{1},...,a_{n})) \\ & = & f^{\mathbf{B}}(F(a_{1}),...,F(a_{n})) \\ & = & f^{\mathbf{B}}(\bar{F}(a_{1}/\ker F),...,\bar{F}(a_{n}/\ker F)) \end{array} \)

con lo cual \(\bar{F}\) cunple (2) de la definicion de homomorfismo \(\Box\)
8.4.3. Producto directo de algebras.

Dadas \(\tau \)-algebras \(\mathbf{A},\mathbf{B},\) definamos una nueva \(\tau \) -algebra \(\mathbf{A}\times \mathbf{B},\) de la siguiente manera

(1) Universo de \(\mathbf{A}\times \mathbf{B}=A\times B\)
(2) \(c^{\mathbf{A}\times \mathbf{B}}=(c^{\mathbf{A}},c^{\mathbf{B}}),\) para cada \(c\in \mathcal{C}\)
(3) \(f^{\mathbf{A}\times \mathbf{B} }((a_{1},b_{1}),...,(a_{n},b_{n}))=(f^{\mathbf{A}}(a_{1},...,a_{n}),f^{ \mathbf{B}}(b_{1},...,b_{n})),\) para cada \(f\in \mathcal{F}_{n}\)
Llamaremos a \(\mathbf{A}\times \mathbf{B}\) el producto directo de \(\mathbf{A}\) y \(\mathbf{B}.\)

Los mapeos

\(\displaystyle \begin{array}{lll} \pi _{1}:A\times B & \rightarrow & A \\ \;\;\;\;\;(a,b) & \rightarrow & a \end{array} \ \ \ \ \ \ \ \ \ \ \ \ \ \ \ \ \ \ \ \ \ \ \ \ \ \ \ \ \ \ \ \ \ \ \ \ \ \ \ \ \ \ \ \ \ \ \ \ \ \begin{array}{lll} \pi _{2}:A\times B & \rightarrow & B \\ \;\;\;\;\;(a,b) & \rightarrow & b \end{array} \)

seran llamados las proyecciones canonicas asociadas al producto \( A\times B\)
Lema 144 Los mapeos \(\pi _{1}:A\times B\rightarrow A\) y \(\pi _{2}:A\times B\rightarrow A\) son homomorfismos
Prueba: Veamos que \(\pi _{1}\) es un homomorfismo. Primero notese que si \(c\in \mathcal{C}\), entonces

\(\displaystyle \pi _{1}(c^{\mathbf{A}\times \mathbf{B}})=\pi _{1}((c^{\mathbf{A}},c^{ \mathbf{B}}))=c^{\mathbf{A}} \)

Sea \(f\in \mathcal{F}_{n}\), con \(n\geq 1\) y sean \( (a_{1},b_{1}),...,(a_{n},b_{n})\in A\times B\). Tenemos que
\(\displaystyle \begin{array}{ccl} \pi _{1}(f^{\mathbf{A}\times \mathbf{B}}((a_{1},b_{1}),...,(a_{n},b_{n})) & = & \pi _{1}((f^{\mathbf{A}}(a_{1},...,a_{n}),f^{\mathbf{B}}(b_{1},...,b_{n})) \\ & = & f^{\mathbf{A}}(a_{1},...,a_{n}) \\ & = & f^{\mathbf{A}}(\pi _{1}(a_{1},b_{1}),...,\pi _{1}(a_{n},b_{n})) \end{array} \)

con lo cual hemos probado que \(\pi _{1}\) cumple (2) de la definicion de homomorfismo \(\Box\)
Lema 145 Para cada \(t\in T^{\tau },\) \(((a_{1},b_{1}),(a_{2},b_{2}),...)\in (A\times B)^{\mathbf{N}},\) se tiene que \(t^{\mathbf{A}\times \mathbf{B} }[((a_{1},b_{1}),(a_{2},b_{2}),...)]=(t^{\mathbf{A}}[(a_{1},a_{2},...)],t^{ \mathbf{B}}[(b_{1},b_{2},...)])\)
8.5. Dos teoremas de reemplazo

Sean \(v_{1},...,v_{n}\) variables, todas distintas. Si \(t\) es un termino de tipo \(\tau \), entonces escribiremos \(t=_{d}t(v_{1},...,v_{n})\) para declarar que las variables que ocurren en \(t\) estan en \(\{v_{1},...,v_{n}\}\) (no necesariamente toda \(v_{j}\) debe ocurrir en \(t\)). Cuando hayamos hecho la declaracion \(t=_{d}t(v_{1},...,v_{n})\), si \(P_{1},...,P_{n}\) son palabras cualesquiera, entonces \(t(P_{1},...,P_{n})\) denotara la palabra que resulta de reemplazar (simultaneamente) cada ocurrencia de \(v_{1}\) en \(t,\) por \( P_{1},\) cada ocurrencia de \(v_{2}\) en \(t,\) por \(P_{2},\) etc. Notese que cuando las palabras \(P_{i}^{\prime }s\) son terminos, \(t(P_{1},...,P_{n})\) es un termino (Lema 123). Ademas notese que si \( t=_{d}t(v_{1},...,v_{n}),\) entonces se da alguna de las siguientes

(1) \(t=c,\) para algun \(c\in \mathcal{C}\)
(2) \(t=v_{j},\) para algun \(j\)
(3) \(t=f(t_{1},...,t_{m}),\) para algun \(f\in \mathcal{F}_{m}\) y \( t_{1},...,t_{m}\) terminos tales que \( t_{1}=_{d}t_{1}(v_{1},...,v_{n}),...,t_{m}=_{d}t_{m}(v_{1},...,v_{n}).\)
Sean \(v_{1},...,v_{n}\) variables, todas distintas y supongamos \( t=_{d}t(v_{1},...,v_{n})\). Sea \(\mathbf{A}\) un modelo de tipo \(\tau \). Sean \( a_{1},...,a_{n}\in A\). Entonces con \(t^{\mathbf{A}}[a_{1}...,a_{n}]\) denotaremos al elemento \(t^{\mathbf{A}}[\vec{b}]\), donde \(\vec{b}\) es una asignacion tal que a cada \(v_{i}\) le asigna el valor \(a_{i}.\) Notese que esta definicion es buena gracias al Lema 131.

Las siguientes propiedades caracterizan la notacion \(t^{A}[\;]\):

(1) Si \(t=c,\) entonces \(t^{\mathbf{A}}[a_{1},....,a_{n}]=c^{A}\)
(2) Si \(t=v_{j},\) entonces \(t^{\mathbf{A}}[a_{1},....,a_{n}]=a_{j}\)
(3) Si \(t=f(t_{1},...,t_{m}),\) entonces
\(\displaystyle t^{\mathbf{A}}[a_{1},....,a_{n}]=f^{\mathbf{A}}(t_{1}^{\mathbf{A} }[a_{1},....,a_{n}],...,t_{m}^{\mathbf{A}}[a_{1},....,a_{n}]) \)

Lema 146 Sean \(w_{1},...,w_{k}\) variables, todas distintas. Sean \( v_{1},...,v_{n}\) variables, todas distintas. Supongamos \( t=_{d}t(w_{1},...,w_{k}),\) \( s_{1}=_{d}s_{1}(v_{1},...,v_{n}),...,s_{k}=_{d}s_{k}(v_{1},...,v_{n}).\) Entonces
(a) \(t(s_{1},...,s_{k})=_{d}t(s_{1},...,s_{k})(v_{1},...,v_{n})\)
(b) Para cada estructura \(\mathbf{A}\) y \(a_{1},....,a_{n}\in A,\) se tiene que
\(\displaystyle t(s_{1},...,s_{k})^{\mathbf{A}}[a_{1},....,a_{n}]=t^{\mathbf{A}}[s_{1}^{ \mathbf{A}}[a_{1},....,a_{n}],...,s_{k}^{\mathbf{A}}[a_{1},....,a_{n}]]. \)
Prueba: Probaremos que valen (a) y (b), por induccion en el \(l\) tal que \(t\in T_{l}^{\tau }.\) El caso \(l=0\) es dejado al lector. Supongamos entonces que valen (a) y (b) siempre que \(t\in T_{l}^{\tau }\) y veamos que entonces valen (a) y (b) cuando \(t\in T_{l+1}^{\tau }-T_{l}^{\tau }\). Hay \(f\in \mathcal{F} _{m}\) y \(t_{1},...,t_{m}\in T_{l}^{\tau }\) tales que \( t_{1}=_{d}t_{1}(w_{1},...,w_{k}),...,t_{m}=_{d}t_{m}(w_{1},...,w_{k})\) y \( t=f(t_{1},...,t_{m})\). Notese que por (a) de la HI tenemos que

\(\displaystyle t_{i}(s_{1},...,s_{k})=_{d}t_{i}(s_{1},...,s_{k})(v_{1},...,v_{n})\text{, } i=1,...,m \)

lo cual ya que
\(\displaystyle t(s_{1},...,s_{k})=f(t_{1}(s_{1},...,s_{k}),...,t_{m}(s_{1},...,s_{k})) \)

nos dice que
\(\displaystyle t(s_{1},...,s_{k})=_{d}t(s_{1},...,s_{k})(v_{1},...,v_{n}) \)

obteniendo asi (a). Para probar (b) notemos que por (b) de la hipotesis inductiva
\(\displaystyle t_{j}(s_{1},...,s_{k})^{\mathbf{A}}[\vec{a}]=t_{j}^{\mathbf{A}}[s_{1}^{ \mathbf{A}}[\vec{a}],...,s_{k}^{\mathbf{A}}[\vec{a}]],j=1,...,m \)

lo cual nos dice que
\(\displaystyle \begin{array}{ccl} t(s_{1},...,s_{k})^{\mathbf{A}}[\vec{a}] & = & f(t_{1}(s_{1},...,s_{k}),...,t_{m}(s_{1},...,s_{k}))^{\mathbf{A}}[\vec{a}] \\ & = & f^{\mathbf{A}}(t_{1}(s_{1},...,s_{k})^{\mathbf{A}}[\vec{a} ],...,t_{m}(s_{1},...,s_{k})^{\mathbf{A}}[\vec{a}]) \\ & = & f^{\mathbf{A}}(t_{1}^{\mathbf{A}}[s_{1}^{\mathbf{A}}[\vec{a} ],...,s_{k}^{\mathbf{A}}[\vec{a}]],...,t_{m}^{\mathbf{A}}[s_{1}^{\mathbf{A}}[ \vec{a}],...,s_{k}^{\mathbf{A}}[\vec{a}]]) \\ & = & t^{\mathbf{A}}[s_{1}^{\mathbf{A}}[\vec{a}],...,s_{k}^{\mathbf{A}}[\vec{ a}]] \end{array} \)

\(\Box\)
« Previous
1
2
3
4
5
6
7
8
9
10
11
12
13
14
15
16
17
18
19
20
21
22
23
24
25
26
27
28
29
30
» Next
×
Lenguaje \(\mathcal{S}^{\Sigma }\)

Entorno para trabajar con el lenguaje \(\mathcal{S}^{\Sigma }\) creado por Gabriel Cerceau:

Descargar!
Close
The JS beautifier will take care of your dirty JavaScript codes, assuring a syntax error-free solution.
