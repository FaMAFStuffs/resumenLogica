Gran Logico

Gran Lógico
Lenguaje \(\mathcal{S}^{\Sigma }\)
Apunte
Contacto
Login
« Previous
1
2
3
4
5
6
7
8
9
10
11
12
13
14
15
16
17
18
19
20
21
22
23
24
25
26
27
28
29
30
» Next
En virtud del lema anterior, diremos que \(\varphi \) se deduce de \( \psi \) por la regla de generalizacion con nombre de constante \(c\) para expresar que \((\psi ,\varphi )\in Generaliz^{\tau }\) y que \(c\) es el unico nombre de constante que ocurre en \(\psi \) y no ocurre en \(\varphi \).

Sea

\(\displaystyle Elec^{\tau }=\{(\exists v\varphi (v),\varphi (e)):\varphi =_{d}\varphi (v),\ Li(\varphi )=\{v\}\ \mathrm{y\ }e\in \mathcal{C}\ \mathrm{no\ ocurre\ en}\ \varphi \} \)

Diremos que \(\varphi \) se deduce de \(\psi \) por la regla de eleccion para expresar que \((\psi ,\varphi )\in Elec^{\tau }\).
Lema 150 Si \(\varphi \) se deduce de \(\psi \) por la regla de eleccion, entonces el nombre de constante \(e\) del cual habla la propiedad que define al conjunto \( Elec^{\tau }\) esta univocamente determinado por el par \((\varphi ,\psi )\).
Prueba: Notese que \(e\) es el unico nombre de constante que ocurre en \(\varphi \) y no ocurre en \(\psi \). \(\Box\)

\( \)

En virtud del lema anterior, diremos que \(\varphi \) se deduce de \( \psi \) por la regla de eleccion con nombre de constante \(e\) para expresar que \((\psi ,\varphi )\in Elec^{\tau }\) y que \(e\) es el unico nombre de constante que ocurre en \(\varphi \) y no ocurre en \(\psi \).

Lema 151 Todas las reglas exepto las reglas de eleccion y generalizacion son universales en el sentido que si \(\varphi \) se deduce de \(\psi _{1},...,\psi _{k}\) por alguna de estas reglas, entonces \(\left( (\psi _{1}\wedge ...\wedge \psi _{k})\rightarrow \varphi \right) \) es una sentencia universalmente valida
Prueba: Veamos que la regla de existencia es universal. Supongamos \(\varphi =_{d}\varphi (v)\), \(t\in T_{c}^{\tau }\) y \(\mathbf{A}\) es una estructura de tipo \(\tau \) tal que \(\mathbf{A}\models \varphi (t)\). Sea \(t^{\mathbf{A}}\) el valor que toma \(t\) en \(\mathbf{A}\). Por el Lema 148 tenemos que \(\mathbf{A}\models \varphi \left[ t^{\mathbf{A}}\right] \), por lo cual tenemos que \(\mathbf{A}\models \exists v\varphi (v)\).

Veamos que la regla de reemplazo es universal. Debemos probar que si \((\psi _{1},\psi _{2},\varphi )\in Reemp^{\tau }=Reemp1^{\tau }\cup Reemp2^{\tau }\) , entonces \(\left( (\psi _{1}\wedge \psi _{2})\rightarrow \varphi \right) \) es una sentencia universalmente valida. El caso en el que \((\psi _{1},\psi _{2},\varphi )\in Reemp1^{\tau }\) es facil y lo dejaremos al lector. Para el caso en el que \((\psi _{1},\psi _{2},\varphi )\in Reemp2^{\tau }\) nos hara falta un resultado un poco mas general. Veamos por induccion en \(k\) que si se dan las siguienes condiciones

- \(\alpha \in F_{k}^{\tau }\) y \(\varphi ,\psi \in F^{\tau }\)
- \(\mathbf{A}\) es una estructura de tipo \(\tau \)
- \(\overline{\alpha }=\) resultado de reemplazar en \(\alpha \) una ocurrencia de \(\varphi \) por \(\psi \),
- \(\mathbf{A}\models \varphi \left[ \vec{a}\right] \) si y solo si \( \mathbf{A}\models \psi \left[ \vec{a}\right] \), para cada \(\vec{a}\in A^{ \mathbf{N}}\)
entonces se da que

- \(\mathbf{A}\models \alpha \left[ \vec{a}\right] \) si y solo si \( \mathbf{A}\models \overline{\alpha }\left[ \vec{a}\right] \), para cada \(\vec{ a}\in A^{\mathbf{N}}\).
CASO \(k=0.\)

Entonces \(\alpha \) es atomica y por lo tanto ya que \(\alpha \) es la unica subformula de \(\alpha \), la situacion es facil de probar.

CASO \(\alpha =\forall x_{i}\alpha _{1}.\)

Si \(\varphi =\alpha \), entonces la situacion es facil de probar. Si \(\varphi \neq \alpha \), entonces la ocurrencia de \(\varphi \) a reemplazar sucede en \(\alpha _{1}\) y por lo tanto \(\overline{\alpha }=\forall x_{i} \overline{\alpha _{1}}.\) Se tiene entonces que para un \(\vec{a}\) dado,

\(\displaystyle \begin{array}{c} \mathbf{A}\models \alpha \left[ \vec{a}\right] \\ \Updownarrow \\ \mathbf{A}\models \alpha _{1}\left[ \downarrow _{i}^{a}\vec{a}\right] ,\text{ para cada }a\in A \\ \Updownarrow \\ \mathbf{A}\models \overline{\alpha _{1}}\left[ \downarrow _{i}^{a}\vec{a} \right] ,\text{ para cada }a\in A \\ \Updownarrow \\ \mathbf{A}\models \overline{\alpha }\left[ \vec{a}\right] \end{array} \)

CASO \(\alpha =(\alpha _{1}\vee \alpha _{2})\).
Si \(\varphi =\alpha \), entonces la situacion es facil de probar. Supongamos \(\varphi \neq \alpha \) y supongamos que la ocurrencia de \(\varphi \) a reemplazar sucede en \(\alpha _{1}\). Entonces \(\overline{\alpha }=( \overline{\alpha _{1}}\vee \alpha _{2})\) y tenemos que

\(\displaystyle \begin{array}{c} \mathbf{A}\models \alpha \left[ \vec{a}\right] \\ \Updownarrow \\ \mathbf{A}\models \alpha _{1}\left[ \vec{a}\right] \text{ o }\mathbf{A} \models \alpha _{2}\left[ \vec{a}\right] \\ \Updownarrow \\ \mathbf{A}\models \overline{\alpha _{1}}\left[ \vec{a}\right] \text{ o } \mathbf{A}\models \alpha _{2}\left[ \vec{a}\right] \\ \Updownarrow \\ \mathbf{A}\models \overline{\alpha }\left[ \vec{a}\right] \end{array} \)

Los demas casos son dejados al lector.
Dejamos al lector el chequeo de la universalidad del resto de las reglas. \(\Box\)

AXIOMAS LOGICOS

Seran llamados axiomas logicos de tipo \(\tau \) a todas las sentencias de alguna de las siguientes formas

\(((\varphi \wedge \psi )\leftrightarrow (\psi \wedge \varphi ))\)
\(((\varphi \vee \psi )\leftrightarrow (\psi \vee \varphi ))\)
\(((\varphi \leftrightarrow \psi )\leftrightarrow (\psi \leftrightarrow \varphi ))\)
\(((t\equiv s)\leftrightarrow (s\equiv t))\)
\((\varphi \leftrightarrow \varphi )\)
\((t\equiv t)\)
\((\varphi \vee \lnot \varphi )\)
\((\varphi \leftrightarrow \lnot \lnot \varphi )\)
\((\varphi \rightarrow \psi )\leftrightarrow (\lnot \varphi \vee \psi )\)
\((\varphi \wedge (\psi \vee \varphi ))\leftrightarrow ((\varphi \wedge \psi )\vee (\varphi \wedge \varphi ))\)
\(((\varphi \wedge \psi )\rightarrow \varphi )\leftrightarrow (\varphi \rightarrow (\psi \rightarrow \varphi ))\)
\((\varphi \leftrightarrow \psi )\leftrightarrow ((\varphi \rightarrow \psi )\wedge (\psi \rightarrow \varphi ))\)
\((\lnot \forall v\varphi \leftrightarrow \exists v\lnot \varphi )\)
\((\lnot \exists v\varphi \leftrightarrow \forall v\lnot \varphi )\)
\(((\varphi \wedge \psi )\wedge \varphi )\leftrightarrow (\varphi \wedge (\psi \wedge \varphi ))\)
\(\lnot (\varphi \wedge \psi )\leftrightarrow (\lnot \varphi \vee \lnot \psi )\)
\(\lnot (\varphi \vee \psi )\leftrightarrow (\lnot \varphi \wedge \lnot \psi )\)
\(\lnot (\varphi \rightarrow \psi )\leftrightarrow (\varphi \wedge \lnot \psi )\)
donde \(v\in Var\), \(t\in T_{c}^{\tau }\) y \(\varphi ,\psi ,\varphi \in S^{\tau }\).

Lema 152 Sea \(\mathbf{\varphi }\in S^{\tau +}\). Hay unicos \(n\geq 1\) y \(\varphi _{1},...,\varphi _{n}\in S^{\tau }\) tales que \( \mathbf{\varphi }=\varphi _{1}...\varphi _{n}\).
Prueba: Solo hay que probar la unicidad la cual sigue de la Proposicion 128 \(\Box\)

Dada \(\mathbf{\varphi }\in S^{\tau +}\), usaremos \(n(\mathbf{\varphi })\) y \( \mathbf{\varphi }_{1},...,\mathbf{\varphi }_{n(\mathbf{\varphi })}\) para denotar los unicos \(n\) y \(\varphi _{1},...,\varphi _{n}\) cuya existencia garantiza el lema anterior. Sea \(Nombres_{1}\) el conjunto formado por las siguientes palabras

\(\displaystyle \begin{array}{rcl} & & \text{EXISTENCIA} \\ & & \text{PARTICULARIZACION} \\ & & \text{ABSURDO} \\ & & \text{EVOCACION} \\ & & \text{CONJUNCIONELIMINACION} \\ & & \text{DISJUNCIONINTRODUCCION} \\ & & \text{ELECCION} \\ & & \text{GENERALIZACION} \end{array} \)

Sea \(Nombres_{2}\) el conjunto formado por las siguientes palabras
\(\displaystyle \begin{array}{rcl} & & \text{MODUSPONENS} \\ & & \text{TRANSITIVIDAD} \\ & & \text{CONJUNCIONINTRODUCCION} \\ & & \text{DISJUNCIONELIMINACION} \\ & & \text{REEMPLAZO} \end{array} \)

Sea
\(\displaystyle Just=Just_{1}\cup Just_{2}\cup Just_{3}\cup Just_{4}\cup Just_{5}\cup Just_{6} \)

donde
\(\displaystyle \begin{array}{rcl} Just_{1} & =& \{\text{CONCLUSION},\text{AXIOMAPROPIO},\text{AXIOMALOGICO}\} \\ Just_{2} & =& \{\text{HIPOTESIS}\bar{k}:k\in \mathbf{N}\} \\ Just_{3} & =& \{\alpha (\bar{k}):k\in \mathbf{N}\text{ y }\alpha \in Nombres_{1}\} \\ Just_{4} & =& \{\alpha (\bar{j},\bar{k}):j,k\in \mathbf{N}\text{ y }\alpha \in Nombres_{2}\} \\ Just_{5} & =& \{\text{DIVISIONPORCASOS}(\bar{j},\bar{k},\bar{l}):j,k,l\in \mathbf{N}\} \\ Just_{6} & =& \{\text{TESIS}\bar{j}\beta :j\in \mathbf{N}\text{ y }\beta \in Just_{3}\cup Just_{4}\cup Just_{5}\} \end{array} \)

Una palabra sera llamada una justificacion si pertenece a \(Just\).
Lema 153 Sea \(\mathbf{J}\in Just^{+}\). Hay unicos \(n\geq 1\) y \(J_{1},...,J_{n}\in Just\) tales que \(\mathbf{J}=J_{1}...J_{n}\).
Prueba: Supongamos \(J_{1},...,J_{n}\), \(J_{1}^{\prime },...,J_{m}^{\prime }\), con \( n,m\geq 1\), son justificaciones tales que \(J_{1}...J_{n}=J_{1}^{\prime }...J_{m}^{\prime }\). Es facil ver que entonces tenemos \(J_{1}=J_{1}^{\prime }\), por lo cual \(J_{2}...J_{n}=J_{2}^{\prime }...J_{m}^{\prime }\). Un argumento inductivo nos dice que entonces \(n=m\) y \(J_{i}=J_{i}^{\prime }\), \( i=1,...,n\) \(\Box\)

Dados numeros naturales \(i\leq j\), usaremos \(\left\langle i,j\right\rangle \) para denotar el conjunto \(\{i,i+1,...,j\}.\) A los conjuntos de la forma \( \left\langle i,j\right\rangle \) los llamaremos bloques. Dada \( \mathbf{J}\in Just^{+}\), usaremos \(n(\mathbf{J})\) y \(\mathbf{J}_{1},..., \mathbf{J}_{n(\mathbf{J})}\) para denotar los unicos \(n\) y \(J_{1},...,J_{n}\) cuya existencia garantiza el lema anterior. Diremos que \(\mathbf{J}\) es balanceada si se dan las siguientes

(1) Por cada \(k\in \mathbf{N}\) a lo sumo hay un \(i\) tal que \(\mathbf{J }_{i}=\) \(\mathrm{HIPOTESIS}\bar{k}\) y a lo sumo hay un \(i\) tal que \(\mathbf{J }_{i}=\) \(\mathrm{TESIS}\bar{k}\alpha \), con \([\alpha ]_{1}\notin Num\)
(2) Si \(\mathbf{J}_{i}=\) \(\mathrm{HIPOTESIS}\bar{k}\) entonces hay un \( l >i\) tal que \(\mathbf{J}_{l}=\) \(\mathrm{TESIS}\bar{k}\alpha \), con \([\alpha ]_{1}\notin Num\)
(3) Si \(\mathbf{J}_{i}=\) \(\mathrm{TESIS}\bar{k}\), con \([\alpha ]_{1}\notin Num\) entonces hay un \(l< i\) tal que \(\mathbf{J}_{l}=\) \(\mathrm{ HIPOTESIS}\bar{k}\)
(4) Si
\(\displaystyle \begin{array}{rcl} \mathbf{J}_{i} & =& \text{HIPOTESIS}\bar{k} \\ \mathbf{J}_{j} & =& \text{TESIS}\bar{k}\alpha \text{, con }[\alpha ]_{1}\notin Num \\ \mathbf{J}_{u} & =& \text{HIPOTESIS}\bar{m} \\ \mathbf{J}_{v} & =& \text{TESIS}\bar{m}\beta \text{, con }[\beta ]_{1}\notin Num \end{array} \)

entonces \(\left\langle i,j\right\rangle \) y \(\left\langle u,v\right\rangle \) son disjuntos o uno contiene al otro
Si \(\mathbf{J}\in Just^{+}\) es balanceada, entonces definamos

\(\displaystyle \mathcal{B}^{\mathbf{J}}=\{\left\langle i,j\right\rangle :\exists k\ \mathbf{ J}_{i}=\text{HIPOTESIS}\bar{k}\text{ y }\mathbf{J}_{j}=\text{TESIS}\bar{k} \alpha \text{, con }[\alpha ]_{1}\notin Num\} \)

Notese que ya que \(\mathbf{J}\) es balanceada, si \(B_{1},B_{2}\in \mathcal{B} ^{\mathbf{J}}\), entonces \(B_{1}\cap B_{2}=\varnothing \) o \(B_{1}\subseteq B_{2} \) o \(B_{2}\subseteq B_{1}\). Un par \((\mathbf{\varphi },\mathbf{J})\in S^{\tau +}\times Just^{+}\) sera llamado adecuado si \(n(\mathbf{ \varphi })=n(\mathbf{J})\) y \(\mathbf{J}\) es balanceada. Sea \((\mathbf{ \varphi },\mathbf{J})\) un par adecuado. Si \(\left\langle i,j\right\rangle \in \mathcal{B}^{\mathbf{J}}\), entonces \(\mathbf{\varphi }_{i}\) sera la hipotesis del bloque \(\left\langle i,j\right\rangle \) en \((\mathbf{ \varphi },\mathbf{J})\) y \(\mathbf{\varphi }_{j}\) sera la tesis del bloque \(\left\langle i,j\right\rangle \) en \((\mathbf{\varphi },\mathbf{J})\). Diremos que \(\mathbf{\varphi }_{i}\) esta bajo la hipotesis \(\mathbf{ \varphi }_{l}\) en \((\mathbf{\varphi },\mathbf{J})\) o que \(\mathbf{ \varphi }_{l}\) es una hipotesis de \(\mathbf{\varphi }_{i}\) en \((\mathbf{\varphi },\mathbf{J})\) cuando haya en \(\mathcal{B}^{\mathbf{J} } \) un bloque de la forma \(\left\langle l,j\right\rangle \) el cual contenga a \(i\). Sean \(i,j\in \left\langle 1,n(\mathbf{\varphi })\right\rangle .\) Diremos que \(i\) es anterior a \(j\) en \((\mathbf{\varphi }, \mathbf{J})\) si \(i< j\) y ademas para todo \(B\in \mathcal{B}^{\mathbf{J}}\) se tiene que \(i\in B\Rightarrow j\in B\). Definamos la relacion \("e\) depende de \(d\) en \((\mathbf{\varphi },\mathbf{J})"\), por las siguientes reglas,
(1) \(e\) depende de \(d\) en \((\mathbf{\varphi }, \mathbf{J})\) cuando hay numeros \(1\leq l< j\leq n(\mathbf{\varphi })\) tales que
(i) \(l\) es anterior a \(j\) en \((\mathbf{\varphi },\mathbf{J})\)
(ii) \(\mathbf{J}_{j}=\alpha \mathrm{ELECCION}(\bar{l})\), con \(\alpha \in \{\varepsilon \}\cup \{\mathrm{TESIS}\bar{k}:k\in \mathbf{N}\}\) y \( \mathbf{\varphi }_{j}\) se deduce por la regla de eleccion de \(\mathbf{ \varphi }_{l}\) con nombre de constante \(e\)
(iii) \(d\) ocurre en \(\mathbf{\varphi }_{l}\).
(2) Si \(e\) depende de \(d\) en \((\mathbf{\varphi },\mathbf{J} ) \) y \(d\) depende de \(h\) en \((\mathbf{\varphi },\mathbf{J})\), entonces \(e\) depende de \(h\) en \((\mathbf{\varphi },\mathbf{ J}).\)
Sea \((\Sigma ,\tau )\) una teoria de primer orden. Sea \(\varphi \) una sentencia de tipo \(\tau \). Una prueba de \(\varphi \) en \((\Sigma ,\tau )\) sera un par adecuado \((\mathbf{\varphi },\mathbf{J})\) tal que:

(1) Hay un conjunto finito \(\mathcal{C}_{1}\), disjunto con \(\mathcal{C }\), tal que \((\mathcal{C}\cup \mathcal{C}_{1},\mathcal{F},\mathcal{R},a)\) es un tipo y cada \(\mathbf{\varphi }_{i}\) es una sentencia de tipo \((\mathcal{C} \cup \mathcal{C}_{1},\mathcal{F},\mathcal{R},a)\)
(2) \(\mathbf{\varphi }_{n(\mathbf{\varphi })}=\varphi \)
(3) Si \(\left\langle i,j\right\rangle \in \mathcal{B}^{\mathbf{J}}\), entonces \(\mathbf{J}_{j+1}=\mathrm{CONCLUSION}\) y \(\mathbf{\varphi }_{j+1}=( \mathbf{\varphi }_{i}\rightarrow \mathbf{\varphi }_{j})\)
(4) Para cada \(i=1,...,n(\mathbf{\varphi }),\) se da una de las siguientes
(a) \(\mathbf{J}_{i}=\mathrm{HIPOTESIS}\bar{k}\) para algun \(k\in \mathbf{N}\)
(b) \(\mathbf{J}_{i}=\mathrm{CONCLUSION}\) y hay un \(j\) tal que \( \left\langle j,i-1\right\rangle \in \mathcal{B}^{\mathbf{J}}\) y \(\mathbf{ \varphi }_{i}=(\mathbf{\varphi }_{j}\rightarrow \mathbf{\varphi }_{i-1})\)
(c) \(\mathbf{J}_{i}=\mathrm{AXIOMALOGICO}\) y \(\mathbf{\varphi }_{i}\) es un axioma logico de tipo \((\mathcal{C}\cup \mathcal{C}_{1},\mathcal{F}, \mathcal{R},a)\}\)
(d) \(\mathbf{J}_{i}=\mathrm{AXIOMAPROPIO}\) y \(\mathbf{\varphi } _{i}\in \Sigma \)
(e) \(\mathbf{J}_{i}\) es de la forma \(\alpha \mathrm{PARTICULARIZACION} (\bar{l})\), con \(l\) anterior a \(i\) y \(\mathbf{\varphi }_{i}\) se deduce de \( \mathbf{\varphi }_{l}\) por la regla de particularizacion
(f) \(\mathbf{J}_{i}\) es de la forma \(\alpha \mathrm{ABSURDO}(\bar{l})\) , con \(l\) anterior a \(i\) y \(\mathbf{\varphi }_{i}\) se deduce de \(\mathbf{ \varphi }_{l}\) por la regla del absurdo
(g) \(\mathbf{J}_{i}\) es de la forma \(\alpha \mathrm{EVOCACION}(\bar{l} )\), con \(l\) anterior a \(i\) y \(\mathbf{\varphi }_{i}\) se deduce de \(\mathbf{ \varphi }_{l}\) por la regla de evocacion
(h) \(\mathbf{J}_{i}\) es de la forma \(\alpha \mathrm{EXISTENCIA}(\bar{l })\), con \(l\) anterior a \(i\) y \(\mathbf{\varphi }_{i}\) se deduce de \(\mathbf{ \varphi }_{l}\) por la regla de existencia
(i) \(\mathbf{J}_{i}\) es de la forma \(\alpha \mathrm{ CONJUNCIONELIMINACION}(\bar{l})\), con \(l\) anterior a \(i\) y \(\mathbf{\varphi } _{i}\) se deduce de \(\mathbf{\varphi }_{l}\) por la regla de conjuncion-eliminacion
(j) \(\mathbf{J}_{i}\) es de la forma \(\alpha \mathrm{ DISJUNCIONINTRODUCCION}(\bar{l})\), con \(l\) anterior a \(i\) y \(\mathbf{\varphi }_{i}\) se deduce de \(\mathbf{\varphi }_{l}\) por la regla de disjuncion-introduccion
(k) \(\mathbf{J}_{i}\) es de la forma \(\alpha \mathrm{MODUSPONENS}( \overline{l_{1}},\overline{l_{2}})\), con \(l_{1}\) y \(l_{2}\) anteriores a \(i\) y \(\mathbf{\varphi }_{i}\) se deduce de \(\mathbf{\varphi }_{l_{1}}\) y \( \mathbf{\varphi }_{l_{2}}\) por la regla de modus ponens
(l) \(\mathbf{J}_{i}\) es de la forma \(\alpha \mathrm{ CONJUNCIONINTRODUCCION}(\overline{l_{1}},\overline{l_{2}})\), con \(l_{1}\) y \( l_{2}\) anteriores a \(i\) y \(\mathbf{\varphi }_{i}\) se deduce de \(\mathbf{ \varphi }_{l_{1}}\) y \(\mathbf{\varphi }_{l_{2}}\) por la regla de conjuncion-introduccion
(m) \(\mathbf{J}_{i}\) es de la forma \(\alpha \mathrm{ DISJUNCIONELIMINACION}(\overline{l_{1}},\overline{l_{2}})\), con \(l_{1}\) y \( l_{2}\) anteriores a \(i\) y \(\mathbf{\varphi }_{i}\) se deduce de \(\mathbf{ \varphi }_{l_{1}}\) y \(\mathbf{\varphi }_{l_{2}}\) por la regla de disjuncion-eliminacion
(n) \(\mathbf{J}_{i}\) es de la forma \(\alpha \mathrm{REEMPLAZO}( \overline{l_{1}},\overline{l_{2}})\), con \(l_{1}\) y \(l_{2}\) anteriores a \(i\) y \(\mathbf{\varphi }_{i}\) se deduce de \(\mathbf{\varphi }_{l_{1}}\) y \( \mathbf{\varphi }_{l_{2}}\) por la regla de reemplazo
(o) \(\mathbf{J}_{i}\) es de la forma \(\alpha \mathrm{TRANSITIVIDAD}( \overline{l_{1}},\overline{l_{2}})\), con \(l_{1}\) y \(l_{2}\) anteriores a \(i\) y \(\mathbf{\varphi }_{i}\) se deduce de \(\mathbf{\varphi }_{l_{1}}\) y \( \mathbf{\varphi }_{l_{2}}\) por la regla de transitividad
(p) \(\mathbf{J}_{i}\) es de la forma \(\alpha \mathrm{DIVISIONPORCASOS}( \overline{l_{1}},\overline{l_{2}},\overline{l_{3}})\), con \(l_{1},l_{2}\) y \( l_{3}\) anteriores a \(i\) y \(\mathbf{\varphi }_{i}\) se deduce de \(\mathbf{ \varphi }_{l_{1}},\mathbf{\varphi }_{l_{2}}\) y \(\mathbf{\varphi }_{l_{3}}\) por la regla de division por casos
(q) \(\mathbf{J}_{i}\) es de la forma \(\alpha \mathrm{ELECCION}(\bar{l} ) \), con \(l\) anterior a \(i\) y \(\mathbf{\varphi }_{i}\) se deduce por la regla de eleccion de \(\mathbf{\varphi }_{l}\), con constante \(e\), la cual no ocurre en \(\mathbf{\varphi }_{1},...,\mathbf{\varphi }_{i-1}\) y no pertenece a \( \mathcal{C}\).
(r) \(\mathbf{J}_{i}\) es de la forma \(\alpha \mathrm{GENERALIZACION}( \bar{l})\), con \(l\) anterior a \(i\) y \(\mathbf{\varphi }_{i}\) se deduce por la regla de generalizacion de \(\mathbf{\varphi }_{l}\), con constante \(c\) tal que:
(i) \(c\not\in \mathcal{C}\)
(ii) \(c\) no es una cte de \(\mathbf{\varphi }\) la cual sea introducida por la aplicacion de la regla de eleccion; es decir para cada \(u\in \{1,...,n(\mathbf{\varphi })\}\), si \(\mathbf{J}_{u}\) es de la forma \(\alpha \mathrm{ELECCION}(\bar{v})\), entonces \(\mathbf{\varphi }_{u}\) no se deduce de \(\mathbf{\varphi }_{v}\) por la regla de eleccion con constante \(c\).
(iii) \(c\) no ocurre en ninguna hipotesis de \(\mathbf{\varphi }_{l}.\)
(iv) Ninguna constante que ocurra en \(\mathbf{\varphi }_{l}\) o en sus hipotesis, depende de \(c.\)
Cuando haya una prueba de \(\varphi \) en \((\Sigma ,\tau )\), diremos que \( \varphi \) es un teorema de la teoria \((\Sigma ,\tau )\) y escribiremos \((\Sigma ,\tau )\vdash \varphi .\)

« Previous
1
2
3
4
5
6
7
8
9
10
11
12
13
14
15
16
17
18
19
20
21
22
23
24
25
26
27
28
29
30
» Next
×
Lenguaje \(\mathcal{S}^{\Sigma }\)

Entorno para trabajar con el lenguaje \(\mathcal{S}^{\Sigma }\) creado por Gabriel Cerceau:

Descargar!
Close
Learn great tips in our HTML blog about web development. Learn how to implement useful features for your website!
