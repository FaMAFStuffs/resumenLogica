Gran Logico

Gran Lógico
Lenguaje \(\mathcal{S}^{\Sigma }\)
Apunte
Contacto
Login
« Previous
1
2
3
4
5
6
7
8
9
10
11
12
13
14
15
16
17
18
19
20
21
22
23
24
25
26
27
28
29
30
» Next

Ejemplos: (a) El siguiente ejemplo muestra que en la sentencia a generalizar no pueden ocurrir ctes que dependan de la cte a generalizar. Sea \(\tau =(\varnothing ,\{f^{1}\},\varnothing ,a).\) Y sea \(\Sigma =\{\forall y\exists x\;f(x)\equiv y\}.\)

\( \begin{array}{llll} 1.\; & \forall y\exists x\;y\equiv f(x) & & \text{AXIOMAPROPIO} \\ 2.\; & \exists x\;y_{0}\equiv f(x) & & \text{PARTICULARIZACION}(1) \\ 3.\; & y_{0}\equiv f(e) & & \text{ELECCION}(2) \\ 4.\; & \forall y\;y\equiv f(e) & & \text{GENERALIZACION}(3) \\ 5. & c\equiv f(e) & & \text{PARTICULARIZACION}(4) \\ 6. & f(e)\equiv d & & \text{PARTICULARIZACION}(4) \\ 7. & c\equiv d & & \text{TRANSITIVIDAD}(5,6) \\ 8. & \forall y\;c\equiv y & & \text{GENERALIZACION}(7) \\ 9. & \forall x\forall y\;x\equiv y & & \text{GENERALIZACION}(8) \end{array} \)

Notese que una estructura \(\mathbf{A}\) de tipo \(\tau \) satisface \( \forall y\exists x\;f(x)\equiv y\) si y solo si \(f^{\mathbf{A}}\) es una funcion sobre.

(b) El siguiente ejemplo muestra que la cte a generalizar no puede ocurrir en hipotesis de la sentencia a la cual se le aplica la generalizacion. Sea \( \tau =(\{1\},\varnothing ,\varnothing ,\varnothing ).\)

\( \begin{array}{llll} 1.\; & c\equiv 1 & & \text{HIPOTESIS}1 \\ 2.\; & \forall x\;x\equiv 1 & & \text{TESIS}1\text{GENERALIZACION}(1) \\ 3.\; & (c\equiv 1\rightarrow \forall x\;x\equiv 1) & & \text{CONCLUSION} \\ 4.\; & \forall y\;\left( y\equiv 1\rightarrow \forall x\;x\equiv 1\right) & & \text{GENERALIZACION}(3) \\ 5.\; & \left( 1\equiv 1\rightarrow \forall x\;x\equiv 1\right) & & \text{ PARTICULARIZACION}(4) \\ 6. & 1\equiv 1 & & \text{AXIOMALOGICO} \\ 7. & \forall x\;x\equiv 1 & & \text{MODUSPONENS}(5,6) \end{array} \)

Lema 154 Sea \((\mathbf{\varphi },\mathbf{J})\) una prueba de \( \varphi \) en \((\Sigma ,\tau )\).
(1) Sea \(m\in \mathbf{N}\) tal que \(\mathbf{J}_{i}\neq \) \(\mathrm{ HIPOTESIS}\bar{m}\), para cada \(i=1,...,n(\mathbf{\varphi })\). Supongamos que \(\mathbf{J}_{i}=\) \(\mathrm{HIPOTESIS}\bar{k}\) y que \(\mathbf{J}_{j}=\) \( \mathrm{TESIS}\bar{k}\alpha \), con \([\alpha ]_{1}\notin Num\). Sea \(\mathbf{ \tilde{J}}\) el resultado de reemplazar en \(\mathbf{J}\) la justificacion \( \mathbf{J}_{i}\) por \(\mathrm{HIPOTESIS}\bar{m}\) y reemplazar la justificacion \(\mathbf{J}_{j}\) por \(\mathrm{TESIS}\bar{m}\alpha \). Entonces \( (\mathbf{\varphi },\mathbf{\tilde{J}})\) es una prueba de \(\varphi \) en \( (\Sigma ,\tau )\).
(2) Sea \(\mathcal{C}_{1}\) el conjunto de nombres de constante que ocurren en alguna \(\mathbf{\varphi }_{i}\) y que no pertenecen a \(\mathcal{C}\) . Sea \(e\in \mathcal{C}_{1}-\mathcal{C}\). Sea \(\tilde{e}\notin \mathcal{C} \cup \mathcal{C}_{1}\) tal que \((\mathcal{C}\cup (\mathcal{C}_{1}-\{e\})\cup \{\tilde{e}\},\mathcal{F},\mathcal{R},a)\) es un tipo. Sea \(\tilde{\varphi} _{i}=\) resultado de reemplazar en \(\mathbf{\varphi }_{i}\) cada ocurrencia de \(e\) por \(\tilde{e}.\) Entonces \((\mathbf{\tilde{\varphi}}_{1}...\mathbf{ \tilde{\varphi}}_{n(\mathbf{\varphi })},\mathbf{J})\) es una prueba de \( \varphi \) en \((\Sigma ,\tau )\).
Prueba: (1) Obvio.

(2) Sean

\(\displaystyle \begin{array}{rcl} \tau _{1} & =& (\mathcal{C}\cup \mathcal{C}_{1},\mathcal{F},\mathcal{R},a) \\ \tau _{2} & =& (\mathcal{C}\cup (\mathcal{C}_{1}-\{e\})\cup \{\tilde{e}\}, \mathcal{F},\mathcal{R},a) \end{array} \)

Para cada \(c\in \mathcal{C}\cup (\mathcal{C}_{1}-\{e\})\) definamos \(\tilde{c} =c\). Notese que el mapeo \(c\rightarrow \tilde{c}\) es una biyeccion entre el conjunto de nombres de constante de \(\tau _{1}\) y el conjunto de nombres de cte de \(\tau _{2}\). Para cada \(t\in T^{\tau _{1}}\) sea \(\tilde{t}=\) resultado de reemplazar en \(t\) cada ocurrencia de \(c\) por \(\tilde{c}\), para cada \(c\in \mathcal{C}\cup \mathcal{C}_{1}\). Analogamente para una formula \( \psi \in F^{\tau _{1}}\), sea \(\tilde{\psi}=\) resultado de reemplazar en \( \psi \) cada ocurrencia de \(c\) por \(\tilde{c}\), para cada \(c\in \mathcal{C} \cup \mathcal{C}_{1}\). Notese que los mapeos \(t\rightarrow \tilde{t}\) y \( \psi \rightarrow \tilde{\psi}\) son biyecciones naturales entre \(T^{\tau _{1}} \) y \(T^{\tau _{2}}\) y entre \(F^{\tau _{1}}\) y \(F^{\tau _{2}}\), respectivamente. Notese que cualesquiera sean \(\psi _{1},\psi _{2}\in F^{\tau _{1}}\), tenemos que \(\psi _{1}\) se deduce de \(\psi _{2}\) por la regla de generalizacion con constante \(c\) sii \(\tilde{\psi}_{1}\) se deduce de \(\tilde{\psi}_{2}\) por la regla de generalizacion con constante \(\tilde{c} \). Para las otras reglas sucede lo mismo. Notese tambien que \(c\) ocurre en \( \psi \) sii \(\tilde{c}\) ocurre en \(\tilde{\psi}.\) Mas aun notese que \(c\) depende de \(d\) en \((\mathbf{\varphi },\mathbf{J})\) sii \(\tilde{c}\) depende de \(\tilde{d}\) en \((\mathbf{\tilde{\varphi}},\mathbf{J})\), donde \(\mathbf{ \tilde{\varphi}}=\widetilde{\mathbf{\varphi }_{1}}...\widetilde{\mathbf{ \varphi }_{n(\mathbf{\varphi })}}\). Ahora es facil chequear que \((\mathbf{ \tilde{\varphi}},\mathbf{J})\) es una prueba de \(\varphi \) en \((\Sigma ,\tau ) \) basandose en que \((\mathbf{\varphi },\mathbf{J})\) es una prueba de \( \varphi \) en \((\Sigma ,\tau )\). \(\Box\)
Una teoria \((\Sigma ,\tau )\) sera inconsistente cuando haya una sentencia \(\varphi \) tal que \((\Sigma ,\tau )\vdash (\varphi \wedge \lnot \varphi ).\) Una teoria \((\Sigma ,\tau )\) sera consistente cuando no sea inconsistente.

Lema 155 Sea \((\Sigma ,\tau )\) una teoria.
(1) Si \((\Sigma ,\tau )\vdash \varphi _{1},...,\varphi _{n}\) y \( (\Sigma \cup \{\varphi _{1},...,\varphi _{n}\},\tau )\vdash \varphi ,\) entonces \((\Sigma ,\tau )\vdash \varphi .\)
(2) Si \((\Sigma ,\tau )\vdash \varphi _{1},...,\varphi _{n}\) y \( \varphi \) se deduce por alguna regla universal a partir de \(\varphi _{1},...,\varphi _{n}\), entonces \((\Sigma ,\tau )\vdash \varphi \).
(3) Si \((\Sigma ,\tau )\) es inconsistente, entonces \((\Sigma ,\tau )\vdash \varphi \), para toda sentencia \(\varphi .\)
(4) Si \((\Sigma ,\tau )\) es consistente y \((\Sigma ,\tau )\vdash \varphi \), entonces \((\Sigma \cup \{\varphi \},\tau )\) es consistente.
(5) \((\Sigma ,\tau )\vdash (\varphi \rightarrow \psi )\) si y solo si \( (\Sigma \cup \{\varphi \},\tau )\vdash \psi \).
(6) Si \((\Sigma ,\tau )\not\vdash \lnot \varphi \), entonces \((\Sigma \cup \{\varphi \},\tau )\) es consistente.
Prueba: (1) Haremos el caso \(n=2.\) Supongamos entonces que \((\Sigma ,\tau )\vdash \varphi _{1},\varphi _{2}\) y \((\Sigma \cup \{\varphi _{1},\varphi _{2}\},\tau )\vdash \varphi \). Para \(i=1,2\), sea \((\varphi _{1}^{i}...\varphi _{n_{i}}^{i},J_{1}^{i}...J_{n_{i}}^{i})\) una prueba de \( \varphi _{i}\) en \((\Sigma ,\tau )\). Sea \((\psi _{1}...\psi _{n},J_{1}...J_{n})\) una prueba de \(\varphi \) en \((\Sigma \cup \{\varphi _{1},\varphi _{2}\},\tau )\). Notese que por el Lema 154 podemos suponer que estas tres pruebas no comparten ningun nombre de constante auxiliar y que tampoco comparten numeros asociados a hipotesis o tesis. Para cada \(i=1,...,n\), definamos \(\widetilde{J_{i}}\) de la siguiente manera.

- Si \(\psi _{i}=\varphi _{1}\) y \(J_{i}=\mathrm{AXIOMAPROPIO}\), entonces \(\widetilde{J_{i}}=\mathrm{EVOCACION}(\overline{n_{1}})\)
- Si \(\psi _{i}=\varphi _{2}\) y \(J_{i}=\mathrm{AXIOMAPROPIO}\), entonces \(\widetilde{J_{i}}=\mathrm{EVOCACION}(\overline{n_{1}+n_{2}})\).
- Si \(\psi _{i}\notin \{\varphi _{1},\varphi _{2}\}\) y \(J_{i}=\mathrm{ AXIOMAPROPIO}\), entonces \(\widetilde{J_{i}}=\mathrm{AXIOMAPROPIO}\).
- Si \(J_{i}=\mathrm{AXIOMALOGICO}\), entonces \(\widetilde{J_{i}}= \mathrm{AXIOMALOGICO}\)
- Si \(J_{i}=\mathrm{CONCLUSION}\), entonces \(\widetilde{J_{i}}=\mathrm{ CONCLUSION}\).
- Si \(J_{i}=\mathrm{HIPOTESIS}\bar{k}\), entonces \(\widetilde{J_{i}}= \mathrm{HIPOTESIS}\bar{k}\)
- Si \(J_{i}=\alpha P(\overline{l_{1}},...,\overline{l_{k}})\), con \( \alpha \in \{\varepsilon \}\cup \{\mathrm{TESIS}\bar{k}:k\in \mathbf{N}\}\), entonces \(\widetilde{J_{i}}=\alpha P(\overline{l_{1}+n_{1}+n_{2}},..., \overline{l_{k}+n_{1}+n_{2}})\)
Para cada \(i=1,...,n_{2}\), definamos \(\widetilde{J_{i}^{2}}\) de la siguiente manera.

- Si \(J_{i}^{2}=\mathrm{AXIOMAPROPIO}\), entonces \(\widetilde{J_{i}^{2} }=\mathrm{AXIOMAPROPIO}\)
- Si \(J_{i}^{2}=\mathrm{AXIOMALOGICO}\), entonces \(\widetilde{J_{i}^{2} }=\mathrm{AXIOMALOGICO}\)
- Si \(J_{i}^{2}=\mathrm{CONCLUSION}\), entonces \(\widetilde{J_{i}^{2}}= \mathrm{CONCLUSION}\).
- Si \(J_{i}^{2}=\mathrm{HIPOTESIS}\bar{k}\), entonces \(\widetilde{ J_{i}^{2}}=\mathrm{HIPOTESIS}\bar{k}\)
- Si \(J_{i}^{2}=\alpha P(\overline{l_{1}},...,\overline{l_{k}})\), con \(\alpha \in \{\varepsilon \}\cup \{\mathrm{TESIS}\bar{k}:k\in \mathbf{N}\}\), entonces \(\widetilde{J_{i}^{2}}=\alpha P(\overline{l_{1}+n_{1}},..., \overline{l_{k}+n_{1}})\)
Es facil chequear que

\(\displaystyle (\varphi _{1}^{1}...\varphi _{n_{1}}^{1}\varphi _{1}^{2}...\varphi _{n_{2}}^{2}\psi _{1}...\psi _{n},J_{1}^{1}...J_{n_{1}}^{1}\widetilde{ J_{1}^{2}}...\widetilde{J_{n_{2}}^{2}}\widetilde{J_{1}}...\widetilde{J_{n}}) \)

es una prueba de \(\varphi \) en \((\Sigma ,\tau )\)
(2) Supongamos que \((\Sigma ,\tau )\vdash \varphi _{1},...,\varphi _{n}\) y que \(\varphi \) se deduce por regla R a partir de \(\varphi _{1},...,\varphi _{n}\), con R universal. Notese que

\(\displaystyle \begin{array}{llll} 1.\; & \varphi _{1} & & \text{AXIOMAPROPIO} \\ 2.\; & \varphi _{2} & & \text{AXIOMAPROPIO} \\ \vdots & \vdots & & \vdots \\ n. & \varphi _{n} & & \text{AXIOMAPROPIO} \\ n+1 & \varphi & & \text{R}(\bar{1},...,\bar{n}) \end{array} \)

es una prueba de \(\varphi \) en \((\Sigma \cup \{\varphi _{1},...,\varphi _{n}\},\tau )\), lo cual por (1) nos dice que \((\Sigma ,\tau )\vdash \varphi \) .
(3) Si \((\Sigma ,\tau )\) es inconsistente, entonces por definicion tenemos que \((\Sigma ,\tau )\vdash \psi \wedge \lnot \psi \) para alguna sentencia \( \psi \). Dada una sentencia cualquiera \(\varphi \) tenemos que \(\varphi \) se deduce por la regla del absurdo a partir de \(\psi \wedge \lnot \psi \) con lo cual (2) nos dice que \((\Sigma ,\tau )\vdash \varphi \)

(4) Supongamos \((\Sigma ,\tau )\) es consistente y \((\Sigma ,\tau )\vdash \varphi \). Si \((\Sigma \cup \{\varphi \},\tau )\) fuera inconsistente, entonces \((\Sigma \cup \{\varphi \},\tau )\vdash \psi \wedge \lnot \psi \), para alguna sentencia \(\psi \), lo cual por (1) nos diria que \((\Sigma ,\tau )\vdash \psi \wedge \lnot \psi \), es decir nos diria que \((\Sigma ,\tau )\) es inconsistente.

(5) Supongamos \((\Sigma ,\tau )\vdash (\varphi \rightarrow \psi )\). Entonces tenemos que \((\Sigma \cup \{\varphi \},\tau )\vdash (\varphi \rightarrow \psi ),\varphi \), lo cual por (2) nos dice que \((\Sigma \cup \{\varphi \},\tau )\vdash \psi \). Supongamos ahora que \((\Sigma \cup \{\varphi \},\tau )\vdash \psi \). Sea \((\varphi _{1}...\varphi _{n},J_{1}...,J_{n})\) una prueba de \(\psi \) en \((\Sigma \cup \{\varphi \},\tau )\). Notese que podemos suponer que \(J_{n}\) es de la forma \(P(\overline{l_{1}},...,\overline{l_{k}})\) . Definimos \(\widetilde{J_{i}}=\) \(\mathrm{TESIS}\bar{m}P(\overline{l_{1}+1} ,...,\overline{l_{k}+1}\), donde \(m\) es tal que ninguna \(J_{i}\) es igual a \( \mathrm{HIPOTESIS}\bar{m}\). Para cada \(i=1,...,n-1\), definamos \(\widetilde{ J_{i}}\) de la siguiente manera.

- Si \(\varphi _{i}=\varphi \) y \(J_{i}=\mathrm{AXIOMAPROPIO}\), entonces \(\widetilde{J_{i}}=\mathrm{EVOCACION}(1)\)
- Si \(\varphi _{i}\neq \varphi \) y \(J_{i}=\mathrm{AXIOMAPROPIO}\), entonces \(\widetilde{J_{i}}=\mathrm{AXIOMAPROPIO}\)
- Si \(J_{i}=\mathrm{AXIOMALOGICO}\), entonces \(\widetilde{J_{i}}= \mathrm{AXIOMALOGICO}\)
- Si \(J_{i}=\mathrm{CONCLUSION}\), entonces \(\widetilde{J_{i}}=\mathrm{ CONCLUSION}\)
- Si \(J_{i}=\mathrm{HIPOTESIS}\bar{k}\) entonces \(\widetilde{J_{i}}= \mathrm{HIPOTESIS}\bar{k}\)
- Si \(J_{i}=\alpha P(\overline{l_{1}},...,\overline{l_{k}})\), con \( \alpha \in \{\varepsilon \}\cup \{\mathrm{TESIS}\bar{k}:k\in \mathbf{N}\}\), entonces \(\widetilde{J_{i}}=\alpha P(\overline{l_{1}+1},...,\overline{l_{k}+1 })\)
Es facil chequear que

\(\displaystyle (\varphi \varphi _{1}...\varphi _{n}(\varphi \rightarrow \psi ),\text{ HIPOTESIS}\bar{m}\widetilde{J_{1}}...\widetilde{J_{n}}\text{CONCLUSION}) \)

es una prueba de \((\varphi \rightarrow \psi )\) en \((\Sigma ,\tau )\) \(\Box\)
9.1. El teorema de correccion

Sea \(\mathbf{A}\) un modelo de tipo \(\tau .\) Diremos que \(\mathbf{A}\) es un modelo de la teoria \((\Sigma ,\tau )\) si se cumple que \(\mathbf{A} \models \varphi \), para cada \(\varphi \in \Sigma .\) Escribiremos \((\Sigma ,\tau )\models \varphi \) cuando \(\varphi \) sea verdadera en todo modelo de \( (\Sigma ,\tau )\).

Como se ha visto en los ejemplos, el concepto de prueba que hemos dado sirve para formalizar las pruebas reales matematicas cuando las mismas son hechas dentro del lenguaje de primer orden. Deberiamos tener cuidado de que en el afan de lograr tal formalizabilidad no hayamos hecho una definicion de prueba demasiado permisiva. Este no es el caso ya que el siguiente teorema nos garantiza que los teoremas de una teoria de primer orden son verdaderos en cada modelo de la teoria.

Teorema 156 (Correccion) \((\Sigma ,\tau )\vdash \varphi \) implica \((\Sigma ,\tau )\models \varphi .\)
No daremos la prueba del teorema anterior ya que es dificultosa. Esto en alguna medida es el precio de tener una nocion de prueba muy expresiva.

Corolario 157 Si \((\Sigma ,\tau )\) tiene un modelo, entonces \((\Sigma ,\tau )\) es consistente.
Prueba: Supongamos \(\mathbf{A}\) es un modelo de \((\Sigma ,\tau ).\) Si \((\Sigma ,\tau )\) fuera inconsistente, tendriamos que hay una \(\varphi \in S^{t}\) tal que \( (\Sigma ,\tau )\vdash (\varphi \wedge \lnot \varphi )\), lo cual por el Teorema de Correccion nos diria que \(\mathbf{A}\models (\varphi \wedge \lnot \varphi )\) \(\Box\)

9.2. El algebra de Lindenbaum

Recordemos que dado un tipo \(\tau \), con \(S^{t}\) denotabamos el conjunto de las sentencias de tipo \(\tau \), es decir

\(\displaystyle S^{t}=\{\varphi \in F^{\tau }:Li(\varphi )=\varnothing \} \)

Sea \((\Sigma ,\tau )\) una teoria. Podemos definir la siguiente relacion sobre \(S^{\tau }\):
\(\displaystyle \varphi \dashv \vdash \psi \text{ si y solo si }(\Sigma ,\tau )\vdash \left( \varphi \leftrightarrow \psi \right) \)

Lema 158 \(\dashv \vdash \) es una relacion de equivalencia.
Prueba: La relacion es reflexiva ya que \((\varphi \leftrightarrow \varphi )\) es un axioma logico. Veamos que es simetrica. Supongamos que \(\varphi \dashv \vdash \psi \), es decir \((\Sigma ,\tau )\vdash \left( \varphi \leftrightarrow \psi \right) \). Ya que \(((\varphi \leftrightarrow \psi )\leftrightarrow (\psi \leftrightarrow \varphi ))\) es un axioma logico, tenemos que \((\Sigma ,\tau )\vdash ((\varphi \leftrightarrow \psi )\leftrightarrow (\psi \leftrightarrow \varphi ))\). Notese que \(\left( \psi \leftrightarrow \varphi \right) \) se deduce de \(((\varphi \leftrightarrow \psi )\leftrightarrow (\psi \leftrightarrow \varphi ))\) y \((\varphi \leftrightarrow \psi )\) por la regla de reemplazo, lo cual por (2) del Lema 155 nos dice que \((\Sigma ,\tau )\vdash \left( \psi \leftrightarrow \varphi \right) \).

Analogamente, usando la regla de transitividad se puede probar que \(\dashv \vdash \) es transitiva. \(\Box\)

Dada una teoria \((\Sigma ,\tau )\) y \(\varphi \in S^{\tau }\), \([\varphi ]\) denotara la clase de \(\varphi \) con respecto a la relacion de equivalencia \( \dashv \vdash \). Definamos la siguiente relacion binaria sobre \(S^{\tau }/ \mathrm{\dashv \vdash }\):

\(\displaystyle \lbrack \varphi ]\leq \lbrack \psi ]\text{ si y solo si }(\Sigma ,\tau )\vdash \left( \varphi \rightarrow \psi \right) . \)

Dejamos al lector verificar que la definicion es inambigua, es decir que es cierta la siguiente propiedad:
- Si \([\varphi ]=[\varphi ^{\prime }]\) y \([\psi ]=[\psi ^{\prime }]\) entonces se da \((\Sigma ,\tau )\vdash \left( \varphi \rightarrow \psi \right) \) si y solo si se da \((\Sigma ,\tau )\vdash \left( \varphi ^{\prime }\rightarrow \psi ^{\prime }\right) \)
Lema 159 Dada una teoria \((\Sigma ,\tau )\), el par \((S^{\tau }/\mathrm{\dashv \vdash } ,\leq )\) es un reticulado en el cual:
\(\displaystyle \begin{array}{rcl} \lbrack \varphi ]\;\mathsf{s\;}[\psi ] & =& [(\varphi \vee \psi )] \\ \lbrack \varphi ]\;\mathsf{i\;}[\psi ] & =& [(\varphi \wedge \psi )] \end{array} \)
Mas aun \((S^{\tau }/\mathrm{\dashv \vdash },\mathsf{s},\mathsf{i})\) es distributivo.
Prueba: Primero que todo deberemos verificar que \(\leq \) es un orden parcial sobre \( S^{\tau }/\dashv \vdash \). Veamos que \(\leq \) es antisimetrica las otras dos propiedades son dejadas al lector. Supongamos que \([\varphi ]\leq \lbrack \psi ]\) y \([\psi ]\leq \lbrack \varphi ].\) Es decir que \((\Sigma ,\tau )\vdash \left( \varphi \rightarrow \psi \right) \), \(\left( \psi \rightarrow \varphi \right) .\) Notese que

\(\displaystyle \begin{array}{llll} 1.\; & \left( \varphi \rightarrow \psi \right) & & \text{AXIOMAPROPIO} \\ 2.\; & \left( \psi \rightarrow \varphi \right) & & \text{AXIOMAPROPIO} \\ 3.\; & ((\varphi \rightarrow \psi )\wedge (\psi \rightarrow \varphi )) & & \text{CONJUNCIONINTRODUCCION}(1,2) \\ 4.\; & (\varphi \leftrightarrow \psi )\leftrightarrow ((\varphi \rightarrow \psi )\wedge (\psi \rightarrow \varphi )) & & \text{AXIOMALOGICO} \\ 5.\; & (\varphi \leftrightarrow \psi ) & & \text{REEMPLAZO}(3,4) \end{array} \)

justifica que \((\{\left( \varphi \rightarrow \psi \right) ,\left( \psi \rightarrow \varphi \right) \},\tau )\vdash (\varphi \leftrightarrow \psi )\) , lo cual por el Lema 155 nos dice que \((\Sigma ,\tau )\vdash (\varphi \leftrightarrow \psi )\), obteniendo \([\varphi ]=[\psi ].\)
Veamos ahora que \([(\varphi \vee \psi )]=[\varphi ]\;\mathsf{s\;}[\psi ].\) Es facil probar que \([\varphi ],[\psi ]\leq \lbrack (\varphi \vee \psi )].\) Supongamos que \([\varphi ],[\psi ]\leq \lbrack \alpha ].\) Es decir que \( (\Sigma ,\tau )\vdash (\varphi \rightarrow \alpha ),(\psi \rightarrow \alpha ).\) Notese que

\(\displaystyle \begin{array}{llll} 1.\; & (\varphi \rightarrow \alpha ) & & \text{AXIOMAPROPIO} \\ 2.\; & (\psi \rightarrow \alpha ) & & \text{AXIOMAPROPIO} \\ 3.\; & (\varphi \vee \psi )\rightarrow \alpha & & \text{ CONJUNCIONINTRODUCCION}(1,2) \end{array} \)

justifica que \((\{(\varphi \rightarrow \alpha ),(\psi \rightarrow \alpha )\},\tau )\vdash (\varphi \vee \psi )\rightarrow \alpha \) lo cual por el Lema 155 nos dice que \((\Sigma ,\tau )\vdash (\varphi \vee \psi )\rightarrow \alpha \), obteniendo que \([(\varphi \vee \psi )]\leq \lbrack \alpha ].\)
Veamos que el reticulado \((S^{\tau }/\mathrm{\dashv \vdash },\mathsf{s}, \mathsf{i})\) es distributivo. Sean \(\varphi ,\psi ,\varphi \in S^{\tau }.\) Ya que

\(\displaystyle (\varphi \wedge (\psi \vee \varphi ))\leftrightarrow ((\varphi \wedge \psi )\vee (\varphi \wedge \varphi )) \)

es un axioma logico, tenemos que
\(\displaystyle \lbrack (\varphi \wedge (\psi \vee \varphi ))]=[((\varphi \wedge \psi )\vee (\varphi \wedge \varphi ))]. \)

Se tiene entonces
\(\displaystyle \begin{array}{lll} \lbrack \varphi ]\;\mathsf{i}\;([\psi ]\;\mathsf{s}\;[\varphi ]) & = & [\varphi ]\;\mathsf{i}\;([(\psi \vee \varphi )]) \\ & = & [(\varphi \wedge (\psi \vee \varphi ))] \\ & = & [((\varphi \wedge \psi )\vee (\varphi \wedge \varphi ))] \\ & = & [(\varphi \wedge \psi )]\;\mathsf{s}\;[(\varphi \wedge \varphi )] \\ & = & ([\varphi ]\;\mathsf{i}\;[\psi ])\;\mathsf{s}\;([\varphi ]\;\mathsf{i} \;[\varphi ]) \end{array} \)

El resto de la prueba es dejado al lector. \(\Box\)
« Previous
1
2
3
4
5
6
7
8
9
10
11
12
13
14
15
16
17
18
19
20
21
22
23
24
25
26
27
28
29
30
» Next
×
Lenguaje \(\mathcal{S}^{\Sigma }\)

Entorno para trabajar con el lenguaje \(\mathcal{S}^{\Sigma }\) creado por Gabriel Cerceau:

Descargar!
Close
