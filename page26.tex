Gran Logico

Gran Lógico
Lenguaje \(\mathcal{S}^{\Sigma }\)
Apunte
Contacto
Login
« Previous
1
2
3
4
5
6
7
8
9
10
11
12
13
14
15
16
17
18
19
20
21
22
23
24
25
26
27
28
29
30
» Next
Una sentencia \(\varphi \) se dice refutable en \((\Sigma ,\tau )\) si \( (\Sigma ,\tau )\vdash \lnot \varphi \).

Lema 160 El poset \((S^{\tau }/\mathrm{\dashv \vdash },\leq )\) tiene \(0\) y \(1\) dados por
\(\displaystyle \begin{array}{lll} 0 & = & \{\varphi \in S^{\tau }:\varphi \text{ es refutable en }(\Sigma ,\tau )\} \\ & = & [\varphi ]\text{, para cada }\varphi \text{ refutable} \\ & & \\ 1 & = & \{\varphi \in S^{\tau }:\varphi \text{ es refutable en }(\Sigma ,\tau )\} \\ & = & [\varphi ]\text{, para cada teorema }\varphi \end{array} \)
Prueba: Veamos que para cada \(\varphi \) tal que \((\Sigma ,\tau )\vdash \lnot \varphi ,\)

\(\displaystyle \lbrack \varphi ]=\{\varphi \in S^{\tau }:(\Sigma ,\tau )\vdash \lnot \varphi \} \)

La inclusion \(\subseteq \) es facil. Supongamos ahora \(\psi \in \{\varphi \in S^{\tau }:(\Sigma ,\tau )\vdash \lnot \varphi \}.\) Es decir que \((\Sigma ,\tau )\vdash \lnot \psi .\) Notese que
\(\displaystyle \begin{array}{llll} 1.\; & \lnot \psi & & \text{AXIOMAPROPIO} \\ 2.\; & \left( \psi \rightarrow \varphi \right) & & \text{ABSURDO}(1) \\ 3.\; & \lnot \varphi & & \text{AXIOMAPROPIO} \\ 4.\; & (\varphi \rightarrow \psi ) & & \text{ABSURDO}(3) \\ 5. & ((\varphi \rightarrow \psi )\wedge (\psi \rightarrow \varphi )) & & \text{CONJUNCIONINTRODUCCION}(5,2) \\ 6. & (\varphi \leftrightarrow \psi )\leftrightarrow ((\varphi \rightarrow \psi )\wedge (\psi \rightarrow \varphi )) & & \text{AXIOMALOGICO} \\ 7. & (\varphi \leftrightarrow \psi ) & & \text{REEMPLAZO}(6,7) \end{array} \)

justifica que \((\{\lnot \varphi ,\lnot \psi \},\tau )\vdash (\varphi \leftrightarrow \psi )\) lo cual por el Lema 155 nos dice que \((\Sigma ,\tau )\vdash (\varphi \leftrightarrow \psi )\), obteniendo que \( \psi \in \lbrack \varphi ].\)
Es facil ver que \(\{\varphi \in S^{\tau }:(\Sigma ,\tau )\vdash \lnot \varphi \}\) es un \(0\) del poset \((S^{\tau }/\mathrm{\dashv \vdash },\leq ).\) Dejamos al lector la prueba analoga de que para cada \(\varphi \) tal que \( (\Sigma ,\tau )\vdash \varphi ,\)

\(\displaystyle \{\varphi \in S^{\tau }:(\Sigma ,\tau )\vdash \varphi \}=[\varphi ], \)

es un \(1\) del poset \((S^{\tau }/\mathrm{\dashv \vdash },\leq ).\) \(\Box\)
Los lemas anteriores garantizan que \((S^{\tau }/\mathrm{\dashv \vdash }, \mathsf{s},\mathsf{i},0,1)\) es un reticulado acotado. Definamos una operacion

\(\displaystyle \begin{array}{rcl} ^{c}:S^{\tau }/\mathrm{\dashv \vdash } & \rightarrow & S^{\tau }/\mathrm{ \dashv \vdash } \\ \lbrack \varphi ] & \rightarrow & [\varphi ]^{c} \end{array} \)

de la siguiente manera
\(\displaystyle \lbrack \varphi ]^{c}=[\lnot \varphi ]. \)

Dejamos al lector verificar que la definicion de \(^{c}\) es inambigua.
Lema 161 \((S^{\tau }/\mathrm{\dashv \vdash },\mathsf{s},\mathsf{i},^{c},0,1)\) es un algebra de Boole
Prueba: En virtud de los lemas anteriores solo falta probar que

\(\displaystyle \begin{array}{rcl} \lbrack \varphi ]\;\mathsf{s}\;[\varphi ]^{c} & =& 1 \\ \lbrack \varphi ]\;\mathsf{i}\;[\varphi ]^{c} & =& 0 \end{array} \)

Dejamos al lector la prueba de estas igualdades. \(\Box\)
Denotaremos con \(\mathcal{A}_{(\Sigma ,\tau )}\) al algebra de Boole \( (S^{\tau }/\mathrm{\dashv \vdash },\mathsf{s},\mathsf{i},^{c},0,1).\) El algebra \(\mathcal{A}_{(\Sigma ,\tau )}\) sera llamada el algebra de Lindenbaum de la teoria \((\Sigma ,\tau )\)

Lema 162 Sean \(\tau =(\mathcal{C},\mathcal{F},\mathcal{R},a)\) y \(\tau ^{\prime }=(\mathcal{C}^{\prime },\mathcal{F}^{\prime },\mathcal{R} ^{\prime },a^{\prime })\) tipos.
(1) Si \(\mathcal{C}\subseteq \mathcal{C}^{\prime }\), \(\mathcal{F} \subseteq \mathcal{F}^{\prime }\), \(\mathcal{R}\subseteq \mathcal{R}^{\prime } \) y \(a^{\prime }\mid _{\mathcal{F}\cup \mathcal{R}}=a\), entonces \((\Sigma ,\tau )\vdash \varphi \) implica \((\Sigma ,\tau ^{\prime })\vdash \varphi \)
(2) Si \(\mathcal{C}\subseteq \mathcal{C}^{\prime }\), \(\mathcal{F}= \mathcal{F}^{\prime }\), \(\mathcal{R}=\mathcal{R}^{\prime }\) y \(a^{\prime }=a\) , entonces \((\Sigma ,\tau ^{\prime })\vdash \varphi \) implica \((\Sigma ,\tau )\vdash \varphi \), cada vez que \(\Sigma \cup \{\varphi \}\subseteq S^{\tau }. \)
Prueba: (1) Supongamos \((\Sigma ,\tau )\vdash \varphi \). Entonces hay una prueba \( (\varphi _{1}...\varphi _{n},J_{1}...J_{n})\) de \(\varphi \) en \((\Sigma ,\tau )\). Sea \(\mathcal{C}_{1}\) el conjunto de nombres de constante que ocurren en alguna \(\varphi _{i}\) y que no pertenecen a \(\mathcal{C}.\) Notese que aplicando varias veces el Lema 154 podemos obtener una prueba \( (\tilde{\varphi}_{1}...\tilde{\varphi}_{n},J_{1}...J_{n})\) de \(\varphi \) en \( (\Sigma ,\tau )\) la cual cumple que los nombres de constante que ocurren en alguna \(\psi _{i}\) y que no pertenecen a \(\mathcal{C}\) no pertenecen a \( \mathcal{C}^{\prime }\). Pero entonces \((\tilde{\varphi}_{1}...\tilde{\varphi} _{n},J_{1}...J_{n})\) es una prueba de \(\varphi \) en \((\Sigma ,\tau ^{\prime })\), con lo cual \((\Sigma ,\tau ^{\prime })\vdash \varphi \)

(2) Supongamos \((\Sigma ,\tau ^{\prime })\vdash \varphi \). Entonces hay una prueba \((\mathbf{\varphi },\mathbf{J})\) de \(\varphi \) en \((\Sigma ,\tau ^{\prime })\). Veremos que \((\mathbf{\varphi },\mathbf{J})\) es una prueba de \( \varphi \) en \((\Sigma ,\tau )\). Ya que \((\mathbf{\varphi },\mathbf{J})\) es una prueba de \(\varphi \) en \((\Sigma ,\tau ^{\prime })\) hay un conjunto finito \(\mathcal{C}_{1}\), disjunto con \(\mathcal{C}^{\prime }\), tal que \(( \mathcal{C}^{\prime }\cup \mathcal{C}_{1},\mathcal{F},\mathcal{R},a)\) es un tipo y cada \(\mathbf{\varphi }_{i}\) es una sentencia de tipo \((\mathcal{C} ^{\prime }\cup \mathcal{C}_{1},\mathcal{F},\mathcal{R},a)\). Notese que \( \widetilde{\mathcal{C}_{1}}=\mathcal{C}_{1}\cup (\mathcal{C}^{\prime }- \mathcal{C})\) es tal que \((\mathcal{C}\cup \widetilde{\mathcal{C}_{1}}, \mathcal{F},\mathcal{R},a)\) es un tipo y cada \(\mathbf{\varphi }_{i}\) es una sentencia de tipo \((\mathcal{C}\cup \widetilde{\mathcal{C}_{1}},\mathcal{F}, \mathcal{R},a)\), con lo cual \((\mathbf{\varphi },\mathbf{J})\) cunple el punto 1. de la definicion de prueba. Todos los otros puntos se cumplen en forma directa, exepto los puntos 4(f) y 4(g)i para los cuales es necesario notar que \(\mathcal{C}\subseteq \mathcal{C}^{\prime }\). \(\Box\)

Lema 163 Sea \((\Sigma ,\tau )\) una teoria y supongamos que \( \tau \) tiene una cantidad infinita de nombres de cte que no ocurren en las sentencias de \(\Sigma \). Entonces para cada formula \(\varphi =_{d}\varphi (v) \), se tiene que \([\forall v\varphi (v)]=\inf (\{[\varphi (t)]:t\) es un termino cerrado\(\})\).
Prueba: Primero notese que \([\forall v\;\varphi (v)]\leq \lbrack \varphi (t)]\), para todo termino cerrado \(t\), ya que podemos dar la siguiente prueba:

\(\displaystyle \begin{array}{cllll} 1. & \forall v\;\varphi (v) & & & \text{HIPOTESIS}1 \\ 2. & \varphi (t) & & & \text{TESIS}1\text{PARTICULARIZACION}(1) \\ 3. & (\forall v\;\varphi (v)\rightarrow \varphi (t)) & & & \text{CONCLUSION } \end{array} \)

Supongamos ahora que \([\psi ]\leq \lbrack \varphi (t)]\), para todo termino cerrado \(t.\) Por hipotesis hay un nombre de cte \(c\in \mathcal{C}\) el cual no ocurre en los elementos de \(\Sigma \cup \{\psi ,\varphi (v)\}.\) Ya que \( [\psi ]\leq \lbrack \varphi (c)]\), hay una prueba \((\varphi _{1}...\varphi _{n},J_{1}...J_{n})\) de \(\left( \psi \rightarrow \varphi (c)\right) \) en \( (\Sigma ,\tau )\). Pero entonces es facil de chequear que la siguiente es una prueba en \((\Sigma ,(\mathcal{C}-\{c\},\mathcal{F},\mathcal{R},a))\) de \( \left( \psi \rightarrow \forall v\;\varphi (v)\right) \):
\(\displaystyle \begin{array}{rlcl} 1. & \varphi _{1} & & J_{1} \\ 2. & \varphi _{2} & & J_{2} \\ \vdots & \vdots & & \vdots \\ n. & \varphi _{n}=\left( \psi \rightarrow \varphi (c)\right) & & J_{n} \\ n+1. & \psi & & \text{HIPOTESIS}\bar{m} \\ n+2. & \varphi (c) & & \text{MODUSPONENS}(\bar{n},\overline{n+1}) \\ n+3. & \forall v\varphi (v) & & \text{TESIS}\bar{m}\text{GENERALIZACION}( \overline{n+2}) \\ n+4. & \left( \psi \rightarrow \forall v\varphi (v)\right) & & \text{ CONCLUSION} \end{array} \)

(con \(m\) elejido suficientemente grande). Por el Lema 162 tenemos entonces que \((\Sigma ,\tau )\vdash \left( \psi \rightarrow \forall v\;\varphi (v)\right) \) \(\Box\)
9.3. Teorema de completitud

A continuacion probaremos un teorema muy importante de la logica de primer orden, el cual nos asegura que nuestro concepto de prueba es lo suficientemente fuerte como para probar toda sentencia la cual sea verdadera en cada uno de los modelos de la teoria en cuestion.

Lema 164 (de Coincidencia): Sean \(\tau \) y \(\tau ^{\prime }\) dos tipos cualesquiera y sea \(\tau _{\cap }\) dado por \(\mathcal{C}_{\cap }=\mathcal{C}\cap \mathcal{C} ^{\prime }\), \(\mathcal{F}_{\cap }=\{f\in \mathcal{F}\cap \mathcal{F}^{\prime }:a(f)=a^{\prime }(f)\}\), \(\mathcal{R}_{\cap }=\{r\in \mathcal{R}\cap \mathcal{R}^{\prime }:a(r)=a^{\prime }(r)\}\) y \(a_{\cap }=a\mid _{\mathcal{F} _{\cap }\cup \mathcal{R}_{\cap }}\). Sean \(\mathbf{A}\) y \(\mathbf{A}^{\prime } \) modelos de tipo \(\tau \) y \(\tau ^{\prime }\) respectivamente. Supongamos que \(A=A^{\prime }\) y que \(c^{\mathbf{A}}=c^{\mathbf{A}^{\prime }}\), para cada \(c\in \mathcal{C}_{\cap }\), \(f^{\mathbf{A}}=f^{\mathbf{A}^{\prime }}\), para cada \(f\in \mathcal{F}_{\cap }\) y \(r^{\mathbf{A}}=r^{\mathbf{A}^{\prime }}\), para cada \(r\in \mathcal{R}_{\cap }\). Entonces
(a) Para cada \(t=_{d}t(\vec{v})\in T^{\tau _{\cap }}\) se tiene que \( t^{\mathbf{A}}[\vec{a}]=t^{\mathbf{A}^{\prime }}[\vec{a}]\), para cada \(\vec{a }\in A^{n}\)
(b) Para cada \(\varphi =_{d}\varphi (\vec{v})\in F^{\tau _{\cap }}\) se tiene que
\(\displaystyle \mathbf{A}\models \varphi \lbrack \vec{a}]\text{ si y solo si }\mathbf{A} ^{\prime }\models \varphi \lbrack \vec{a}]\text{.} \)
(c) Si \(\Sigma \cup \{\varphi \}\subseteq S^{\tau _{\cap }}\), entonces
\(\displaystyle (\Sigma ,\tau )\models \varphi \text{ sii }(\Sigma ,\tau ^{\prime })\models \varphi \text{.} \)
Prueba: (a) y (b) son directos por induccion.

(c) Supongamos que \((\Sigma ,\tau )\models \varphi \). Sea \(\mathbf{A} ^{\prime }\) un modelo de \(\tau ^{\prime }\) tal que \(\mathbf{A}^{\prime }\models \Sigma \). Sea \(a\in A^{\prime }\) un elemento fijo. Sea \(\mathbf{A}\) el modelo de tipo \(\tau \) definido de la siguiente manera

- universo de \(\mathbf{A}=\) \(A^{\prime }\)
- \(c^{\mathbf{A}}=c^{\mathbf{A}^{\prime }}\), para cada \(c\in \mathcal{ C}_{\cap }\),
- \(f^{\mathbf{A}}=f^{\mathbf{A}^{\prime }}\), para cada \(f\in \mathcal{ F}_{\cap }\)
- \(r^{\mathbf{A}}=r^{\mathbf{A}^{\prime }}\), para cada \(r\in \mathcal{ R}_{\cap }\)
- \(c^{\mathbf{A}}=a\), para cada \(c\in \mathcal{C}-\widetilde{\mathcal{ C}}\)
- \(f^{\mathbf{A}}(a_{1},...,a_{a(f)})=a\), para cada \(f\in \mathcal{F}- \mathcal{F}_{\cap }\), \(a_{1},...,a_{a(f)}\in A^{\prime }\)
- \(r^{\mathbf{A}}=\varnothing \), para cada \(r\in \mathcal{R-R}_{\cap }\)
Ya que \(\mathbf{A}^{\prime }\models \Sigma \), (b) nos dice que \( \mathbf{A}\models \Sigma \), lo cual nos dice que \(\mathbf{A}\models \varphi \) . Nuevamente por (b) tenemos que \(\mathbf{A}^{\prime }\models \varphi \), con lo cual hemos probado que \((\Sigma ,\tau ^{\prime })\models \varphi \) \(\Box\)

Ejercicio: Sean \(\tau \) y \(\tau ^{\prime }\) dos tipos cualesquiera y sea \(\tau _{\cap }\) definido como en el lema anterior. Pruebe que \(T^{\tau _{\cap }}=T^{\tau }\cap T^{\tau ^{\prime }}\) y \(F^{\tau _{\cap }}=F^{\tau }\cap F^{\tau ^{\prime }}\)

Lema 165 Sea \(\tau \) un tipo. Hay una sucesion de formulas
\(\displaystyle \gamma _{1},\gamma _{2},... \)

tal que:
(1) \(\left\vert Li(\gamma _{j})\right\vert \leq 1\), para cada \( j=1,2,...\)
(2) si \(\left\vert Li(\gamma )\right\vert \leq 1\), entonces \(\gamma =\gamma _{j}\), para algun \(j\in \mathbf{N}\)
Prueba: Notese que las formulas de tipo \(\tau \) son palabras de algun alfabeto finito \(A\). Dado un orden total estricto \(< \) para \(A\), podemos definir

\(\displaystyle \begin{array}{rcl} \gamma _{1} & =& \min\nolimits_{\alpha }^{< }\left( \alpha \in F^{\tau }\wedge \left\vert Li(\alpha )\right\vert \leq 1\right) \\ \gamma _{t+1} & =& \min\nolimits_{\alpha }^{< }\left( \alpha \in F^{\tau }\wedge \left\vert Li(\alpha )\right\vert \leq 1\wedge (\forall i\in \omega )_{i\leq t}\alpha \neq \gamma _{i}\right) \end{array} \)

Claramente esta sucesion cumple (1) y es facil ver que tambien se cumple la propiedad (2). \(\Box\)
Teorema 166 (Completitud) (Godel) \((\Sigma ,\tau )\models \varphi \) implica \( (\Sigma ,\tau )\vdash \varphi .\)
Prueba: Primero probaremos completitud para el caso en que \(\tau \) tiene una cantidad infinita de nombres de cte que no ocurren en las sentencias de \( \Sigma \). Lo probaremos por el absurdo, es decir supongamos que \(\varphi _{0} \) es tal que \((\Sigma ,\tau )\models \varphi _{0}\) y \((\Sigma ,\tau )\not\vdash \varphi _{0}.\) Notese que ya que \((\Sigma ,\tau )\not\vdash \varphi _{0}\), tenemos que \([\lnot \varphi _{0}]\not=0^{\mathcal{A}_{(\Sigma ,\tau )}}.\) Para cada \(j\in \mathbf{N}\), sea \(w_{j}\in Var\) tal que \( Li(\gamma _{j})\subseteq \{w_{j}\}\). Para cada \(j\), declaremos \(\gamma _{j}=_{d}\gamma _{j}(w_{j})\). Notese que por el Lema 163 tenemos que \(\inf \{[\gamma _{j}(t)]:t\in T_{c}^{\tau }\}=[\forall w_{j}\gamma _{j}(w_{j})]\), para cada \(j=1,2,...\). Por el Teorema de Rasiova y Sikorski tenemos que hay un filtro primo de \(\mathcal{A}_{(\Sigma ,\tau )}\) , \(\mathcal{U}\) el cual cumple:

(a) \([\lnot \varphi _{0}]\in \mathcal{U}\)
(b) para cada \(j\in \mathbf{N}\), \(\{[\gamma _{j}(t)]:t\in T_{c}^{\tau }\}\subseteq \mathcal{U}\) implica que \([\forall w_{j}\gamma _{j}(w_{j})]\in \mathcal{U}\)
Ya que la sucesion de las \(\gamma _{i}\) cubre todas las formulas con a lo sumo una variable libre, podemos reescribir la propiedad (b) de la siguiente manera

(b)\(^{\prime }\) para cada \(\varphi =_{d}\varphi (v)\in F^{\tau }\), si \(\{[\varphi (t)]:t\in T_{c}^{\tau }\}\subseteq \mathcal{U}\) entonces \( [\forall v\varphi (v)]\in \mathcal{U}\)
Definamos sobre \(T_{c}^{\tau }\) la siguiente relacion:

\(\displaystyle t\bowtie s\text{ si y solo si }[(t\equiv s)]\in \mathcal{U}\text{.} \)

Veamos entonces que:
(1) \(\bowtie \) es de equivalencia.
(2) Para cada \(\varphi =_{d}\varphi (v_{1},...,v_{n})\in F^{\tau }\), \( t_{1},...,t_{n},s_{1},...,s_{n}\in T_{c}^{\tau }\), si \(t_{1}\bowtie s_{1}\), \( t_{2}\bowtie s_{2}\), \(...\), \(t_{n}\bowtie s_{n}\), entonces \([\varphi (t_{1},...,t_{n})]\in \mathcal{U}\) si y solo si \([\varphi (s_{1},...,s_{n})]\in \mathcal{U}\).
(3) Para cada \(f\in \mathcal{F}_{n}\), \( t_{1},...,t_{n},s_{1},...,s_{n}\in T_{c}^{\tau }\),
\(\displaystyle t_{1}\bowtie s_{1},t_{2}\bowtie s_{2},...,\;t_{n}\bowtie s_{n}\text{ implica }f(t_{1},...,t_{n})\bowtie f(s_{1},...,s_{n}). \)

Probaremos (2). Notese que

\(\displaystyle (\Sigma ,\tau )\vdash \left( (t_{1}\equiv s_{1})\wedge (t_{2}\equiv s_{2})\wedge ...\wedge (t_{n}\equiv s_{n})\wedge \varphi (t_{1},...,t_{n})\right) \rightarrow \varphi (s_{1},...,s_{n}) \)

lo cual nos dice que
\(\displaystyle \lbrack (t_{1}\equiv s_{1})]\;\mathsf{i\;}[(t_{2}\equiv s_{2})]\;\mathsf{i\;} ...\;\mathsf{i\;}[(t_{n}\equiv s_{n})]\;\mathsf{i\;}[\varphi (t_{1},...,t_{n})]\leq \lbrack \varphi (s_{1},...,s_{n})] \)

de lo cual se desprende que
\(\displaystyle \lbrack \varphi (t_{1},...,t_{n})]\in \mathcal{U}\text{ implica }[\varphi (s_{1},...,s_{n})]\in \mathcal{U} \)

ya que \(\mathcal{U}\) es un filtro. La otra implicacion es analoga.
Para probar (3) podemos tomar \(\varphi =\left( f(v_{1},...,v_{n})\equiv f(s_{1},...,s_{n})\right) \) y aplicar (2).

Definamos ahora un modelo \(\mathbf{A}_{\mathcal{U}}\) de tipo \(\tau \) de la siguiente manera:

- Universo de \(\mathbf{A}_{\mathcal{U}}=T_{c}^{\tau }/\mathrm{\bowtie }\)
- \(f^{\mathbf{A}_{\mathcal{U}}}(t_{1}/\mathrm{\bowtie },...,t_{n}/ \mathrm{\bowtie })=f(t_{1},...,t_{n})/\mathrm{\bowtie }\), \(f\in \mathcal{F} _{n}\), \(t_{1},...,t_{n}\in T_{c}^{\tau }\;\)
- \(r^{\mathbf{A}_{\mathcal{U}}}=\{(t_{1}/\mathrm{\bowtie },...,t_{n}/ \mathrm{\bowtie }):[r(t_{1},...,t_{n})]\in \mathcal{U}\}\), \(r\in \mathcal{R} _{n}.\)
Notese que la definicion de \(f^{\mathbf{A}_{\mathcal{U}}}\) es inambigua por (3). Probaremos las siguientes propiedades basicas:

(4) Para cada \(t=_{d}t(v_{1},...,v_{n})\in T^{\tau }\), \( t_{1},...,t_{n}\in T_{c}^{\tau }\), tenemos que
\(\displaystyle t^{\mathbf{A}_{\mathcal{U}}}[t_{1}/\mathrm{\bowtie },...,t_{n}/\mathrm{ \bowtie }]=t(t_{1},...,t_{n})/\mathrm{\bowtie } \)

(5) Para cada \(\varphi =_{d}\varphi (v_{1},...,v_{n})\in F^{\tau }\), \( t_{1},...,t_{n}\in T_{c}^{\tau }\), tenemos que
\(\displaystyle \mathbf{A}_{\mathcal{U}}\models \varphi \lbrack t_{1}/\mathrm{\bowtie } ,...,t_{n}/\mathrm{\bowtie }]\text{ si y solo si }[\varphi (t_{1},...,t_{n})]\in \mathcal{U}. \)

La prueba de (4) es directa por induccion. Probaremos (5) por induccion en el \(k\) tal que \(\varphi \in F_{k}^{\tau }\). El caso \(k=0\) es dejado al lector. Supongamos (5) vale para \(\varphi \in F_{k}^{\tau }\). Sea \(\varphi =_{d}\varphi (v_{1},...,v_{n})\in F_{k+1}^{\tau }-F_{k}^{\tau }.\) Hay varios casos:

CASO \(\varphi (v_{1},...,v_{n})=\left( \varphi _{1}(v_{1},...,v_{n})\vee \varphi _{2}(v_{1},...,v_{n})\right) .\)

Tenemos

\(\displaystyle \begin{array}{c} \mathbf{A}_{\mathcal{U}}\models \varphi \lbrack t_{1}/\mathrm{\bowtie } ,...,t_{n}/\mathrm{\bowtie }] \\ \Updownarrow \\ \mathbf{A}_{\mathcal{U}}\models \varphi _{1}[t_{1}/\mathrm{\bowtie } ,...,t_{n}/\mathrm{\bowtie }]\text{ o }\mathbf{A}_{\mathcal{U}}\models \varphi _{2}[t_{1}/\mathrm{\bowtie },...,t_{n}/\mathrm{\bowtie }] \\ \Updownarrow \\ \lbrack \varphi _{1}(t_{1},...,t_{n})]\in \mathcal{U}\text{ o }[\varphi _{2}(t_{1},...,t_{n})]\in \mathcal{U} \\ \Updownarrow \\ \lbrack \varphi _{1}(t_{1},...,t_{n})]\ \mathsf{s\ }[\varphi _{2}(t_{1},...,t_{n})]\in \mathcal{U} \\ \Updownarrow \\ \lbrack \left( \varphi _{1}(t_{1},...,t_{n})\vee \varphi _{2}(t_{1},...,t_{n})\right) ]\in \mathcal{U} \\ \Updownarrow \\ \lbrack \varphi (t_{1},...,t_{n})]\in \mathcal{U}. \end{array} \)

CASO \(\varphi (v_{1},...,v_{n})=\forall v\varphi _{1}(v_{1},...,v_{n},v).\)
Tenemos

\(\displaystyle \begin{array}{c} \mathbf{A}_{\mathcal{U}}\models \varphi \lbrack t_{1}/\mathrm{\bowtie } ,...,t_{n}/\mathrm{\bowtie }] \\ \Updownarrow \\ \mathbf{A}_{\mathcal{U}}\models \varphi _{1}[t_{1}/\mathrm{\bowtie } ,...,t_{n}/\mathrm{\bowtie },t/\mathrm{\bowtie }]\text{, para todo }t\in T_{c}^{\tau } \\ \Updownarrow \\ \lbrack \varphi _{1}(t_{1},...,t_{n},t)]\in \mathcal{U}\text{, para todo } t\in T_{c}^{\tau } \\ \Updownarrow \\ \lbrack \forall v\varphi _{1}(t_{1},...,t_{n},v)]\in \mathcal{U} \\ \Updownarrow \\ \lbrack \varphi (t_{1},...,t_{n})]\in \mathcal{U}. \end{array} \)

CASO \(\varphi (v_{1},...,v_{n})=\exists v\varphi _{1}(v_{1},...,v_{n},v).\)
Tenemos

\(\displaystyle \begin{array}{c} \mathbf{A}_{\mathcal{U}}\models \varphi \lbrack t_{1}/\mathrm{\bowtie } ,...,t_{n}/\mathrm{\bowtie }] \\ \Updownarrow \\ \mathbf{A}_{\mathcal{U}}\models \varphi _{1}[t_{1}/\mathrm{\bowtie } ,...,t_{n}/\mathrm{\bowtie },t/\mathrm{\bowtie }]\text{, para algun }t\in T_{c}^{\tau } \\ \Updownarrow \\ \lbrack \varphi _{1}(t_{1},...,t_{n},t)]\in \mathcal{U}\text{, para algun } t\in T_{c}^{\tau } \\ \Updownarrow \\ \lbrack \varphi _{1}(t_{1},...,t_{n},t)]^{c}\not\in \mathcal{U}\text{, para algun }t\in T_{c}^{\tau } \\ \Updownarrow \\ \lbrack \lnot \varphi _{1}(t_{1},...,t_{n},t)]\not\in \mathcal{U}\text{, para algun }t\in T_{c}^{\tau } \\ \Updownarrow \\ \lbrack \forall v\;\lnot \varphi _{1}(t_{1},...,t_{n},v)]\not\in \mathcal{U} \\ \Updownarrow \\ \lbrack \forall v\;\lnot \varphi _{1}(t_{1},...,t_{n},v)]^{c}\in \mathcal{U} \\ \Updownarrow \\ \lbrack \lnot \forall v\;\lnot \varphi _{1}(t_{1},...,t_{n},v)]\in \mathcal{U } \\ \Updownarrow \\ \lbrack \varphi (t_{1},...,t_{n})]\in \mathcal{U}. \end{array} \)

Pero ahora notese que (5) en particular nos dice que para cada sentencia \( \psi \in S^{\tau }\), \(\mathbf{A}_{\mathcal{U}}\models \psi \) si y solo si \( [\psi ]\in \mathcal{U}.\) De esta forma llegamos a que \(\mathbf{A}_{\mathcal{U }}\models \Sigma \) y \(\mathbf{A}_{\mathcal{U}}\models \lnot \varphi _{0}\), lo cual contradice la suposicion de que \((\Sigma ,\tau )\models \varphi _{0}. \)
Ahora supongamos que \(\tau \) es cualquier tipo. Sean \(s_{1}\) y \(s_{2}\) un par de simbolos no pertenecientes a la lista

\(\displaystyle \forall \ \ \exists \ \ \lnot \ \ \vee \ \ \wedge \ \ \rightarrow \ \ \leftrightarrow \ \ (\ \ )\ \ ,\ \equiv \ \ \mathsf{X}\ \ \mathit{0}\ \ \mathit{1}\ \ ...\ \ \mathit{9}\ \ \mathbf{0}\ \ \mathbf{1}\ \ ...\ \ \mathbf{9} \)

y tales que ninguno ocurra en alguna palabra de \(\mathcal{C}\cup \mathcal{F} \cup \mathcal{R}.\) Si \((\Sigma ,\tau )\models \varphi \), entonces usando el Lema de Coincidencia se puede ver que \((\Sigma ,(\mathcal{C}\cup \{s_{1}s_{2}s_{1},s_{1}s_{2}s_{2}s_{1},...\},\mathcal{F},\mathcal{R} ,a))\models \varphi \), por lo cual
\(\displaystyle (\Sigma ,(\mathcal{C}\cup \{s_{1}s_{2}s_{1},s_{1}s_{2}s_{2}s_{1},...\}, \mathcal{F},\mathcal{R},a))\vdash \varphi . \)

Pero por Lema 162, tenemos que \((\Sigma ,\tau )\vdash \varphi .\) \(\Box\)
Corolario 167 Toda teoria consistente tiene un modelo.
Prueba: Supongamos \((\Sigma ,\tau )\) es consistente y no tiene modelos. Entonces \( (\Sigma ,\tau )\models \left( \varphi \wedge \lnot \varphi \right) \), con lo cual por completitud \((\Sigma ,\tau )\vdash \left( \varphi \wedge \lnot \varphi \right) \), lo cual es absurdo. \(\Box\)

Corolario 168 (Teorema de Compacidad)
(a) Si \((\Sigma ,\tau )\) es tal que \((\Sigma _{0},\tau )\) tiene un modelo, para cada subconjunto finito \(\Sigma _{0}\subseteq \Sigma \), entonces \((\Sigma ,\tau )\) tiene un modelo
(b) Si \((\Sigma ,\tau )\models \varphi \), entonces hay un subconjunto finito \(\Sigma _{0}\subseteq \Sigma \) tal que \((\Sigma _{0},\tau )\models \varphi \).
Prueba: (a) Si \((\Sigma ,\tau )\) fuera inconsistente habria un subconjunto finito \( \Sigma _{0}\subseteq \Sigma \) tal que la teoria \((\Sigma _{0},\tau )\) es inconsistente (\(\Sigma _{0}\) puede ser formado con los axiomas de \(\Sigma \) usados en una prueba que atestigue que \((\Sigma ,\tau )\vdash \left( \varphi \wedge \lnot \varphi \right) \)). O sea que \((\Sigma ,\tau )\) es consistente por lo cual tiene un modelo.

(b) Si \((\Sigma ,\tau )\models \varphi \), entonces por completitud, \((\Sigma ,\tau )\vdash \varphi \). Pero entonces hay un subconjunto finito \(\Sigma _{0}\subseteq \Sigma \) tal que \((\Sigma _{0},\tau )\vdash \varphi \), es decir tal que \((\Sigma _{0},\tau )\models \varphi \) (correccion). \(\Box\)

« Previous
1
2
3
4
5
6
7
8
9
10
11
12
13
14
15
16
17
18
19
20
21
22
23
24
25
26
27
28
29
30
» Next
×
Lenguaje \(\mathcal{S}^{\Sigma }\)

Entorno para trabajar con el lenguaje \(\mathcal{S}^{\Sigma }\) creado por Gabriel Cerceau:

Descargar!
Close
The JS beautifier will take care of your dirty JavaScript codes, assuring a syntax error-free solution.
