Gran Logico

Gran Lógico
Lenguaje \(\mathcal{S}^{\Sigma }\)
Apunte
Contacto
Login
« Previous
1
2
3
4
5
6
7
8
9
10
11
12
13
14
15
16
17
18
19
20
21
22
23
24
25
26
27
28
29
30
» Next
10. La aritmetica de Peano

A continuacion estudiaremos una teoria de primer orden la cual ha sido paradigmatica en el desarrollo de la logica. Sea \(\tau _{A}=(\{0,1\},\{+^{2},.^{2}\},\{\leq ^{2}\},a)\). Para cada \(\psi =_{d}\psi (v_{1},...,v_{n},v)\), formula de \(\tau _{A}\), sea \(Ind_{\psi }\) la siguiente sentencia

\(\displaystyle \forall v_{1}...\forall v_{n}\ ((\psi (\vec{v},0)\wedge \forall v\ (\psi ( \vec{v},v)\rightarrow \psi (\vec{v},v+1)))\rightarrow \forall v\ \psi (\vec{v },v)) \)

Sea \(\Sigma _{A}\) formado por todas las sentencias de la forma \(Ind_{\psi }\) junto con las siguientes
\(\forall x_{1}\forall x_{2}\forall x_{3}\;x_{1}+(x_{2}+x_{3})\equiv (x_{1}+x_{2})+x_{3}\)
\(\forall x_{1}\forall x_{2}\;x_{2}+x_{1}\equiv x_{1}+x_{2}\)
\(\forall x_{1}\forall x_{2}\forall x_{3}\;x_{1}.(x_{2}.x_{3})\equiv (x_{1}.x_{2}).x_{3}\)
\(\forall x_{1}\forall x_{2}\;x_{2}.x_{1}\equiv x_{1}.x_{2}\)
\(\forall x_{1}\;x_{1}+0\equiv x_{1}\)
\(\forall x_{1}\;x_{1}.0\equiv 0\)
\(\forall x_{1}\;x_{1}.1\equiv x_{1}\)
\(\forall x_{1}\forall x_{2}\forall x_{3}\;x_{1}.(x_{2}+x_{3})\equiv (x_{1}.x_{2})+(x_{1}.x_{3})\)
\(\forall x_{1}\forall x_{2}\forall x_{3}\;(x_{1}+x_{3}\equiv x_{2}+x_{3}\rightarrow x_{1}\equiv x_{2})\)
\(\forall x_{1}\;x_{1}\leq x_{1}\)
\(\forall x_{1}\forall x_{2}\forall x_{3}\;((x_{1}\leq x_{2}\wedge x_{2}\leq x_{3})\rightarrow x_{1}\leq x_{3})\)
\(\forall x_{1}\forall x_{2}\;((x_{1}\leq x_{2}\wedge x_{2}\leq x_{1})\rightarrow x_{1}\equiv x_{2})\)
\(\forall x_{1}\forall x_{2}\;(x_{1}\leq x_{2}\vee x_{2}\leq x_{1})\)
\(\forall x_{1}\forall x_{2}\;(x_{1}\leq x_{2}\leftrightarrow \exists x_{3}\;x_{2}\equiv x_{1}+x_{3})\)
\(0< 1\)
La teoria \((\Sigma _{A},\tau _{A})\) sera llamada Aritmetica de Peano y la denotaremos con \(Arit\). Denotemos con \(\mathbf{ \omega }\) a la estructura de tipo \(\tau _{A}\) que tiene a \(\omega \) como universo e interpreta los nombres de \(\tau _{A}\) en la manera usual.

Lema 175 \(\mathbf{\omega }\) es un modelo de \(Arit\)
Prueba: Sea \(\psi =_{d}\psi (v_{1},...,v_{n},v)\), f\'{o}rmula de \(\tau _{A}\). Veremos que \(\mathbf{\omega }\vDash Ind_{\psi }\). Sea

\(\displaystyle \varphi =((\psi (\vec{v},0)\wedge \forall v\ (\psi (\vec{v},v)\rightarrow \psi (\vec{v},v+1)))\rightarrow \forall v\ \psi (\vec{v},v)) \)

Declaremos \(\varphi =_{d}\varphi (v_{1},...,v_{n})\). Notese que \(\mathbf{ \omega }\vDash Ind_{\psi }\) si y solo si para cada \(a_{1},...,a_{n}\in \omega \) se tiene que \(\mathbf{\omega }\vDash \varphi \lbrack \vec{a}]\). Sean \(a_{1},...,a_{n}\in \omega \) fijos. Probaremos que \(\mathbf{\omega } \vDash \varphi \lbrack \vec{a}]\). Notar que si
\(\displaystyle \mathbf{\omega }\nvDash (\psi (\vec{v},0)\wedge \forall v\ (\psi (\vec{v} ,v)\rightarrow \psi (\vec{v},v+1)))[\vec{a}] \)

entonces \(\mathbf{\omega }\vDash \varphi \lbrack \vec{a}]\) por lo cual podemos hacer solo el caso en que
\(\displaystyle \mathbf{\omega }\vDash (\psi (\vec{v},0)\wedge \forall v\ (\psi (\vec{v} ,v)\rightarrow \psi (\vec{v},v+1)))[\vec{a}] \)

Sea \(S=\{a\in \omega :\mathbf{\omega }\vDash \psi (\vec{v},v)[\vec{a},a]\}\). Ya que \(\mathbf{\omega }\vDash \psi (\vec{v},0)[\vec{a}]\), es facil ver usando el lema de reemplazo que \(\mathbf{\omega }\vDash \psi (\vec{v},v)[ \vec{a},0]\), lo cual nos dice que \(0\in S\). Ya que \(\mathbf{\omega }\vDash (\forall v\ (\psi (\vec{v},v)\rightarrow \psi (\vec{v},v+1)))[\vec{a}]\), tenemos que
(1) Para cada \(a\in \omega \), si \(\mathbf{\omega }\vDash \psi (\vec{v} ,v)[\vec{a},a]\), entonces \(\mathbf{\omega }\vDash \psi (\vec{v},v+1)[\vec{a} ,a]\).
Pero por el lema de reemplazo, tenemos que \(\mathbf{\omega }\vDash \psi ( \vec{v},v+1)[\vec{a},a]\) sii \(\mathbf{\omega }\vDash \psi (\vec{v},v)[\vec{a} ,a+1]\), lo cual nos dice que

(2) Para cada \(a\in \omega \), si \(\mathbf{\omega }\vDash \psi (\vec{v} ,v)[\vec{a},a]\), entonces \(\mathbf{\omega }\vDash \psi (\vec{v},v)[\vec{a} ,a+1]\).
Ya que (2) nos dice que \(a\in S\) implica \(a+1\in S\), tenemos que \(S=\omega \) ya que \(0\in S\). Es decir que para cada \(a\in \omega \), se da que \(\mathbf{ \omega }\vDash \psi (\vec{v},v)[\vec{a},a]\) lo cual nos dice que \(\mathbf{ \omega }\vDash \forall v\ \psi (\vec{v},v)[\vec{a}]\).

Es rutina probar que \(\mathbf{\omega }\) satisface los otros 15 axiomas de \( Arit\). \(\Box\)

El modelo \(\mathbf{\omega }\) es llamado el modelo standard de \(Arit\) . Definamos el mapeo \(\widehat{\ \ \ }:\omega \rightarrow \{(\;)\;,\;+\;0\;1\}^{\ast }\) de la siguiente manera

\(\displaystyle \begin{array}{rcl} \widehat{0} & =& 0 \\ \widehat{1} & =& 1 \\ \widehat{n+1} & =& +(\widehat{n},1)\text{, para cada }n\geq 1 \end{array} \)


Proposición 176 Hay un modelo de \(Arit\) el cual no es isomorfo a \(\mathbf{\omega }\)
Prueba: Sea \(\tau =(\{0,1,\blacktriangle \},\{+^{2},.^{2}\},\{\leq ^{2}\},a)\) y sea \( \Sigma =\Sigma _{A}\cup \{\lnot (\widehat{n}\equiv \blacktriangle ):n\in \omega \}\). Por el Teorema de Compacidad la teoria \((\Sigma ,\tau )\) tiene un modelo \(\mathbf{A}=(A,i)\). Ya que

\(\displaystyle \mathbf{A}\vDash \lnot (\widehat{n}\equiv \blacktriangle )\text{, para cada } n\in \omega \)

tenemos que
\(\displaystyle i(\blacktriangle )\neq \widehat{n}^{\mathbf{A}}\text{, para cada }n\in \omega \)

Por el Lema de Coincidencia la estructura \(\mathbf{B}=(A,i\mid _{\{0,1,+,.,\leq \}})\) es un modelo de \(Arit\). Ademas dicho lema nos garantiza que \(\widehat{n}^{\mathbf{B}}=\widehat{n}^{\mathbf{A}}\), para cada \(n\in \omega \), por lo cual tenemos que
\(\displaystyle i(\blacktriangle )\neq \widehat{n}^{\mathbf{B}}\text{, para cada }n\in \omega \)

Veamos que \(\mathbf{B}\) no es isomorfo a \(\mathbf{\omega }\). Supongamos \( F:\omega \rightarrow A\) es un isomorfismo de \(\mathbf{\omega }\) en \(\mathbf{B }\). Es facil de probar por induccion en \(n\) que \(F(n)=\widehat{n}^{\mathbf{B} }\), para cada \(n\in \omega \). Pero esto produce un absurdo ya que nos dice que \(i(\blacktriangle )\) no esta en la imagen de \(F\). \(\Box\)
Ejercicio: Dado un modelo \(\mathbf{A}\) de \(Arit\) y elementos \( a,b\in A\), diremos que \(a\) divide a \(b\) en \(\mathbf{A}\) cuando haya un \(c\in A\) tal que \(b=.^{\mathbf{A}}(c,a).\) Un elemento \(a\in A\) sera llamado primo en \(\mathbf{A}\) si \(a\neq 1^{\mathbf{A} }\) y sus unicos divisores son \(1^{\mathbf{A}}\) y \(a\). Pruebe que hay un modelo de \(Arit\), \(\mathbf{A}\), en el cual hay infinitos primos no pertenecientes a \(\{\widehat{n}^{\mathbf{A}}:n\in \omega \}\).

Lema 177 Las siguientes sentencias son teoremas de la aritmetica de Peano:
(1) \(\forall x\;0\leq x\)
(2) \(\forall x\;(x\leq 0\rightarrow x\equiv 0)\)
(3) \(\forall x\forall y\;(x+y\equiv 0\rightarrow (x\equiv 0\wedge y\equiv 0))\)
(4) \(\forall x\;(\lnot (x\equiv 0)\rightarrow \exists z\ (x\equiv z+1))\)
(5) \(\forall x\forall y\;(x< y\rightarrow x+1\leq y)\)
(6) \(\forall x\forall y\;(x< y+1\rightarrow x\leq y)\)
(7) \(\forall x\forall y\;(x\leq y+1\rightarrow (x\leq y\vee x\equiv y+1))\)
(8) \(\forall x\forall y\;(\lnot y\equiv 0\rightarrow \exists q\exists r\;x\equiv q.y+r\wedge r< y)\)
Prueba:

\(\displaystyle \begin{array}{lllll} \;1. & x_{0}\leq 0 & & & \text{HIPOTESIS}1 \\ \;2. & \forall x\;0\leq x & & & \text{TEOREMA} \\ \;3. & 0\leq x_{0} & & & \text{PARTICULARIZACION}(2) \\ \;4. & x_{0}\leq 0\wedge 0\leq x_{0} & & & \text{CONJUNCIONINTRODUCCION} (1,3) \\ \;5. & \forall x_{1}\forall x_{2}\;((x_{1}\leq x_{2}\wedge x_{2}\leq x_{1})\rightarrow x_{1}\equiv x_{2}) & & & \text{AXIOMAPROPIO} \\ \;6. & \forall x_{2}\;((x_{0}\leq x_{2}\wedge x_{2}\leq x_{0})\rightarrow x_{0}\equiv x_{2}) & & & \text{PARTICULARIZACION}(5) \\ \;7. & ((x_{0}\leq 0\wedge 0\leq x_{0})\rightarrow x_{0}\equiv 0) & & & \text{PARTICULARIZACION}(6) \\ \;8. & x_{0}\equiv 0 & & & \text{TESIS}1\text{MODUSPONENS}(4,7) \\ \;9. & x_{0}\leq 0\rightarrow x_{0}\equiv 0 & & & \text{CONCLUSION} \\ 10. & \forall x\ (x\leq 0\rightarrow x\equiv 0) & & & \text{GENERALIZACION} (9) \end{array} \)

\(\displaystyle \begin{array}{lllll} \;1. & x_{0}+y_{0}\equiv 0 & & & \text{HIPOTESIS}1 \\ & (x_{0}+y_{0}\equiv 0\leftrightarrow 0\equiv x_{0}+y_{0}) & & & \\ & 0\equiv x_{0}+y_{0} & & & \\ \;2. & \exists u\ (0\equiv x_{0}+u) & & & \text{TEOREMA} \\ \;3. & \forall x_{1}\forall x_{2}\;(x_{1}\leq x_{2}\leftrightarrow \exists x_{3}\;x_{2}\equiv x_{1}+x_{3}) & & & \text{PARTICULARIZACION}(2) \\ \;4. & x_{0}\leq 0 & & & \text{CONJUNCIONINTRODUCCION}(1,3) \\ \;5. & \forall x_{1}\forall x_{2}\;((x_{1}\leq x_{2}\wedge x_{2}\leq x_{1})\rightarrow x_{1}\equiv x_{2}) & & & \text{AXIOMAPROPIO} \\ \;6. & \forall x_{2}\;((x_{0}\leq x_{2}\wedge x_{2}\leq x_{0})\rightarrow x_{0}\equiv x_{2}) & & & \text{PARTICULARIZACION}(5) \\ \;7. & ((x_{0}\leq 0\wedge 0\leq x_{0})\rightarrow x_{0}\equiv 0) & & & \text{PARTICULARIZACION}(6) \\ \;8. & x_{0}\equiv 0 & & & \text{TESIS}1\text{MODUSPONENS}(4,7) \\ \;9. & x_{0}\leq 0\rightarrow x_{0}\equiv 0 & & & \text{CONCLUSION} \\ 10. & \forall x\ (x\leq 0\rightarrow x\equiv 0) & & & \text{GENERALIZACION} (9) \end{array} \)

\(\Box\)
Lema 178 Sean \(n,m\in \omega \). Las siguientes sentencias son teoremas de la aritmetica de Peano:
(a) \((+(\widehat{n},\widehat{m})\equiv \widehat{n+m})\)
(b) \((.(\widehat{n},\widehat{m})\equiv \widehat{n.m})\)
(c) \(\forall x\;(x\leq \widehat{n}\rightarrow (x\equiv \widehat{0} \vee x\equiv \widehat{1}\vee ...\vee x\equiv \widehat{n}))\)
Lema 179 Para cada termino cerrado \(t\), tenemos que \(Arit\vdash (t\equiv \widehat{t^{ \mathbf{\omega }}})\)
Lema 180 Si \(\varphi \) es una sentencia atomica o negacion de atomica y \(\mathbf{ \omega }\models \varphi \), entonces \(Arit\vdash \varphi \).
Prueba: Hay dos casos. Supongamos \(\varphi =(t\equiv s)\), con \(t,s\) terminos cerrados. Ya que \(\mathbf{\omega }\models \varphi \), tenemos que \(t^{\mathbf{ \omega }}=s^{\mathbf{\omega }}\) y por lo tanto \(\widehat{t^{\mathbf{\omega }} }=\widehat{s^{\mathbf{\omega }}}\). Por el lema anterior tenemos que \( Arit\vdash (t\equiv \widehat{t^{\mathbf{\omega }}}),(s\equiv \widehat{s^{ \mathbf{\omega }}})\) lo cual, ya que \(\widehat{t^{\mathbf{\omega }}}\) y \( \widehat{s^{\mathbf{\omega }}}\) son el mismo termino nos dice por la regla de transitividad que \(Arit\vdash (t\equiv s)\). Supongamos \(\varphi =(t\leq s) \), con \(t,s\) terminos cerrados. Ya que \(\mathbf{\omega }\models \varphi \) , tenemos que \(t^{\mathbf{\omega }}\leq s^{\mathbf{\omega }}\) y por lo tanto hay un \(k\in \omega \) tal que \(t^{\mathbf{\omega }}+k=s^{\mathbf{\omega }}\). Se tiene entonces que \(\widehat{t^{\mathbf{\omega }}+k}=\widehat{s^{\mathbf{ \omega }}}\). Por el lema anterior tenemos que \(Arit\vdash +(\widehat{t^{ \mathbf{\omega }}},\widehat{k})\equiv \widehat{t^{\mathbf{\omega }}+k}\) lo cual nos dice que

\(\displaystyle Arit\vdash +(\widehat{t^{\mathbf{\omega }}},\widehat{k})\equiv \widehat{s^{ \mathbf{\omega }}} \)

Pero el lema anterior nos dice que
\(\displaystyle Arit\vdash (t\equiv \widehat{t^{\mathbf{\omega }}}),(s\equiv \widehat{s^{ \mathbf{\omega }}}) \)

y por lo tanto la regla de reemplazo nos asegura que \(Arit\vdash +(t, \widehat{k})\equiv s\). Ya que
\(\displaystyle \forall x_{1}\forall x_{2}\;(x_{1}\leq x_{2}\leftrightarrow \exists x_{3}\;x_{2}\equiv x_{1}+x_{3}) \)

es un axioma de \(Arit\), tenemos que \(Arit\vdash (t\leq s)\). \(\Box\)
El siguiente lema muestra que en \(Arit\) se pueden probar ciertas sentencias las cuales emulan el principio de induccion completa.

Lema 181 Sea \(\varphi =_{d}\varphi (\vec{v},v)\in F^{\tau _{A}}\). Supongamos \(v\) es sustituible por \(w\) en \(\varphi \) y \(w\notin \{v_{1},...,v_{n}\}.\) Entonces:
\(\displaystyle Arit\vdash \forall \vec{v}((\varphi (\vec{v},0)\wedge \forall v(\forall w(w< v\rightarrow \varphi (\vec{v},w))\rightarrow \varphi (\vec{v} ,v)))\rightarrow \forall v\varphi (\vec{v},v)) \)
Prueba: Sea \(\tilde{\varphi}=\forall w(w\leq v\rightarrow \varphi (\vec{v},w))\). Notar que \(\tilde{\varphi}=_{d}\tilde{\varphi}(\vec{v},v)\). Sea \( IndCom_{\varphi }\) la sentencia

\(\displaystyle \forall \vec{v}((\varphi (\vec{v},0)\wedge \forall v(\forall w(w< v\rightarrow \varphi (\vec{v},w))\rightarrow \varphi (\vec{v} ,v)))\rightarrow \forall v\varphi (\vec{v},v)) \)

Salvo por el uso de algunos teoremas simples y el uso simultaneo de las reglas de particularizacion y generalizacion, la siguiente es la prueba buscada
\(\displaystyle \begin{array}{lllll} \;1. & (\varphi (\vec{c},0)\wedge \forall v(\forall w(w< v\rightarrow \varphi (\vec{c},w))\rightarrow \varphi (\vec{c},v)) & & & \text{HIPOTESIS}1 \\ \;2. & \;\;\;w_{0}\leq 0 & & & \text{HIPOTESIS}2 \\ \;3. & \;\;\;\forall x\;(x\leq 0\rightarrow x\equiv 0) & & & \text{TEOREMA} \\ \;4. & \;\;\;w_{0}\leq 0\rightarrow w_{0}\equiv 0 & & & \text{ PARTICULARIZACION}(3) \\ \;5. & \;\;\;w_{0}\equiv 0 & & & \text{MODUSPONENS}(2,4) \\ \;6. & \;\;\;\varphi (\vec{c},0) & & & \text{CONJUNCIONELIMINACION}(1) \\ \;7. & \;\;\;\varphi (\vec{c},w_{0}) & & & \text{TESIS}2\text{REEMPLAZO} (5,6) \\ \;8. & w_{0}\leq 0\rightarrow \varphi (\vec{c},w_{0}) & & & \text{ CONCLUSION} \\ \;9. & \tilde{\varphi}(\vec{c},0) & & & \text{GENERALIZACION}(8) \\ 10. & \;\;\;\tilde{\varphi}(\vec{c},v_{0}) & & & \text{HIPOTESIS}3 \\ 11. & \;\;\;\;\;\;w_{0}< v_{0}+1 & & & \text{HIPOTESIS}4 \\ 12. & \;\;\;\;\;\;\forall x,y\;x< y+1\rightarrow x\leq y & & & \text{TEOREMA } \\ 13. & \;\;\;\;\;\;w_{0}< v_{0}+1\rightarrow w_{0}\leq v_{0} & & & \text{ PARTICULARIZACION}(12) \\ 14. & \;\;\;\;\;\;w_{0}\leq v_{0} & & & \text{MODUSPONENS}(11,13) \\ 15. & \;\;\;\;\;\;w_{0}\leq v_{0}\rightarrow \varphi (\vec{c},w_{0}) & & & \text{PARTICULARIZACION}(10) \\ 16. & \;\;\;\;\;\;\varphi (\vec{c},w_{0}) & & & \text{TESIS}4\text{ MODUSPONENS}(14,15) \\ 17. & \;\;\;w_{0}< v_{0}+1\rightarrow \varphi (\vec{c},w_{0}) & & & \text{ CONCLUSION} \\ 18. & \;\;\;\forall w\;w< v_{0}+1\rightarrow \varphi (\vec{c},w) & & & \text{GENERALIZACION}(17) \\ 19. & \;\;\;\forall v(\forall w(w< v\rightarrow \varphi (\vec{c} ,w))\rightarrow \varphi (\vec{c},v)) & & & \text{CONJUNCIONELIMINACION}(1) \\ 20. & \;\;\;(\forall w(w< v_{0}+1\rightarrow \varphi (\vec{c},w))\rightarrow \varphi (\vec{c},v_{0}+1)) & & & \text{PARTICULARIZACION}(19) \\ 21. & \;\;\;\varphi (\vec{c},v_{0}+1) & & & \text{MODUSPONENS}(18,20) \\ 22. & \;\;\;\;\;\;w_{0}\leq v_{0}+1 & & & \text{HIPOTESIS}5 \\ 23. & \;\;\;\;\;\;\forall x,y\;x\leq y+1\rightarrow (x\leq y\vee x\equiv y+1) & & & \text{TEOREMA} \\ 24. & \;\;\;\;\;\;w_{0}\leq v_{0}+1\rightarrow (w_{0}\leq v_{0}\vee w_{0}\equiv v_{0}+1) & & & \text{PARTICULARIZACION}(23) \\ 25. & \;\;\;\;\;\;(w_{0}\leq v_{0}\vee w_{0}\equiv v_{0}+1) & & & \text{ MODUSPONENS}(22,24) \\ 26. & \;\;\;\;\;\;w_{0}\leq v_{0}\rightarrow \varphi (\vec{c},w_{0}) & & & \text{PARTICULARIZACION}(10) \\ 27. & \;\;\;\;\;\;\;\;\;w_{0}\equiv v_{0}+1 & & & \text{HIPOTESIS}6 \\ 28. & \;\;\;\;\;\;\;\;\;\varphi (\vec{c},w_{0}) & & & \text{TESIS}6\text{ REEMPLAZO}(21,27) \\ 29. & \;\;\;w_{0}\equiv v_{0}+1\rightarrow \varphi (\vec{c},w_{0}) & & & \text{CONCLUSION} \\ 30. & \;\;\;\;\;\;\varphi (\vec{c},w_{0}) & & & \text{TESIS}5\text{ DISJUNCIONELIMINACION}(25,26,29) \\ 31. & \;\;\;w_{0}\leq v_{0}+1\rightarrow \varphi (\vec{c},w_{0}) & & & \text{CONCLUSION} \\ 32. & \;\;\;\tilde{\varphi}(\vec{c},v_{0}+1) & & & \text{TESIS}3\text{ GENERALIZACION}(31) \\ 33. & \tilde{\varphi}(\vec{c},v_{0})\rightarrow \tilde{\varphi}(\vec{c} ,v_{0}+1) & & & \text{CONCLUSION} \\ 34. & \forall v\tilde{\varphi}(\vec{c},v)\rightarrow \tilde{\varphi}(\vec{c} ,v+1) & & & \text{GENERALIZACION}(33) \\ 35. & \tilde{\varphi}(\vec{c},0)\wedge \forall v\tilde{\varphi}(\vec{c} ,v)\rightarrow \tilde{\varphi}(\vec{c},v+1) & & & \text{ CONJUNCIONINTRODUCCION}(9,34) \\ 36. & Ind_{\tilde{\varphi}} & & & \text{AXIOMAPROPIO} \\ 37. & (\tilde{\varphi}(\vec{c},0)\wedge \forall v(\tilde{\varphi}(\vec{c} ,v)\rightarrow \tilde{\varphi}(\vec{c},v+1))\rightarrow \forall v\tilde{ \varphi}(\vec{c},v) & & & \text{PARTICULARIZACION}(36) \\ 38. & \forall v\tilde{\varphi}(\vec{c},v) & & & \text{MODUSPONENS}(35,37) \\ 39. & \tilde{\varphi}(\vec{c},v_{0}) & & & \text{PARTICULARIZACION}(38) \\ 40. & v_{0}\leq v_{0}\rightarrow \varphi (\vec{c},v_{0}) & & & \text{ PARTICULARIZACION}(39) \\ 41. & \forall x\;x\leq x & & & \text{AXIOMAPROPIO} \\ 42. & v_{0}\leq v_{0} & & & \text{PARTICULARIZACION}(41) \\ 43. & \varphi (\vec{c},v_{0}) & & & \text{MODUSPONENS}(40,42) \\ 44. & \forall v\varphi (\vec{c},v) & & & \text{TESIS}1\text{GENERALIZACION} (43) \\ 45. & (\varphi (\vec{c},0)\wedge \forall v(\forall w(w< v\rightarrow \varphi ( \vec{c},w))\rightarrow \varphi (\vec{c},v)))\rightarrow \forall v\varphi ( \vec{c},v) & & & \text{CONCLUSION} \\ 46. & IndCom_{\varphi } & & & \text{GENERALIZACION}(45) \end{array} \)

\(\Box\)
« Previous
1
2
3
4
5
6
7
8
9
10
11
12
13
14
15
16
17
18
19
20
21
22
23
24
25
26
27
28
29
30
» Next
×
Lenguaje \(\mathcal{S}^{\Sigma }\)

Entorno para trabajar con el lenguaje \(\mathcal{S}^{\Sigma }\) creado por Gabriel Cerceau:

Descargar!
Close
