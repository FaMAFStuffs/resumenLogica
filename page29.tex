Gran Logico

Gran Lógico
Lenguaje \(\mathcal{S}^{\Sigma }\)
Apunte
Contacto
Login
« Previous
1
2
3
4
5
6
7
8
9
10
11
12
13
14
15
16
17
18
19
20
21
22
23
24
25
26
27
28
29
30
» Next
10.1. Analisis de recursividad del lenguaje de primer orden

En esta seccion estudiaremos la recursividad de la sintaxis del lenguaje de primer orden de la teoria \(Arit\). Analizaremos el concepto de prueba formal en una teoria de la forma \((\Sigma ,\tau _{A})\), donde \(\Sigma \) es un conjunto recursivamente enumerable. Para hacer mas concreto el tratamiento supondremos que los nombres de constante auxiliares en las pruebas formales estaran siempre en el conjunto

\(\displaystyle Aux=\{\triangle \Box \triangle ,\triangle \Box \Box \triangle ,\triangle \Box \Box \Box \triangle ,...\} \)

esto no afectara nuestro analisis ya que es claro que toda prueba formal de una teoria de la forma \((\Sigma ,\tau _{A})\) puede ser reemplazada por una que sus nombres de contante auxiliares esten en \(Aux\). Es decir que las sentencias involucradas en las pruebas formales que consideraremos seran sentencias de tipo \(\tau _{A}^{e}\) donde
\(\displaystyle \tau _{A}^{e}=(\{0,1\}\cup Aux,\{+^{2},.^{2}\},\{\leq \},a) \)

Sea \(A\) el alfabeto formado por los siguientes simbolos
\(\displaystyle \forall \ \ \exists \ \ \lnot \ \ \vee \ \ \wedge \ \ \rightarrow \ \ \leftrightarrow \ \ (\ \ )\ \ ,\ \ \equiv \ \ 0\ \ 1\ \ +\ \ .\ \ \leq \ \ \triangle \ \ \Box \ \ \mathsf{X}\ \ \mathit{0}\ \ \mathit{1}\ \ ...\ \ \mathit{9}\ \ \mathbf{0}\ \ \mathbf{1}\ \ ...\ \ \mathbf{9} \)

Notese que los simbolos del alfabeto \(A\) son justamente los simbolos que ocurren en las formulas de tipo \(\tau _{A}^{e}\).
Lema 182 Los conjuntos \(T^{\tau _{A}^{e}},F^{\tau _{A}^{e}},T^{\tau _{A}}\) y \(F^{\tau _{A}}\) son \(A\)-p.r.
Prueba: Veamos que \(T^{\tau _{A}^{e}}\) es \(A\)-p.r.. Fijemos un orden total estricto \( < \) sobre \(A\). Sea \(P=\lambda x[\ast ^{< }(x)\in T^{\tau _{A}^{e}}]\). Notese que \(P(0)=0\) y \(P(x+1)=1\) si y solo si se da alguna de las siguientes

- \(\ast ^{< }(x+1)\in \{0,1\}\cup Aux\)
- \((\exists u,v\in \omega )\ast ^{< }(x+1)=+(\mathrm{\ast }^{< }(u), \mathrm{\ast }^{< }(v))\wedge (P^{\downarrow }(x))_{u+1}\wedge (P^{\downarrow }(x))_{v+1}\)
- \((\exists u,v\in \omega )\ast ^{< }(x+1)=\mathrm{.}(\mathrm{\ast } ^{< }(u),\mathrm{\ast }^{< }(v))\wedge (P^{\downarrow }(x))_{u+1}\wedge (P^{\downarrow }(x))_{v+1}\)
Por el Lema 48 tenemos que \(P\) es \(A\)-p.r., por lo cual \(\chi _{T^{\tau _{A}^{e}}}=P\circ \#^{< }\) lo es. Notese que

\(\displaystyle t\in T^{\tau _{A}}\text{ sii }t\in T^{\tau _{A}^{e}}\wedge \triangle \text{ no ocurre en }t\wedge \Box \text{ no ocurre en }t \)

por lo cual \(T^{\tau _{A}}\) es \(A\)-p.r. \(\Box\)
Lema 183 Los siguientes predicados son \(A\)-p.r.
(1) \("v\) ocurre libremente en \(\varphi \) a partir de \(i":\omega \times Var\times F^{\tau _{A}^{e}}\rightarrow \omega \)
(2) \("v\in Li(\varphi )":Var\times F^{\tau _{A}^{e}}\rightarrow \omega \)
(3) \("v\) es sustituible por \(t\) en \(\varphi ":Var\times T^{\tau _{A}^{e}}\times F^{\tau _{A}^{e}}\rightarrow \omega \)
Prueba: (1). Veamos que \(P:\omega \times Var\times F^{\tau _{A}^{e}}\rightarrow \omega \), dado por

\(\displaystyle P(i,v,\varphi )=\left\{ \begin{array}{ccl} 1 & & \text{si }v\mathit{\ }\text{ocurre libremente en}\mathit{\ }\varphi \text{ a partir de }i \\ 0 & & \text{caso contrario} \end{array} \right. \)

es \(B\)-p.r.. Sea \(R:\omega \times Var\rightarrow \omega \) el predicado dado por \(R(x,v)=1\) si y solo si \(\ast ^{< }((x)_{1})\in F^{\tau _{A}^{e}}\) y \(v \mathit{\ }\)ocurre libremente en\(\mathit{\ }\ast ^{< }((x)_{1})\) a partir de \( (x)_{2}\). Sea \(\mathrm{Nex}=\{\wedge ,\vee ,\rightarrow ,\leftrightarrow \}\) . Notese que \(F_{0}^{\tau _{A}^{e}}\) es \(A\)-p.r. ya que
\(\displaystyle F_{0}^{\tau _{A}^{e}}=F^{\tau _{A}^{e}}\cap (A-\{\forall ,\exists ,\lnot ,\vee ,\wedge ,\rightarrow ,\leftrightarrow \})^{\ast } \)

Notese que \(R(0,v)=0\), para cada \(v\in Var\) y que \(R(x+1,v)=1\) si y solo si \( (x+1)_{2}\geq 1\) y se da alguna de las siguientes:
- \(\ast ^{< }((x+1)_{1})\in F_{0}^{\tau _{A}^{e}}\wedge v\) ocurre en \( \ast ^{< }((x+1)_{1})\) a partir de \((x+1)_{2}\)
- \((\exists \varphi _{1},\varphi _{2}\in F^{\tau _{A}^{e}})(\exists \eta \in \mathrm{Nex})\ast ^{< }((x+1)_{1})=(\varphi _{1}\eta \varphi _{2})\wedge \)
\(\ \ \ \ \ \ \ \ \ \ \ \ \ \ \ \ \ \ \ \ \ \ \ \ \ \left( (R^{\downarrow }(x,v))_{\left\langle \#^{< }(\varphi _{1}),(x+1)_{2}-1\right\rangle +1}\vee (R^{\downarrow }(x,v))_{\left\langle \#^{< }(\varphi _{2}),(x+1)_{2}-\left\vert (\varphi _{1}\eta \right\vert \right\rangle +1}\right) \)

- \((\exists \varphi _{1}\in F^{\tau _{A}^{e}})\ast ^{< }((x+1)_{1})=\lnot \varphi _{1}\wedge (R^{\downarrow }(x,v))_{\left\langle \#^{< }(\varphi _{1}),(x+1)_{2}-1\right\rangle +1}\)
- \((\exists \varphi _{1}\in F^{\tau _{A}^{e}})(\exists w\in Var)(Q\in \{\forall ,\exists \})\;w\neq v\wedge \)
\(\ \ \ \ \ \ \ \ \ \ \ \ \ \ \ \ \ \ \ \ \ \ \ \ \ \ \ \ \ \ \ \ \ \ \ \ \ \ast ^{< }((x+1)_{1})=Qw\varphi _{1}\wedge (R^{\downarrow }(x,v))_{\left\langle \#^{< }(\varphi _{1}),(x+1)_{2}-\left\vert (Qw\right\vert \right\rangle +1}\)

Es decir que por el Lema 48 tenemos que \(R\) es \(A\) -p.r.. Notese que para \((i,v,\varphi )\in \omega \times Var\times F^{\tau _{A}^{e}}\), tenemos \(P(i,v,\varphi )=R(\left\langle \#^{< }(\varphi ),i\right\rangle ,v)\). Ahora es facil obtener la funcion \(P\) haciendo composiciones adecuadas con \(R\). \(\Box\)

Dados \(v\in Var\) y \(t,s\in T^{\tau }\), usaremos \(\downarrow _{v}^{t}(s)\) para denotar el resultado de reemplazar simultaneamente cada ocurrencia de \( v \) en \(s\) por \(t\). Similarmente, si \(\varphi \in F^{\tau }\), usaremos \( \downarrow _{v}^{t}(\varphi )\) para denotar el resultado de reemplazar simultaneamente cada ocurrencia libre de \(v\) en \(\varphi \) por \(t\)

Lema 184 Las funciones \(\lambda svt[\downarrow _{v}^{t}(s)]\) y \(\lambda \varphi vt[\downarrow _{v}^{t}(\varphi )]\) son \(A\)-p.r..
Prueba: Sea \(< \) un orden total estricto sobre \(A\). Sea \(h:\omega \times Var\times T^{\tau _{A}^{e}}\rightarrow \omega \) dada por

\(\displaystyle h(x,v,t)=\left\{ \begin{array}{ccc} \#^{< }(\downarrow _{v}^{t}(\ast ^{< }(x))) & & \text{si }\ast ^{< }(x)\in T^{\tau _{A}^{e}} \\ 0 & & \text{caso contrario} \end{array} \right. \)

Sea \(P:\omega \times \omega \times Var\times T^{\tau _{A}^{e}}\times A^{\ast }\rightarrow \omega \) tal que \(P(A,x,v,t,\alpha )=1\) si y solo si se da alguna de las siguientes
- \(\ast ^{< }(x+1)\notin T^{\tau _{A}^{e}}\wedge \alpha =\varepsilon \)
- \(\ast ^{< }(x+1)=v\wedge \alpha =t\)
- \(\ast ^{< }(x+1)\in (\{0,1\}\cup Aux)-\{v\}\wedge \alpha =\ast ^{< }(x+1)\)
- \((\exists r,s\in T^{\tau _{A}^{e}})\ast ^{< }(x+1)=+(r,s)\wedge \alpha =+(\ast ^{< }((A)_{\#^{< }(r)+1}),\ast ^{< }((A)_{\#^{< }(s)+1}))\)
- \((\exists r,s\in T^{\tau _{A}^{e}})\ast ^{< }(x+1)=\mathrm{.} (r,s)\wedge \alpha =\mathrm{.}(\ast ^{< }((A)_{\#^{< }(r)+1}),\ast ^{< }((A)_{\#^{< }(s)+1}))\)
Notese que \(P(h^{\downarrow }(x,v,t),x,v,t,\alpha )=1\) si y solo si ya sea \(\ast ^{< }(x+1)\notin T^{\tau }\) y \(\alpha =\varepsilon \) o \(\ast ^{< }(x+1)\in T^{\tau }\) y \(\alpha =\mathrm{\downarrow }_{v}^{t}(\ast ^{< }(x+1))\). Tenemos entonces

\(\displaystyle \begin{array}{rcl} h(0,v,t) & =& 0 \\ h(x+1,v,t) & =& \#^{< }(\min_{\alpha }^{< }P(h^{\downarrow }(x,v,t),x,v,t,\alpha )), \end{array} \)

por lo cual el Lema 48 nos dice que \(h\) es \(A\)-p.r. Ahora es facil obtener la funcion \(\downarrow _{v}^{t}(s):T^{\tau _{A}^{e}}\times Var\times T^{\tau _{A}^{e}}\rightarrow T^{\tau _{A}^{e}}\) haciendo composiciones adecuadas con \(h\). \(\Box\)
Lema 185 El predicado \(R:F^{\tau _{A}^{e}}\times F^{\tau _{A}^{e}}\times F^{\tau _{A}^{e}}\times F^{\tau _{A}^{e}}\rightarrow \omega \) , dado por
\(\displaystyle R(\varphi ,\tilde{\varphi},\psi _{1},\psi _{2})=\left\{ \begin{array}{cccl} \begin{array}{c} 1 \\ \; \\ \ \end{array} & & & \begin{array}{cl} \text{si }\tilde{\varphi}= & \text{resultado de reemplazar algunas} \\ & \text{(posiblemente }0\text{) ocurrencias de} \\ & \psi _{1}\text{ en }\varphi \text{ por }\psi _{2} \end{array} \\ 0 & & & \text{ caso contrario} \end{array} \right. \)
es \(A\)-p.r..
Prueba: Sea \(\mathrm{Nex}=\{\wedge ,\vee ,\rightarrow ,\leftrightarrow \}\). Sea \(< \) un orden total estricto sobre \(A\). Notese que \(R(\varphi ,\tilde{\varphi} ,\psi _{1},\psi _{2})=1\) sii se da alguna de las siguientes

- \(\varphi =\tilde{\varphi}\)
- \((\varphi =\psi _{1}\wedge \tilde{\varphi}=\psi _{2})\)
- \((\exists \varphi _{1},\varphi _{2},\tilde{\varphi}_{1},\tilde{ \varphi}_{2}\in F^{\tau _{A}^{e}})(\exists \eta \in \mathrm{Nex})\varphi =(\varphi _{1}\eta \varphi _{2})\wedge \tilde{\varphi}=(\tilde{\varphi} _{1}\eta \tilde{\varphi}_{2})\wedge \)
\(\;\;\;\;\;\;\;\;\;\;\;\;\;\;\;\;\;\;\;\;\;\;\;\;\;\;\;\;\;\;\;\;\;\;\;R( \varphi _{1},\tilde{\varphi}_{1},\psi _{1},\psi _{2})\wedge R(\varphi _{2}, \tilde{\varphi}_{2},\psi _{1},\psi _{2})\)

- \((\exists \varphi _{1},\tilde{\varphi}_{1}\in F^{\tau _{A}^{e}})\varphi =\lnot \varphi _{1}\wedge \tilde{\varphi}=\lnot \tilde{ \varphi}_{1}\wedge R(\varphi _{1},\tilde{\varphi}_{1},\psi _{1},\psi _{2})\)
- \((\exists \varphi _{1},\tilde{\varphi}_{1}\in F^{\tau _{A}^{e}})(\exists v\in Var)(\exists Q\in \{\forall ,\exists \})\varphi =Qv\varphi _{1}\wedge \)
\(\ \ \ \ \ \ \ \ \ \ \ \ \ \ \ \ \ \ \ \ \ \ \ \ \ \ \ \ \ \ \ \ \ \ \ \ \ \ \ \ \ \ \ \ \ \ \ \ \ \ \ \ \ \ \ \ \ \ \ \ \ \ \tilde{\varphi}=Qv\tilde{ \varphi}_{1}\wedge R(\varphi _{1},\tilde{\varphi}_{1},\psi _{1},\psi _{2})\)

Se puede usar lo anterior para ver que \(R^{\prime }:\omega \times F^{\tau _{A}^{e}}\times F^{\tau _{A}^{e}}\rightarrow \omega \), dado por

\(\displaystyle R^{\prime }(x,\psi _{1},\psi _{2})=\left\{ \begin{array}{cc} \begin{array}{c} 1 \\ \; \end{array} & \begin{array}{c} \text{si }\ast ^{< }((x)_{1}),\ast ^{< }((x)_{2})\in F^{\tau _{A}^{e}}\text{ y }\ast ^{< }((x)_{2})=\text{resultado de} \\ \text{reemplazar algunas ocurrencias de }\psi _{1}\text{ en }\ast ^{< }((x)_{1})\text{ por }\psi _{2} \end{array} \\ 0 & \text{caso contrario} \end{array} \right. \)

es \(A\)-p.r., via el Lema 48. Finalmente \(R\) puede obtenerse haciendo composiciones adecuadas con \(R^{\prime }\). \(\Box\)

Lema 186 Sea \(B\) un alfabeto finito. Sea \(S\subseteq B^{\ast }\) un conjunto \(B\)-p.r.. El conjunto \(S^{+}\) es \(B\)-p.r.
Prueba: Notese que \(\alpha \in S^{+}\) si y solo si

\(\displaystyle (\exists z\in \mathbf{N})(\forall i\in \mathbf{N})_{i\leq Lt(z)}\ast ^{< }((z)_{i})\in S\wedge \alpha =\mathrm{\subset }_{i=1}^{Lt(z)}\ast ^{< }((z)_{i}) \)

Dejamos al lector completar los detalles faltantes. \(\Box\)
Lema 187 Los conjuntos \(ModPon^{\tau _{A}^{e}}\), \(Elec^{\tau _{A}^{e}}\), \(Reem^{\tau _{A}^{e}}\), \(Reem_{2}^{\tau _{A}^{e}}\), etc, son \(A\)-p.r..
Prueba: Veamos que \(Reem_{2}^{\tau _{A}^{e}}\) es \(A\)-p.r.. Sea \(Q:F^{\tau _{A}^{e}}\times F^{\tau _{A}^{e}}\times F^{\tau _{A}^{e}}\rightarrow \omega \) el predicado tal que \(Q(\varphi ,\psi ,\sigma )=1\) si y solo si

\((\exists \alpha \in (\forall Var)^{+})(\exists \psi _{1},\psi _{2}\in F^{\tau _{A}^{e}})\ \psi =\alpha (\psi _{1}\leftrightarrow \psi _{2})\wedge \)
\(\ \ \ \ \ \ \ \ \ \ \ \ \ \ \ \ \ \ \ \ \ \ \ Li(\psi _{1})=Li(\psi _{2})\wedge \left( (\forall v\in Var)\ v\notin Li(\psi _{1})\vee v\text{ ocurre en }\alpha \right) \)
\(\ \ \ \ \ \ \ \ \ \ \ \ \ \ \ \ \ \ \ \ \ \ \ \ \ \ \ \ \ \ \ \ \ \ \ \ \ \ \ \ \ \ \ \ \ \ \ \ \ \ \ \ \ \ \ \ \ \ \ \ \ \ \ \ \ \ \ \ \ \ \ \ \ \ \ \ \ \ \ \ \ \ \ \ \ \ \ \ \ \ \ \ \ \ \ \ \ \ \ \ \ \ \ \ \ \ \ \wedge R(\varphi ,\sigma ,\psi _{1},\psi _{2})\)
(\(R\) es el predicado dado por el Lema 185). Es facil ver que \(Q\) es \(A\)-p.r. y que \(Reem_{2}^{\tau _{A}^{e}}=Q\mid _{S^{\tau _{A}^{e}}\times S^{\tau _{A}^{e}}\times S^{\tau _{A}^{e}}}\). \(\Box\)

Lema 188 El predicado \("\psi \) se deduce de \(\varphi \) por generalizacion con constante \(c":S^{\tau _{A}^{e}}\times S^{\tau _{A}^{e}}\times Aux\rightarrow \omega \) es \(A\)-p.r..
Prueba: Notese que \(\psi \) se deduce de \(\varphi \) por generalizacion con constante \( c\) si y solo si hay una formula \(\gamma \) y una variable \(v\) tales que

- \(Li(\gamma )=\{v\}\)
- cada ocurrencia de \(v\) en \(\gamma \) es libre
- \(c\) no ocurre en \(\gamma \)
- \(\varphi =\mathrm{\downarrow }_{v}^{c}(\gamma )\wedge \psi =\forall v\gamma \)
El lector podra usando esta equivalencia facilmente justificar que el predicado en cuestion es \(A\)-p.r.. \(\Box\)

Lema 189 El predicado \("\psi \) se deduce de \(\varphi \) por eleccion con constante \( e":S^{\tau _{A}^{e}}\times S^{\tau _{A}^{e}}\times Aux\rightarrow \omega \) es \(A\)-p.r..
Definamos

\(\displaystyle AxLog^{\tau _{A}^{e}}=\{\varphi \in S^{\tau _{A}^{e}}:\varphi \text{ es un axioma logico}\} \)

Lema 190 \(AxLog^{\tau _{A}^{e}}\) es \(A\)-p.r..
Recordemos que dada \(\mathbf{\varphi }\in S^{\tau _{A}^{e}+}\), usamos \(n( \mathbf{\varphi })\) y \(\mathbf{\varphi }_{1},...,\mathbf{\varphi }_{n( \mathbf{\varphi })}\) para denotar los unicos \(n\) y \(\varphi _{1},...,\varphi _{n}\) tales que \(\mathbf{\varphi }=\varphi _{1}...\varphi _{n}\) (la unicidad es garantizada en Lema 152). Extendamos esta notacion definiendo \(\mathbf{\varphi }_{i}=\varepsilon \) para \(i=0\) o \(i >n( \mathbf{\varphi })\).

Lema 191 Las funciones
\(\displaystyle \begin{array}{ccc} S^{\tau _{A}^{e}+} & \rightarrow & \omega \\ \mathbf{\varphi } & \rightarrow & n(\mathbf{\varphi }) \end{array} \ \ \ \ \ \ \ \ \ \ \ \ \ \ \ \ \ \ \ \ \ \begin{array}{ccc} \omega \times S^{\tau _{A}^{e}+} & \rightarrow & S^{\tau _{A}^{e}}\cup \{\varepsilon \} \\ (i,\mathbf{\varphi }) & \rightarrow & \mathbf{\varphi }_{i} \end{array} \)
son \(A\)-p.r.
Recordemos que dada \(\mathbf{J}\in Just^{+}\), usamos \(n(\mathbf{J})\) y \( \mathbf{J}_{1},...,\mathbf{J}_{n(\mathbf{J})}\) para denotar los unicos \(n\) y \(J_{1},...,J_{n}\) tales que \(\mathbf{J}=J_{1}...J_{n}\) (la unicidad es garantizada en Lema 153). Extendamos esta notacion definiendo \(\mathbf{J}_{i}=\varepsilon \) para \(i=0\) o \(i >n(\mathbf{J })\).

Sea \(B\) el alfabeto que consiste en todos los simbolos que ocurren en alguna palabra de \(Just\). Es decir \(B\) consiste de los simbolos

\(\displaystyle (\ \ )\ \ ,\ \ 0\ \ 1\ \ 2\ \ 3\ \ 4\ \ 5\ \ 6\ \ 7\ \ 8\ \ 9\ \ \text{H I P O T E S N M C L }... \)

donde los ptos suspensivos indican que debemos agregar todas las letras mayusculas faltantes que aparesen en el nonbre de alguna regla.
Lema 192 \(Just\) es \(B\)-p.r. Las funciones
\(\displaystyle \begin{array}{ccc} Just^{+} & \rightarrow & \omega \\ \mathbf{J} & \rightarrow & n(\mathbf{J}) \end{array} \ \ \ \ \ \ \ \ \ \ \ \ \ \ \ \ \ \begin{array}{ccc} \omega \times Just^{+} & \rightarrow & Just\cup \{\varepsilon \} \\ (i,\mathbf{J}) & \rightarrow & \mathbf{J}_{i} \end{array} \)
son \(B\)-p.r.
Prueba: Sean

\(\displaystyle \begin{array}{rcl} R & =& \{\gamma :\gamma \text{ es el nombre de alguna regla}\} \\ T & =& \{\varepsilon \}\cup \{\text{TESIS}\bar{k}:k\in \mathbf{N}\} \end{array} \)

Notese que \(Just\) es la union de una cantidad finita de conjuntos. Uno de ellos es el conjunto
\(\displaystyle L=\{\beta \gamma (\overline{l_{1}},...,\overline{l_{k}}):\gamma \in R\text{, cada }l_{j}\in \mathbf{N}\text{ y }\beta \in T\} \)

Veremos que \(L\) es \(B\)-p.r., y dejamos al lector lor restantes. Dejamos al lector la prueba de que \(T\) es \(B\)-p.r.. Notese que \(\alpha \in L\) sii
\(\displaystyle (\exists \beta \in T)(\exists \gamma \in R)(\exists z\in \mathbf{N})\ \alpha =\beta \gamma (\left( \subset _{i=1}^{Lt(z)-1}\overline{(z)_{i}},\right) \overline{(z)_{Lt(z)}}) \)

Se deja al lector dar cotas de los cuantificadores, para poder aplicar el lema de cuantificacion acotada. \(\Box\)
Lema 193 El conjunto \(\{\mathbf{J}\in Just^{+}:\mathbf{J}\) es balanceada\(\}\) es \(B\) -p.r.
« Previous
1
2
3
4
5
6
7
8
9
10
11
12
13
14
15
16
17
18
19
20
21
22
23
24
25
26
27
28
29
30
» Next
×
Lenguaje \(\mathcal{S}^{\Sigma }\)

Entorno para trabajar con el lenguaje \(\mathcal{S}^{\Sigma }\) creado por Gabriel Cerceau:

Descargar!
Close
Read HTML code related articles on our blog to get the best tips on web content composing.
